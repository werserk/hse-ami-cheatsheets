\documentclass[12pt]{article}
\usepackage[utf8]{inputenc}
\usepackage[russian]{babel}
\usepackage{amsmath,amssymb,mathtools}
\usepackage{geometry}
\geometry{margin=2.5cm}
\usepackage{/home/werserk/Projects/hse-ami-cheatsheets/assets/styles/hse-unified}

\begin{document}

\ExamVariantTitle{Демонстрационный вариант Б}

\begin{examproblems}
\item (10 баллов) Решите разностное уравнение
\[
y_{t+3}+4y_{t+2}+5y_{t+1}+2y_{t}=12(-2)^{t}.
\]

\item (20 баллов) Решите систему дифференциальных уравнений
\[
\begin{cases}
\dot{x}=z+1,\\[4pt]
\dot{y}=y^{2} e^{3x},\\[4pt]
\dot{z}=(1+z)^{2},
\end{cases}
\qquad
\text{в области } y>0,\; z>-1,
\]
используя найденный первый интеграл.

\item (20 баллов) Даны две задачи Коши для уравнения
\[
y \frac{\partial z}{\partial x}-x \frac{\partial z}{\partial y}=0:
\]
в задаче а) дано $z=2y$ при $x=1$, а в задаче б) $z=2y$ при $x=1+y$.
В обеих задачах решение ищем в окрестности точки $(1,0)$. Найдите решение этих задач, если это возможно. Проверьте выполнение условий теоремы существования и единственности решения для подобных задач Коши (для уравнений в частных производных).

\item (30 баллов)
\begin{enumerate}
\item[а) (15 баллов)] Дано уравнение
\[
(x+2y)y'' - 3y' + y\sqrt{1-x}=0.
\]
Пусть $y_{1}(x)$ — решение этого уравнения, удовлетворяющее начальным условиям $y_{1}(0)=0$, $y_{1}'(0)=1$, соответственно $y_{2}(x)$ — другое решение этого уравнения, удовлетворяющее условиям $y_{2}(0)=3$, $y_{2}'(0)=2$. Составляют ли эти решения фундаментальную систему? Обоснуйте свой ответ. Найдите значение вронскиана, построенного по этим решениям в точке $x=-1$.

\item[б) (15 баллов)] Рассмотрите уравнение
\[
x^{2}y''+2x^{2} y'+\Bigl(\tfrac{1}{3}x^{2}-1\Bigr)y=0
\]
на полуоси $x>0$. Подобрав функцию $u(x)$ для подстановки $y=ux$, приведите уравнение к виду $z''+q(x)z=0$, и докажите, что решение $y(x)$ имеет не более одного нуля.
\item (20 баллов) Исследуйте фазовый портрет (определите тип траекторий) нелинейной системы уравнений
\[
\begin{cases}
\dot{x}=ay+x(x^{2}+y^{2}),\\[4pt]
\dot{y}=-ax+y(x^{2}+y^{2}),
\end{cases}
\]
в окрестности начала координат для всех значений параметра $a$.
\textit{Указание.} Система интегрируется, если перейти к полярным координатам. Каким положением равновесия является начало координат для линеаризованной системы?
\end{enumerate}
\end{examproblems}

\end{document}
