\section{M1. ЛОС-разностные: характеристический многочлен, частное решение, резонанс}

\subsection*{1. Тип экзаменационной задачи (полное условие)}
Найдите общее решение:
\[
y_{t+3}-y_{t+2}+4y_{t+1}-4y_{t} \;=\; 26\cdot 3^{t} + 10t + 9,
\]
где \(t\in\mathbb Z\), \((y_t)_{t\in\mathbb Z}\subset\mathbb R\) (или \(\mathbb C\)).

\subsection*{2. Универсальный алгоритм (визуальные формулы и детерминированные шаги)}

\textbf{Исходные данные и обозначения (ввод).} Дано ЛОС порядка \(n\in\mathbb N\):
\[
a_n y_{t+n} + a_{n-1} y_{t+n-1} + \dots + a_1 y_{t+1} + a_0 y_t = f(t),
\]
где \(a_n\neq 0,\ a_k\in\mathbb R\ (\text{или }\mathbb C),\ t\in\mathbb Z\).
Вводим: \(\chi(r):=r^n+b_{n-1}r^{n-1}+\dots+b_1 r + b_0\) — характеристический многочлен (после нормировки \(a_n=1\));
\(k_\chi(\lambda)\in\mathbb N\) — кратность корня \(\lambda\) в \(\chi\);
\(P_d(t)\in\mathbb R[t]\) — произвольный полином степени \(\le d\);
\(Q_{\lambda,\theta}(r):=r^2-2\lambda\cos\theta\,r+\lambda^2\).

\paragraph{Шаг 0.} \textbf{Привести уравнение к канонической форме.}\\
Разделить на \(a_n\) (если \(a_n\neq 1\)) и написать
\[
y_{t+n} + b_{n-1} y_{t+n-1} + \dots + b_1 y_{t+1} + b_0 y_t = f(t).
\]

\paragraph{Шаг 1.} \textbf{Построить \(\chi(r)\) и зафиксировать кратности корней.}\\
Выписать \(\chi(r)=r^{n}+b_{n-1}r^{n-1}+\dots+b_1 r + b_0\), найти все \(\lambda_j\) и \(k_\chi(\lambda_j)\).

\paragraph{Шаг 2.} \textbf{Записать общее решение однородной части \(y^{(h)}_t\).}\\
Для каждого корня \(\lambda\) кратности \(s=k_\chi(\lambda)\) включить базис
\[
t^0\lambda^{t},\ t^1\lambda^{t},\dots,t^{s-1}\lambda^{t};
\]
для пары \(\lambda=\rho e^{i\theta}\), \(\bar\lambda=\rho e^{-i\theta}\) — реальный базис \(\rho^{t}\cos(\theta t),\ \rho^{t}\sin(\theta t)\).

\textbf{Таблица соответствий (множитель \(\Rightarrow\) вклад в \(y^{(h)}\)):}
\[
\begin{array}{|c|c|}
\hline
\textbf{Множитель} & \textbf{Вклад в } \textbf{y(h)} \\
\hline
(r-\lambda)^s & P_{s-1}(t)\lambda^{t} \\
\hline
(r^2-2\rho\cos\theta\,r+\rho^2)^s & P_{s-1}(t)\rho^{t}\cos(\theta t),\ P_{s-1}(t)\rho^{t}\sin(\theta t) \\
\hline
\end{array}
\]

\paragraph{Шаг 3.} \textbf{Выбрать пробную форму \(y^{(p)}_t\) по атомам \(f(t)\) и признакам резонанса через \(\chi\).}\\
Разложить \(f(t)\) на атомы и применить правила из таблицы:

\begin{center}
\begin{tabular}{|c|c|c|}
\hline
\textbf{Атом} & \textbf{Резонанс?} & \textbf{Вклад в } \textbf{y(p)} \\
\hline
\(\lambda^{t}\) & \(k_\chi(\lambda) = 0\)? & \(A\,\lambda^{t}\) \\
\hline
\(P_d(t)\) & \(k_\chi(1) = 0\)? & \(c_0 + c_1 t + \dots + c_d t^d\) \\
\hline
\(\lambda^{t}P_d(t)\) & \(k_\chi(\lambda) = 0\)? & \(\lambda^{t}(c_0 + c_1 t + \dots + c_d t^d)\) \\
\hline
\(\lambda^{t}\cos(\theta t)\) & \(Q_{\lambda,\theta} \mid \chi\)? & \(\lambda^{t}(A\cos(\theta t)+B\sin(\theta t))\) \\
\(\lambda^{t}\sin(\theta t)\) & & \\
\hline
\textbf{При резонансе:} & \textbf{любая форма} & \textbf{умножить на } \(t^{s}\) \\
\hline
\end{tabular}
\end{center}

\paragraph{Шаг 4.} \textbf{Определить коэффициенты пробной формы.}\\
Подставить \(y^{(p)}\) в уравнение, сгруппировать по независимым типам (\(\lambda^t\), \(t^k\), \(\lambda^t\cos/\sin\)) и решить линейную систему на коэффициенты.

\paragraph{Шаг 5.} \textbf{Собрать общий ответ и учесть начальные условия (при наличии).}\\
Записать \(y_t=y^{(h)}_t+y^{(p)}_t\). При наличии \(y_0,\dots,y_{n-1}\) подставить соответствующие \(t\) и решить систему для констант при \(y^{(h)}\).

\subsection*{3. Сопроводительные материалы (таблицы и обозначения)}

\textbf{Атом → пробная форма (до резонанса):}
\[
\lambda^{t}\mapsto A\,\lambda^{t},\qquad
P_d(t)\mapsto c_0 + c_1 t + \dots + c_d t^d,\qquad
\lambda^{t}P_d(t)\mapsto \lambda^{t}(c_0 + c_1 t + \dots + c_d t^d),
\]
\[
\lambda^{t}\cos(\theta t),\ \lambda^{t}\sin(\theta t)\mapsto \lambda^{t}\big(A\cos(\theta t)+B\sin(\theta t)\big).
\]
\textbf{Правило резонанса (через \(\chi\)):} \(s=k_\chi(1)\) для \(P_d(t)\); \(s=k_\chi(\lambda)\) для \(\lambda^{t}P_d(t)\); если \(Q_{\lambda,\theta}\mid\chi\), умножить триг-форму на \(t^{s}\).

\subsection*{4. Применение алгоритма к объявленной задаче}

\[
y_{t+3}-y_{t+2}+4y_{t+1}-4y_{t} \;=\; 26\cdot 3^{t} + 10t + 9.
\]

\paragraph{Шаг 0.} \textbf{Канонический вид зафиксирован.}\\
Уравнение уже записано как \(y_{t+3}+(-1)y_{t+2}+4y_{t+1}+(-4)y_t=f(t)\), нормировка не требуется.

\paragraph{Шаг 1.} \textbf{Построить \(\chi(r)\) и кратности корней.}\\
\(\chi(r)=r^3-r^2+4r-4=(r-1)(r^2+4)\); корни \(1\), \(\pm 2i\), все кратности равны 1: \(k_\chi(1)=1\), \(k_\chi(\pm 2i)=1\).

\paragraph{Шаг 2.} \textbf{Записать \(y^{(h)}_t\) по найденному спектру.}\\
\[
y^{(h)}_t=C_1\cdot 1^{t}+2^{t}\Big(C_2\cos\tfrac{\pi t}{2}+C_3\sin\tfrac{\pi t}{2}\Big).
\]

\paragraph{Шаг 3.} \textbf{Выбрать \(y^{(p)}_t\) по атомам RHS и признакам резонанса на \(\chi\).}\\
\(f(t)=26\cdot 3^t + P_1(t)\), где \(P_1(t)=10t+9\). 
\begin{itemize}
\item Для \(3^t\): \(k_\chi(3)=0\) (3 не корень) \(\Rightarrow A\cdot 3^t\)
\item Для \(P_1(t)\): \(k_\chi(1)=1\) (1 — корень кратности 1) \(\Rightarrow t(\tilde a t+\tilde b)=\tilde a t^2+\tilde b t\)
\end{itemize}
Итого
\[
y^{(p)}_t=A\cdot 3^t + a\,t^2 + b\,t.
\]

\paragraph{Шаг 4.} \textbf{Найти коэффициенты пробной формы, учитывая разложение по типам.}\\
Подстановка даёт
\[
L[y^{(p)}]=26A\cdot 3^{t}+10a\,t+(9a+5b)\stackrel{!}{=}26\cdot 3^{t}+10t+9
\Rightarrow A=1,\ a=1,\ b=0.
\]
Следовательно, \(y^{(p)}_t=3^{t}+t^{2}\).

\paragraph{Шаг 5.} \textbf{Собрать общий ответ и отметить, как добавляются начальные условия.}\\
\[
\boxed{\,y_t=C_1+2^{t}\Big(C_2\cos\tfrac{\pi t}{2}+C_3\sin\tfrac{\pi t}{2}\Big)+3^{t}+t^{2}\, }.
\]
При наличии \(y_0,y_1,y_2\) — подставить \(t=0,1,2\) и решить систему для \(C_1,C_2,C_3\).
