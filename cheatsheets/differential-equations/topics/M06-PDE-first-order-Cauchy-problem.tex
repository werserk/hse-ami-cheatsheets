\section{M6. ПЧП 1-го порядка (задача Коши): нехарактеристичность, построение F по данным}

\subsection*{1. Тип экзаменационной задачи (полное условие)}
Даны две задачи Коши для уравнения
\[
y\,z_x-x\,z_y=0:
\]
а) \(z=2y\) при \(x=1\);\quad б) \(z=2y\) при \(x=1+y\).
Искать решение в окрестности \((1,0)\). Проверить условия теоремы существования–единственности.

\subsection*{2. Универсальный алгоритм (визуальные формулы и детерминированные шаги)}

\textbf{Исходные данные и обозначения (ввод).} Дано квазилинейное ПЧП 1-го порядка \(a(x,y)z_x+b(x,y)z_y=0\), где \(a,b\in C^1(\Omega\subset\mathbb R^2)\), и начальные данные на кривой \(\gamma:s\mapsto(x(s),y(s))\): \(z(\gamma(s))=\varphi(s)\).

Вводим: \(I_1(x,y)\) — первый интеграл (инвариант); \(\Delta(s)\) — определитель нехарактеристичности; \(\gamma'(s)\) — касательный вектор к кривой; \(F\) — произвольная функция.

\paragraph{Шаг 0.} \textbf{Найти характеристики.}\\
Решить систему \(\displaystyle \frac{dy}{dx}=\frac{b}{a}\) и найти первый интеграл \(I_1(x,y)=C_1\).

\paragraph{Шаг 1.} \textbf{Записать общее решение.}\\
Общее решение имеет вид \(z(x,y)=F(I_1(x,y))\), где \(F\) — произвольная функция.

\paragraph{Шаг 2.} \textbf{Сшить с начальными данными.}\\
Подставить кривую \(\gamma\) в общее решение: \(F(I_1(\gamma(s)))=\varphi(s)\).
Если \(\Delta\neq0\), то \(s=\sigma(I)\) локально и
\[
z(x,y)=\varphi(\sigma(I_1(x,y))).
\]

\paragraph{Шаг 3.} \textbf{Проверить нехарактеристичность.}\\
Вычислить \(\Delta(s)=a(\gamma)y'(s)-b(\gamma)x'(s)\).
Проверить условие \((a,b)\not\parallel \gamma'(s)\ \Leftrightarrow\ \Delta\neq0\).

\paragraph{Шаг 4.} \textbf{Сформулировать итог.}\\
\(\Delta\neq0\Rightarrow\) единственность; \(\Delta=0\Rightarrow\) ветвление или неединственность.

\subsection*{3. Сопроводительные материалы (таблицы и обозначения)}

\textbf{Быстрые инварианты:}
\begin{center}
\begin{tabular}{|c|c|c|}
\hline
\textbf{Коэффициенты } $(a,b)$ & \textbf{Уравнение } $\dfrac{dy}{dx}=\dfrac{b}{a}$ & \textbf{Инвариант } $I_1(x,y)$ \\
\hline
$(y,\ -x)$ & $-\dfrac{x}{y}$ & $x^2+y^2$ \\
\hline
$(x,\ y)$ & $\dfrac{y}{x}$ & $\dfrac{y}{x}$ \\
\hline
$(\alpha x,\ \beta y)$ & $\dfrac{\beta y}{\alpha x}$ & $\dfrac{y}{x^{\beta/\alpha}}$ \\
\hline
$(\alpha x+\beta y,\ \gamma x+\delta y)$ & $\dfrac{\gamma x+\delta y}{\alpha x+\beta y}$ & линейная замена $\Rightarrow \dfrac{\eta}{\xi^{\lambda_2/\lambda_1}}$ \\
\hline
\end{tabular}
\end{center}

\textbf{Условие нехарактеристичности:} \(\Delta(s)=a(\gamma)y'(s)-b(\gamma)x'(s)\neq0\).

\textbf{Правила диагностики:} В виде \(g(x,y)=0\): \(a g_x+b g_y\neq0\) на \(\gamma\).

\subsection*{4. Применение алгоритма к объявленной задаче}

\textbf{Дано:} \(y\,z_x-x\,z_y=0\) с двумя задачами Коши в окрестности \((1,0)\).

\paragraph{Шаг 0.} \textbf{Найти характеристики.}\\
\(a=y,\ b=-x\Rightarrow dy/dx=-x/y\Rightarrow I_1=x^2+y^2\).

\paragraph{Шаг 1.} \textbf{Записать общее решение.}\\
Общее решение: \(z=F(x^2+y^2)\).

\paragraph{Шаг 2.} \textbf{Сшить с начальными данными.}\\
\textbf{(а) }\(x=1,\ z=2y\):\\
\(I_1|_{x=1}=1+y^2,\quad \Delta=y\cdot1-(-1)\cdot0=y\).\\
В \((1,0)\): \(\Delta=0\) (характеристическая).\\
Инверсия многозначна: \(y=\pm\sqrt{I-1}\Rightarrow\)
\[
\boxed{\,z=2\,\operatorname{sgn}(y)\sqrt{x^2+y^2-1}\,}
\]
(неединственность у \(y=0\)).

\textbf{(б) }\(x=1+y,\ z=2y\):\\
\(I_1|_{x=1+y}=1+2y+2y^2,\quad \Delta=2y+1\).\\
В \((1,0)\): \(\Delta=1\neq0\) (нехарактеристическая).\\
\(I=1+2s+2s^2\Rightarrow s=\frac{-1+\sqrt{2I-1}}{2}\) (ветвь у \(s\approx0\)).
\[
\boxed{\,z(x,y)=-1+\sqrt{\,2(x^2+y^2)-1\,}\,}
\]
(единственно в окрестности \((1,0)\)).

