\documentclass[11pt,a4paper]{article}
\usepackage[utf8]{inputenc}
\usepackage[T2A]{fontenc}
\usepackage[russian]{babel}
\usepackage{amsmath, amsfonts, amssymb}
\usepackage{../../../assets/styles/hse-unified}
\usepackage{geometry}
\usepackage{parskip}
\usepackage{fancyhdr}

% Команда для отображения by werserk (чёрный цвет, без ссылки)
\newcommand{\byWerserk}{\textbf{by werserk}}

% Установим единый цвет ссылок через пакет стилей
\HSESetAccent{000000}

\geometry{margin=1.8cm}
\pagestyle{fancy}
\fancyhf{}
\fancyfoot[L]{\byWerserk}
\fancyfoot[R]{\thepage}
\renewcommand{\headrulewidth}{0pt}

% --- Каталог тем для этого предмета ---
\newcommand{\topicsBase}{preparation/topics}

\begin{document}

\HSETitle{Подготовка: Дифференциальные уравнения}{Полная версия с разборами тем и ссылками}

\vspace{1cm}

% --- Оглавление ---
{
	\hypersetup{linkcolor=black}
	\tableofcontents
}

\newpage

% --- Подключайте темы ниже ---
% Примеры (создайте соответствующие файлы в preparation/topics/):
\section{Линейные разностные уравнения с постоянными коэффициентами (ЛОС)}

\subsection*{1. Тип экзаменационной задачи (полное условие)}
Найдите общее решение:
\[
y_{t+3}-y_{t+2}+4y_{t+1}-4y_{t} \;=\; 26\cdot 3^{t} + 10t + 9,
\]
где \(t\in\mathbb Z\), \((y_t)_{t\in\mathbb Z}\subset\mathbb R\) (или \(\mathbb C\)).

\subsection*{2. Универсальный алгоритм (визуальные формулы и детерминированные шаги)}

\textbf{Исходные данные и обозначения (ввод).} Дано ЛОС порядка \(n\in\mathbb N\):
\[
a_n y_{t+n} + a_{n-1} y_{t+n-1} + \dots + a_1 y_{t+1} + a_0 y_t = f(t),
\]
где \(a_n\neq 0,\ a_k\in\mathbb R\ (\text{или }\mathbb C),\ t\in\mathbb Z\).
Вводим: \(\chi(r):=r^n+b_{n-1}r^{n-1}+\dots+b_1 r + b_0\) — характеристический многочлен (после нормировки \(a_n=1\));
\(k_\chi(\lambda)\in\mathbb N\) — кратность корня \(\lambda\) в \(\chi\);
\(P_d(t)\in\mathbb R[t]\) — произвольный полином степени \(\le d\);
\(Q_{\lambda,\theta}(r):=r^2-2\lambda\cos\theta\,r+\lambda^2\).

\paragraph{Шаг 0.} \textbf{Привести уравнение к канонической форме.}\\
Разделить на \(a_n\) (если \(a_n\neq 1\)) и написать
\[
y_{t+n} + b_{n-1} y_{t+n-1} + \dots + b_1 y_{t+1} + b_0 y_t = f(t).
\]

\paragraph{Шаг 1.} \textbf{Построить \(\chi(r)\) и зафиксировать кратности корней.}\\
Выписать \(\chi(r)=r^{n}+b_{n-1}r^{n-1}+\dots+b_1 r + b_0\), найти все \(\lambda_j\) и \(k_\chi(\lambda_j)\).

\paragraph{Шаг 2.} \textbf{Записать общее решение однородной части \(y^{(h)}_t\).}\\
Для каждого корня \(\lambda\) кратности \(s=k_\chi(\lambda)\) включить базис
\[
t^0\lambda^{t},\ t^1\lambda^{t},\dots,t^{s-1}\lambda^{t};
\]
для пары \(\lambda=\rho e^{i\theta}\), \(\bar\lambda=\rho e^{-i\theta}\) — реальный базис \(\rho^{t}\cos(\theta t),\ \rho^{t}\sin(\theta t)\).

\paragraph{Шаг 3.} \textbf{Выбрать пробную форму \(y^{(p)}_t\) по атомам \(f(t)\) и признакам резонанса через \(\chi\).}\\
Разложить \(f(t)\) на атомы и применить правила из таблицы:

\begin{center}
\begin{tabular}{|c|c|c|}
\hline
\textbf{Атом} & \textbf{Резонанс?} & \textbf{Вклад в пробную форму} \\
\hline
\(\lambda^{t}\) & \(k_\chi(\lambda) = 0\)? & \(A\,\lambda^{t}\) \\
\hline
\(P_d(t)\) & \(k_\chi(1) = 0\)? & \(\sum_{k=0}^d c_k t^k\) \\
\hline
\(\lambda^{t}P_d(t)\) & \(k_\chi(\lambda) = 0\)? & \(\lambda^{t}\sum_{k=0}^d c_k t^k\) \\
\hline
\(\lambda^{t}\cos(\theta t)\) & \(Q_{\lambda,\theta} \mid \chi\)? & \(\lambda^{t}(A\cos(\theta t)+B\sin(\theta t))\) \\
\(\lambda^{t}\sin(\theta t)\) & & \\
\hline
\textbf{При резонансе:} & \textbf{любая форма} & \textbf{умножить на } \(t^{s}\) \\
\hline
\end{tabular}
\end{center}

\paragraph{Шаг 4.} \textbf{Определить коэффициенты пробной формы.}\\
Подставить \(y^{(p)}\) в уравнение, сгруппировать по независимым типам (\(\lambda^t\), \(t^k\), \(\lambda^t\cos/\sin\)) и решить линейную систему на коэффициенты.

\paragraph{Шаг 5.} \textbf{Собрать общий ответ и учесть начальные условия (при наличии).}\\
Записать \(y_t=y^{(h)}_t+y^{(p)}_t\). При наличии \(y_0,\dots,y_{n-1}\) подставить соответствующие \(t\) и решить систему для констант при \(y^{(h)}\).

\subsection*{3. Сопроводительные материалы (таблицы и обозначения)}

\textbf{Атом → пробная форма (до резонанса):}
\[
\lambda^{t}\mapsto A\,\lambda^{t},\qquad
P_d(t)\mapsto \sum_{k=0}^d c_k t^k,\qquad
\lambda^{t}P_d(t)\mapsto \lambda^{t}\sum_{k=0}^d c_k t^k,
\]
\[
\lambda^{t}\cos(\theta t),\ \lambda^{t}\sin(\theta t)\mapsto \lambda^{t}\big(A\cos(\theta t)+B\sin(\theta t)\big).
\]
\textbf{Правило резонанса (через \(\chi\)):} \(s=k_\chi(1)\) для \(P_d(t)\); \(s=k_\chi(\lambda)\) для \(\lambda^{t}P_d(t)\); если \(Q_{\lambda,\theta}\mid\chi\), умножить триг-форму на \(t^{s}\).

\subsection*{4. Применение алгоритма к объявленной задаче}

\[
y_{t+3}-y_{t+2}+4y_{t+1}-4y_{t} \;=\; 26\cdot 3^{t} + 10t + 9.
\]

\paragraph{Шаг 0.} \textbf{Канонический вид зафиксирован.}\\
Уравнение уже записано как \(y_{t+3}+(-1)y_{t+2}+4y_{t+1}+(-4)y_t=f(t)\), нормировка не требуется.

\paragraph{Шаг 1.} \textbf{Построить \(\chi(r)\) и кратности корней.}\\
\(\chi(r)=r^3-r^2+4r-4=(r-1)(r^2+4)\); корни \(1\), \(\pm 2i\), все кратности равны 1: \(k_\chi(1)=1\), \(k_\chi(\pm 2i)=1\).

\paragraph{Шаг 2.} \textbf{Записать \(y^{(h)}_t\) по найденному спектру.}\\
\[
y^{(h)}_t=C_1\cdot 1^{t}+2^{t}\Big(C_2\cos\tfrac{\pi t}{2}+C_3\sin\tfrac{\pi t}{2}\Big).
\]

\paragraph{Шаг 3.} \textbf{Выбрать \(y^{(p)}_t\) по атомам RHS и признакам резонанса на \(\chi\).}\\
\(f(t)=26\cdot 3^t + P_1(t)\), где \(P_1(t)=10t+9\). 
\begin{itemize}
\item Для \(3^t\): \(k_\chi(3)=0\) (3 не корень) \(\Rightarrow A\cdot 3^t\)
\item Для \(P_1(t)\): \(k_\chi(1)=1\) (1 — корень кратности 1) \(\Rightarrow t(\tilde a t+\tilde b)=\tilde a t^2+\tilde b t\)
\end{itemize}
Итого
\[
y^{(p)}_t=A\cdot 3^t + a\,t^2 + b\,t.
\]

\paragraph{Шаг 4.} \textbf{Найти коэффициенты пробной формы, учитывая разложение по типам.}\\
Подстановка даёт
\[
L[y^{(p)}]=26A\cdot 3^{t}+10a\,t+(9a+5b)\stackrel{!}{=}26\cdot 3^{t}+10t+9
\Rightarrow A=1,\ a=1,\ b=0.
\]
Следовательно, \(y^{(p)}_t=3^{t}+t^{2}\).

\paragraph{Шаг 5.} \textbf{Собрать общий ответ и отметить, как добавляются начальные условия.}\\
\[
\boxed{\,y_t=C_1+2^{t}\Big(C_2\cos\tfrac{\pi t}{2}+C_3\sin\tfrac{\pi t}{2}\Big)+3^{t}+t^{2}\, }.
\]
При наличии \(y_0,y_1,y_2\) — подставить \(t=0,1,2\) и решить систему для \(C_1,C_2,C_3\).

\section{ПЧП первого порядка. Инварианты характеристик}\label{sec:pde-first-order}

\HSEDefinition{\Term{Линейное ПЧП первого порядка} в $\mathbb{R}^3$ — уравнение вида $a(x,y,z)\,u_x + b(x,y,z)\,u_y + c(x,y,z)\,u_z = 0$, где $a,b,c$ — непрерывные функции. Метод характеристик приводит к ОДУ-системе $\dot x = a(x,y,z),\ \dot y = b(x,y,z),\ \dot z = c(x,y,z)$.}

% Подключаем все подглавы по ПЧП первого порядка
\subsection{Вводная информация и единый алгоритм}\label{sec:pde-intro-algorithm}

Рассматриваем линейное ПЧП первого порядка в $\mathbb{R}^3$:
\[
a(x,y,z)\,u_x + b(x,y,z)\,u_y + c(x,y,z)\,u_z \;=\; 0,
\]
где $a,b,c$ — непрерывны. Метод характеристик приводит к ОДУ-системе
\[
\dot x = a(x,y,z),\quad \dot y = b(x,y,z),\quad \dot z = c(x,y,z).
\]

\HSEDefinition{\Term{Инвариант (первый интеграл)} $I(x,y,z)$ — это $C^1$-функция, постоянная вдоль характеристик, т.е.
\[
\frac{d}{ds}I(x(s),y(s),z(s))= \nabla I \cdot (a,b,c) = a I_x + b I_y + c I_z = 0.
\]
Общее решение ПЧП имеет вид $u=F(I_1,I_2)$, где $I_1,I_2$ — два независимых инварианта.}

\subsubsection{Единый алгоритм (5 шагов)}

\begin{enumerate}
\item \textbf{Цель.} Распознать форму и записать \emph{характеристики}.

    \textbf{Действие.} Выписать $\dot x=a,\ \dot y=b,\ \dot z=c$ и безпараметрические равенства
    $\displaystyle \frac{dx}{a}=\frac{dy}{b}=\frac{dz}{c}$.

\item \textbf{Цель.} Найти \emph{первый инвариант $I_1$}.

    \textbf{Действие.} Решить одну пару, напр. $\displaystyle \frac{dy}{dx}=\frac{b}{a}$, получить семейство кривых уровня
    $\Phi_1(x,y)=\text{const}$, положить $I_1=\Phi_1$ (или любой эквивалент).

\item \textbf{Цель.} Найти \emph{второй инвариант $I_2$}.

    \textbf{Действие.} Зафиксировать $I_1=\text{const}$ (т.е. связь $y=\Psi(x;I_1)$) и решить ОДУ по $z$ из
    $\displaystyle \frac{dz}{dx}=\frac{c(x,\Psi(x;I_1),z)}{a(x,\Psi(x;I_1))}$.
    \begin{itemize}
    \item Если $c\equiv 0$, взять $I_2=z$.
    \item Если $c=\alpha(x,y)\,z+\beta(x,y)$, получаем линейное ОДУ $z'=\tilde{\alpha}(x;I_1)z+\tilde{\beta}(x;I_1)$.
    \emph{Интегрирующий множитель} $M(x;I_1)=\exp\!\big(-\int \tilde{\alpha}\,dx\big)$, и
    \[
    \boxed{\,I_2 = z\,M(x;I_1) - \int \tilde{\beta}(x;I_1)\,M(x;I_1)\,dx\,}
    \]
    — константа вдоль характеристики.
    \item Если отделяется $z$ — разделить переменные и получить интегральный инвариант.
    \end{itemize}

\item \textbf{Цель.} Сформировать \emph{общее решение}.

    \textbf{Действие.} Записать $u=F(I_1,I_2)$.

\item \textbf{Цель.} Проверить \emph{независимость} $I_1,I_2$.

    \textbf{Действие.} Убедиться, что $dI_1\wedge dI_2\neq 0$ (или $\nabla I_1\times \nabla I_2\neq 0$) в рабочей области.
\end{enumerate}

\subsection{Как быстро находить $I_1$: детерминированные «детекторы»}\label{sec:pde-first-invariant}

Сначала можно убрать общий ненулевой множитель: $(a,b)\sim (\tilde a,\tilde b)$ дают те же инварианты.

\begin{itemize}
\item \textbf{Радиальный тип:} $(a,b)=(x,y)$ $\Rightarrow$ $\displaystyle \frac{dy}{dx}=\frac{y}{x}$ $\Rightarrow$ $I_1=\dfrac{y}{x}$.

\item \textbf{Вращение:} $(a,b)=(y,-x)$ $\Rightarrow$ $y\,dy=-x\,dx$ $\Rightarrow$ $I_1=x^2+y^2$.

\item \textbf{Диагональный линейный:} $(a,b)=(\alpha x,\beta y)$ $\Rightarrow$ $I_1=\dfrac{y}{x^{\beta/\alpha}}$.

\item \textbf{Общий линейный однородный:} $(a,b)=(\alpha x+\beta y,\ \gamma x+\delta y)$.
Через левые собственные векторы $M^\top$: взять $\xi=\mathbf w_1\!\cdot\!(x,y)$, $\eta=\mathbf w_2\!\cdot\!(x,y)$, тогда
$I_1=\eta/\xi^{\lambda_2/\lambda_1}$.
Альтернатива: подстановка $v=y/x$ всегда даёт $\dfrac{dv}{dx}=\dfrac{\Phi(v)}{x}$, интегрируется в логарифмах.

\item \textbf{Однородность одного порядка $d$:} если $a(\lambda x,\lambda y)=\lambda^d a$ и $b(\lambda x,\lambda y)=\lambda^d b$, то
$\,v=y/x$ ведёт к $I_1=x\,G(v)$.
\end{itemize}

\subsection{Как добирать $I_2$: стандартные ветки}\label{sec:pde-second-invariant}

\begin{itemize}
\item $c\equiv 0$ $\Rightarrow$ $I_2=z$.

\item $c=\mu(x,y)\,z$ $\Rightarrow$ $z'=\tilde{\alpha}(x;I_1)\,z$ $\Rightarrow$ $I_2=z\,\exp\!\big(-\int \tilde{\alpha}\,dx\big)$.

\item $c=\mu(x,y)\,z+\nu(x,y)$ $\Rightarrow$ линейное ОДУ, формула выше с $\tilde{\beta}\not\equiv0$.

\item $c=\mu(x,y)$ (не зависит от $z$) $\Rightarrow$ $I_2=z-\int \frac{\mu(x,\Psi(x;I_1))}{a(x,\Psi(x;I_1))}\,dx$.
\end{itemize}


\section{Нелинейные 2D-системы: равновесия, линеаризация, негиперболика}\label{sec:nonlinear-2d-systems}

\HSEDefinition{\Term{Нелинейная 2D-система} — автономная система вида $\dot{x} = f(x, y)$, $\dot{y} = g(x,y)$, где $f, g \in C^1$. Метод линеаризации применяется для анализа положений равновесия и их типов.}

\subsection{Где применяется метод линеаризации (признаки «наш случай»)}
\begin{itemize}
    \item Дано: автономная система $\dot{x} = f(x, y)$, $\dot{y} = g(x,y)$, $f, g \in C^1$.
    \item Спрашивают: положения равновесия, их тип и эскиз фазового портрета в окрестности.
    \item В точке(ах) равновесия Якоби $J = \begin{pmatrix} f_x & f_y \\ g_x & g_y \end{pmatrix}$ удовлетворяет $\det J \neq 0$ (гиперболическая точка).
    \item Если $\det J = 0$ или $D = 0$ --- это уже не МЗ (негиперболика/граница случаев).
\end{itemize}

\subsection{Единый 5-шаговый алгоритм (используем во всех примерах)}

\subsubsection{Шаг 1. Цель: найти все равновесия.}
Действие: решить $f(x, y) = 0, g(x, y) = 0$.

\subsubsection{Шаг 2. Цель: получить линеаризацию.}
Действие: в каждой найденной точке вычислить Якоби $J$.

\subsubsection{Шаг 3. Цель: классифицировать тип точки по числам.}
Действие (единственное ветвление строго по знакам):
\begin{itemize}
    \item Если $\det J < 0 \to$ седло (неустойч.).
    \item Если $\det J > 0$:
    \begin{itemize}
        \item посчитать $D = \text{tr}^2 - 4 \det$.
        \item если $D > 0$: $\text{tr} < 0 \to$ устойч. узел, $\text{tr} > 0 \to$ неустойч. узел;
        \item если $D < 0$: $\text{tr} < 0 \to$ устойч. фокус, $\text{tr} > 0 \to$ неустойч. фокус.
    \end{itemize}
\end{itemize}

\subsubsection{Шаг 4. Цель: зафиксировать направления и устойчивость.}
Действие: указать «куда текут» траектории (в/из точки) и, при седле, назвать две устойчивые/неустойчивые сепаратрисы (вдоль собственных направлений $J$).

\subsubsection{Шаг 5. Цель: нарисовать локальный эскиз.}
Действие: около каждой точки нанести тип (узел/фокус/седло), стрелки по устойчивости, грубо ориентируясь на нулевые изоклины $f = 0, g = 0$ для знаков $\dot{x}, \dot{y}$.

\HSENote{Этот алгоритм является универсальным для анализа нелинейных 2D-систем и позволяет систематически подходить к решению задач на классификацию равновесных точек.}

% \input{\topicsBase/03-systems}
% \input{\topicsBase/04-numerical}

\end{document}


