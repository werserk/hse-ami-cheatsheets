\documentclass[11pt,a4paper]{article}
\usepackage[utf8]{inputenc}
\usepackage[T2A]{fontenc}
\usepackage[russian]{babel}
\usepackage{amsmath, amsfonts, amssymb}
\usepackage{geometry}
\usepackage{parskip}
\usepackage{fancyhdr}
\usepackage{hyperref}
\hypersetup{
    colorlinks=true,
    linkcolor=black,
    urlcolor=black,
    citecolor=black
}

% Команда для отображения by werserk (чёрный цвет, без ссылки)
\newcommand{\byWerserk}{\textbf{by werserk}}


\geometry{margin=1.8cm}
\pagestyle{fancy}
\fancyhf{}
\fancyfoot[L]{\byWerserk}
\fancyfoot[R]{\thepage}
\renewcommand{\headrulewidth}{0pt}

% --- Каталог тем для этого предмета ---
\newcommand{\topicsBase}{topics}

\begin{document}

% \HSETitle{Подготовка: Дифференциальные уравнения}{Полная версия с разборами тем и ссылками}
\title{Подготовка: Дифференциальные уравнения}
\author{Полная версия с разборами тем и ссылками}
\maketitle
\thispagestyle{fancy}

\vspace{1cm}

% --- Оглавление ---
\tableofcontents

\newpage

% --- Подключайте темы ниже ---
% Примеры (создайте соответствующие файлы в topics/):
\section{Разностные уравнения}\label{sec:difference-equations}

\HSEDefinition{\Term{Разностное уравнение} — соотношение между элементами последовательности (или векторной последовательности), задающее правило перехода от шага $t$ к $t+1$ или к нескольким последующим шагам. В этом разделе: ЛОРУ (линейные однородные разностные уравнения) и их расширения.}

% Подключаем все подглавы по разностным уравнениям
\subsection{Однородные линейные разностные уравнения}\label{subsec:homogeneous}

\begin{center}
\fbox{\parbox{0.9\textwidth}{
\textbf{Пример.} Решите однородное линейное разностное уравнение:
\begin{align}
y_{t+3} - 3y_{t+2} + 6y_{t+1} - 4y_t = 0
\end{align}
}}
\end{center}

\textbf{Определение.} Линейное однородное разностное уравнение порядка $k$ с постоянными коэффициентами:
\begin{align}
a_t + c_1 a_{t-1} + c_2 a_{t-2} + \dots + c_k a_{t-k} = 0, \quad c_k \neq 0
\end{align}

Пара «уравнение + $k$ начальных условий» задаёт единственное решение.

\textbf{Идея решения: метод характеристических корней.} Полагаем $a_t = r^t$ $\Rightarrow$
\begin{align}
r^t (1 + c_1 r^{-1} + c_2 r^{-2} + \dots + c_k r^{-k}) = 0 \iff r^k + c_1 r^{k-1} + \dots + c_k = 0
\end{align}

т.е. характеристический многочлен $\chi(r) = r^k + c_1 r^{k-1} + \dots + c_k$. Его корни целиком описывают форму общего решения.

\textit{Обозначения:} $p_j(t), q_j(t)$ — полиномы по $t$ степени $\le j$.

\begin{table}[h!]
\centering
\caption{Выбор формы решения по типу корней характеристического многочлена}
\label{tab:form-choices}
\begin{tabular}{|l|l|}
\hline
\textbf{Условия на корни} & \textbf{Вклад в решение} \\
\hline
Действительный корень $r$ кратности $m\ge 1$ &
$p_{m-1}(t)\, r^{\,t}$ \\
\hline
Комплексно-сопряжённая пара $\rho e^{\pm i\theta}$ кратности $s\ge 1$ &
$\rho^{\,t}\big(p_{s-1}(t)\cos(\theta t)+q_{s-1}(t)\sin(\theta t)\big)$ \\
\hline
\end{tabular}

\vspace{0.5em}
\emph{Итоговое общее решение — сумма форм всех корней:}
\begin{align*}
a_t = \sum_{j} p_{m_j-1}(t)\, r_j^{\,t}
\; + \sum_{k} \rho_k^{\,t}\big(p_{s_k-1}(t)\cos(\theta_k t)+q_{s_k-1}(t)\sin(\theta_k t)\big),
\end{align*}
где $r_j$ — действительные корни кратности $m_j$, $\rho_k e^{\pm i\theta_k}$ — комплексно-сопряжённые корни кратности $s_k$.
Сумма кратностей всех корней равна порядку $k$.
\end{table}

\textbf{Начальные условия.} Подставляем $t = 0, 1, \dots, k - 1$ в общий вид, решаем линейную систему на $\alpha$-коэффициенты.

\begin{center}
\fbox{\parbox{0.9\textwidth}{
\textbf{Алгоритм.}
\begin{enumerate}
\item \textbf{Нормализация.} Привести уравнение к виду $a_t+\sum_{j=1}^{k} c_j a_{t-j}=0$, $c_k\neq 0$.
\item \textbf{Характеристический многочлен.} Записать $\chi(r)=r^{k}+c_1 r^{k-1}+\dots+c_k$.
\item \textbf{Корни и кратности.} Найти корни $r$ и их кратности $m$ (\,$\sum m = k$\,).
\item \textbf{Общий вид решения (см.~таблицу~\ref{tab:form-choices}).}
Для каждого корня/пары взять соответствующий вклад из таблицы и сложить их.
\item \textbf{Подгонка под начальные условия.} Подставить $k$ заданных значений подряд и решить линейную систему для постоянных.
\end{enumerate}
}}
\end{center}

% Ссылки на задачи: добавьте URL/refs сюда
% Например: Сборник И. Иванов, гл. 2; задачи 1-20.

\subsection{Минимальное ЛОУ: метод аннигиляторов}\label{subsec:min-lre}

\begin{center}
\fbox{\parbox{0.92\textwidth}{
\textbf{TL;DR:} минимальное ЛОУ (минимальная однородная линейная рекуррент с постоянными коэффициентами), которое имеет данные последовательности в качестве решений, строится так:
\begin{enumerate}
\item к каждой заданной последовательности приписать аннигилятор (многочлен от $E$);
\item взять НОК этих аннигиляторов как многочлен $L(\lambda)$;
\item развернуть $L(E)\,y=0$ в явную рекурренту. Степень $L$ — минимальный порядок.
\end{enumerate}
}}
\end{center}

\subsubsection*{Методика (детерминированно)}

Пусть даны частные решения $y^{(1)},\ldots,y^{(m)}$.

\paragraph{Шаг 1. Атом $\to$ аннигилятор}
Для каждой последовательности выпиши минимальный аннигилирующий многочлен:
\begin{table}[h!]
\centering
\caption{Атом $\to$ аннигилятор}
\label{tab:atom-to-annihilator}
\begin{tabular}{|l|l|}
\hline
\textbf{Атом (последовательность)} & \textbf{Минимальный аннигилятор $L(\lambda)$} \\
\hline
$r^t$ & $(\lambda - r)$ \\
\hline
$t^k r^t$ & $(\lambda - r)^{k+1}$ \\
\hline
$\rho^t \cos(\omega t)$, $\rho^t \sin(\omega t)$ & $Q_{\rho,\omega}(\lambda)=\lambda^2-2\rho\cos\omega\,\lambda+\rho^2$ \\
\hline
$t^k \, \rho^t \, \cos/\sin(\omega t)$ & $Q_{\rho,\omega}(\lambda)^{k+1}$ \\
\hline
$t^k$ & $(\lambda-1)^{k+1}$ \\
\hline
$(-1)^t$ & $(\lambda+1)$ \\
\hline
\end{tabular}
\end{table}

\paragraph{Шаг 2. Собрать общий аннигилятор}
Возьмём НОК (наименьший общий кратный) всех многочленов из шага 1:
\[
 L(\lambda)=\operatorname{lcm}\big(L_1(\lambda),\ldots,L_m(\lambda)\big).
\]
При одинаковых базах/частотах выбирается максимальная кратность (а не сумма).

\paragraph{Шаг 3. Развернуть в рекуррент}
Если $L(\lambda)=\lambda^k+c_1\lambda^{k-1}+\cdots+c_k$, то искомое уравнение:
\[
 \boxed{\,y_{t+k}+c_1 y_{t+k-1}+\cdots+c_k y_t=0\,}.
\]

\paragraph{Минимальность.}
Любой многочлен $P(E)$, который зануляет все данные последовательности, обязан делиться на $L(E)$. Поэтому $\deg L$ — минимально возможный порядок.

\bigskip

\begin{center}
\fbox{\parbox{0.92\textwidth}{
\subsubsection*{Простой пример}

Дано:
\[
y_t^{(1)}=3^t,\qquad y_t^{(2)}=(-2)^t.
\]

\paragraph{Шаг 1.} Аннигиляторы: $(\lambda-3)$ и $(\lambda+2)$.

\paragraph{Шаг 2.} НОК:
\[
(\lambda-3)(\lambda+2)=\lambda^2-\lambda-6.
\]

\paragraph{Шаг 3.} Рекуррентное соотношение (развёртка):
\[
\boxed{\,y_{t+2}-y_{t+1}-6y_t=0\,}.
\]

Проверка: последовательности $3^t$ и $(-2)^t$ действительно являются решениями; порядок $2$ минимален.
}}
\end{center}

\bigskip

\begin{center}
\fbox{\parbox{0.92\textwidth}{
\subsubsection*{Пример посложнее}

Дано:
\[
y^{(1)}_t=2^t,\qquad y^{(2)}_t=t2^t,\qquad y^{(3)}_t=(-1)^t,\qquad y^{(4)}_t=3^t\cos\frac{\pi t}{3}.
\]

\subsubsection*{Шаг 1. Аннигиляторы}
\begin{itemize}
  \item Для $2^t$: $(\lambda-2)$.
  \item Для $t2^t$: $(\lambda-2)^{2}$ (кратность на 1 больше из-за множителя $t$).
  \item Для $(-1)^t$: $(\lambda+1)$.
  \item Для $3^t\cos\frac{\pi t}{3}$:
  \[
  Q_{3,\pi/3}(\lambda)=\lambda^2-2\cdot 3\cos\frac{\pi}{3}\,\lambda+3^2
  =\lambda^2-3\lambda+9.
  \]
\end{itemize}

\subsubsection*{Шаг 2. НОК}
Учитываем максимальную кратность по базе $2$, значит
\[
L(\lambda)=(\lambda-2)^2(\lambda+1)(\lambda^2-3\lambda+9).
\]

\subsubsection*{Шаг 3. Развёртка}
Сначала
\[
(\lambda-2)^2(\lambda+1)=(\lambda^2-4\lambda+4)(\lambda+1)=\lambda^3-3\lambda^2+4.
\]
Умножаем на $\lambda^2-3\lambda+9$:
\[
L(\lambda)=(\lambda^3-3\lambda^2+4)(\lambda^2-3\lambda+9)
=\lambda^5-6\lambda^4+18\lambda^3-23\lambda^2-12\lambda+36.
\]

Соответствующая рекуррентная формула (коэффициенты берём по степеням $\lambda$) будет
\[
\boxed{\,y_{t+5}-6y_{t+4}+18y_{t+3}-23y_{t+2}-12y_{t+1}+36y_t=0\,.}
\]

\paragraph{Комментарий.} Это и есть минимальное ЛОУ, annihilator которого равен $L(\lambda)$.
}}
\end{center}

\input{\topicsBase/03-inhomogeneous-difference}
\subsection{Системы разностных уравнений}\label{subsec:systems}

\HSEExample{\textbf{Пример.} Решите систему разностных уравнений: $\begin{cases} x_{t+1} = 4x_t + y_t \\ y_{t+1} = 2y_t \end{cases}$, с начальными условиями $\mathbf{x}_0 = \begin{pmatrix} 1 \\ 1 \end{pmatrix}$.}

\HSEDefinition{Система линейных разностных уравнений первого порядка с постоянными коэффициентами: $\mathbf{x}_{t+1} = A\mathbf{x}_t$, где задано $\mathbf{x}_0$, $A \in \mathbb{R}^{n \times n}$, $\mathbf{x}_t \in \mathbb{R}^n$. Цель: найти $\mathbf{x}_t = A^t\mathbf{x}_0$.}

\textbf{Идея решения:} возведение матрицы $A$ в степень $t$. Для этого используем спектральное разложение матрицы.

\textit{Обозначения:} $\lambda_i$ — собственные значения, $\mathbf{v}_i$ — собственные векторы, $\chi_A(\lambda) = \det(\lambda I - A)$ — характеристический многочлен.

\begin{table}[h!]
\centering
\caption{Выбор метода решения по типу собственных значений}
\label{tab:method-selection}
\begin{tabular}{|l|l|}
\hline
\textbf{Условия на собственные значения} & \textbf{Рекомендуемый метод} \\
\hline
Разные действительные корни, полный базис собственных векторов & Диагонализация \\
\hline
Повторный корень, недостаточно собственных векторов & Жорданова форма \\
\hline
Комплексно-сопряжённая пара & Реальный блок поворота \\
\hline
Матрица $2 \times 2$ (любой случай) & Кэли–Гамильтон \\
\hline
\end{tabular}
\end{table}

\bigskip
\hrule
\bigskip

\subsubsection*{1. Диагонализация}

\textbf{Условие применения:} матрица $A$ имеет $n$ линейно независимых собственных векторов (диагонализуема).

\textbf{Теорема.} Если $A$ диагонализуема, то $A = S\Lambda S^{-1}$, где $\Lambda = \text{diag}(\lambda_1, \ldots, \lambda_n)$ — диагональная матрица собственных значений, $S = [\mathbf{v}_1, \ldots, \mathbf{v}_n]$ — матрица собственных векторов.

\HSEAlgorithm{\textbf{Алгоритм диагонализации.}\begin{enumerate}\item Характеристический многочлен: $\chi_A(\lambda) = \det(\lambda I - A)$.\item Собственные значения: решить $\chi_A(\lambda) = 0$.\item Собственные векторы: для каждого $\lambda_i$ решить $(A - \lambda_i I)\mathbf{v}_i = \mathbf{0}$.\item Проверка диагонализуемости: $\det S \neq 0$.\item Диагонализация: $A = S\Lambda S^{-1}$.\item Возведение в степень: $A^t = S\Lambda^t S^{-1}$.\item Решение: $\mathbf{x}_t = A^t\mathbf{x}_0$.\end{enumerate}}

\HSEExample{\textbf{Пример.} Та же система. Решение: \; $A = \begin{pmatrix} 4 & 1 \\ 0 & 2 \end{pmatrix}$. Найдём собственные значения и векторы, построим $S,\Lambda, S^{-1}$ и получим $A^t = \begin{pmatrix} 4^t & \frac{4^t - 2^t}{2} \\ 0 & 2^t \end{pmatrix}$, откуда $\mathbf{x}_t = \begin{pmatrix} \frac{3 \cdot 4^t - 2^t}{2} \\ 2^t \end{pmatrix}$.}

\bigskip
\hrule
\bigskip

\subsubsection*{2. Жорданова форма (повторный корень)}

\textbf{Условие применения:} матрица $A$ имеет повторное собственное значение, но недостаточно собственных векторов для диагонализации.

\textbf{Теорема.} Если $A$ имеет единственное собственное значение $\lambda$ кратности $n$, то $A = \lambda I + N$, где $N$ — нильпотентная матрица ($N^m = 0$ для некоторого $m \leq n$).

\textbf{Ключевая идея:} используем биномиальную формулу для $(I + \lambda^{-1}N)^t$.

\HSEAlgorithm{\textbf{Алгоритм Жордановой формы.}\begin{enumerate}\item Собственное значение $\lambda$.\item Нильпотентная матрица $N = A - \lambda I$.\item Индекс нильпотентности: $N^m=0$.\item Формула: $A^t = \lambda^t \sum_{k=0}^{m-1} \binom{t}{k} (\lambda^{-1}N)^k$.\item Решение: $\mathbf{x}_t = A^t\mathbf{x}_0$.\end{enumerate}}

\HSEExample{\textbf{Пример.} Система $\begin{cases} x_{t+1} = 2x_t + y_t \\ y_{t+1} = 2y_t \end{cases}$. Для $A = \begin{pmatrix} 2 & 1 \\ 0 & 2 \end{pmatrix}$ имеем $\lambda=2$, $N=\begin{pmatrix}0 & 1 \\ 0 & 0\end{pmatrix}$, $m=2$, поэтому $A^t = 2^t\begin{pmatrix} 1 & t/2 \\ 0 & 1 \end{pmatrix}$ и $\mathbf{x}_t = 2^t \begin{pmatrix} 1 + t/2 \\ 1 \end{pmatrix}$.}

\bigskip
\hrule
\bigskip

\subsubsection*{3. Комплексная пара (реальный блок)}

\textbf{Условие применения:} матрица $A$ $2 \times 2$ имеет комплексно-сопряжённые собственные значения $\lambda = \rho e^{\pm i\theta}$.

\textbf{Теорема.} Для матрицы $A$ $2 \times 2$ с комплексными корнями $\lambda = \rho e^{\pm i\theta}$ справедливо:
\[
A^t = \rho^t \begin{pmatrix} \cos(t\theta) & -\sin(t\theta) \\ \sin(t\theta) & \cos(t\theta) \end{pmatrix}
\]

\textbf{Ключевая идея:} комплексные корни соответствуют повороту с масштабированием в вещественном пространстве.

\HSEAlgorithm{\textbf{Алгоритм для комплексной пары.}\begin{enumerate}\item Проверка: $(\operatorname{tr}A)^2 - 4\det A < 0$.\item $\rho = \sqrt{\det A}$, $\cos\theta = \frac{\operatorname{tr}A}{2\rho}$.\item $A^t = \rho^t \begin{pmatrix} \cos(t\theta) & -\sin(t\theta) \\ \sin(t\theta) & \cos(t\theta) \end{pmatrix}$.\item Решение: $\mathbf{x}_t = A^t\mathbf{x}_0$.\end{enumerate}}

\HSEExample{\textbf{Пример.} $A = \begin{pmatrix} 1 & -1 \\ 1 & 1 \end{pmatrix}$: $(\operatorname{tr}A)^2-4\det A = -4 < 0$, $\rho=\sqrt{2}$, $\theta=\pi/4$. Тогда $A^t = (\sqrt{2})^t R(t\theta)$ и $\mathbf{x}_t = (\sqrt{2})^t \begin{pmatrix} \cos\tfrac{\pi t}{4} \\ \sin\tfrac{\pi t}{4} \end{pmatrix}$.}

\bigskip
\hrule
\bigskip

\subsubsection*{4. Кэли–Гамильтон (универсальный метод)}

\textbf{Условие применения:} универсальный метод для матриц любого размера, особенно удобен для $2 \times 2$.

\textbf{Теорема Кэли–Гамильтона.} Матрица $A$ удовлетворяет своему характеристическому уравнению: $\chi_A(A) = 0$.

\textbf{Ключевая идея:} используем тождество $\chi_A(A) = 0$ для построения рекуррентного соотношения на степени матрицы.

\HSEAlgorithm{\textbf{Алгоритм Кэли–Гамильтона (2×2).}\begin{enumerate}\item $\chi_A(\lambda) = \lambda^2 - (\operatorname{tr}A)\lambda + \det A$.\item Рекуррентное: $A^{t+2} = (\operatorname{tr}A)A^{t+1} - (\det A)A^t$.\item Представление: $A^t = \alpha_t A + \beta_t I$.\item Решить на $\alpha_t,\beta_t$ и получить $\mathbf{x}_t$.\end{enumerate}}

\HSEExample{\textbf{Пример.} Для $A = \begin{pmatrix} 3 & 2 \\ -2 & -1 \end{pmatrix}$: $\chi_A(\lambda) = (\lambda-1)^2$, откуда $A^{t+2}=2A^{t+1}-A^t$, $A^t = tA + (1-t)I$, и $\mathbf{x}_t = \begin{pmatrix} 4t + 1 \\ 1 - 4t \end{pmatrix}$.}

\HSEBox{Общий алгоритм решения систем разностных уравнений}{\begin{enumerate}\item Анализ матрицы: $\operatorname{tr}A$, $\det A$, $\chi_A(\lambda)$.\item Выбор метода по типу спектра.\item Получить $A^t$ соответствующим методом.\item Решить $\mathbf{x}_t = A^t\mathbf{x}_0$.\item Проверка: $A^0=I$, $A^1=A$.\end{enumerate}}

\textbf{Полезные проверки:}
\begin{itemize}
\item \textbf{Начальные условия:} $A^0 = I$, $A^1 = A$.
\item \textbf{Жорданова форма:} если $A = \lambda I + N$, проверить $N^m = 0$.
\item \textbf{Комплексная пара:} $\det A = \rho^2$, $\text{tr}\,A = 2\rho\cos\theta$.
\item \textbf{Биномиальные коэффициенты:} не забыть $\binom{t}{k}$ в формуле Жордана.
\end{itemize}

% Ссылки на задачи: добавьте URL/refs сюда
% Например: Сборник И. Иванов, гл. 3; задачи 1-15.


% \section{ПЧП первого порядка. Инварианты характеристик}\label{sec:pde-first-order}

\HSEDefinition{\Term{Линейное ПЧП первого порядка} в $\mathbb{R}^3$ — уравнение вида $a(x,y,z)\,u_x + b(x,y,z)\,u_y + c(x,y,z)\,u_z = 0$, где $a,b,c$ — непрерывные функции. Метод характеристик приводит к ОДУ-системе $\dot x = a(x,y,z),\ \dot y = b(x,y,z),\ \dot z = c(x,y,z)$.}

% Подключаем все подглавы по ПЧП первого порядка
\subsection{Вводная информация и единый алгоритм}\label{sec:pde-intro-algorithm}

Рассматриваем линейное ПЧП первого порядка в $\mathbb{R}^3$:
\[
a(x,y,z)\,u_x + b(x,y,z)\,u_y + c(x,y,z)\,u_z \;=\; 0,
\]
где $a,b,c$ — непрерывны. Метод характеристик приводит к ОДУ-системе
\[
\dot x = a(x,y,z),\quad \dot y = b(x,y,z),\quad \dot z = c(x,y,z).
\]

\HSEDefinition{\Term{Инвариант (первый интеграл)} $I(x,y,z)$ — это $C^1$-функция, постоянная вдоль характеристик, т.е.
\[
\frac{d}{ds}I(x(s),y(s),z(s))= \nabla I \cdot (a,b,c) = a I_x + b I_y + c I_z = 0.
\]
Общее решение ПЧП имеет вид $u=F(I_1,I_2)$, где $I_1,I_2$ — два независимых инварианта.}

\subsubsection{Единый алгоритм (5 шагов)}

\begin{enumerate}
\item \textbf{Цель.} Распознать форму и записать \emph{характеристики}.

    \textbf{Действие.} Выписать $\dot x=a,\ \dot y=b,\ \dot z=c$ и безпараметрические равенства
    $\displaystyle \frac{dx}{a}=\frac{dy}{b}=\frac{dz}{c}$.

\item \textbf{Цель.} Найти \emph{первый инвариант $I_1$}.

    \textbf{Действие.} Решить одну пару, напр. $\displaystyle \frac{dy}{dx}=\frac{b}{a}$, получить семейство кривых уровня
    $\Phi_1(x,y)=\text{const}$, положить $I_1=\Phi_1$ (или любой эквивалент).

\item \textbf{Цель.} Найти \emph{второй инвариант $I_2$}.

    \textbf{Действие.} Зафиксировать $I_1=\text{const}$ (т.е. связь $y=\Psi(x;I_1)$) и решить ОДУ по $z$ из
    $\displaystyle \frac{dz}{dx}=\frac{c(x,\Psi(x;I_1),z)}{a(x,\Psi(x;I_1))}$.
    \begin{itemize}
    \item Если $c\equiv 0$, взять $I_2=z$.
    \item Если $c=\alpha(x,y)\,z+\beta(x,y)$, получаем линейное ОДУ $z'=\tilde{\alpha}(x;I_1)z+\tilde{\beta}(x;I_1)$.
    \emph{Интегрирующий множитель} $M(x;I_1)=\exp\!\big(-\int \tilde{\alpha}\,dx\big)$, и
    \[
    \boxed{\,I_2 = z\,M(x;I_1) - \int \tilde{\beta}(x;I_1)\,M(x;I_1)\,dx\,}
    \]
    — константа вдоль характеристики.
    \item Если отделяется $z$ — разделить переменные и получить интегральный инвариант.
    \end{itemize}

\item \textbf{Цель.} Сформировать \emph{общее решение}.

    \textbf{Действие.} Записать $u=F(I_1,I_2)$.

\item \textbf{Цель.} Проверить \emph{независимость} $I_1,I_2$.

    \textbf{Действие.} Убедиться, что $dI_1\wedge dI_2\neq 0$ (или $\nabla I_1\times \nabla I_2\neq 0$) в рабочей области.
\end{enumerate}

\subsection{Как быстро находить $I_1$: детерминированные «детекторы»}\label{sec:pde-first-invariant}

Сначала можно убрать общий ненулевой множитель: $(a,b)\sim (\tilde a,\tilde b)$ дают те же инварианты.

\begin{itemize}
\item \textbf{Радиальный тип:} $(a,b)=(x,y)$ $\Rightarrow$ $\displaystyle \frac{dy}{dx}=\frac{y}{x}$ $\Rightarrow$ $I_1=\dfrac{y}{x}$.

\item \textbf{Вращение:} $(a,b)=(y,-x)$ $\Rightarrow$ $y\,dy=-x\,dx$ $\Rightarrow$ $I_1=x^2+y^2$.

\item \textbf{Диагональный линейный:} $(a,b)=(\alpha x,\beta y)$ $\Rightarrow$ $I_1=\dfrac{y}{x^{\beta/\alpha}}$.

\item \textbf{Общий линейный однородный:} $(a,b)=(\alpha x+\beta y,\ \gamma x+\delta y)$.
Через левые собственные векторы $M^\top$: взять $\xi=\mathbf w_1\!\cdot\!(x,y)$, $\eta=\mathbf w_2\!\cdot\!(x,y)$, тогда
$I_1=\eta/\xi^{\lambda_2/\lambda_1}$.
Альтернатива: подстановка $v=y/x$ всегда даёт $\dfrac{dv}{dx}=\dfrac{\Phi(v)}{x}$, интегрируется в логарифмах.

\item \textbf{Однородность одного порядка $d$:} если $a(\lambda x,\lambda y)=\lambda^d a$ и $b(\lambda x,\lambda y)=\lambda^d b$, то
$\,v=y/x$ ведёт к $I_1=x\,G(v)$.
\end{itemize}

\subsection{Как добирать $I_2$: стандартные ветки}\label{sec:pde-second-invariant}

\begin{itemize}
\item $c\equiv 0$ $\Rightarrow$ $I_2=z$.

\item $c=\mu(x,y)\,z$ $\Rightarrow$ $z'=\tilde{\alpha}(x;I_1)\,z$ $\Rightarrow$ $I_2=z\,\exp\!\big(-\int \tilde{\alpha}\,dx\big)$.

\item $c=\mu(x,y)\,z+\nu(x,y)$ $\Rightarrow$ линейное ОДУ, формула выше с $\tilde{\beta}\not\equiv0$.

\item $c=\mu(x,y)$ (не зависит от $z$) $\Rightarrow$ $I_2=z-\int \frac{\mu(x,\Psi(x;I_1))}{a(x,\Psi(x;I_1))}\,dx$.
\end{itemize}


% \section{Линейные ОДУ второго порядка. Снятие $y'$, вронскиан, быстрые выводы о нулях}\label{sec:linear-ode-second-order}

\HSEDefinition{\Term{Однородное линейное уравнение второго порядка} — уравнение вида $y''+p(x)\,y'(x)+q(x)\,y(x)=0$, где $p,q\in C(I)$. Основные инструменты: снятие $y'$ заменой $y=\phi z$ и формула Абеля для вронскиана.}

% Подключаем все подглавы по линейным ОДУ второго порядка
\subsection{Вводная информация и единый алгоритм}\label{sec:ode2-intro-algorithm}

Рассматриваем однородное линейное уравнение второго порядка
\[
y''+p(x)\,y'(x)+q(x)\,y(x)=0, \qquad p,q\in C(I).
\]

Два основных инструмента:

\begin{itemize}
\item \textbf{Снятие $y'$} заменой $y=\phi z$, где
\[
\phi(x)=\exp\!\Bigl(-\frac12\int p(x)\,dx\Bigr),
\quad
z \text{ удовлетворяет } z''+Q(x)z=0,
\]
\[
\boxed{\,Q(x)=q(x)-\frac{p'(x)}{2}-\frac{p(x)^2}{4}\,}.
\]

\item \textbf{Формула Абеля (вронскиан)}: если $y_1,y_2$ — решения,
\[
\boxed{\,W(x):=\det\!\begin{pmatrix}y_1&y_2\\y_1'&y_2'\end{pmatrix}
= W(x_0)\exp\!\Bigl(-\int_{x_0}^{x} p(t)\,dt\Bigr)\,}.
\]
Отсюда: $W\not\equiv0 \iff y_1,y_2$ фундаментальны.
\end{itemize}

\subsubsection{Единый алгоритм (5 шагов)}

\begin{enumerate}
\item \textbf{Цель.} Привести к стандартному виду и зафиксировать $p,q$.

    \textbf{Действие.} При необходимости разделить исходное уравнение на коэффициент при $y''$; выписать $p,q$.

\item \textbf{Цель.} Снять $y'$ и получить нормальную форму.

    \textbf{Действие.} Положить $\displaystyle \phi=\exp\!\big(-\tfrac12\int p\,dx\big)$, $y=\phi z$.
    Тогда $z''+Qz=0$ с $Q=q-\tfrac{p'}{2}-\tfrac{p^2}{4}$.

\item \textbf{Цель.} Посчитать/сравнить вронскиан (линейная независимость).

    \textbf{Действие.} По Абелю $\displaystyle W(x)=W(x_0)\exp\!\bigl(-\int_{x_0}^{x} p\bigr)$.
    Если даны $y_1(x_0),y_1'(x_0),y_2(x_0),y_2'(x_0)$, то $W(x_0)$ считаем мгновенно.

\item \textbf{Цель.} Сделать краткий качественный вывод.

    \textbf{Действие.}
    \begin{itemize}
    \item \emph{«$\le 1$ нуля на интервале»}: если на интервале $J$ выполнено $Q\le 0$, то у нетривиального решения $z$ не может быть двух нулей в $J$ (интегральный аргумент, см. ниже) $\Rightarrow$ у $y=\phi z$ также $\le1$ нуля.
    \item \emph{«Положительный максимум невозможен»}: если $q(x)<0$ на $I$, то локальный максимум решения $y$ на $I$ не может быть $>0$ (экстремум-тест).
    \end{itemize}

\item \textbf{Цель.} Записать компактный итог.

    \textbf{Действие.} Указать $Q(x)$, формулу $W(x)$ (или значение), и сформулировать требуемое свойство/классификацию.
\end{enumerate}

\subsection{Два готовых мини-инструмента (доказательные скетчи)}\label{sec:ode2-mini-tools}

\subsubsection{«Не более одного нуля», если $Q\le 0$}

Пусть $z''+Qz=0$ на $J$ и $Q\le0$. Допустим у нетривиального $z$ два нуля $a<b$. Тогда
\[
\int_a^b z z''\,dx + \int_a^b Q z^2\,dx = 0
\;\Rightarrow\;
-\int_a^b (z')^2\,dx + \int_a^b Q z^2\,dx = 0,
\]
где интегрирование по частям и $z(a)=z(b)=0$.
Правая часть $\le -\int_a^b (z')^2\,dx<0$ — противоречие. Значит, максимум один ноль.

\subsubsection{«Положительный максимум невозможен», если $q<0$}

Пусть $y''+p y'+q y=0$, $q<0$. В локальном максимуме $x_0$: $y'(x_0)=0$, $y''(x_0)\le 0$.
Подстановка даёт $y''(x_0)=-q(x_0)y(x_0)$. Если $y(x_0)>0$, то $y''(x_0)>0$ — противоречие.
Значит любой максимум $\le 0$.

\subsection{Примеры (одинаковая схема 5 шагов)}\label{sec:ode2-examples}

\subsubsection{Пример 1 (Эйлер—Коши; $x>0$)}

\[
y''+\frac{2}{x}y' - \frac{3}{x^2}y=0.
\]

\textbf{Шаг 1.} \textbf{Цель.} $p,q$. \textbf{Действие.} $p=\frac{2}{x}$, $q=-\frac{3}{x^2}$.

\textbf{Шаг 2.} \textbf{Цель.} $Q$. \textbf{Действие.} $\phi=x^{-1}$,
$
Q=q-\frac{p'}{2}-\frac{p^2}{4}
=-\frac{3}{x^2}-\Bigl(-\frac{1}{x^2}\Bigr)-\frac{1}{x^2}=-\frac{3}{x^2}.
$

\textbf{Шаг 3.} \textbf{Цель.} $W(x)$. \textbf{Действие.} $W(x)=W(1)\exp\!\bigl(-\int_1^x\frac{2}{t}dt\bigr)=W(1)/x^2$.

\textbf{Шаг 4.} \textbf{Цель.} Вывод о нулях. \textbf{Действие.} $Q\le0$ на $x>0$ $\Rightarrow$ у нетривиального решения $\le1$ нуля на $(0,\infty)$.

\textbf{Шаг 5.} \textbf{Цель.} Итог. \textbf{Действие.} Нормальная форма $z''-\frac{3}{x^2}z=0$, $W(x)=W(1)/x^2$, «$\le1$ нуля».

\subsubsection{Пример 2 (постоянные коэффициенты)}

\[
y''+4y'+3y=0.
\]

\textbf{Шаг 1.} $p=4,\ q=3$.

\textbf{Шаг 2.} \textbf{Цель.} $Q$. \textbf{Действие.} $\phi=e^{-2x}$, $Q=3-0-4=-1$, т.е. $z''-z=0$.

\textbf{Шаг 3.} \textbf{Цель.} $W(x)$. \textbf{Действие.} $W(x)=W(0)\,e^{-4x}$.

\textbf{Шаг 4.} \textbf{Цель.} Нули. \textbf{Действие.} $Q=-1<0$ $\Rightarrow$ у нетривиального решения $\le1$ нуля.

\textbf{Шаг 5.} \textbf{Цель.} Итог. \textbf{Действие.} $z=\alpha e^x+\beta e^{-x}$, $y=e^{-2x}z$, $W(x)=W(0)e^{-4x}$, «не осциллирует».

\subsubsection{Пример 3 (переменный коэффициент $p$, подобранный $q$)}

\[
y''+x\,y'+\Bigl(\frac{x^2}{4}-1\Bigr)\,y=0\qquad (x\in\mathbb R).
\]

\textbf{Шаг 1.} $p=x,\ q=\tfrac{x^2}{4}-1$.

\textbf{Шаг 2.} \textbf{Цель.} $Q$. \textbf{Действие.} $\phi=e^{-x^2/4}$,
\[
Q=q-\frac{p'}{2}-\frac{p^2}{4}
=\Bigl(\frac{x^2}{4}-1\Bigr)-\frac12-\frac{x^2}{4}=-\frac{3}{2}.
\]
Получаем $z''-\tfrac32 z=0$.

\textbf{Шаг 3.} \textbf{Цель.} $W(x)$. \textbf{Действие.} $W(x)=W(0)\,e^{-\int_0^x t\,dt}=W(0)\,e^{-x^2/2}$.

\textbf{Шаг 4.} \textbf{Цель.} Нули. \textbf{Действие.} $Q=-\tfrac32<0$ $\Rightarrow$ $\le1$ нуля.

\textbf{Шаг 5.} \textbf{Цель.} Итог. \textbf{Действие.} Нормальная форма постоянного знака, вронскиан гауссов, решение не осциллирует.

\subsubsection{Пример 4 (вронскиан в конкретной точке)}

\[
(x+2)\,y''-3\,y'+\sqrt{\,1-x\,}\,y=0,\qquad x\in(-2,1).
\]

Пусть $y_1,y_2$ — решения с начальными данными
\[
y_1(0)=0,\ y_1'(0)=1;\qquad y_2(0)=3,\ y_2'(0)=2.
\]

\textbf{Шаг 1.} \textbf{Цель.} $p,q$. \textbf{Действие.} Делим на $x+2$: $p(x)=-\dfrac{3}{x+2}$, $q(x)=\dfrac{\sqrt{1-x}}{x+2}$.

\textbf{Шаг 2.} \textbf{Цель.} $Q$. \textbf{Действие.} Не обязательно считать: достаточно для $W$.

\textbf{Шаг 3.} \textbf{Цель.} $W(x)$. \textbf{Действие.} $W(0)=\det\begin{pmatrix}0&3\\1&2\end{pmatrix}=-3\neq0$ (фундаментальная пара).
По Абелю
\[
W(-1)=W(0)\exp\!\Bigl(-\int_0^{-1}\!p(t)\,dt\Bigr)
= -3\exp\!\Bigl(-\int_0^{-1}\!\frac{-3}{t+2}\,dt\Bigr)
= -3\cdot\Bigl(\frac{1}{2}\Bigr)^{\!3}=-\frac{3}{8}.
\]

\textbf{Шаг 4.} \textbf{Цель.} Вывод. \textbf{Действие.} Пара фундаментальна, $W(-1)=-3/8$.

\textbf{Шаг 5.} \textbf{Цель.} Итог. \textbf{Действие.} Независимость подтверждена в любой точке интервала.

\subsection{Памятка и типичные ловушки}\label{sec:ode2-memo-traps}

\subsubsection{Памятка: что делать на экзамене (минимум выбора)}

\begin{enumerate}
\item Записать $p,q$ (после деления на коэффициент при $y''$).
\item Снять $y'$: $\phi=\exp(-\tfrac12\int p)$, найти $Q=q-\tfrac{p'}{2}-\tfrac{p^2}{4}$.
\item Нужен вронскиан? Сразу Абель: $W(x)=W(x_0)\exp(-\int p)$; если даны $y_i(x_0),y_i'(x_0)$ — подставить.
\item Нужен качественный вывод? Если $Q\le 0$ на данном промежутке — пишем «$\le 1$ нуля» (аргумент A).
Если дано $q<0$ — пишем «положительный максимум невозможен» (аргумент B).
\item Итог: короткая формулировка ($Q$, $W$, свойство).
\end{enumerate}

\subsubsection{Типичные ловушки}

\begin{itemize}
\item Забыли разделить на коэффициент при $y''$ — $p,q$ посчитаны неверно.
\item Ошибка знака в $Q$: внимательно к $-\tfrac{p'}{2}$ и $-\tfrac{p^2}{4}$.
\item Вронскиан: не пытайтесь дифференцировать $W$ напрямую — используйте Абеля.
\item Для утверждений о нулях не нужно решать уравнение: достаточно знака $Q$ (после снятия $y'$).
\end{itemize}


% \section{Нелинейные 2D-системы: равновесия, линеаризация, негиперболика}\label{sec:nonlinear-2d-systems}

\HSEDefinition{\Term{Нелинейная 2D-система} — автономная система вида $\dot{x} = f(x, y)$, $\dot{y} = g(x,y)$, где $f, g \in C^1$. Метод линеаризации применяется для анализа положений равновесия и их типов.}

\subsection{Где применяется метод линеаризации (признаки «наш случай»)}
\begin{itemize}
    \item Дано: автономная система $\dot{x} = f(x, y)$, $\dot{y} = g(x,y)$, $f, g \in C^1$.
    \item Спрашивают: положения равновесия, их тип и эскиз фазового портрета в окрестности.
    \item В точке(ах) равновесия Якоби $J = \begin{pmatrix} f_x & f_y \\ g_x & g_y \end{pmatrix}$ удовлетворяет $\det J \neq 0$ (гиперболическая точка).
    \item Если $\det J = 0$ или $D = 0$ --- это уже не МЗ (негиперболика/граница случаев).
\end{itemize}

\subsection{Единый 5-шаговый алгоритм (используем во всех примерах)}

\textbf{Шаг 1. Цель: найти все равновесия.}

Действие: решить $f(x, y) = 0, g(x, y) = 0$.

\textbf{Шаг 2. Цель: получить линеаризацию.}

Действие: в каждой найденной точке вычислить Якоби $J$.

\textbf{Шаг 3. Цель: классифицировать тип точки по числам.}

Действие (единственное ветвление строго по знакам):
\begin{itemize}
    \item Если $\det J < 0 \to$ седло (неустойч.).
    \item Если $\det J > 0$:
    \begin{itemize}
        \item посчитать $D = \text{tr}^2 - 4 \det$.
        \item если $D > 0$: $\text{tr} < 0 \to$ устойч. узел, $\text{tr} > 0 \to$ неустойч. узел;
        \item если $D < 0$: $\text{tr} < 0 \to$ устойч. фокус, $\text{tr} > 0 \to$ неустойч. фокус.
    \end{itemize}
\end{itemize}

\textbf{Шаг 4. Цель: зафиксировать направления и устойчивость.}

Действие: указать «куда текут» траектории (в/из точки) и, при седле, назвать две устойчивые/неустойчивые сепаратрисы (вдоль собственных направлений $J$).

\textbf{Шаг 5. Цель: нарисовать локальный эскиз.}

Действие: около каждой точки нанести тип (узел/фокус/седло), стрелки по устойчивости, грубо ориентируясь на нулевые изоклины $f = 0, g = 0$ для знаков $\dot{x}, \dot{y}$.

\HSENote{Этот алгоритм является универсальным для анализа нелинейных 2D-систем и позволяет систематически подходить к решению задач на классификацию равновесных точек.}

% \input{\topicsBase/03-systems}
% \input{\topicsBase/04-numerical}

\end{document}


