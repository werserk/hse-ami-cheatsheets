\documentclass[11pt,a4paper]{article}
\usepackage[utf8]{inputenc}
\usepackage[T2A]{fontenc}
\usepackage[russian]{babel}
\usepackage{amsmath, amsfonts, amssymb, mathtools}
\usepackage{geometry}
\usepackage{parskip}
\usepackage{fancyhdr}
\usepackage{hyperref}
\hypersetup{
    colorlinks=true,
    linkcolor=black,
    urlcolor=black,
    citecolor=black
}

% Команда для отображения by werserk (чёрный цвет, без ссылки)
\newcommand{\byWerserk}{\textbf{by werserk}}


\geometry{margin=1.8cm}
\pagestyle{fancy}
\fancyhf{}
\fancyfoot[L]{\byWerserk}
\fancyfoot[R]{\thepage}
\renewcommand{\headrulewidth}{0pt}

% --- Каталог тем для этого предмета ---
\newcommand{\topicsBase}{topics}

\begin{document}

% \HSETitle{Подготовка: Дифференциальные уравнения}{Полная версия с разборами тем и ссылками}
\title{Подготовка: Дифференциальные уравнения}
\author{Полная версия с разборами тем и ссылками}
\maketitle
\thispagestyle{fancy}

\vspace{1cm}

% --- Оглавление ---
\tableofcontents

\newpage

% --- Подключайте темы ниже ---
% Примеры (создайте соответствующие файлы в topics/):
\section{Линейные разностные уравнения с постоянными коэффициентами (ЛОС)}

\subsection*{1. Тип экзаменационной задачи (полное условие)}
Найдите общее решение:
\[
y_{t+3}-y_{t+2}+4y_{t+1}-4y_{t} \;=\; 26\cdot 3^{t} + 10t + 9,
\]
где \(t\in\mathbb Z\), \((y_t)_{t\in\mathbb Z}\subset\mathbb R\) (или \(\mathbb C\)).

\subsection*{2. Универсальный алгоритм (визуальные формулы и детерминированные шаги)}

\textbf{Исходные данные и обозначения (ввод).} Дано ЛОС порядка \(n\in\mathbb N\):
\[
a_n y_{t+n} + a_{n-1} y_{t+n-1} + \dots + a_1 y_{t+1} + a_0 y_t = f(t),
\]
где \(a_n\neq 0,\ a_k\in\mathbb R\ (\text{или }\mathbb C),\ t\in\mathbb Z\).
Вводим: \(\chi(r):=r^n+b_{n-1}r^{n-1}+\dots+b_1 r + b_0\) — характеристический многочлен (после нормировки \(a_n=1\));
\(k_\chi(\lambda)\in\mathbb N\) — кратность корня \(\lambda\) в \(\chi\);
\(P_d(t)\in\mathbb R[t]\) — произвольный полином степени \(\le d\);
\(Q_{\lambda,\theta}(r):=r^2-2\lambda\cos\theta\,r+\lambda^2\).

\paragraph{Шаг 0.} \textbf{Привести уравнение к канонической форме.}\\
Разделить на \(a_n\) (если \(a_n\neq 1\)) и написать
\[
y_{t+n} + b_{n-1} y_{t+n-1} + \dots + b_1 y_{t+1} + b_0 y_t = f(t).
\]

\paragraph{Шаг 1.} \textbf{Построить \(\chi(r)\) и зафиксировать кратности корней.}\\
Выписать \(\chi(r)=r^{n}+b_{n-1}r^{n-1}+\dots+b_1 r + b_0\), найти все \(\lambda_j\) и \(k_\chi(\lambda_j)\).

\paragraph{Шаг 2.} \textbf{Записать общее решение однородной части \(y^{(h)}_t\).}\\
Для каждого корня \(\lambda\) кратности \(s=k_\chi(\lambda)\) включить базис
\[
t^0\lambda^{t},\ t^1\lambda^{t},\dots,t^{s-1}\lambda^{t};
\]
для пары \(\lambda=\rho e^{i\theta}\), \(\bar\lambda=\rho e^{-i\theta}\) — реальный базис \(\rho^{t}\cos(\theta t),\ \rho^{t}\sin(\theta t)\).

\textbf{Таблица соответствий (множитель \(\Rightarrow\) вклад в \(y^{(h)}\)):}
\[
\begin{array}{|c|c|}
\hline
\textbf{Множитель} & \textbf{Вклад в } \textbf{y(h)} \\
\hline
(r-\lambda)^s & P_{s-1}(t)\lambda^{t} \\
\hline
(r^2-2\rho\cos\theta\,r+\rho^2)^s & P_{s-1}(t)\rho^{t}\cos(\theta t),\ P_{s-1}(t)\rho^{t}\sin(\theta t) \\
\hline
\end{array}
\]

\paragraph{Шаг 3.} \textbf{Выбрать пробную форму \(y^{(p)}_t\) по атомам \(f(t)\) и признакам резонанса через \(\chi\).}\\
Разложить \(f(t)\) на атомы и применить правила из таблицы:

\begin{center}
\begin{tabular}{|c|c|c|}
\hline
\textbf{Атом} & \textbf{Резонанс?} & \textbf{Вклад в } \textbf{y(p)} \\
\hline
\(\lambda^{t}\) & \(k_\chi(\lambda) = 0\)? & \(A\,\lambda^{t}\) \\
\hline
\(P_d(t)\) & \(k_\chi(1) = 0\)? & \(c_0 + c_1 t + \dots + c_d t^d\) \\
\hline
\(\lambda^{t}P_d(t)\) & \(k_\chi(\lambda) = 0\)? & \(\lambda^{t}(c_0 + c_1 t + \dots + c_d t^d)\) \\
\hline
\(\lambda^{t}\cos(\theta t)\) & \(Q_{\lambda,\theta} \mid \chi\)? & \(\lambda^{t}(A\cos(\theta t)+B\sin(\theta t))\) \\
\(\lambda^{t}\sin(\theta t)\) & & \\
\hline
\textbf{При резонансе:} & \textbf{любая форма} & \textbf{умножить на } \(t^{s}\) \\
\hline
\end{tabular}
\end{center}

\paragraph{Шаг 4.} \textbf{Определить коэффициенты пробной формы.}\\
Подставить \(y^{(p)}\) в уравнение, сгруппировать по независимым типам (\(\lambda^t\), \(t^k\), \(\lambda^t\cos/\sin\)) и решить линейную систему на коэффициенты.

\paragraph{Шаг 5.} \textbf{Собрать общий ответ и учесть начальные условия (при наличии).}\\
Записать \(y_t=y^{(h)}_t+y^{(p)}_t\). При наличии \(y_0,\dots,y_{n-1}\) подставить соответствующие \(t\) и решить систему для констант при \(y^{(h)}\).

\subsection*{3. Сопроводительные материалы (таблицы и обозначения)}

\textbf{Атом → пробная форма (до резонанса):}
\[
\lambda^{t}\mapsto A\,\lambda^{t},\qquad
P_d(t)\mapsto c_0 + c_1 t + \dots + c_d t^d,\qquad
\lambda^{t}P_d(t)\mapsto \lambda^{t}(c_0 + c_1 t + \dots + c_d t^d),
\]
\[
\lambda^{t}\cos(\theta t),\ \lambda^{t}\sin(\theta t)\mapsto \lambda^{t}\big(A\cos(\theta t)+B\sin(\theta t)\big).
\]
\textbf{Правило резонанса (через \(\chi\)):} \(s=k_\chi(1)\) для \(P_d(t)\); \(s=k_\chi(\lambda)\) для \(\lambda^{t}P_d(t)\); если \(Q_{\lambda,\theta}\mid\chi\), умножить триг-форму на \(t^{s}\).

\subsection*{4. Применение алгоритма к объявленной задаче}

\[
y_{t+3}-y_{t+2}+4y_{t+1}-4y_{t} \;=\; 26\cdot 3^{t} + 10t + 9.
\]

\paragraph{Шаг 0.} \textbf{Канонический вид зафиксирован.}\\
Уравнение уже записано как \(y_{t+3}+(-1)y_{t+2}+4y_{t+1}+(-4)y_t=f(t)\), нормировка не требуется.

\paragraph{Шаг 1.} \textbf{Построить \(\chi(r)\) и кратности корней.}\\
\(\chi(r)=r^3-r^2+4r-4=(r-1)(r^2+4)\); корни \(1\), \(\pm 2i\), все кратности равны 1: \(k_\chi(1)=1\), \(k_\chi(\pm 2i)=1\).

\paragraph{Шаг 2.} \textbf{Записать \(y^{(h)}_t\) по найденному спектру.}\\
\[
y^{(h)}_t=C_1\cdot 1^{t}+2^{t}\Big(C_2\cos\tfrac{\pi t}{2}+C_3\sin\tfrac{\pi t}{2}\Big).
\]

\paragraph{Шаг 3.} \textbf{Выбрать \(y^{(p)}_t\) по атомам RHS и признакам резонанса на \(\chi\).}\\
\(f(t)=26\cdot 3^t + P_1(t)\), где \(P_1(t)=10t+9\). 
\begin{itemize}
\item Для \(3^t\): \(k_\chi(3)=0\) (3 не корень) \(\Rightarrow A\cdot 3^t\)
\item Для \(P_1(t)\): \(k_\chi(1)=1\) (1 — корень кратности 1) \(\Rightarrow t(\tilde a t+\tilde b)=\tilde a t^2+\tilde b t\)
\end{itemize}
Итого
\[
y^{(p)}_t=A\cdot 3^t + a\,t^2 + b\,t.
\]

\paragraph{Шаг 4.} \textbf{Найти коэффициенты пробной формы, учитывая разложение по типам.}\\
Подстановка даёт
\[
L[y^{(p)}]=26A\cdot 3^{t}+10a\,t+(9a+5b)\stackrel{!}{=}26\cdot 3^{t}+10t+9
\Rightarrow A=1,\ a=1,\ b=0.
\]
Следовательно, \(y^{(p)}_t=3^{t}+t^{2}\).

\paragraph{Шаг 5.} \textbf{Собрать общий ответ и отметить, как добавляются начальные условия.}\\
\[
\boxed{\,y_t=C_1+2^{t}\Big(C_2\cos\tfrac{\pi t}{2}+C_3\sin\tfrac{\pi t}{2}\Big)+3^{t}+t^{2}\, }.
\]
При наличии \(y_0,y_1,y_2\) — подставить \(t=0,1,2\) и решить систему для \(C_1,C_2,C_3\).

\section{Синтез разностного уравнения по заданным решениям}

\subsection*{1. Тип экзаменационной задачи (полное условие)}
\textbf{Задача.} Построить \emph{линейное однородное разностное уравнение с постоянными коэффициентами} минимально возможного порядка, частными решениями которого являются
\[
y^{(1)}_{t}=3^{\,t},\qquad y^{(2)}_{t}=2^{\,t}\,\sin\!\frac{\pi t}{3}.
\]
(Решение здесь не приводится; это контекст для главы.)

\subsection*{2. Универсальный алгоритм (визуальные формулы и детерминированные шаги)}

\textbf{Исходные данные и обозначения (ввод).} Дано множество частных решений \(\{y^{(k)}_t\}_{k=1}^K\) ЛОС. Требуется построить характеристический полином \(p(\lambda)\) минимального порядка \(N\) такой, что все \(y^{(k)}_t\) являются решениями уравнения \(p(L)[y_t]=0\), где \(L\) — оператор сдвига \(Ly_t=y_{t+1}\).

Вводим: \(\alpha\in\mathbb{R}\) — основание экспоненты; \(\omega\in\mathbb{R}\) — частота тригонометрических функций; \(s\in\mathbb{N}_0\) — степень полинома \(t^s\); \(p(\lambda)\in\mathbb{R}[\lambda]\) — характеристический полином.

\paragraph{Шаг 0.} \textbf{Распознать «атом» каждого данного решения.}\\
Для каждого \(y^{(k)}_t\) определить одну из форм:
\(\alpha^t\); \(t^{s}\alpha^t\); \(\alpha^t\cos(\omega t)\) или \(\alpha^t\sin(\omega t)\); \(t^{s}\alpha^t\cos(\omega t)\) или \(t^{s}\alpha^t\sin(\omega t)\).

\paragraph{Шаг 1.} \textbf{Получить характеристический множитель(и) для каждого атома.}\\
По таблице соответствий заменить атом на множитель \(p(\lambda)\) с учётом кратности \((s+1)\).

\paragraph{Шаг 2.} \textbf{Собрать общий характеристический полином минимального порядка.}\\
Перемножить \emph{разные} множители (комплексные корни берутся парой \(\Rightarrow\) реальный квадратичный множитель). Повторы дают максимальную кратность.

\paragraph{Шаг 3.} \textbf{Записать разностное уравнение.}\\
Привести \(p(\lambda)\) к виду \(\lambda^N+a_{N-1}\lambda^{N-1}+\dots+a_1\lambda+a_0\) и выписать
\[
y_{t+N}+a_{N-1}y_{t+N-1}+\dots+a_1y_{t+1}+a_0y_t=0.
\]

\paragraph{Шаг 4.} \textbf{Проверить минимальность и корректность.}\\
Убедиться, что \(N\) равен сумме степеней множителей; проверить зануление \(p(\lambda)\) на атомах (для тригонометрических — на \(\lambda=\alpha e^{\pm i\omega}\)).

\subsection*{3. Сопроводительные материалы (таблицы и обозначения)}

\textbf{Таблица соответствий (атом \(\Rightarrow\) множитель \(\Rightarrow\) кратность):}
\[
\begin{array}{|c|c|c|}
\hline
\textbf{Атом} & \textbf{Множитель} & \textbf{Кратность} \\
\hline
\alpha^t & (\lambda-\alpha) & 1 \\
\hline
t^{s}\alpha^t & (\lambda-\alpha)^{s+1} & s+1 \\
\hline
\alpha^t\cos(\omega t), \alpha^t\sin(\omega t) & \lambda^2-2\alpha\cos\omega\,\lambda+\alpha^2 & 1 \\
\hline
t^{s}\alpha^t\cos(\omega t), t^{s}\alpha^t\sin(\omega t) & (\lambda^2-2\alpha\cos\omega\,\lambda+\alpha^2)^{s+1} & s+1 \\
\hline
\end{array}
\]

\textbf{Быстрые значения \(\cos\omega\):}
\[
\cos\frac{\pi}{6}=\frac{\sqrt3}{2},\quad
\cos\frac{\pi}{4}=\frac{\sqrt2}{2},\quad
\cos\frac{\pi}{3}=\frac12,\quad
\cos\frac{\pi}{2}=0.
\]

\textbf{Правила сборки:} (i) Пара \(\{\cos,\sin\}\) с одинаковыми \(\alpha,\omega\) даёт один и тот же квадратичный множитель (не удваивать). (ii) При нескольких степенях \(t^{s}\) берётся максимальная кратность.

\subsection*{4. Применение алгоритма к объявленной задаче}

\textbf{Дано:} \(y^{(1)}_t=3^{t}\), \(y^{(2)}_t=2^{t}\sin\!\frac{\pi t}{3}\).

\paragraph{Шаг 0.} \textbf{Распознать «атом» каждого данного решения.}\\
Атомы: \(3^t\) (\(\alpha=3\)); \(2^t\sin(\pi t/3)\) (\(\alpha=2,\ \omega=\pi/3\)).

\paragraph{Шаг 1.} \textbf{Получить характеристический множитель(и) для каждого атома.}\\
Множители: \((\lambda-3)\) и \(\lambda^2-2\cdot 2\cos(\pi/3)\lambda+2^2=\lambda^2-2\lambda+4\).

\paragraph{Шаг 2.} \textbf{Собрать общий характеристический полином минимального порядка.}\\
Сборка: \(p(\lambda)=(\lambda-3)(\lambda^2-2\lambda+4)\).

\paragraph{Шаг 3.} \textbf{Записать разностное уравнение.}\\
Развёртка: \(p(\lambda)=\lambda^3-5\lambda^2+10\lambda-12\).
Соответствующее ЛОС:
\[
\boxed{\,y_{t+3}-5y_{t+2}+10y_{t+1}-12y_t=0\,}.
\]

\paragraph{Шаг 4.} \textbf{Проверить минимальность и корректность.}\\
Минимальность: порядок \(N=3\); проверка \(p(3)=0\) и \(\lambda=2e^{\pm i\pi/3}\) зануляют квадратичный множитель.

\section{Нелинейные 2D-системы: равновесия, линеаризация, классификация}

\subsection*{1. Тип экзаменационной задачи (полное условие)}
\textbf{Условие.} Найдите положения равновесия автономной системы уравнений, определите их характер, и нарисуйте фазовые портреты в окрестности положений равновесия
\[
\begin{cases}
\dot{x}=2-2\sqrt{1+x+y},\\[4pt]
\dot{y}=\exp\!\Bigl(\tfrac{5}{4}x+2y+y^{2}\Bigr)-1.
\end{cases}
\]

\subsection*{2. Универсальный алгоритм (визуальные формулы и детерминированные шаги)}

\textbf{Исходные данные и обозначения (ввод).} Дана автономная система \(\dot{x}=f(x,y)\), \(\dot{y}=g(x,y)\), где \(f,g\in C^1(\mathbb{R}^2)\). Требуется найти положения равновесия \((x_0,y_0)\) такие, что \(f(x_0,y_0)=0\), \(g(x_0,y_0)=0\), и классифицировать их характер.

Вводим: \(J(x,y)\) — матрица Якоби; \(\operatorname{tr}J=f_x+g_y\) — след; \(\det J=f_x g_y-f_y g_x\) — определитель; \(D=\operatorname{tr}^2-4\det\) — дискриминант; \(\lambda_{1,2}\) — собственные значения \(J\).

\paragraph{Шаг 0.} \textbf{Найти положения равновесия.}\\
Решить систему \(f(x,y)=0\), \(g(x,y)=0\) и найти все точки \((x_0,y_0)\) такие, что \(f(x_0,y_0)=0\), \(g(x_0,y_0)=0\).

\paragraph{Шаг 1.} \textbf{Составить матрицу Якоби.}\\
Вычислить частные производные и составить
\[
J(x,y)=\begin{pmatrix} f_x & f_y\\[2pt] g_x & g_y \end{pmatrix}.
\]

\paragraph{Шаг 2.} \textbf{Вычислить инварианты в каждой точке равновесия.}\\
Для каждой точки \((x_0,y_0)\) вычислить:
\[
\operatorname{tr}J(x_0,y_0),\quad \det J(x_0,y_0),\quad D=\operatorname{tr}^2-4\det.
\]

\paragraph{Шаг 3.} \textbf{Классифицировать тип точки по детектору.}\\
Применить правила из таблицы классификации по знакам \(\det\), \(D\), \(\operatorname{tr}\).

\paragraph{Шаг 4.} \textbf{Определить устойчивость и направления.}\\
По знаку \(\operatorname{tr}\) и типу точки зафиксировать вход/выход; для седла отметить две сепаратрисы вдоль собственных направлений \(J\).

\paragraph{Шаг 5.} \textbf{Нарисовать фазовый портрет.}\\
Нанести типы точек и стрелки; при необходимости использовать изоклины \(f=0\), \(g=0\) для знаков \(\dot x\), \(\dot y\).

\subsection*{3. Сопроводительные материалы (таблицы и обозначения)}

\textbf{Таблица классификации равновесий:}
\[
\begin{array}{|c|c|c|}
\hline
\textbf{Условие} & \textbf{Тип точки} & \textbf{Устойчивость} \\
\hline
\det<0 & \text{седло} & \text{неустойчивая} \\
\hline
\det>0,\ D>0,\ \operatorname{tr}<0 & \text{узел} & \text{устойчивый} \\
\hline
\det>0,\ D>0,\ \operatorname{tr}>0 & \text{узел} & \text{неустойчивый} \\
\hline
\det>0,\ D<0,\ \operatorname{tr}<0 & \text{фокус} & \text{устойчивый} \\
\hline
\det>0,\ D<0,\ \operatorname{tr}>0 & \text{фокус} & \text{неустойчивый} \\
\hline
\det>0,\ D=0 \text{ или } \det=0 & \text{негиперболика} & \text{см. главу M10} \\
\hline
\end{array}
\]

\textbf{Быстрые производные (частые атомы):}
\[
\begin{aligned}
&f(x,y)=A-B\sqrt{\,\Phi(x,y)\,}\!:\quad
f_x=-\frac{B}{2}\Phi^{-1/2}\Phi_x,\ \ f_y=-\frac{B}{2}\Phi^{-1/2}\Phi_y;\\
&g(x,y)=e^{\Psi(x,y)}-1:\quad g_x=e^{\Psi}\Psi_x,\ \ g_y=e^{\Psi}\Psi_y.
\end{aligned}
\]

\textbf{Правила упрощения:} Если в равновесии \(g=0\), то \(e^{\Psi}=1\) и \(g_x=\Psi_x\), \(g_y=\Psi_y\); если \(1+x+y=1\), то \(\sqrt{1+x+y}=1\) и \(f_x=f_y=-1\).

\subsection*{4. Применение алгоритма к объявленной задаче}

\textbf{Дано:} \(\dot{x}=2-2\sqrt{1+x+y}\), \(\dot{y}=\exp\!\Bigl(\tfrac{5}{4}x+2y+y^{2}\Bigr)-1\).

\paragraph{Шаг 0.} \textbf{Найти положения равновесия.}\\
\(f=0\ \Rightarrow\ \sqrt{1+x+y}=1\ \Rightarrow\ x+y=0\). \quad
\(g=0\ \Rightarrow\ \tfrac{5}{4}x+2y+y^2=0\). Подставляя \(y=-x\):
\[
x^2-\tfrac{3}{4}x=0\ \Rightarrow\ x\in\{0,\tfrac{3}{4}\}.
\]
Точки равновесия: \((0,0)\) и \((\tfrac{3}{4},-\tfrac{3}{4})\).

\paragraph{Шаг 1.} \textbf{Составить матрицу Якоби.}\\
При \(x+y=0\) и \(\Psi=0\) имеем
\[
J(x,y)=
\begin{pmatrix}
-1 & -1\\[2pt]
\tfrac{5}{4} & 2+2y
\end{pmatrix}.
\]
Значит \(J(0,0)=\begin{pmatrix}-1&-1\\[2pt]\tfrac{5}{4}&2\end{pmatrix}\), \(J(\tfrac{3}{4},-\tfrac{3}{4})=\begin{pmatrix}-1&-1\\[2pt]\tfrac{5}{4}&\tfrac{1}{2}\end{pmatrix}\).

\paragraph{Шаг 2.} \textbf{Вычислить инварианты в каждой точке равновесия.}\\
\[
\begin{aligned}
&(0,0):\ \operatorname{tr}=1,\ \det=-\tfrac{3}{4}<0;\\
&(\tfrac{3}{4},-\tfrac{3}{4}):\ \operatorname{tr}=-\tfrac{1}{2},\ \det=\tfrac{3}{4}>0,\ 
D=\tfrac{1}{4}-3=-\tfrac{11}{4}<0.
\end{aligned}
\]

\paragraph{Шаг 3.} \textbf{Классифицировать тип точки по детектору.}\\
\[
\begin{aligned}
&(0,0):\ \det<0\ \Rightarrow\ \text{седло (неустойчивая)};\\
&(\tfrac{3}{4},-\tfrac{3}{4}):\ \det>0,\ D<0,\ \operatorname{tr}<0\ \Rightarrow\ \text{фокус устойчивый}.
\end{aligned}
\]

\paragraph{Шаг 4.} \textbf{Определить устойчивость и направления.}\\
В \((0,0)\) — две сепаратрисы по собственным направлениям \(J\); в \((\tfrac{3}{4},-\tfrac{3}{4})\) — спиральное вхождение.

\paragraph{Шаг 5.} \textbf{Нарисовать фазовый портрет.}\\
Эскиз: седло в \((0,0)\) с «крестом» сепаратрис; устойчивый фокус в \((\tfrac{3}{4},-\tfrac{3}{4})\) со стрелками внутрь. Изоклина \(x+y=0\) помогает ориентировать знаки \(\dot x\).

\[
\boxed{\,\text{Две точки равновесия: седло }(0,0)\text{ и устойчивый фокус }(\tfrac{3}{4},-\tfrac{3}{4})\,}
\]


\section{Линейные ОДУ 2-го порядка: нормальная форма, вронскиан, короткие доказательства}

\subsection*{1. Тип экзаменационной задачи (полное условие)}
\textbf{Стейтмент.}
Пусть функции \(p(x),q(x)\) непрерывны на \(\mathbb R\) и \(q(x)<0\) для всех \(x\).
Пусть \(y(x)\) — нетривиальное решение
\[
y''+p(x)y'(x)+q(x)y(x)=0.
\]
Покажите, что если решение принимает максимальное значение в некоторой точке, то это значение не может быть больше \(0\).

\subsection*{2. Универсальный алгоритм (визуальные формулы и детерминированные шаги)}

\textbf{Исходные данные и обозначения (ввод).} Дано линейное ОДУ 2-го порядка \(y''+p(x)y'+q(x)y=0\), где \(p,q\in C(\mathbb{R})\). Требуется доказать качественные свойства решений (экстремумы, нули, устойчивость).

Вводим: \(\phi(x)\) — интегрирующий множитель; \(Q(x)\) — эффективный потенциал; \(W(x)\) — вронскиан; \(z(x)\) — решение в нормальной форме; \(x_0\) — точка экстремума или нуля.

\paragraph{Шаг 0.} \textbf{Нормализация: увидеть \(p,q\).}\\
Привести уравнение к виду \(y''+p(x)y'+q(x)y=0\) и зафиксировать знаки \(p(x)\), \(q(x)\).

\paragraph{Шаг 1.} \textbf{Нормальная форма: убрать \(y'\) при необходимости.}\\
Взять
\[
\phi(x)=\exp\!\Bigl(-\tfrac12\!\int p(x)\,dx\Bigr),\quad y=\phi z,
\]
тогда
\[
z''+Q(x)z=0,\qquad Q(x)=q-\frac{p'}{2}-\frac{p^2}{4}.
\]

\paragraph{Шаг 2.} \textbf{Вронскиан: независимость/масштаб.}\\
Формула Абеля:
\[
W(x)=W(x_0)\,\exp\!\Bigl(-\!\int_{x_0}^{x}p(t)\,dt\Bigr).
\]

\paragraph{Шаг 3.} \textbf{Локальные/качественные выводы: «максимум/минимум/нули».}\\
\begin{itemize}
  \item \emph{Триггер «экстремум».} В точке максимума \(x_0\): \(y'(x_0)=0,\ y''(x_0)\le0\). Подставить в уравнение.
  \item \emph{Триггер «\(\le1\) нуля».} Перейти к \(z''+Qz=0\); при \(Q\le0\):
  \[
  \int_a^b zz''\,dx+\int_a^b Qz^2\,dx=0
  \Rightarrow -\!\int_a^b (z')^2\,dx+\!\int_a^b Qz^2\,dx=0,
  \]
  что невозможно при двух нулях.
\end{itemize}

\paragraph{Шаг 4.} \textbf{Итог: короткая формулировка.}\\
Выписать использованные \(\phi,Q\) и/или \(W\) и сформулировать вывод.

\subsection*{3. Сопроводительные материалы (таблицы и обозначения)}

\textbf{Детектор ветки Шага 3:}
\[
\begin{array}{|c|c|}
\hline
\textbf{Признак в условии} & \textbf{Действие} \\
\hline
\text{Есть «максимум/минимум», дан знак } q & \text{Экстремум-тест: } y'=0\text{, знак } y''\text{, подстановка в ОДУ} \\
\hline
\text{Требуется «не более одного нуля»} & \text{Шаг 1 } \Rightarrow z''+Qz=0\text{, при } Q\le0 \\
& \text{интегральный аргумент} \\
\hline
\text{Нужно проверить фундаментальность пары} & \text{Абель: } W(x)=W(x_0)e^{-\int p} \\
\hline
\end{array}
\]

\textbf{Памятка формул M4:}
\[
\phi(x)=\exp\!\Bigl(-\tfrac12\!\int p\Bigr),\qquad
Q=q-\tfrac12 p'-\tfrac14 p^2,\qquad
W(x)=W(x_0)\exp\!\Bigl(-\!\int_{x_0}^{x}p(t)\,dt\Bigr).
\]

\textbf{Правила экстремума:} В точке локального максимума \(x_0\): \(y'(x_0)=0\), \(y''(x_0)\le0\); в точке локального минимума: \(y'(x_0)=0\), \(y''(x_0)\ge0\).

\subsection*{4. Применение алгоритма к объявленной задаче}

\textbf{Дано:} \(y''+p(x)y'+q(x)y=0\), где \(q(x)<0\) для всех \(x\), и \(y(x)\) — нетривиальное решение с максимумом в точке \(x_0\).

\paragraph{Шаг 0.} \textbf{Нормализация: увидеть \(p,q\).}\\
Уравнение уже в виде \(y''+p\,y'+q\,y=0\) с \(q(x)<0\) для всех \(x\).

\paragraph{Шаг 1.} \textbf{Нормальная форма: убрать \(y'\) при необходимости.}\\
Переход к \(z\) не требуется для данного доказательства.

\paragraph{Шаг 2.} \textbf{Вронскиан: независимость/масштаб.}\\
Вронскиан не нужен для данного доказательства.

\paragraph{Шаг 3.} \textbf{Локальные/качественные выводы: «максимум/минимум/нули».}\\
В точке локального максимума \(x_0\):
\(y'(x_0)=0,\ y''(x_0)\le0\). Подставляя в уравнение:
\[
y''(x_0)=-p(x_0)\,y'(x_0)-q(x_0)\,y(x_0)= -q(x_0)\,y(x_0).
\]
При \(q(x_0)<0\) из \(y(x_0)>0\) следовало бы \(y''(x_0)>0\), что противоречит максимуму.
Значит \(y(x_0)\le0\).

\paragraph{Шаг 4.} \textbf{Итог: короткая формулировка.}\\
\[
\boxed{\,\text{Положительный локальный максимум невозможен при } q(x)<0\,}
\]


\section{ПЧП 1-го порядка (задача Коши по кривой)}

\subsection*{1. Тип экзаменационной задачи (полное условие)}
Даны две задачи Коши для уравнения
\[
y\,z_x-x\,z_y=0:
\]
а) \(z=2y\) при \(x=1\);\quad б) \(z=2y\) при \(x=1+y\).
Искать решение в окрестности \((1,0)\). Проверить условия теоремы существования–единственности.

\subsection*{2. Универсальный алгоритм (визуальные формулы и детерминированные шаги)}

\textbf{Исходные данные и обозначения (ввод).} Дано квазилинейное ПЧП 1-го порядка \(a(x,y)z_x+b(x,y)z_y=0\), где \(a,b\in C^1(\Omega\subset\mathbb R^2)\), и начальные данные на кривой \(\gamma:s\mapsto(x(s),y(s))\): \(z(\gamma(s))=\varphi(s)\).

Вводим: \(I_1(x,y)\) — первый интеграл (инвариант); \(\Delta(s)\) — определитель нехарактеристичности; \(\gamma'(s)\) — касательный вектор к кривой; \(F\) — произвольная функция.

\paragraph{Шаг 0.} \textbf{Найти характеристики.}\\
Решить систему \(\displaystyle \frac{dy}{dx}=\frac{b}{a}\) и найти первый интеграл \(I_1(x,y)=C_1\).

\paragraph{Шаг 1.} \textbf{Записать общее решение.}\\
Общее решение имеет вид \(z(x,y)=F(I_1(x,y))\), где \(F\) — произвольная функция.

\paragraph{Шаг 2.} \textbf{Сшить с начальными данными.}\\
Подставить кривую \(\gamma\) в общее решение: \(F(I_1(\gamma(s)))=\varphi(s)\).
Если \(\Delta\neq0\), то \(s=\sigma(I)\) локально и
\[
z(x,y)=\varphi(\sigma(I_1(x,y))).
\]

\paragraph{Шаг 3.} \textbf{Проверить нехарактеристичность.}\\
Вычислить \(\Delta(s)=a(\gamma)y'(s)-b(\gamma)x'(s)\).
Проверить условие \((a,b)\not\parallel \gamma'(s)\ \Leftrightarrow\ \Delta\neq0\).

\paragraph{Шаг 4.} \textbf{Сформулировать итог.}\\
\(\Delta\neq0\Rightarrow\) единственность; \(\Delta=0\Rightarrow\) ветвление или неединственность.

\subsection*{3. Сопроводительные материалы (таблицы и обозначения)}

\textbf{Быстрые инварианты:}
\[
\begin{array}{|c|c|c|}
\hline
\textbf{Коэффициенты } (a,b) & \textbf{Уравнение } \frac{dy}{dx}=\frac{b}{a} & \textbf{Инвариант } I_1(x,y) \\
\hline
(y,\ -x) & -\frac{x}{y} & x^2+y^2 \\
\hline
(x,\ y) & \frac{y}{x} & \frac{y}{x} \\
\hline
(\alpha x,\ \beta y) & \frac{\beta y}{\alpha x} & \dfrac{y}{x^{\beta/\alpha}} \\
\hline
(\alpha x+\beta y,\ \gamma x+\delta y) & \frac{\gamma x+\delta y}{\alpha x+\beta y} & \text{линейная замена}\Rightarrow \frac{\eta}{\xi^{\lambda_2/\lambda_1}} \\
\hline
\end{array}
\]

\textbf{Условие нехарактеристичности:} \(\Delta(s)=a(\gamma)y'(s)-b(\gamma)x'(s)\neq0\).

\textbf{Правила диагностики:} В виде \(g(x,y)=0\): \(a g_x+b g_y\neq0\) на \(\gamma\).

\subsection*{4. Применение алгоритма к объявленной задаче}

\textbf{Дано:} \(y\,z_x-x\,z_y=0\) с двумя задачами Коши в окрестности \((1,0)\).

\paragraph{Шаг 0.} \textbf{Найти характеристики.}\\
\(a=y,\ b=-x\Rightarrow dy/dx=-x/y\Rightarrow I_1=x^2+y^2\).

\paragraph{Шаг 1.} \textbf{Записать общее решение.}\\
Общее решение: \(z=F(x^2+y^2)\).

\paragraph{Шаг 2.} \textbf{Сшить с начальными данными.}\\
\textbf{(а) }\(x=1,\ z=2y\):\\
\(I_1|_{x=1}=1+y^2,\quad \Delta=y\cdot1-(-1)\cdot0=y\).\\
В \((1,0)\): \(\Delta=0\) (характеристическая).\\
Инверсия многозначна: \(y=\pm\sqrt{I-1}\Rightarrow\)
\[
\boxed{\,z=2\,\operatorname{sgn}(y)\sqrt{x^2+y^2-1}\,}
\]
(неединственность у \(y=0\)).

\textbf{(б) }\(x=1+y,\ z=2y\):\\
\(I_1|_{x=1+y}=1+2y+2y^2,\quad \Delta=2y+1\).\\
В \((1,0)\): \(\Delta=1\neq0\) (нехарактеристическая).\\
\(I=1+2s+2s^2\Rightarrow s=\frac{-1+\sqrt{2I-1}}{2}\) (ветвь у \(s\approx0\)).
\[
\boxed{\,z(x,y)=-1+\sqrt{\,2(x^2+y^2)-1\,}\,}
\]
(единственно в окрестности \((1,0)\)).


\section{Глава M6. Системы разностных: диагонализуемые матрицы, Phi t равно A в степени t, вариация постоянных}

\subsection*{1) Тип экзаменационной задачи}
\textbf{Условие.}
\[
\begin{pmatrix} x_{t+1}\\[2pt] y_{t+1} \end{pmatrix}
= A \begin{pmatrix} x_{t}\\[2pt] y_{t} \end{pmatrix}
+ \begin{pmatrix} 1\\[2pt] -1 \end{pmatrix}, 
\qquad 
A=\frac12\begin{pmatrix} -1 & 3\\[2pt] 3 & -1 \end{pmatrix}.
\]
(а) Найти фундаментальную матрицу \(\Phi_t\).
(б) Полагая \(\binom{x_t}{y_t}=\Phi_t\binom{c_1^{\,t}}{c_2^{\,t}}\), выписать уравнения для \(c_1^{\,t},c_2^{\,t}\) (не решать).

\subsection*{2) Универсальный алгоритм (формулы)}
\textbf{Ввод.} \(A\in\mathbb R^{n\times n}\), \(x_t\in\mathbb R^n\), \(b_t\in\mathbb R^n\). \(\ \Phi_t:=A^t\).
Спектр: \(A=V\Lambda V^{-1}\), \(\Lambda=\operatorname{diag}(\lambda_1,\dots,\lambda_n)\).

\paragraph{Шаг 1. Спектр.}
Найти \(\lambda_j\) и базис \(\{v_j\}\): \((A-\lambda_j I)v_j=0\). \(\sum_j \dim\ker(A-\lambda_j I)=n\Rightarrow\) диагонализуемо.

\paragraph{Шаг 2. A в степени t.}
\[
A^t=V\Lambda^tV^{-1},\quad \Lambda^t=\operatorname{diag}(\lambda_1^t,\dots,\lambda_n^t).
\]
Если \(\lambda=\rho e^{\pm i\theta}\): на \(\mathbb R^2\) блок \(S=\begin{smallmatrix}a&-b\\ b&a\end{smallmatrix}\), \(a+ib=\lambda\), 
\[
S^t=\rho^t\begin{pmatrix}\cos(\theta t)&-\sin(\theta t)\\ \sin(\theta t)&\cos(\theta t)\end{pmatrix}.
\]

\paragraph{Шаг 3. Phi t и однородная система.}
\[
x_{t+1}=Ax_t\ \Rightarrow\ x_t=\Phi_t x_0,\quad \Phi_t=A^t.
\]

\paragraph{Шаг 4. Вариация постоянных.}
Полагаем \(x_t=\Phi_t c^{\,t}\). Тогда
\[
\Phi_{t+1}c^{\,t+1}=\Phi_{t+1}c^{\,t}+b_t\ \Rightarrow\ 
\boxed{\,c^{\,t+1}-c^{\,t}=\Phi_{t+1}^{-1}b_t\,}.
\]
Эквивалентно: \(x_t=A^t x_0+\sum_{k=0}^{t-1}A^{t-1-k}b_k\).

\paragraph{Шаг 5. Частные случаи.}
Если \(b_t\equiv b\) и \(I-A\) обратима: \(x_t=A^t(x_0-(I-A)^{-1}b)+(I-A)^{-1}b\).
Если \(\lambda<0\): \(\lambda^t=(-1)^t|\lambda|^t\).
Пара \(\rho e^{\pm i\theta}\): блок \(\rho^tR(\theta t)\).

\subsection*{3) Сопроводительные материалы}

\begin{center}
\begin{tabular}{|c|c|}
\hline
Спектр \(A\) & Формула для \(A^t\)\\
\hline
\(\lambda_j\in\mathbb R\) простые & \(V\operatorname{diag}(\lambda_1^t,\dots,\lambda_n^t)V^{-1}\)\\
\hline
\(\rho e^{\pm i\theta}\) & \(W\bigl(\rho^t\begin{smallmatrix}\cos\theta t&-\sin\theta t\\ \sin\theta t&\cos\theta t\end{smallmatrix}\bigr)W^{-1}\)\\
\hline
смешанный & блочно по строкам выше\\
\hline
\end{tabular}
\end{center}

\[
\Phi_t^{-1}=V\,\operatorname{diag}(\lambda_1^{-t},\dots,\lambda_n^{-t})\,V^{-1}.
\]

\subsection*{4) Применение алгоритма к условию}

\paragraph{Шаг 1.} \(\widehat{A}=\begin{smallmatrix}-1&3\\ 3&-1\end{smallmatrix}\Rightarrow \sigma(\widehat{A})=\{2,-4\}\), \(v_1=(1,1)\), \(v_2=(1,-1)\).  
\(\sigma(A)=\{1,-2\}\) (диагонализуемо).

\paragraph{Шаг 2.}
\[
V=\begin{pmatrix}1&1\\ 1&-1\end{pmatrix},\quad
\Lambda=\operatorname{diag}(1,-2),\quad
V^{-1}=\tfrac12\begin{pmatrix}1&1\\ 1&-1\end{pmatrix}.
\]
\[
\Phi_t=A^t=\tfrac12\begin{pmatrix}1+(-2)^t & 1-(-2)^t\\ 1-(-2)^t & 1+(-2)^t\end{pmatrix}.
\]

\paragraph{Шаг 3.} \(x_t=\Phi_t x_0\).

\paragraph{Шаг 4.}
\[
c^{\,t+1}-c^{\,t}=\Phi_{t+1}^{-1}b,\quad
\Phi_{t+1}^{-1}=V\operatorname{diag}\!\bigl(1,(-2)^{-(t+1)}\bigr)V^{-1}.
\]
\[
\Phi_{t+1}^{-1}b=\frac12\!\begin{pmatrix}1+(-2)^{-(t+1)} & 1-(-2)^{-(t+1)}\\ 1-(-2)^{-(t+1)} & 1+(-2)^{-(t+1)}\end{pmatrix}\!\!\binom{1}{-1}
=\binom{(-\tfrac12)^{\,t+1}}{-(-\tfrac12)^{\,t+1}}.
\]
\[
\boxed{\,c_1^{\,t+1}-c_1^{\,t}=(-\tfrac12)^{\,t+1},\qquad c_2^{\,t+1}-c_2^{\,t}=-(-\tfrac12)^{\,t+1}\,}.
\]

\paragraph{Шаг 5.} \((I-A)\) необратима (есть \(\lambda=1\)) \(\Rightarrow\) стационарная формула неприменима; используем вариацию постоянных как выше.
\section{Нелинейные 2D-системы: линеаризация, классификация по tr, det, D}

\subsection*{1. Тип экзаменационной задачи (полное условие)}
\textbf{Стейтмент.}
Найдите положения равновесия автономной системы, определите их характер и набросайте фазовые портреты в окрестности равновесий:
\[
\begin{cases}
\dot{x}=2-2\sqrt{\,1+x+y\,},\\[4pt]
\dot{y}=\exp\!\bigl(\tfrac{5}{4}x+2y+y^{2}\bigr)-1.
\end{cases}
\]

\subsection*{2. Универсальный алгоритм (визуальные формулы и детерминированные шаги)}

\textbf{Исходные данные и обозначения (ввод).} Дана автономная система \(\dot{x}=f(x,y)\), \(\dot{y}=g(x,y)\), где \(f,g\in C^1(\mathbb{R}^2)\). Требуется найти положения равновесия \((x_*,y_*)\) такие, что \(f(x_*,y_*)=0\), \(g(x_*,y_*)=0\), и классифицировать их характер по линеаризации.

Вводим: \(J\) — матрица Якоби; \(\operatorname{tr}J=f_x+g_y\) — след; \(\det J=f_x g_y-f_y g_x\) — определитель; \(D=\operatorname{tr}^2-4\det\) — дискриминант; \(\lambda_{1,2}\) — собственные значения \(J\).

\paragraph{Шаг 0.} \textbf{Найти положения равновесия.}\\
Решить систему \(f(x,y)=0\), \(g(x,y)=0\) и найти все точки \((x_*,y_*)\) такие, что \(f(x_*,y_*)=0\), \(g(x_*,y_*)=0\).

\paragraph{Шаг 1.} \textbf{Вычислить матрицу Якоби.}\\
Вычислить частные производные и составить
\[
J=\begin{pmatrix}
f_x & f_y\\[2pt]
g_x & g_y
\end{pmatrix}\Big|_{(x_*,y_*)}.
\]

\paragraph{Шаг 2.} \textbf{Вычислить инварианты в каждой точке равновесия.}\\
Посчитать
\[
\operatorname{tr}J=f_x+g_y,\qquad
\det J=f_xg_y-f_yg_x,\qquad
D=\operatorname{tr}^2-4\det,
\]
и применить таблицу классификации.

\paragraph{Шаг 3.} \textbf{Определить стабильность и направления.}\\
\begin{itemize}
  \item \(\det<0\): седло (неустойчиво).
  \item \(\det>0,\,D>0\): узел; знак \(\operatorname{tr}\) даёт устойчивость.
  \item \(\det>0,\,D<0\): фокус; знак \(\operatorname{tr}\) даёт устойчивость.
\end{itemize}

\paragraph{Шаг 4.} \textbf{Нарисовать локальный эскиз.}\\
Нанести тип точки и стрелки вход/выход; для седла — сепаратрисы по собственным векторам \(J\).

\textbf{Примечание.} Если \(\det J\neq 0\) (гиперболическая точка), линеаризация локально адекватна типу (Хартман–Гробман).

\subsection*{3. Сопроводительные материалы (таблицы и обозначения)}

\textbf{Классификация по \(\det,\ \operatorname{tr},\ D\):}
\[
\begin{array}{|c|c|c|}
\hline
\textbf{Условие} & \textbf{Тип точки} & \textbf{Устойчивость} \\
\hline
\det<0 & \text{седло} & \text{неустойчивая} \\
\hline
\det>0,\ D>0,\ \operatorname{tr}<0 & \text{узел} & \text{устойчивый} \\
\hline
\det>0,\ D>0,\ \operatorname{tr}>0 & \text{узел} & \text{неустойчивый} \\
\hline
\det>0,\ D<0,\ \operatorname{tr}<0 & \text{фокус} & \text{устойчивый} \\
\hline
\det>0,\ D<0,\ \operatorname{tr}>0 & \text{фокус} & \text{неустойчивый} \\
\hline
\end{array}
\]

\textbf{Детектор гиперболичности:} \(\det J\neq 0\quad\Rightarrow\quad\text{линеаризация достаточна для локального типа.}\)

\textbf{Правила границ:} Границы \(\det=0\) или \(D=0\) — вне рамок M7 (негиперболика).

\subsection*{4. Применение алгоритма к объявленной задаче}

\textbf{Дано:} \(\dot{x}=2-2\sqrt{1+x+y}\), \(\dot{y}=\exp\!\bigl(\tfrac{5}{4}x+2y+y^{2}\bigr)-1\).

\paragraph{Шаг 0.} \textbf{Найти положения равновесия.}\\
\(f=0 \Rightarrow x+y=0\).
\(g=0 \Rightarrow \tfrac{5}{4}x+2y+y^2=0\).
Совместно: точки \((0,0)\) и \((\tfrac{3}{4},-\tfrac{3}{4})\).

\paragraph{Шаг 1.} \textbf{Вычислить матрицу Якоби.}\\
\[
f_x=f_y=-\frac{1}{\sqrt{1+x+y}},\quad
g_x=\tfrac{5}{4}e^{\Phi},\quad
g_y=(2+2y)e^{\Phi},\ \ \Phi=\tfrac{5}{4}x+2y+y^2.
\]
В равновесиях \(\sqrt{1+x+y}=1,\ e^{\Phi}=1\).

\paragraph{Шаг 2.} \textbf{Вычислить инварианты в каждой точке равновесия.}\\
Для \((0,0)\): \(J=\begin{pmatrix}-1&-1\\[2pt]\tfrac{5}{4}&2\end{pmatrix}\), \(\operatorname{tr}=1\), \(\det=-\tfrac{3}{4}<0\).\\
Для \((\tfrac{3}{4},-\tfrac{3}{4})\): \(J=\begin{pmatrix}-1&-1\\[2pt]\tfrac{5}{4}&\tfrac{1}{2}\end{pmatrix}\), \(\operatorname{tr}=-\tfrac{1}{2}\), \(\det=\tfrac{3}{4}>0\), \(D=\tfrac{1}{4}-3=-\tfrac{11}{4}<0\).

\paragraph{Шаг 3.} \textbf{Определить стабильность и направления.}\\
Для \((0,0)\): \(\det<0\) \(\Rightarrow\) \textbf{седло (неустойчивая)}.\\
Для \((\tfrac{3}{4},-\tfrac{3}{4})\): \(\det>0,\ D<0,\ \operatorname{tr}<0\) \(\Rightarrow\) \textbf{устойчивый фокус}.

\paragraph{Шаг 4.} \textbf{Нарисовать локальный эскиз.}\\
Седло в \((0,0)\): одна устойчивая и одна неустойчивая сепаратриса.
Фокус в \((\tfrac{3}{4},-\tfrac{3}{4})\): затухающие спирали.

\[
\boxed{\,\text{Две точки равновесия: седло }(0,0)\text{ и устойчивый фокус }(\tfrac{3}{4},-\tfrac{3}{4})\,}
\]

% \section{Линейные ОДУ второго порядка. Снятие $y'$, вронскиан, быстрые выводы о нулях}\label{sec:linear-ode-second-order}

\HSEDefinition{\Term{Однородное линейное уравнение второго порядка} — уравнение вида $y''+p(x)\,y'(x)+q(x)\,y(x)=0$, где $p,q\in C(I)$. Основные инструменты: снятие $y'$ заменой $y=\phi z$ и формула Абеля для вронскиана.}

% Подключаем все подглавы по линейным ОДУ второго порядка
\subsection{Вводная информация и единый алгоритм}\label{sec:ode2-intro-algorithm}

Рассматриваем однородное линейное уравнение второго порядка
\[
y''+p(x)\,y'(x)+q(x)\,y(x)=0, \qquad p,q\in C(I).
\]

Два основных инструмента:

\begin{itemize}
\item \textbf{Снятие $y'$} заменой $y=\phi z$, где
\[
\phi(x)=\exp\!\Bigl(-\frac12\int p(x)\,dx\Bigr),
\quad
z \text{ удовлетворяет } z''+Q(x)z=0,
\]
\[
\boxed{\,Q(x)=q(x)-\frac{p'(x)}{2}-\frac{p(x)^2}{4}\,}.
\]

\item \textbf{Формула Абеля (вронскиан)}: если $y_1,y_2$ — решения,
\[
\boxed{\,W(x):=\det\!\begin{pmatrix}y_1&y_2\\y_1'&y_2'\end{pmatrix}
= W(x_0)\exp\!\Bigl(-\int_{x_0}^{x} p(t)\,dt\Bigr)\,}.
\]
Отсюда: $W\not\equiv0 \iff y_1,y_2$ фундаментальны.
\end{itemize}

\subsubsection{Единый алгоритм (5 шагов)}

\begin{enumerate}
\item \textbf{Цель.} Привести к стандартному виду и зафиксировать $p,q$.

    \textbf{Действие.} При необходимости разделить исходное уравнение на коэффициент при $y''$; выписать $p,q$.

\item \textbf{Цель.} Снять $y'$ и получить нормальную форму.

    \textbf{Действие.} Положить $\displaystyle \phi=\exp\!\big(-\tfrac12\int p\,dx\big)$, $y=\phi z$.
    Тогда $z''+Qz=0$ с $Q=q-\tfrac{p'}{2}-\tfrac{p^2}{4}$.

\item \textbf{Цель.} Посчитать/сравнить вронскиан (линейная независимость).

    \textbf{Действие.} По Абелю $\displaystyle W(x)=W(x_0)\exp\!\bigl(-\int_{x_0}^{x} p\bigr)$.
    Если даны $y_1(x_0),y_1'(x_0),y_2(x_0),y_2'(x_0)$, то $W(x_0)$ считаем мгновенно.

\item \textbf{Цель.} Сделать краткий качественный вывод.

    \textbf{Действие.}
    \begin{itemize}
    \item \emph{«$\le 1$ нуля на интервале»}: если на интервале $J$ выполнено $Q\le 0$, то у нетривиального решения $z$ не может быть двух нулей в $J$ (интегральный аргумент, см. ниже) $\Rightarrow$ у $y=\phi z$ также $\le1$ нуля.
    \item \emph{«Положительный максимум невозможен»}: если $q(x)<0$ на $I$, то локальный максимум решения $y$ на $I$ не может быть $>0$ (экстремум-тест).
    \end{itemize}

\item \textbf{Цель.} Записать компактный итог.

    \textbf{Действие.} Указать $Q(x)$, формулу $W(x)$ (или значение), и сформулировать требуемое свойство/классификацию.
\end{enumerate}

\subsection{Два готовых мини-инструмента (доказательные скетчи)}\label{sec:ode2-mini-tools}

\subsubsection{«Не более одного нуля», если $Q\le 0$}

Пусть $z''+Qz=0$ на $J$ и $Q\le0$. Допустим у нетривиального $z$ два нуля $a<b$. Тогда
\[
\int_a^b z z''\,dx + \int_a^b Q z^2\,dx = 0
\;\Rightarrow\;
-\int_a^b (z')^2\,dx + \int_a^b Q z^2\,dx = 0,
\]
где интегрирование по частям и $z(a)=z(b)=0$.
Правая часть $\le -\int_a^b (z')^2\,dx<0$ — противоречие. Значит, максимум один ноль.

\subsubsection{«Положительный максимум невозможен», если $q<0$}

Пусть $y''+p y'+q y=0$, $q<0$. В локальном максимуме $x_0$: $y'(x_0)=0$, $y''(x_0)\le 0$.
Подстановка даёт $y''(x_0)=-q(x_0)y(x_0)$. Если $y(x_0)>0$, то $y''(x_0)>0$ — противоречие.
Значит любой максимум $\le 0$.

\subsection{Примеры (одинаковая схема 5 шагов)}\label{sec:ode2-examples}

\subsubsection{Пример 1 (Эйлер—Коши; $x>0$)}

\[
y''+\frac{2}{x}y' - \frac{3}{x^2}y=0.
\]

\textbf{Шаг 1.} \textbf{Цель.} $p,q$. \textbf{Действие.} $p=\frac{2}{x}$, $q=-\frac{3}{x^2}$.

\textbf{Шаг 2.} \textbf{Цель.} $Q$. \textbf{Действие.} $\phi=x^{-1}$,
$
Q=q-\frac{p'}{2}-\frac{p^2}{4}
=-\frac{3}{x^2}-\Bigl(-\frac{1}{x^2}\Bigr)-\frac{1}{x^2}=-\frac{3}{x^2}.
$

\textbf{Шаг 3.} \textbf{Цель.} $W(x)$. \textbf{Действие.} $W(x)=W(1)\exp\!\bigl(-\int_1^x\frac{2}{t}dt\bigr)=W(1)/x^2$.

\textbf{Шаг 4.} \textbf{Цель.} Вывод о нулях. \textbf{Действие.} $Q\le0$ на $x>0$ $\Rightarrow$ у нетривиального решения $\le1$ нуля на $(0,\infty)$.

\textbf{Шаг 5.} \textbf{Цель.} Итог. \textbf{Действие.} Нормальная форма $z''-\frac{3}{x^2}z=0$, $W(x)=W(1)/x^2$, «$\le1$ нуля».

\subsubsection{Пример 2 (постоянные коэффициенты)}

\[
y''+4y'+3y=0.
\]

\textbf{Шаг 1.} $p=4,\ q=3$.

\textbf{Шаг 2.} \textbf{Цель.} $Q$. \textbf{Действие.} $\phi=e^{-2x}$, $Q=3-0-4=-1$, т.е. $z''-z=0$.

\textbf{Шаг 3.} \textbf{Цель.} $W(x)$. \textbf{Действие.} $W(x)=W(0)\,e^{-4x}$.

\textbf{Шаг 4.} \textbf{Цель.} Нули. \textbf{Действие.} $Q=-1<0$ $\Rightarrow$ у нетривиального решения $\le1$ нуля.

\textbf{Шаг 5.} \textbf{Цель.} Итог. \textbf{Действие.} $z=\alpha e^x+\beta e^{-x}$, $y=e^{-2x}z$, $W(x)=W(0)e^{-4x}$, «не осциллирует».

\subsubsection{Пример 3 (переменный коэффициент $p$, подобранный $q$)}

\[
y''+x\,y'+\Bigl(\frac{x^2}{4}-1\Bigr)\,y=0\qquad (x\in\mathbb R).
\]

\textbf{Шаг 1.} $p=x,\ q=\tfrac{x^2}{4}-1$.

\textbf{Шаг 2.} \textbf{Цель.} $Q$. \textbf{Действие.} $\phi=e^{-x^2/4}$,
\[
Q=q-\frac{p'}{2}-\frac{p^2}{4}
=\Bigl(\frac{x^2}{4}-1\Bigr)-\frac12-\frac{x^2}{4}=-\frac{3}{2}.
\]
Получаем $z''-\tfrac32 z=0$.

\textbf{Шаг 3.} \textbf{Цель.} $W(x)$. \textbf{Действие.} $W(x)=W(0)\,e^{-\int_0^x t\,dt}=W(0)\,e^{-x^2/2}$.

\textbf{Шаг 4.} \textbf{Цель.} Нули. \textbf{Действие.} $Q=-\tfrac32<0$ $\Rightarrow$ $\le1$ нуля.

\textbf{Шаг 5.} \textbf{Цель.} Итог. \textbf{Действие.} Нормальная форма постоянного знака, вронскиан гауссов, решение не осциллирует.

\subsubsection{Пример 4 (вронскиан в конкретной точке)}

\[
(x+2)\,y''-3\,y'+\sqrt{\,1-x\,}\,y=0,\qquad x\in(-2,1).
\]

Пусть $y_1,y_2$ — решения с начальными данными
\[
y_1(0)=0,\ y_1'(0)=1;\qquad y_2(0)=3,\ y_2'(0)=2.
\]

\textbf{Шаг 1.} \textbf{Цель.} $p,q$. \textbf{Действие.} Делим на $x+2$: $p(x)=-\dfrac{3}{x+2}$, $q(x)=\dfrac{\sqrt{1-x}}{x+2}$.

\textbf{Шаг 2.} \textbf{Цель.} $Q$. \textbf{Действие.} Не обязательно считать: достаточно для $W$.

\textbf{Шаг 3.} \textbf{Цель.} $W(x)$. \textbf{Действие.} $W(0)=\det\begin{pmatrix}0&3\\1&2\end{pmatrix}=-3\neq0$ (фундаментальная пара).
По Абелю
\[
W(-1)=W(0)\exp\!\Bigl(-\int_0^{-1}\!p(t)\,dt\Bigr)
= -3\exp\!\Bigl(-\int_0^{-1}\!\frac{-3}{t+2}\,dt\Bigr)
= -3\cdot\Bigl(\frac{1}{2}\Bigr)^{\!3}=-\frac{3}{8}.
\]

\textbf{Шаг 4.} \textbf{Цель.} Вывод. \textbf{Действие.} Пара фундаментальна, $W(-1)=-3/8$.

\textbf{Шаг 5.} \textbf{Цель.} Итог. \textbf{Действие.} Независимость подтверждена в любой точке интервала.

\subsection{Памятка и типичные ловушки}\label{sec:ode2-memo-traps}

\subsubsection{Памятка: что делать на экзамене (минимум выбора)}

\begin{enumerate}
\item Записать $p,q$ (после деления на коэффициент при $y''$).
\item Снять $y'$: $\phi=\exp(-\tfrac12\int p)$, найти $Q=q-\tfrac{p'}{2}-\tfrac{p^2}{4}$.
\item Нужен вронскиан? Сразу Абель: $W(x)=W(x_0)\exp(-\int p)$; если даны $y_i(x_0),y_i'(x_0)$ — подставить.
\item Нужен качественный вывод? Если $Q\le 0$ на данном промежутке — пишем «$\le 1$ нуля» (аргумент A).
Если дано $q<0$ — пишем «положительный максимум невозможен» (аргумент B).
\item Итог: короткая формулировка ($Q$, $W$, свойство).
\end{enumerate}

\subsubsection{Типичные ловушки}

\begin{itemize}
\item Забыли разделить на коэффициент при $y''$ — $p,q$ посчитаны неверно.
\item Ошибка знака в $Q$: внимательно к $-\tfrac{p'}{2}$ и $-\tfrac{p^2}{4}$.
\item Вронскиан: не пытайтесь дифференцировать $W$ напрямую — используйте Абеля.
\item Для утверждений о нулях не нужно решать уравнение: достаточно знака $Q$ (после снятия $y'$).
\end{itemize}


% \input{\topicsBase/04-numerical}

\end{document}


