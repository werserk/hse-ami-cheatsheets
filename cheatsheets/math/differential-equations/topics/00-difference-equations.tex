\section{Линейные разностные уравнения с постоянными коэффициентами (ЛОС)}

\subsection*{1. Пример задачи из экзамена}
Найдите общее решение:
\[
y_{t+3}-y_{t+2}+4y_{t+1}-4y_{t} \;=\; 26\cdot 3^{t} + 10t + 9.
\]

\subsection*{2. Универсальный алгоритм}
Пусть задано ЛОС порядка \(n\):
\[
\sum_{k=0}^{n} a_k y_{t+k} = f(t), \quad a_n\neq0.
\]

\paragraph{Шаг 0.} \textbf{Привести к канонической форме (ведущий коэффициент = 1), если нужно.}

\paragraph{Шаг 1.} \textbf{Построить характеристический многочлен.}
\[
P(r)=r^{n}+b_{n-1}r^{n-1}+\dots+b_0.
\]
Найти корни \(r_i\) с кратностями \(m_i\).

\paragraph{Шаг 2.} \textbf{Записать общее решение однородного уравнения.}\\
Для корня \(r\) кратности \(s\) включаем в базис
\[
t^{0}r^{t},\ t^{1}r^{t},\dots,t^{s-1}r^{t}.
\]
Комплексные пары дают реальные базисы \(r^{t}\cos(\theta t),\, r^{t}\sin(\theta t)\).

\paragraph{Шаг 3.} \textbf{Выбрать пробную форму для частного решения \(y^{(p)}\) по типу \(f(t)\).}\\
Правило: для каждого атома \(f(t)\) взять стандартную пробную форму (см. таблицу ниже) и умножить на \(t^{m}\), где \(m\) — кратность корня характеристического многочлена, соответствующего этому атому.

\paragraph{Шаг 4.} \textbf{Определить коэффициенты в \(y^{(p)}\).}\\
Подставить \(y^{(p)}\) в уравнение, приравнять по независимым типам и решить линейную систему для неизвестных.

\paragraph{Шаг 5.} \textbf{Общий ответ и начальные условия.}\\
\(y_t=y^{(h)}_t+y^{(p)}_t\). При наличии начальных данных решить систему для констант \(C_i\).

\subsection*{3. Таблицы и шпаргалки}

\paragraph{Таблица 1. Атом \(f(t)\) $\mapsto$ пробная форма (без учёта резонанса)}
\begin{center}
\begin{tabular}{|c|c|}
\hline
\textbf{Атом} & \textbf{Пробная форма \(y^{(p)}\) (до умножения на \(t^s\))} \\
\hline
\(\alpha^{t}\) & \(A\alpha^{t}\) \\
\hline
\(t^{d}\) & \(\displaystyle\sum_{k=0}^{d} c_k t^k\) \\
\hline
\(\alpha^{t}P(t)\) & \(\alpha^{t}\sum_{k=0}^d c_k t^k\) \\
\hline
\(\alpha^{t}\cos(\beta t)\) or \(\alpha^{t}\sin(\beta t)\) & \(\alpha^{t}\bigl(A\cos(\beta t)+B\sin(\beta t)\bigr)\) \\
\hline
Сумма атомов & Сумма соответствующих пробных форм \\
\hline
\end{tabular}
\end{center}

\vspace{6pt}
\paragraph{Таблица 2. Правило резонанса} Если атом соответствует корню \(r=\alpha e^{i\theta}\) характеристического многочлена и этот корень имеет кратность \(m\), умножаем пробную форму на \(t^{m}\).

\vspace{6pt}
\paragraph{Таблица 3. Комплексные корни} Корень \(re^{\pm i\theta}\) даёт реальные базисы \(r^{t}\cos(\theta t),\ r^{t}\sin(\theta t)\).

\subsection*{4. Применение алгоритма к задаче}

\paragraph{Шаг 0.} Каноническая форма уже задана:
\[
y_{t+3}-y_{t+2}+4y_{t+1}-4y_{t}=26\cdot3^{t}+10t+9.
\]

\paragraph{Шаг 1.} Характеристический многочлен:
\[
P(r)=r^{3}-r^{2}+4r-4=(r-1)(r^{2}+4).
\]
Корни: \(r=1\) (кратность 1), \(r=\pm 2i\) (кратности 1).

\paragraph{Шаг 2.} Общее решение однородного:
\[
y^{(h)}_t=C_1 + 2^{t}\bigl(C_2\cos\tfrac{\pi t}{2}+C_3\sin\tfrac{\pi t}{2}\bigr).
\]

\paragraph{Шаг 3.} Правая часть разбивается:
\[
26\cdot 3^{t}\quad\text{(атом }\alpha^{t}\text{ с }\alpha=3\text{)};\qquad 10t+9\quad\text{(полином степени 1, эквивалент }1^{t}\text{).}
\]
— Для \(3^{t}\): пробуем \(A3^{t}\). Корень \(3\) не соответствует корням характеристического многочлена \(\Rightarrow\) множитель \(t^{0}\).
— Для \(10t+9\): базовая пробная форма — полином степени 1 \((a t + b)\), но т.к. \(r=1\) — корень характеристики кратности \(1\), умножаем на \(t^{1}\). Итак пробная форма:
\[
y^{(p)}_t = A\cdot 3^{t} + a t^{2} + b t.
\]

\paragraph{Шаг 4.} Подставляем \(y^{(p)}\) в левую часть, получаем:
\[
L[y^{(p)}]=26A\cdot 3^{t} + 10a\,t + (9a+5b).
\]
Равняем с \(26\cdot 3^{t} + 10t + 9\), получаем:
\[
26A=26\Rightarrow A=1;\quad 10a=10\Rightarrow a=1;\quad 9a+5b=9\Rightarrow b=0.
\]
Значит \(y^{(p)}_t=3^{t}+t^{2}\).

\paragraph{Шаг 5.} Общее решение:
\[
\boxed{\,y_t=C_1 + 2^{t}\bigl(C_2\cos\tfrac{\pi t}{2}+C_3\sin\tfrac{\pi t}{2}\bigr) + 3^{t} + t^{2}\, }.
\]
Если заданы начальные условия \(y_0,y_1,y_2\), подставляем их и решаем систему для \(C_1,C_2,C_3\).

\subsection*{5. Советы}
\begin{itemize}
  \item Всегда факторизуйте характеристический многочлен в начале.
  \item Не забывайте правило резонанса (умножение на \(t^{m}\)).
  \item Для тригонометрических членов используйте форму с \(\cos\) и \(\sin\).
  \item При равенстве по типам (\(\alpha^{t}\), полином по \(t\) и т.д.) приравнивайте коэффициенты — это даёт линейную систему.
\end{itemize}
