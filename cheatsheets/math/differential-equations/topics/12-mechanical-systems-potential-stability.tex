\section{Механические системы и устойчивость потенциала}

\subsection*{1. Тип экзаменационной задачи (полное условие)}
Дано: \(\ddot{\mathbf x}=-\nabla V(\mathbf x)\), \(V\in C^{1}(\mathbb R^{n})\), 
\(V(\mathbf 0)=\min V\), \(V(\mathbf x)>0\) при \(\mathbf x\neq \mathbf 0\). 
\textbf{Требуется:} найти положение равновесия и доказать его устойчивость.

\subsection*{2. Универсальный алгоритм (визуальные формулы и детерминированные шаги)}

\textbf{Исходные данные и обозначения (ввод).} Дана механическая система \(\ddot{\mathbf x}=-\nabla V(\mathbf x)\).
Вводим переменные: \(\mathbf x\in\mathbb R^{n}\) — координаты, \(\mathbf v=\dot{\mathbf x}\) — скорости.
Вводим энергию: \(E(\mathbf x,\mathbf v)=\tfrac12\|\mathbf v\|^{2}+V(\mathbf x)\).
Вводим гессиан: \(H=\nabla^{2}V(\mathbf 0)\) — матрица вторых производных потенциала в точке равновесия.

\paragraph{Шаг 0.} \textbf{Проверить потенциальность системы.}\\
Проверить: \(\mathbf f(\mathbf x)=-\nabla V(\mathbf x)\ \Leftrightarrow\ \partial_{x_j}f_i=\partial_{x_i}f_j\ \ (\forall i,j)\).

\paragraph{Шаг 1.} \textbf{Найти положения равновесия.}\\
\(\dot{\mathbf x}=\mathbf 0,\ \ddot{\mathbf x}=\mathbf 0\ \Longleftrightarrow\ \mathbf v=\mathbf 0,\ \nabla V(\mathbf x)=\mathbf 0\). 
При строгом минимуме \(V\) в \(\mathbf 0\): \(\boxed{\mathbf x_*=\mathbf 0}\).

\paragraph{Шаг 2.} \textbf{Проверить сохранение энергии.}\\
\(\dot E=\dot{\mathbf x}\cdot\ddot{\mathbf x}+\nabla V\cdot\dot{\mathbf x}
=\mathbf v\cdot(-\nabla V)+\nabla V\cdot\mathbf v=0
\ \Rightarrow\ \boxed{E(t)\equiv E(0)}\).

\paragraph{Шаг 3.} \textbf{Доказать положительную определённость энергии.}\\
\(V(\mathbf 0)=0,\ V(\mathbf x)>0\ (\mathbf x\neq 0)\Rightarrow 
\boxed{E(\mathbf x,\mathbf v)\ge 0,\ E=0\Leftrightarrow (\mathbf x,\mathbf v)=(\mathbf 0,\mathbf 0)}\).

\paragraph{Шаг 4.} \textbf{Использовать субуровни энергии для оценки траекторий.}\\
\(m(\varepsilon):=\min_{\|\mathbf x\|=\varepsilon}V(\mathbf x)>0\). 
Если \(E(0)<m(\varepsilon)\), то \(E(t)\equiv E(0)<m(\varepsilon)\) и \(\|\mathbf x(t)\|<\varepsilon\) \(\forall t\ge0\).

\paragraph{Шаг 5.} \textbf{Доказать устойчивость по Ляпунову.}\\
\(\forall\,\varepsilon>0\ \exists\,\delta(\varepsilon)>0:\ \|(\mathbf x(0),\mathbf v(0))\|<\delta\Rightarrow \|\mathbf x(t)\|<\varepsilon\ \forall t\ge0\). 
\[
\boxed{\ (\mathbf 0,\mathbf 0)\ \text{устойчиво по Ляпунову (не асимптотически)}\ }
\]

\subsection*{3. Сопроводительные материалы (таблицы и обозначения)}

\textbf{Детектор потенциальности:}
\[
\mathbf f(\mathbf x)=-\nabla V(\mathbf x)\ \Leftrightarrow\ \partial_{x_j}f_i=\partial_{x_i}f_j\ \ (\forall i,j)
\]

\textbf{Локальная квадратичная аппроксимация потенциала:}
\[
\nabla V(\mathbf 0)=\mathbf 0,\ H=\nabla^{2}V(\mathbf 0),\ H\succ0\Rightarrow 
V(\mathbf x)=\tfrac12\,\mathbf x^{\!\top}H\mathbf x+o(\|\mathbf x\|^{2})
\]
\[
\Rightarrow \exists m>0: V(\mathbf x)\ge \tfrac m2\|\mathbf x\|^{2}\ \text{(в малой окрестности)}
\]

\textbf{Энергетический кандидат Ляпунова:}
\[
\boxed{E(\mathbf x,\mathbf v)=\tfrac12\|\mathbf v\|^{2}+V(\mathbf x)},\quad \boxed{\dot E=0}
\]

\textbf{Свойства субуровней энергии:}
\(\mathcal L_{c}:=\{(\mathbf x,\mathbf v):E(\mathbf x,\mathbf v)\le c\}\) — замкнутые множества; при малых \(c\) лежат в окрестности \((\mathbf 0,\mathbf 0)\).

\textbf{Критерии устойчивости:}
\begin{center}
\begin{tabular}{|c|c|}
\hline
\textbf{Условие} & \textbf{Вывод} \\ \hline
\(V(\mathbf 0)=0\), \(V(\mathbf x)>0\) при \(\mathbf x\neq 0\) & устойчивость по Ляпунову \\ \hline
\(H=\nabla^{2}V(\mathbf 0)\succ 0\) & локальная устойчивость \\ \hline
\(\dot E=0\) & консервативная система \\ \hline
\end{tabular}
\end{center}

\subsection*{4. Применение алгоритма к объявленной задаче}

\[
\ddot{\mathbf x}=-\nabla V(\mathbf x),\quad V(\mathbf 0)=\min V,\quad V(\mathbf x)>0\ \text{при}\ \mathbf x\neq \mathbf 0
\]

\paragraph{Шаг 0.} \textbf{Проверить потенциальность системы.}\\
Система задана в потенциальной форме: \(\mathbf f(\mathbf x)=-\nabla V(\mathbf x)\).

\paragraph{Шаг 1.} \textbf{Найти положения равновесия.}\\
\(\nabla V(\mathbf 0)=\mathbf 0\Rightarrow \mathbf x_*=\mathbf 0\).

\paragraph{Шаг 2.} \textbf{Проверить сохранение энергии.}\\
\(E=\tfrac12\|\dot{\mathbf x}\|^{2}+V(\mathbf x)\), \(\dot E=0\).

\paragraph{Шаг 3.} \textbf{Доказать положительную определённость энергии.}\\
\(V(\mathbf x)\ge0\), \(=0\Leftrightarrow \mathbf x=\mathbf 0\Rightarrow E\ge0\), нуль только в \((\mathbf 0,\mathbf 0)\).

\paragraph{Шаг 4.} \textbf{Использовать субуровни энергии для оценки траекторий.}\\
\(m(\varepsilon)=\min_{\|\mathbf x\|=\varepsilon}V(\mathbf x)>0\), \(E(0)<m(\varepsilon)\Rightarrow \|\mathbf x(t)\|<\varepsilon\).

\paragraph{Шаг 5.} \textbf{Доказать устойчивость по Ляпунову.}\\
\(\boxed{(\mathbf 0,\mathbf 0)\ \text{устойчиво по Ляпунову}}\).

