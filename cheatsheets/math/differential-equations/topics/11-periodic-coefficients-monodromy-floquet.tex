\section{Периодические коэффициенты, сдвиг, монодромия, Флоке}

\subsection*{1. Тип экзаменационной задачи (полное условие)}
Пусть \(q\in C(\mathbb R)\) периодична с периодом \(1\). Пусть \(y\not\equiv0\) — решение
\[
y''(x)+q(x)\,y(x)=0,\qquad y(0)=0,\quad y(1)=0.
\]
Доказать: \(\exists\,C\in\mathbb R:\ \boxed{\,y(x+1)=C\,y(x)\ \forall x\,}\).

\subsection*{2. Универсальный алгоритм (визуальные формулы и детерминированные шаги)}

\textbf{Исходные данные и обозначения (ввод).} Дано уравнение \(y''(x)+q(x)\,y(x)=0\), где \(q(x+T)=q(x)\) с периодом \(T>0\).
Вводим векторную форму: \(X=\binom{y}{y'}\), \(X'=A(x)X\), где \(A(x)=\begin{psmallmatrix}0&1\\ -q(x)&0\end{psmallmatrix}\).
Вводим фундаментальную матрицу: \(Y'(x)=A(x)Y(x)\), \(Y(0)=I\).
Вводим монодромию: \(M:=Y(T)\).
По формуле Абеля: \(\boxed{\det M=1}\).

\paragraph{Шаг 0.} \textbf{Проверить периодичность коэффициентов.}\\
Убедиться, что \(q(x+T)=q(x)\) для некоторого \(T>0\).

\paragraph{Шаг 1.} \textbf{Применить сдвиг к решению.}\\
Определить \(y_T(x):=y(x+T)\). Тогда \(y_T''+q\,y_T=0\) (так как \(q\) периодична).

\paragraph{Шаг 2.} \textbf{Найти 1D-подпространство при граничных условиях.}\\
Если \(y(0)=0\), выбрать базисное решение \(u\): \(u(0)=0,\ u'(0)=1\).
Тогда \(\mathcal S_0=\{\alpha u\ :\ \alpha\in\mathbb R\}\).

\paragraph{Шаг 3.} \textbf{Доказать скалярность сдвига на \(\mathcal S_0\).}\\
Если \(y(0)=y(T)=0\) и \(y=\beta u\neq0\), то \(u(T)=0\) и \(\exists\,\alpha:\ u(x+T)=\alpha\,u(x)\).
Сравнивая производные в нуле: \(\alpha=\dfrac{u'(T)}{u'(0)}=u'(T)\).

\paragraph{Шаг 4.} \textbf{Вывести формулу для коэффициента \(C\).}\\
\[
\boxed{\,y(x+T)=C\,y(x),\quad C=u'(T)=\dfrac{y'(T)}{y'(0)}\,}
\]

\paragraph{Шаг 5.} \textbf{Связать с матрицей монодромии.}\\
Выбрать базис \((u,v)\): \(u(0)=0,u'(0)=1;\ v(0)=1,v'(0)=0\).
\[
M=Y(T)=\begin{pmatrix}v(T)&u(T)\\ v'(T)&u'(T)\end{pmatrix}
=\begin{pmatrix}v(T)&0\\ v'(T)&C\end{pmatrix}
\]
Из \(\det M=1\): \(v(T)=C^{-1}\).
Собственные числа: \(\rho_1=C,\ \rho_2=C^{-1}\).

\subsection*{3. Сопроводительные материалы (таблицы и обозначения)}

\textbf{Основные свойства монодромии:}
\[
Y(x+T)=Y(x)M,\quad \det M=1,\quad \sigma(M)=\{\rho_1,\rho_2\},\ \rho_1\rho_2=1
\]

\textbf{Классификация по собственным числам монодромии:}
\begin{center}
\begin{tabular}{|c|c|c|}
\hline
\textbf{Условие} & \(\rho_{1,2}\) & \textbf{Поведение решений} \\ \hline
\(|\operatorname{tr}M|<2\) & \(e^{\pm i\theta}\) & ограниченные осцилляции \\ \hline
\(|\operatorname{tr}M|=2\) & \(\pm1\) & пороговый случай \\ \hline
\(|\operatorname{tr}M|>2\) & \(\rho_{1,2}\in\mathbb R\), \(\rho_1=\rho_2^{-1}\), \(\rho_1>1\) & рост/затухание \\ \hline
\end{tabular}
\end{center}

\textbf{Специальный случай \(y(0)=y(T)=0\):}
\[
u(T)=0\ \Rightarrow\ 
M=\begin{pmatrix}C^{-1}&0\\ *&C\end{pmatrix},\qquad
\boxed{\,y(x+T)=C\,y(x),\ C=u'(T)=\dfrac{y'(T)}{y'(0)}\,}
\]

\textbf{Формула Абеля для периодических систем:}
\[
\boxed{\,\det Y(x)=\det Y(0)\exp\!\left(\int_0^x \operatorname{tr}A(s)\,ds\right)\,}
\]
Для \(A(x)=\begin{psmallmatrix}0&1\\ -q(x)&0\end{psmallmatrix}\): \(\operatorname{tr}A(x)=0\), поэтому \(\det M=1\).

\subsection*{4. Применение алгоритма к объявленной задаче}

\[
y''(x)+q(x)\,y(x)=0,\qquad y(0)=0,\quad y(1)=0,\quad T=1
\]

\paragraph{Шаг 0.} \textbf{Проверить периодичность коэффициентов.}\\
\(q(x+1)=q(x)\) — условие выполнено.

\paragraph{Шаг 1.} \textbf{Применить сдвиг к решению.}\\
\(y_1(x)=y(x+1)\Rightarrow y_1''+q\,y_1=0\).

\paragraph{Шаг 2.} \textbf{Найти 1D-подпространство при граничных условиях.}\\
\(u(0)=0,\ u'(0)=1\Rightarrow \mathcal S_0=\{\alpha u\}\).

\paragraph{Шаг 3.} \textbf{Доказать скалярность сдвига на \(\mathcal S_0\).}\\
\(y=\beta u\neq0,\ y(1)=0\Rightarrow u(1)=0\).

\paragraph{Шаг 4.} \textbf{Вывести формулу для коэффициента \(C\).}\\
\(u(x+1)=C\,u(x),\ C=u'(1)\Rightarrow y(x+1)=C\,y(x)\).

\paragraph{Шаг 5.} \textbf{Связать с матрицей монодромии.}\\
\(M=Y(1)=\begin{psmallmatrix}v(1)&0\\ v'(1)&C\end{psmallmatrix}\), \(\det M=1\Rightarrow v(1)=C^{-1}\), \(\sigma(M)=\{C,C^{-1}\}\).

Итог:
\[
\boxed{\,y(x+1)=\dfrac{y'(1)}{y'(0)}\,y(x)\,}
\]

