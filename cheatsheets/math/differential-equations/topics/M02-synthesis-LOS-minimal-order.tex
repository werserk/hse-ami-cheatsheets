\section{M2. Синтез ЛОС по заданным частным решениям (минимальный порядок)}

\subsection*{1. Тип экзаменационной задачи (полное условие)}
\textbf{Задача.} Построить \emph{линейное однородное разностное уравнение с постоянными коэффициентами} минимально возможного порядка, частными решениями которого являются
\[
y^{(1)}_{t}=3^{\,t},\qquad y^{(2)}_{t}=2^{\,t}\,\sin\!\frac{\pi t}{3}.
\]
(Решение здесь не приводится; это контекст для главы.)

\subsection*{2. Универсальный алгоритм (визуальные формулы и детерминированные шаги)}

\textbf{Исходные данные и обозначения (ввод).} Дано множество частных решений \(\{y^{(k)}_t\}_{k=1}^K\) ЛОС. Требуется построить характеристический полином \(p(\lambda)\) минимального порядка \(N\) такой, что все \(y^{(k)}_t\) являются решениями уравнения \(p(L)[y_t]=0\), где \(L\) — оператор сдвига \(Ly_t=y_{t+1}\).

Вводим: \(\alpha\in\mathbb{R}\) — основание экспоненты; \(\omega\in\mathbb{R}\) — частота тригонометрических функций; \(s\in\mathbb{N}_0\) — степень полинома \(t^s\); \(p(\lambda)\in\mathbb{R}[\lambda]\) — характеристический полином.

\paragraph{Шаг 0.} \textbf{Распознать «атом» каждого данного решения.}\\
Для каждого \(y^{(k)}_t\) определить одну из форм:
\(\alpha^t\); \(t^{s}\alpha^t\); \(\alpha^t\cos(\omega t)\) или \(\alpha^t\sin(\omega t)\); \(t^{s}\alpha^t\cos(\omega t)\) или \(t^{s}\alpha^t\sin(\omega t)\).

\paragraph{Шаг 1.} \textbf{Получить характеристический множитель(и) для каждого атома.}\\
По таблице соответствий заменить атом на множитель \(p(\lambda)\) с учётом кратности \((s+1)\).

\paragraph{Шаг 2.} \textbf{Собрать общий характеристический полином минимального порядка.}\\
Перемножить \emph{разные} множители (комплексные корни берутся парой \(\Rightarrow\) реальный квадратичный множитель). Повторы дают максимальную кратность.

\paragraph{Шаг 3.} \textbf{Записать разностное уравнение.}\\
Привести \(p(\lambda)\) к виду \(\lambda^N+a_{N-1}\lambda^{N-1}+\dots+a_1\lambda+a_0\) и выписать
\[
y_{t+N}+a_{N-1}y_{t+N-1}+\dots+a_1y_{t+1}+a_0y_t=0.
\]

\paragraph{Шаг 4.} \textbf{Проверить минимальность и корректность.}\\
Убедиться, что \(N\) равен сумме степеней множителей; проверить зануление \(p(\lambda)\) на атомах (для тригонометрических — на \(\lambda=\alpha e^{\pm i\omega}\)).

\subsection*{3. Сопроводительные материалы (таблицы и обозначения)}

\textbf{Таблица соответствий (атом \(\Rightarrow\) множитель \(\Rightarrow\) кратность):}
\begin{center}
\begin{tabular}{|c|c|c|}
\hline
\textbf{Атом} & \textbf{Множитель} & \textbf{Кратность} \\
\hline
$\alpha^t$ & $(\lambda-\alpha)$ & $1$ \\
\hline
$t^{s}\alpha^t$ & $(\lambda-\alpha)^{s+1}$ & $s+1$ \\
\hline
$\alpha^t\cos(\omega t),\ \alpha^t\sin(\omega t)$ & $\lambda^2-2\alpha\cos\omega\,\lambda+\alpha^2$ & $1$ \\
\hline
$t^{s}\alpha^t\cos(\omega t),\ t^{s}\alpha^t\sin(\omega t)$ & $(\lambda^2-2\alpha\cos\omega\,\lambda+\alpha^2)^{s+1}$ & $s+1$ \\
\hline
\end{tabular}
\end{center}

\textbf{Быстрые значения \(\cos\omega\):}
\[
\cos\frac{\pi}{6}=\frac{\sqrt3}{2},\quad
\cos\frac{\pi}{4}=\frac{\sqrt2}{2},\quad
\cos\frac{\pi}{3}=\frac12,\quad
\cos\frac{\pi}{2}=0.
\]

\textbf{Правила сборки:} (i) Пара \(\{\cos,\sin\}\) с одинаковыми \(\alpha,\omega\) даёт один и тот же квадратичный множитель (не удваивать). (ii) При нескольких степенях \(t^{s}\) берётся максимальная кратность.

\subsection*{4. Применение алгоритма к объявленной задаче}

\textbf{Дано:} \(y^{(1)}_t=3^{t}\), \(y^{(2)}_t=2^{t}\sin\!\frac{\pi t}{3}\).

\paragraph{Шаг 0.} \textbf{Распознать «атом» каждого данного решения.}\\
Атомы: \(3^t\) (\(\alpha=3\)); \(2^t\sin(\pi t/3)\) (\(\alpha=2,\ \omega=\pi/3\)).

\paragraph{Шаг 1.} \textbf{Получить характеристический множитель(и) для каждого атома.}\\
Множители: \((\lambda-3)\) и \(\lambda^2-2\cdot 2\cos(\pi/3)\lambda+2^2=\lambda^2-2\lambda+4\).

\paragraph{Шаг 2.} \textbf{Собрать общий характеристический полином минимального порядка.}\\
Сборка: \(p(\lambda)=(\lambda-3)(\lambda^2-2\lambda+4)\).

\paragraph{Шаг 3.} \textbf{Записать разностное уравнение.}\\
Развёртка: \(p(\lambda)=\lambda^3-5\lambda^2+10\lambda-12\).
Соответствующее ЛОС:
\[
\boxed{\,y_{t+3}-5y_{t+2}+10y_{t+1}-12y_t=0\,}.
\]

\paragraph{Шаг 4.} \textbf{Проверить минимальность и корректность.}\\
Минимальность: порядок \(N=3\); проверка \(p(3)=0\) и \(\lambda=2e^{\pm i\pi/3}\) зануляют квадратичный множитель.
