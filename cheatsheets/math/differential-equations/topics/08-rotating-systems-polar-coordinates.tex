\section{Вращающиеся системы и полярные координаты: \((x,y)\to(r,\theta)\), \(\dot r,\dot\theta\), классификация}

\subsection*{1. Тип экзаменационной задачи (полное условие)}
Исследовать фазовый портрет
\[
\begin{cases}
\dot{x}=a\,y+x(x^{2}+y^{2}),\\
\dot{y}=-a\,x+y(x^{2}+y^{2}),
\end{cases}
\]
в окрестности \((0,0)\) для всех \(a\in\mathbb R\). \textit{Указание:} перейти в полярные координаты. Определить тип начала координат для линеаризованной системы.

\subsection*{2. Универсальный алгоритм (визуальные формулы и детерминированные шаги)}

\textbf{Исходные данные и обозначения (ввод).} Дана система \(\dot x=f(x,y),\ \dot y=g(x,y)\) в декартовых координатах.
Вводим полярные координаты: \(x=r\cos\theta,\ y=r\sin\theta\), где \(r>0\).
Вводим функции: \(H(x,y):=\frac{x f+y g}{x^{2}+y^{2}}\) — радиальная компонента; \(A(x,y):=-\,\frac{x g-y f}{x^{2}+y^{2}}\) — угловая компонента.
Тогда \(\dot r=r\,H,\ \dot\theta=A\).

\paragraph{Шаг 0.} \textbf{Перейти к полярным координатам.}\\
Использовать формулы преобразования:
\[
\boxed{\ \dot r=\frac{x f+y g}{r},\qquad \dot\theta=\frac{x g-y f}{r^{2}}\ }
\]
или через введённые функции:
\[
\boxed{\ \dot r=r\,H,\quad \dot\theta=A\ }
\]

\paragraph{Шаг 1.} \textbf{Вычислить функции \(H\) и \(A\).}\\
Найти \(H(x,y)=\frac{x f+y g}{x^{2}+y^{2}}\) и \(A(x,y)=-\,\frac{x g-y f}{x^{2}+y^{2}}\).

\paragraph{Шаг 2.} \textbf{Определить порядки малости у \(r=0\).}\\
Разложить \(H(r,\theta)=h_k(\theta)\,r^{k}+o(r^{k})\) и \(A(r,\theta)=a_0+ a_1(\theta)\,r+\dots\) при \(r\to 0\).

\paragraph{Шаг 3.} \textbf{Классифицировать тип равновесия по знаку \(h_k\).}\\
Анализировать \(\dot r=r\,H\):
\begin{itemize}
\item Если \(k=0\) и \(h_0\neq 0\): \(\dot r\sim h_0 r\)
  \begin{itemize}
  \item \(h_0<0\): устойчивый фокус
  \item \(h_0>0\): неустойчивый фокус
  \end{itemize}
\item Если \(k\ge 1\): \(\dot r\sim h_k(\theta)\,r^{k+1}\) — негиперболическое равновесие
  \begin{itemize}
  \item \(h_k(\theta)>0\): радиальный разлёт
  \item \(h_k(\theta)<0\): радиальное притяжение
  \end{itemize}
\end{itemize}
Анализировать \(\dot\theta \sim a_0\): при \(a_0\neq 0\) — равномерное вращение (знак \(a_0\) задаёт направление).

\paragraph{Шаг 4.} \textbf{Проверить линеаризацию.}\\
Вычислить якобиан \(J(0,0)=\begin{psmallmatrix}f_x&f_y\\ g_x&g_y\end{psmallmatrix}\).
Если \(\sigma(J)=\{\pm i a_0\}\) — центр для линейной части, устойчивость определяет \(H\).

\paragraph{Шаг 5.} \textbf{Построить эскиз фазового портрета.}\\
Нарисовать стрелки по знакам \(\operatorname{sgn}(\dot r)\) и \(\operatorname{sgn}(\dot\theta)\).
Кривая \(\dot r=0\Leftrightarrow H=0\) — радиальные барьеры.

\subsection*{3. Сопроводительные материалы (таблицы и обозначения)}

\textbf{Основные формулы преобразования:}
\[
\boxed{\ x f+y g=r^{2}H\ },\qquad
\boxed{\ x g-y f=-\,r^{2}A\ }.
\]

\textbf{Классификация по порядку малости \(H\):}
\[
H(r,\theta)\sim
\begin{cases}
h_0\ (\neq0) &\Rightarrow\ \dot r\sim h_0 r\ \Rightarrow\
\begin{cases}
h_0<0:\ \text{устойчивый фокус},\\
h_0>0:\ \text{неустойчивый фокус},
\end{cases}\\[6pt]
h_k(\theta)\,r^{k},\ k\ge1 &\Rightarrow\ \dot r\sim h_k(\theta)\,r^{k+1}\ \text{(негиперболика)}
\end{cases}
\]

\textbf{Специальный случай:} Если \(f=a\,y+x\,\Phi,\ g=-a\,x+y\,\Phi\), то
\[
\boxed{\ \dot r=r\,\Phi,\quad \dot\theta=-a\ }.
\]

\subsection*{4. Применение алгоритма к объявленной задаче}

\[
\begin{cases}
\dot{x}=a\,y+x(x^{2}+y^{2}),\\
\dot{y}=-a\,x+y(x^{2}+y^{2}),
\end{cases}
\]

\paragraph{Шаг 0.} \textbf{Перейти к полярным координатам.}\\
Используем формулы: \(f(x,y)=a\,y+x(x^{2}+y^{2})\), \(g(x,y)=-a\,x+y(x^{2}+y^{2})\).

\paragraph{Шаг 1.} \textbf{Вычислить функции \(H\) и \(A\).}\\
\[
H=\frac{x(ay+xr^{2})+y(-ax+yr^{2})}{r^{2}}=\frac{ar^{2}\sin\theta\cos\theta+ar^{2}\sin\theta\cos\theta+r^{4}}{r^{2}}=r^{2}
\]
\[
A=-\frac{x(-ax+yr^{2})-y(ay+xr^{2})}{r^{2}}=-\frac{-ar^{2}\cos^{2}\theta-ar^{2}\sin^{2}\theta}{r^{2}}=a
\]

\paragraph{Шаг 2.} \textbf{Определить порядки малости у \(r=0\).}\\
\(H=r^{2}\Rightarrow k=2,\ h_2\equiv1>0\); \(A\equiv a\).

\paragraph{Шаг 3.} \textbf{Классифицировать тип равновесия.}\\
\(\dot r=r^{3}>0\) при \(r>0\Rightarrow\) радиальный разлёт (негиперболическая неустойчивость).
\(\dot\theta=a\Rightarrow\) равномерное вращение (знак \(a\) задаёт направление).

\paragraph{Шаг 4.} \textbf{Проверить линеаризацию.}\\
\(J(0,0)=\begin{psmallmatrix}0&a\\ -a&0\end{psmallmatrix}\), \(\sigma(J)=\{\pm i a\}\Rightarrow\) центр для линейной части (\(a\neq0\)); истинная динамика — разлёт из-за \(r^{3}\).

\paragraph{Шаг 5.} \textbf{Построить эскиз фазового портрета.}\\
Эскиз: расходящиеся спирали при \(a\neq0\); при \(a=0\) — чисто радиальный разлёт (\(\dot r=r^{3},\ \dot\theta\equiv0\)).

\boxed{\text{При }a\neq0\text{: расходящиеся спирали; при }a=0\text{: радиальный разлёт}}

