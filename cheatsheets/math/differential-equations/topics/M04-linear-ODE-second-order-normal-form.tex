\section{M4. Линейные ОДУ 2-го порядка: снятие $y'$, вронскиан, нормальная форма $z''+Qz=0$, выводы про нули}

\subsection*{1. Тип экзаменационной задачи (полное условие)}
Пусть функции $p(x),q(x)\in C(\mathbb R)$ и $q(x)<0$ для всех $x\in\mathbb R$. Пусть $y(x)\not\equiv0$ — решение
\[
y''(x)+p(x)\,y'(x)+q(x)\,y(x)=0.
\]
Докажите: если $y$ имеет локальный максимум в точке $x_0$, то $y(x_0)\le 0$.

\subsection*{2. Универсальный алгоритм (визуальные формулы и детерминированные шаги)}

\textbf{Исходные данные и обозначения (ввод).}
Рассматривается линейное ОДУ второго порядка
\[
y''+p(x)\,y'(x)+q(x)\,y(x)=0,\quad x\in I\subset\mathbb R,
\]
где $y:I\to\mathbb R$ (или $\mathbb C$), $p,q\in C(I)$.
Вводим обозначения: $\phi(x)\in C^1(I)$ — интегрирующий множитель для снятия $y'$; $z=\phi^{-1}y$; $Q(x)$ — эффективный потенциал в нормальной форме; $W[y_1,y_2]=y_1y_2'-y_2y_1'$ — вронскиан пары решений.

\paragraph{Шаг 0.} \textbf{Диагноз линейности и фиксация коэффициентов.}\\
Привести уравнение к виду $y''+p(x)y'(x)+q(x)y(x)=0$; зафиксировать $p,q$.
Если коэффициент при $y''$ зависит от $y$ или $y'$ (не только от $x$) — это \emph{не M4}.

\paragraph{Шаг 1.} \textbf{Нормальная форма (снять $y'$).}\\
Взять
\[
\phi(x)=\exp\!\Bigl(-\tfrac12\!\int p(x)\,dx\Bigr),\qquad y=\phi z,
\]
тогда
\[
z''+Q(x)\,z=0,\qquad Q(x)=q(x)-\frac{p'(x)}{2}-\frac{p(x)^2}{4}.
\]

\paragraph{Шаг 2.} \textbf{Вронскиан и независимость (Абель).}\\
Для двух решений $y_1,y_2$:
\[
W'=-p\,W,\qquad W(x)=W(x_0)\,\exp\!\Bigl(-\!\int_{x_0}^{x}p(t)\,dt\Bigr).
\]
Критерий: $W(x_0)\neq0\Rightarrow y_1,y_2$ линейно независимы на интервале.

\paragraph{Шаг 3.} \textbf{Качественные выводы по триггерам.}\\
\begin{itemize}
\item Триггер «экстремум»: в $x_0$ максимума $y'(x_0)=0,\ y''(x_0)\le0$; подставить в ОДУ.
\item Триггер «$\le 1$ нуля»: перейти к $z''+Qz=0$; при $Q\le0$ интегральный приём
\[
\int_a^b zz''\,dx+\int_a^b Qz^2\,dx=0\Rightarrow -\int_a^b (z')^2\,dx+\int_a^b Qz^2\,dx=0,
\]
что исключает два нуля у нетривиального решения.
\end{itemize}

\paragraph{Шаг 4.} \textbf{Итог (коробочная формулировка).}\\
Записать использованные $\phi,Q$ или $W$ и сформулировать вывод в $\boxed{\cdots}$.

\subsection*{3. Сопроводительные материалы (таблицы и обозначения)}

\textbf{Детектор линейности (сторожок).}
\[
\begin{array}{|c|c|}
\hline
\text{Признак} & \text{Вывод} \\
\hline
a_2(x)y''+a_1(x)y'+a_0(x)y=0\ \text{с }a_i=a_i(x) & \text{линейное (наш класс M4)} \\
\hline
\text{Коэффициент при }y''\text{ зависит от }y\text{ или }y' & \text{не M4 (нельзя Абеля/ФС)} \\
\hline
\end{array}
\]

\textbf{Памятка формул M4.}
\[
\phi(x)=\exp\!\Bigl(-\tfrac12\!\int p\Bigr),\qquad
Q=q-\tfrac12 p'-\tfrac14 p^2,\qquad
W(x)=W(x_0)\exp\!\Bigl(-\!\int_{x_0}^{x}p\Bigr).
\]

\textbf{Детектор ветки шага 3 (какой приём включать).}
\[
\begin{array}{|c|c|}
\hline
\text{Признак в условии} & \text{Действие} \\
\hline
\text{есть максимум/минимум и знак }q & \text{экстремум-тест: } y'=0,\ \text{подстановка в ОДУ} \\
\hline
\text{нужно «}\le 1\text{ нуля»} & z''+Qz=0,\ Q\le0\ \Rightarrow\ \text{интегральный приём} \\
\hline
\text{независимость пары} & \text{формула Абеля для }W \\
\hline
\end{array}
\]

\subsection*{4. Применение алгоритма к объявленной задаче}

\[
y''+p(x)y'+q(x)y=0,\qquad q(x)<0\ \forall x\in\mathbb R,\quad y\not\equiv0.
\]

\paragraph{Шаг 0.} \textbf{Диагноз линейности и фиксация коэффициентов.}\\
Уравнение уже в виде $y''+p y'+q y=0$; задан знак $q<0$.

\paragraph{Шаг 1.} \textbf{Нормальная форма (снять $y'$).}\\
Для экстремума переход не обязателен; оставляем исходную форму.

\paragraph{Шаг 2.} \textbf{Вронскиан и независимость (Абель).}\\
Не используется в данном выводе.

\paragraph{Шаг 3.} \textbf{Качественные выводы по триггерам.}\\
Триггер «экстремум»: в точке локального максимума $x_0$ имеем $y'(x_0)=0$, $y''(x_0)\le0$. Подстановка даёт
\[
y''(x_0)=-p(x_0)y'(x_0)-q(x_0)y(x_0)=-q(x_0)y(x_0).
\]
При $q(x_0)<0$ из $y(x_0)>0$ следовало бы $y''(x_0)>0$ — противоречие. Значит $y(x_0)\le0$.

\paragraph{Шаг 4.} \textbf{Итог (коробочная формулировка).}\\
\[
\boxed{\,\text{Положительный локальный максимум невозможен при } q(x)<0\, }.
\]