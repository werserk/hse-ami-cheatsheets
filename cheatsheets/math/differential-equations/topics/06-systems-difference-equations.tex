\section{Глава M6. Системы разностных: диагонализуемые матрицы, Phi t равно A в степени t, вариация постоянных}

\subsection*{1) Тип экзаменационной задачи}
\textbf{Условие.}
\[
\begin{pmatrix} x_{t+1}\\[2pt] y_{t+1} \end{pmatrix}
= A \begin{pmatrix} x_{t}\\[2pt] y_{t} \end{pmatrix}
+ \begin{pmatrix} 1\\[2pt] -1 \end{pmatrix}, 
\qquad 
A=\frac12\begin{pmatrix} -1 & 3\\[2pt] 3 & -1 \end{pmatrix}.
\]
(а) Найти фундаментальную матрицу \(\Phi_t\).
(б) Полагая \(\binom{x_t}{y_t}=\Phi_t\binom{c_1^{\,t}}{c_2^{\,t}}\), выписать уравнения для \(c_1^{\,t},c_2^{\,t}\) (не решать).

\subsection*{2) Универсальный алгоритм (формулы)}
\textbf{Ввод.} \(A\in\mathbb R^{n\times n}\), \(x_t\in\mathbb R^n\), \(b_t\in\mathbb R^n\). \(\ \Phi_t:=A^t\).
Спектр: \(A=V\Lambda V^{-1}\), \(\Lambda=\operatorname{diag}(\lambda_1,\dots,\lambda_n)\).

\paragraph{Шаг 1. Спектр.}
Найти \(\lambda_j\) и базис \(\{v_j\}\): \((A-\lambda_j I)v_j=0\). \(\sum_j \dim\ker(A-\lambda_j I)=n\Rightarrow\) диагонализуемо.

\paragraph{Шаг 2. A в степени t.}
\[
A^t=V\Lambda^tV^{-1},\quad \Lambda^t=\operatorname{diag}(\lambda_1^t,\dots,\lambda_n^t).
\]
Если \(\lambda=\rho e^{\pm i\theta}\): на \(\mathbb R^2\) блок \(S=\begin{smallmatrix}a&-b\\ b&a\end{smallmatrix}\), \(a+ib=\lambda\), 
\[
S^t=\rho^t\begin{pmatrix}\cos(\theta t)&-\sin(\theta t)\\ \sin(\theta t)&\cos(\theta t)\end{pmatrix}.
\]

\paragraph{Шаг 3. Phi t и однородная система.}
\[
x_{t+1}=Ax_t\ \Rightarrow\ x_t=\Phi_t x_0,\quad \Phi_t=A^t.
\]

\paragraph{Шаг 4. Вариация постоянных.}
Полагаем \(x_t=\Phi_t c^{\,t}\). Тогда
\[
\Phi_{t+1}c^{\,t+1}=\Phi_{t+1}c^{\,t}+b_t\ \Rightarrow\ 
\boxed{\,c^{\,t+1}-c^{\,t}=\Phi_{t+1}^{-1}b_t\,}.
\]
Эквивалентно: \(x_t=A^t x_0+\sum_{k=0}^{t-1}A^{t-1-k}b_k\).

\paragraph{Шаг 5. Частные случаи.}
Если \(b_t\equiv b\) и \(I-A\) обратима: \(x_t=A^t(x_0-(I-A)^{-1}b)+(I-A)^{-1}b\).
Если \(\lambda<0\): \(\lambda^t=(-1)^t|\lambda|^t\).
Пара \(\rho e^{\pm i\theta}\): блок \(\rho^tR(\theta t)\).

\subsection*{3) Сопроводительные материалы}

\begin{center}
\begin{tabular}{|c|c|}
\hline
Спектр \(A\) & Формула для \(A^t\)\\
\hline
\(\lambda_j\in\mathbb R\) простые & \(V\operatorname{diag}(\lambda_1^t,\dots,\lambda_n^t)V^{-1}\)\\
\hline
\(\rho e^{\pm i\theta}\) & \(W\bigl(\rho^t\begin{smallmatrix}\cos\theta t&-\sin\theta t\\ \sin\theta t&\cos\theta t\end{smallmatrix}\bigr)W^{-1}\)\\
\hline
смешанный & блочно по строкам выше\\
\hline
\end{tabular}
\end{center}

\[
\Phi_t^{-1}=V\,\operatorname{diag}(\lambda_1^{-t},\dots,\lambda_n^{-t})\,V^{-1}.
\]

\subsection*{4) Применение алгоритма к условию}

\paragraph{Шаг 1.} \(\widehat{A}=\begin{smallmatrix}-1&3\\ 3&-1\end{smallmatrix}\Rightarrow \sigma(\widehat{A})=\{2,-4\}\), \(v_1=(1,1)\), \(v_2=(1,-1)\).  
\(\sigma(A)=\{1,-2\}\) (диагонализуемо).

\paragraph{Шаг 2.}
\[
V=\begin{pmatrix}1&1\\ 1&-1\end{pmatrix},\quad
\Lambda=\operatorname{diag}(1,-2),\quad
V^{-1}=\tfrac12\begin{pmatrix}1&1\\ 1&-1\end{pmatrix}.
\]
\[
\Phi_t=A^t=\tfrac12\begin{pmatrix}1+(-2)^t & 1-(-2)^t\\ 1-(-2)^t & 1+(-2)^t\end{pmatrix}.
\]

\paragraph{Шаг 3.} \(x_t=\Phi_t x_0\).

\paragraph{Шаг 4.}
\[
c^{\,t+1}-c^{\,t}=\Phi_{t+1}^{-1}b,\quad
\Phi_{t+1}^{-1}=V\operatorname{diag}\!\bigl(1,(-2)^{-(t+1)}\bigr)V^{-1}.
\]
\[
\Phi_{t+1}^{-1}b=\frac12\!\begin{pmatrix}1+(-2)^{-(t+1)} & 1-(-2)^{-(t+1)}\\ 1-(-2)^{-(t+1)} & 1+(-2)^{-(t+1)}\end{pmatrix}\!\!\binom{1}{-1}
=\binom{(-\tfrac12)^{\,t+1}}{-(-\tfrac12)^{\,t+1}}.
\]
\[
\boxed{\,c_1^{\,t+1}-c_1^{\,t}=(-\tfrac12)^{\,t+1},\qquad c_2^{\,t+1}-c_2^{\,t}=-(-\tfrac12)^{\,t+1}\,}.
\]

\paragraph{Шаг 5.} \((I-A)\) необратима (есть \(\lambda=1\)) \(\Rightarrow\) стационарная формула неприменима; используем вариацию постоянных как выше.