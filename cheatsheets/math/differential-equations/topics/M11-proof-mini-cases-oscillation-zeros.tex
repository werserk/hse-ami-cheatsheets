\section{M11. Доказательные мини-кейсы: осцилляция и нули решений (пример: Бессель)}

\subsection*{1. Тип экзаменационной задачи (полное условие)}
Докажите, что всякое нетривиальное решение уравнения Бесселя
\[
x^{2}y''+x y'+\Bigl(x^{2}-\tfrac12\Bigr)y=0
\]
имеет бесконечно много нулей на промежутке \(x>0\).

\subsection*{2. Универсальный алгоритм (визуальные формулы и детерминированные шаги)}

\textbf{Исходные данные и обозначения (ввод).}
Рассматривается \(y''+p(x)y'+q(x)y=0\), \(p,q\in C^{1}[X_0,\infty)\).
Нормальная форма: \(y=\phi z\), где
\[
\phi(x)=\exp\!\Bigl(-\tfrac12\int p(x)\,dx\Bigr),\qquad
Q(x)=q(x)-\frac{p'(x)}{2}-\frac{p(x)^{2}}{4},
\]
и \(z\) удовлетворяет \(z''+Q(x)z=0\).

\paragraph{Шаг 1.} \textbf{Привести к нормальной форме \(z''+Q(x)z=0\).}\\
Вычислить \(p,q\); взять \(\phi=\exp(-\tfrac12\int p)\), положить \(y=\phi z\) и \(Q=q-\tfrac12 p'-\tfrac14 p^{2}\).

\paragraph{Шаг 2.} \textbf{Зафиксировать численный детектор по \(Q\) на хвосте.}\\
Найти \(X_1\ge X_0\) и \(\alpha>0\) такие, что на \([X_1,\infty)\) выполнено, например, \(Q(x)\ge \alpha^{2}\) или \(Q(x)\le 0\).

\paragraph{Шаг 3.} \textbf{Выбрать эталон для сравнения.}\\
При \(Q\ge \alpha^{2}>0\) — эталон \(v''+\alpha^{2}v=0\); при \(Q\le 0\) — эталон \(v''=0\).

\paragraph{Шаг 4.} \textbf{Сравнение на \([X_1,\infty)\) и вывод о нулях.}\\
Если \(Q\ge \alpha^{2}\), то у \(z\) бесконечно много нулей; если \(Q\le 0\), то у \(z\) не более одного нуля.

\paragraph{Шаг 5.} \textbf{Перевести вывод к \(y\) и зафиксировать область.}\\
Так как \(\phi\neq 0\) на \([X_1,\infty)\), нули \(y\) совпадают с нулями \(z\).

\subsection*{3. Сопроводительные материалы (таблицы и обозначения)}

\textbf{Нормализация:}\quad
\(\displaystyle \phi=\exp\!\Bigl(-\frac12\int p\Bigr),\ 
Q=q-\frac{p'}{2}-\frac{p^{2}}{4},\ 
z=\dfrac{y}{\phi}.\)

\begin{center}
\begin{tabular}{|c|c|c|}
\hline
\textbf{Условие на } \(Q\) & \textbf{Эталон} & \textbf{Вывод о нулях} \\
\hline
\(Q(x)\ge \alpha^{2}>0\) & \(v''+\alpha^{2}v=0\) & беск. много нулей у \(z\) \\
\hline
\(Q(x)\le 0\) & \(v''=0\) & не более одного нуля у \(z\) \\
\hline
\end{tabular}
\end{center}

\subsection*{4. Применение алгоритма к объявленной задаче}

\paragraph{Шаг 1.} \textbf{Нормальная форма.}\\
\(x^{2}y''+xy'+(x^{2}-\tfrac12)y=0 \Rightarrow
y''+\tfrac{1}{x}y'+\bigl(1-\tfrac{1}{2x^{2}}\bigr)y=0\).
\[
\phi=x^{-1/2},\qquad
Q(x)=1-\frac{1}{4x^{2}}.
\]

\paragraph{Шаг 2.} \textbf{Детектор по \(Q\).}\\
Для \(x\ge 1\): \(Q(x)\ge \tfrac{3}{4}\), берём \(\alpha=\sqrt{3}/2\).

\paragraph{Шаг 3.} \textbf{Эталон.}\\
\(v''+\alpha^{2}v=0\).

\paragraph{Шаг 4.} \textbf{Сравнение и вывод.}\\
На \([1,\infty)\) выполнено \(Q\ge \alpha^{2}>0\) \(\Rightarrow\) у любого нетривиального решения \(z\) бесконечно много нулей.

\paragraph{Шаг 5.} \textbf{Возврат к \(y\).}\\
\(\phi=x^{-1/2}\neq 0\) при \(x>0\) \(\Rightarrow\) у \(y\) бесконечно много нулей на \(x>0\).
