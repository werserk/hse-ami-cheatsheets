\section{Линейные ОДУ 2-го порядка: нормальная форма, вронскиан, короткие доказательства}

\subsection*{1. Тип экзаменационной задачи (полное условие)}
\textbf{Стейтмент.}
Пусть функции \(p(x),q(x)\) непрерывны на \(\mathbb R\) и \(q(x)<0\) для всех \(x\).
Пусть \(y(x)\) — нетривиальное решение
\[
y''+p(x)y'(x)+q(x)y(x)=0.
\]
Покажите, что если решение принимает максимальное значение в некоторой точке, то это значение не может быть больше \(0\).

\subsection*{2. Универсальный алгоритм (визуальные формулы и детерминированные шаги)}

\textbf{Исходные данные и обозначения (ввод).} Дано линейное ОДУ 2-го порядка \(y''+p(x)y'+q(x)y=0\), где \(p,q\in C(\mathbb{R})\). Требуется доказать качественные свойства решений (экстремумы, нули, устойчивость).

Вводим: \(\phi(x)\) — интегрирующий множитель; \(Q(x)\) — эффективный потенциал; \(W(x)\) — вронскиан; \(z(x)\) — решение в нормальной форме; \(x_0\) — точка экстремума или нуля.

\paragraph{Шаг 0.} \textbf{Нормализация: увидеть \(p,q\).}\\
Привести уравнение к виду \(y''+p(x)y'+q(x)y=0\) и зафиксировать знаки \(p(x)\), \(q(x)\).

\paragraph{Шаг 1.} \textbf{Нормальная форма: убрать \(y'\) при необходимости.}\\
Взять
\[
\phi(x)=\exp\!\Bigl(-\tfrac12\!\int p(x)\,dx\Bigr),\quad y=\phi z,
\]
тогда
\[
z''+Q(x)z=0,\qquad Q(x)=q-\frac{p'}{2}-\frac{p^2}{4}.
\]

\paragraph{Шаг 2.} \textbf{Вронскиан: независимость/масштаб.}\\
Формула Абеля:
\[
W(x)=W(x_0)\,\exp\!\Bigl(-\!\int_{x_0}^{x}p(t)\,dt\Bigr).
\]

\paragraph{Шаг 3.} \textbf{Локальные/качественные выводы: «максимум/минимум/нули».}\\
\begin{itemize}
  \item \emph{Триггер «экстремум».} В точке максимума \(x_0\): \(y'(x_0)=0,\ y''(x_0)\le0\). Подставить в уравнение.
  \item \emph{Триггер «\(\le1\) нуля».} Перейти к \(z''+Qz=0\); при \(Q\le0\):
  \[
  \int_a^b zz''\,dx+\int_a^b Qz^2\,dx=0
  \Rightarrow -\!\int_a^b (z')^2\,dx+\!\int_a^b Qz^2\,dx=0,
  \]
  что невозможно при двух нулях.
\end{itemize}

\paragraph{Шаг 4.} \textbf{Итог: короткая формулировка.}\\
Выписать использованные \(\phi,Q\) и/или \(W\) и сформулировать вывод.

\subsection*{3. Сопроводительные материалы (таблицы и обозначения)}

\textbf{Детектор ветки Шага 3:}
\[
\begin{array}{|c|c|}
\hline
\textbf{Признак в условии} & \textbf{Действие} \\
\hline
\text{Есть «максимум/минимум», дан знак } q & \text{Экстремум-тест: } y'=0\text{, знак } y''\text{, подстановка в ОДУ} \\
\hline
\text{Требуется «не более одного нуля»} & \text{Шаг 1 } \Rightarrow z''+Qz=0\text{, при } Q\le0 \\
& \text{интегральный аргумент} \\
\hline
\text{Нужно проверить фундаментальность пары} & \text{Абель: } W(x)=W(x_0)e^{-\int p} \\
\hline
\end{array}
\]

\textbf{Памятка формул M4:}
\[
\phi(x)=\exp\!\Bigl(-\tfrac12\!\int p\Bigr),\qquad
Q=q-\tfrac12 p'-\tfrac14 p^2,\qquad
W(x)=W(x_0)\exp\!\Bigl(-\!\int_{x_0}^{x}p(t)\,dt\Bigr).
\]

\textbf{Правила экстремума:} В точке локального максимума \(x_0\): \(y'(x_0)=0\), \(y''(x_0)\le0\); в точке локального минимума: \(y'(x_0)=0\), \(y''(x_0)\ge0\).

\subsection*{4. Применение алгоритма к объявленной задаче}

\textbf{Дано:} \(y''+p(x)y'+q(x)y=0\), где \(q(x)<0\) для всех \(x\), и \(y(x)\) — нетривиальное решение с максимумом в точке \(x_0\).

\paragraph{Шаг 0.} \textbf{Нормализация: увидеть \(p,q\).}\\
Уравнение уже в виде \(y''+p\,y'+q\,y=0\) с \(q(x)<0\) для всех \(x\).

\paragraph{Шаг 1.} \textbf{Нормальная форма: убрать \(y'\) при необходимости.}\\
Переход к \(z\) не требуется для данного доказательства.

\paragraph{Шаг 2.} \textbf{Вронскиан: независимость/масштаб.}\\
Вронскиан не нужен для данного доказательства.

\paragraph{Шаг 3.} \textbf{Локальные/качественные выводы: «максимум/минимум/нули».}\\
В точке локального максимума \(x_0\):
\(y'(x_0)=0,\ y''(x_0)\le0\). Подставляя в уравнение:
\[
y''(x_0)=-p(x_0)\,y'(x_0)-q(x_0)\,y(x_0)= -q(x_0)\,y(x_0).
\]
При \(q(x_0)<0\) из \(y(x_0)>0\) следовало бы \(y''(x_0)>0\), что противоречит максимуму.
Значит \(y(x_0)\le0\).

\paragraph{Шаг 4.} \textbf{Итог: короткая формулировка.}\\
\[
\boxed{\,\text{Положительный локальный максимум невозможен при } q(x)<0\,}
\]

