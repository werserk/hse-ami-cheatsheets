\section{M9. Первые интегралы в 3D-системах ОДУ: поиск двух независимых и проверка}

\subsection*{1. Тип экзаменационной задачи (полное условие)}
Найдите два независимых первых интеграла и проверьте их независимость для системы
\[
\dot x = y z,\qquad
\dot y = z,\qquad
\dot z = -y,
\]
где $(x,y,z)\in\mathbb R^3$. Укажите область, где полученная пара интегралов независима.

\subsection*{2. Универсальный алгоритм (визуальные формулы и детерминированные шаги)}

\textbf{Исходные данные и обозначения (ввод).}
Пусть
\[
\dot x=f_1(x,y,z),\quad \dot y=f_2(x,y,z),\quad \dot z=f_3(x,y,z),
\]
$f_i\in C^1$. Первый интеграл $I$ удовлетворяет $\dfrac{d}{dt}I=\nabla I\cdot (f_1,f_2,f_3)=0$.

\paragraph{Шаг 1.} \textbf{Выделить «быструю» 2D-подсистему.}\\
Проверить пары $(y,z)$, $(x,y)$, $(x,z)$ на простой вид (вращение, масштаб, линейная/однородная структура) и выбрать удобную.

\paragraph{Шаг 2.} \textbf{Построить первый интеграл $I_1$ для выбранной пары.}\\
По детекторам (см. таблицу D): при вращении $(\alpha z,-\alpha y)$ взять $I_1=y^2+z^2$; при масштабе $(\alpha y,\beta z)$ взять $I_1=z/y^{\beta/\alpha}$; в линейном случае перейти к базису собственных векторов. Проверить $\dot I_1=0$.

\paragraph{Шаг 3.} \textbf{Построить второй интеграл $I_2$ через «линейно-квадратичный анзац».}\\
Если $f_1=f_1(y,z)$, попробовать $I_2=x-F(y,z)$ с $F=a y^2+b z^2+c\,yz$ и потребовать
\[
\dot I_2=\dot x-(F_y\dot y+F_z\dot z)\equiv 0
\quad\Longleftrightarrow\quad
(2a y+c z)f_2+(2b z+c y)f_3\equiv f_1.
\]
Решить линейную систему на $a,b,c$. При неудаче циклически переставить переменные.

\paragraph{Шаг 4.} \textbf{Проверить независимость пары $(I_1,I_2)$.}\\
Убедиться, что $\nabla I_1\times \nabla I_2\neq 0$ (или $dI_1\wedge dI_2\neq 0$) в интересующей области.

\paragraph{Шаг 5.} \textbf{Сформулировать результат.}\\
Записать $\boxed{I_1=C_1,\ I_2=C_2}$ и область валидности (где независимость не нарушается).

\subsection*{3. Сопроводительные материалы (таблицы и обозначения)}

\begin{center}
\begin{tabular}{|c|c|c|}
\hline
\textbf{Пара} & \textbf{Признак} & \textbf{Инвариант } $I_1$ \\
\hline
$(\dot y,\dot z)=(\alpha z,-\alpha y)$ & вращение & $y^2+z^2$ \\
\hline
$(\dot y,\dot z)=(\alpha y,\beta z)$ & масштаб & $z/y^{\beta/\alpha}$ \\
\hline
$(\dot y,\dot z)=(Ay+Bz,Cy+Dz)$ & линейная & $\eta/\xi^{\lambda_2/\lambda_1}$ в базисе левых СВ \\
\hline
\end{tabular}
\end{center}

\vspace{4pt}

\noindent\textbf{Q-анзац для $I_2$:} при $f_1=f_1(y,z)$ берем $I_2=x-F(y,z)$, $F=a y^2+b z^2+c\,yz$ и решаем
\[
(2a y+c z)\,f_2+(2b z+c y)\,f_3\equiv f_1.
\]
Если не решается — переставляем роли переменных и повторяем.

\subsection*{4. Применение алгоритма к объявленной задаче}

\[
\dot x = y z,\qquad \dot y = z,\qquad \dot z = -y.
\]

\paragraph{Шаг 1.} \textbf{Выделяем пару.}\\
$(\dot y,\dot z)=(z,-y)$ — вращение.

\paragraph{Шаг 2.} \textbf{Первый интеграл $I_1$.}\\
$I_1=y^2+z^2$, так как $\dot I_1=2y\dot y+2z\dot z=2yz+2z(-y)=0$.

\paragraph{Шаг 3.} \textbf{Второй интеграл $I_2$ через анзац.}\\
$f_1(y,z)=yz$. Ищем $I_2=x-F(y,z)$, $F=a y^2+b z^2+c\,yz$ из
\[
(2a y+c z)z+(2b z+c y)(-y)\equiv yz.
\]
Получаем $a=\tfrac12,\ b=0,\ c=0$, значит $F=\tfrac12 y^2$ и
\[
I_2=x-\tfrac12 y^2.
\]

\paragraph{Шаг 4.} \textbf{Независимость.}\\
$\nabla I_1=(0,2y,2z),\ \nabla I_2=(1,-y,0)$,
\[
\nabla I_1\times\nabla I_2=(0,-2z,-2y)\neq 0 \ \text{при}\ (y,z)\neq(0,0).
\]

\paragraph{Шаг 5.} \textbf{Результат.}\\
\[
\boxed{\,I_1=y^2+z^2=C_1,\qquad I_2=x-\tfrac12 y^2=C_2\,}
\]
Независимость выполняется на множестве $\{(y,z)\neq(0,0)\}$.

