\section{ПЧП 1-го порядка: задача Коши}

\subsection*{1. Тип экзаменационной задачи (полное условие)}
Даны две задачи Коши для уравнения
\[
y\,\frac{\partial z}{\partial x}-x\,\frac{\partial z}{\partial y}=0:
\]
а) \(z=2y\) при \(x=1\);\quad б) \(z=2y\) при \(x=1+y\).
В обеих задачах решение ищем в окрестности точки \((1,0)\).
Найдите решение этих задач, если это возможно. Проверьте условия теоремы существования и единственности (нехарактеристичность начальной линии).

\subsection*{2. Универсальный алгоритм (визуальные формулы и детерминированные шаги)}

\textbf{Исходные данные и обозначения (ввод).} Дано ПЧП 1-го порядка в виде \(a\,u_x+b\,u_y+c\,u_z=0\).
Вводим вектор коэффициентов: \(A=(a,b,c)\).
Вводим начальное многообразие: \(\Sigma: S=0\) (3D) или \(\Gamma: g=0\) (2D).
Вводим нормаль: \(n=\nabla S\) (3D) или \(n=(g_x,g_y)\) (2D).
Вводим инварианты: \(I_1,I_2\) — первоинтегралы системы характеристик.

\paragraph{Шаг 0.} \textbf{Нормализовать уравнение и проверить нехарактеристичность.}\\
Привести ПЧП к виду \(a\,u_x+b\,u_y+c\,u_z=0\), положить \(A=(a,b,c)\).
Проверить тест нехарактеристичности:
\begin{itemize}
\item 3D: \(A\cdot n\neq0\) на \(\Sigma\)
\item 2D: \(a\,g_x+b\,g_y\neq0\) на \(\Gamma\)
\end{itemize}

\paragraph{Шаг 1.} \textbf{Найти систему характеристик.}\\
\[
\dot x=a,\qquad \dot y=b,\qquad \dot z=c,\qquad \frac{d}{ds}u(x(s),y(s),z(s))=0
\]

\paragraph{Шаг 2.} \textbf{Найти инварианты (первоинтегралы).}\\
Из \(\tfrac{dx}{a}=\tfrac{dy}{b}=\tfrac{dz}{c}\) находим первоинтегралы \(I_1,I_2\) (или \(I_1\) в 2D).
Общий вид решения: \(u=F(I_1,I_2)\) (3D) или \(u=F(I_1)\) (2D).

\paragraph{Шаг 3.} \textbf{Определить функцию \(F\) по начальным данным.}\\
Ограничиваем инварианты на начальное многообразие:
\(F(I_1|_{\Sigma},I_2|_{\Sigma})=u|_{\Sigma}\) (3D) или \(F(I_1|_{\Gamma})=u|_{\Gamma}\) (2D).
Локальная обратимость отображения параметров в инварианты эквивалентна нехарактеристичности.

\paragraph{Шаг 4.} \textbf{Записать решение и указать область единственности.}\\
Записать \(u\) через найденный \(F\) и указать область единственности (где тест нехарактеристичности выполняется).

\subsection*{3. Сопроводительные материалы (таблицы и обозначения)}

\textbf{Тест нехарактеристичности:}
\begin{itemize}
\item 3D: \(A\cdot n\neq0\) на \(\Sigma\) \(\Rightarrow\) локальная единственность
\item 2D: \(a\,g_x+b\,g_y\neq0\) на \(\Gamma\) \(\Rightarrow\) локальная единственность
\end{itemize}

\textbf{Быстрые детекторы инвариантов (для \((x,y)\)):}
\begin{center}
\begin{tabular}{|c|c|}
\hline
\textbf{Тип поля} & \textbf{Инвариант} \\ \hline
Вращение \((a,b)=(y,-x)\) & \(I_1=x^2+y^2\) \\ \hline
Масштаб \((a,b)=(x,y)\) & \(I_1=y/x\) \\ \hline
Диагональ \((\alpha x,\beta y)\) & \(I_1=y/x^{\beta/\alpha}\) \\ \hline
Общий линейный случай & по собственным векторам \(M^\top\) или через \(v=y/x\) \\ \hline
\end{tabular}
\end{center}

\textbf{Система характеристик:}
\[
\frac{dx}{a}=\frac{dy}{b}=\frac{dz}{c}=ds
\]

\textbf{Общий вид решения:}
\begin{itemize}
\item 3D: \(u=F(I_1,I_2)\), где \(I_1,I_2\) — независимые инварианты
\item 2D: \(u=F(I_1)\), где \(I_1\) — инвариант
\end{itemize}

\subsection*{4. Применение алгоритма к объявленной задаче}

\[
y\,\frac{\partial z}{\partial x}-x\,\frac{\partial z}{\partial y}=0
\]

\paragraph{Шаг 0.} \textbf{Нормализовать уравнение и проверить нехарактеристичность.}\\
\(A=(y,-x)\) (поле вращения), значит \(I_1=x^2+y^2\) и \(z=F(I_1)\).

\paragraph{Шаг 1.} \textbf{Найти систему характеристик.}\\
\(\dot x=y\), \(\dot y=-x\), \(\dot z=0\) — окружности \(x^2+y^2=\text{const}\).

\paragraph{Шаг 2.} \textbf{Найти инварианты.}\\
\(I_1=x^2+y^2\) — единственный инвариант, \(z=F(x^2+y^2)\).

\paragraph{Шаг 3.} \textbf{Определить функцию \(F\) по начальным данным.}\\
\begin{itemize}
\item \textbf{(а) Данные \(z=2y\) при \(x=1\):}\\
Тест: \(g=x-1\Rightarrow a\,g_x+b\,g_y=y\). В \((1,0)\): \(=0\) (характеристично).\\
На \(x=1\): \(F(1+y^2)=2y\) — не функция одного аргумента около \(y=0\).\\
Итог: единственности нет; возможны ветви, например
\[
z(x,y)=\pm\,2\sqrt{x^2+y^2-1}\quad (x^2+y^2>1)
\]

\item \textbf{(б) Данные \(z=2y\) при \(x=1+y\):}\\
Тест: \(g=x-1-y\Rightarrow a\,g_x+b\,g_y=y+x\). В \((1,0)\): \(=1\neq0\) (нехарактеристично).\\
На \(x=1+y\): \(I_1=1+2y+2y^2\), поэтому \(F(1+2y+2y^2)=2y\).\\
Локально (около \(y=0\)) \(y\mapsto 1+2y+2y^2\) обратима, и
\[
z(x,y)=F(x^2+y^2)=-1+\tfrac12\sqrt{\,8(x^2+y^2)-4\,}
\]
(ветвь выбрана так, чтобы \(z(1,0)=0\) и на начальной кривой \(z=2y\)).
\end{itemize}

\paragraph{Шаг 4.} \textbf{Записать решение и указать область единственности.}\\
\begin{itemize}
\item \textbf{(а):} Решение не единственно в окрестности \((1,0)\)
\item \textbf{(б):} Единственность — в области, где начальная кривая пересекает каждую окружность \(x^2+y^2=\mathrm{const}\) единожды
\end{itemize}

