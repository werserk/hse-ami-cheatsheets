\section{Негиперболические равновесия: полярные координаты и инвариантные лучи}

\subsection*{1. Тип экзаменационной задачи (полное условие)}
Исследуйте фазовый портрет (определите тип траекторий) нелинейной системы
\[
\begin{cases}
\dot{x}=a\,y+x(x^{2}+y^{2}),\\[2pt]
\dot{y}=-a\,x+y(x^{2}+y^{2}),
\end{cases}
\]
в окрестности начала координат для всех значений параметра \(a\).
\textit{Указание.} Переход к полярным координатам.

\subsection*{2. Универсальный алгоритм (визуальные формулы и детерминированные шаги)}

\textbf{Исходные данные и обозначения (ввод).} Дана система \(\dot x=f(x,y),\ \dot y=g(x,y)\).
Вводим якобиан: \(J(0,0)=\begin{pmatrix}f_x&f_y\\ g_x&g_y\end{pmatrix}\) в точке \((0,0)\).
Вводим собственные значения: \(\lambda_{1,2}\) — корни характеристического уравнения \(\det(J-\lambda I)=0\).
Вводим полярные координаты: \(x=r\cos\theta,\ y=r\sin\theta\), где \(r>0\).
Вводим функции: \(\text{Rad}=x f+y g\) — радиальная компонента; \(\text{Tan}=x g-y f\) — тангенциальная компонента.

\paragraph{Шаг 0.} \textbf{Диагностировать негиперболичность.}\\
Вычислить \(J(0,0)\) и \(\lambda_{1,2}\). Если \(\Re\lambda=0\) или \(\det J=0\) — негиперболическое равновесие \(\Rightarrow\) переходить к полярным координатам.

\paragraph{Шаг 1.} \textbf{Проверить представимость в виде «вращение+радиал».}\\
Проверить, можно ли представить систему как:
\[
(f,g)=\Omega(r^{2})\,J\!\binom{x}{y}+R(r^{2})\!\binom{x}{y},\qquad
J=\begin{pmatrix}0&1\\-1&0\end{pmatrix},\ r^{2}=x^{2}+y^{2}
\]

\paragraph{Шаг 2.} \textbf{Перейти к полярным координатам.}\\
Использовать формулы:
\[
\dot r=\frac{x f+y g}{r}=\frac{\text{Rad}}{r},\qquad
\dot\theta=\frac{x g-y f}{r^{2}}=\frac{\text{Tan}}{r^{2}}
\]
Получить \(\dot r=R(r)\), \(\dot\theta=\Omega(r)\).

\paragraph{Шаг 3.} \textbf{Определить порядки малости у \(r=0\).}\\
Разложить: \(\dot r=\alpha r^{k}+o(r^{k})\ (k\ge 2)\), \(\dot\theta=\beta+\gamma r^{\ell}+o(r^{\ell})\).

\paragraph{Шаг 4.} \textbf{Классифицировать тип равновесия по знакам \(\alpha\) и \(\beta\).}\\
\begin{itemize}
\item \(\beta\ne 0,\ \alpha>0\) — неустойчивый спиральный источник
\item \(\beta\ne 0,\ \alpha<0\) — устойчивый спиральный сток
\item \(\beta=0,\ \alpha>0\) — радиальный источник (инвариантные лучи \(\theta=\mathrm{const}\))
\item \(\beta=0,\ \alpha<0\) — радиальный сток
\end{itemize}

\paragraph{Шаг 5.} \textbf{Построить эскиз фазового портрета.}\\
Указать знак вращения по \(\beta\) и монотонность \(r(t)\) по знаку \(\alpha\); отметить инвариантные кривые при \(\beta=0\).

\subsection*{3. Сопроводительные материалы (таблицы и обозначения)}

\textbf{Быстрый полярный тест:}
\[
\text{Rad}=x f+y g,\quad \text{Tan}=x g-y f
\]
Если \(\text{Rad}=r^{2}\cdot \tilde R(r^{2})\) и \(\text{Tan}=r^{2}\cdot \tilde \Omega(r^{2})\), то
\(\dot r=\tilde R(r)\), \(\dot\theta=\tilde \Omega(r)\).

\textbf{Таблица локальной классификации:}
\begin{center}
\begin{tabular}{|c|c|c|c|}
\hline
\textbf{Случай} & \(\dot r\) & \(\dot\theta\) & \textbf{Тип у } \(r=0\) \\ \hline
Спиральный источник & \(\alpha r^{k},\ \alpha>0\) & \(\beta\ne 0\) & неустойчивый спираль \\ \hline
Спиральный сток & \(\alpha r^{k},\ \alpha<0\) & \(\beta\ne 0\) & устойчивый спираль \\ \hline
Радиальный источник & \(\alpha r^{k},\ \alpha>0\) & \(\beta=0\) & лучи инвариантны, исход \\ \hline
Радиальный сток & \(\alpha r^{k},\ \alpha<0\) & \(\beta=0\) & лучи инвариантны, вход \\ \hline
\end{tabular}
\end{center}

\textbf{Формулы преобразования в полярные координаты:}
\[
\boxed{\ \dot r=\frac{x f+y g}{r}\ },\qquad
\boxed{\ \dot\theta=\frac{x g-y f}{r^{2}}\ }
\]

\subsection*{4. Применение алгоритма к объявленной задаче}

\[
\begin{cases}
\dot{x}=a\,y+x(x^{2}+y^{2}),\\[2pt]
\dot{y}=-a\,x+y(x^{2}+y^{2}),
\end{cases}
\]

\paragraph{Шаг 0.} \textbf{Диагностировать негиперболичность.}\\
\(J(0,0)=\begin{pmatrix}0&a\\ -a&0\end{pmatrix}\Rightarrow \lambda_{1,2}=\pm i a\) — негиперболическое равновесие.

\paragraph{Шаг 1.} \textbf{Проверить представимость в виде «вращение+радиал».}\\
\((f,g)=a\,J\binom{x}{y}+r^{2}\binom{x}{y}\) — система имеет вид «вращение+радиал».

\paragraph{Шаг 2.} \textbf{Перейти к полярным координатам.}\\
\[
\dot r=\frac{x(ay+xr^{2})+y(-ax+yr^{2})}{r}=\frac{r^{4}}{r}=r^{3}
\]
\[
\dot\theta=\frac{x(-ax+yr^{2})-y(ay+xr^{2})}{r^{2}}=\frac{-ar^{2}}{r^{2}}=-a
\]

\paragraph{Шаг 3.} \textbf{Определить порядки малости у \(r=0\).}\\
\(\alpha=1>0,\ k=3\); \(\beta=-a\).

\paragraph{Шаг 4.} \textbf{Классифицировать тип равновесия.}\\
\begin{itemize}
\item Если \(a\ne 0\): \(\beta=-a\ne 0,\ \alpha=1>0\) — неустойчивый спиральный источник (вращение со знаком \(-a\))
\item Если \(a=0\): \(\beta=0,\ \alpha=1>0\) — радиальный источник (\(\theta=\mathrm{const}\) инвариантно, \(r\) растёт)
\end{itemize}

\paragraph{Шаг 5.} \textbf{Построить эскиз фазового портрета.}\\
При \(a\ne 0\): из нуля выходят спирали, \(r(t)\) монотонно растёт. При \(a=0\): исход по лучам \(\theta=\mathrm{const}\).

\boxed{\text{При }a\neq0\text{: неустойчивый спиральный источник; при }a=0\text{: радиальный источник}}

