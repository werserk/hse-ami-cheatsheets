\section{Нелинейные 2D-системы: линеаризация, классификация по tr, det, D}

\subsection*{1. Тип экзаменационной задачи (полное условие)}
\textbf{Стейтмент.}
Найдите положения равновесия автономной системы, определите их характер и набросайте фазовые портреты в окрестности равновесий:
\[
\begin{cases}
\dot{x}=2-2\sqrt{\,1+x+y\,},\\[4pt]
\dot{y}=\exp\!\bigl(\tfrac{5}{4}x+2y+y^{2}\bigr)-1.
\end{cases}
\]

\subsection*{2. Универсальный алгоритм (визуальные формулы и детерминированные шаги)}

\textbf{Исходные данные и обозначения (ввод).} Дана автономная система \(\dot{x}=f(x,y)\), \(\dot{y}=g(x,y)\), где \(f,g\in C^1(\mathbb{R}^2)\). Требуется найти положения равновесия \((x_*,y_*)\) такие, что \(f(x_*,y_*)=0\), \(g(x_*,y_*)=0\), и классифицировать их характер по линеаризации.

Вводим: \(J\) — матрица Якоби; \(\operatorname{tr}J=f_x+g_y\) — след; \(\det J=f_x g_y-f_y g_x\) — определитель; \(D=\operatorname{tr}^2-4\det\) — дискриминант; \(\lambda_{1,2}\) — собственные значения \(J\).

\paragraph{Шаг 0.} \textbf{Найти положения равновесия.}\\
Решить систему \(f(x,y)=0\), \(g(x,y)=0\) и найти все точки \((x_*,y_*)\) такие, что \(f(x_*,y_*)=0\), \(g(x_*,y_*)=0\).

\paragraph{Шаг 1.} \textbf{Вычислить матрицу Якоби.}\\
Вычислить частные производные и составить
\[
J=\begin{pmatrix}
f_x & f_y\\[2pt]
g_x & g_y
\end{pmatrix}\Big|_{(x_*,y_*)}.
\]

\paragraph{Шаг 2.} \textbf{Вычислить инварианты в каждой точке равновесия.}\\
Посчитать
\[
\operatorname{tr}J=f_x+g_y,\qquad
\det J=f_xg_y-f_yg_x,\qquad
D=\operatorname{tr}^2-4\det,
\]
и применить таблицу классификации.

\paragraph{Шаг 3.} \textbf{Определить стабильность и направления.}\\
\begin{itemize}
  \item \(\det<0\): седло (неустойчиво).
  \item \(\det>0,\,D>0\): узел; знак \(\operatorname{tr}\) даёт устойчивость.
  \item \(\det>0,\,D<0\): фокус; знак \(\operatorname{tr}\) даёт устойчивость.
\end{itemize}

\paragraph{Шаг 4.} \textbf{Нарисовать локальный эскиз.}\\
Нанести тип точки и стрелки вход/выход; для седла — сепаратрисы по собственным векторам \(J\).

\textbf{Примечание.} Если \(\det J\neq 0\) (гиперболическая точка), линеаризация локально адекватна типу (Хартман–Гробман).

\subsection*{3. Сопроводительные материалы (таблицы и обозначения)}

\textbf{Классификация по \(\det,\ \operatorname{tr},\ D\):}
\[
\begin{array}{|c|c|c|}
\hline
\textbf{Условие} & \textbf{Тип точки} & \textbf{Устойчивость} \\
\hline
\det<0 & \text{седло} & \text{неустойчивая} \\
\hline
\det>0,\ D>0,\ \operatorname{tr}<0 & \text{узел} & \text{устойчивый} \\
\hline
\det>0,\ D>0,\ \operatorname{tr}>0 & \text{узел} & \text{неустойчивый} \\
\hline
\det>0,\ D<0,\ \operatorname{tr}<0 & \text{фокус} & \text{устойчивый} \\
\hline
\det>0,\ D<0,\ \operatorname{tr}>0 & \text{фокус} & \text{неустойчивый} \\
\hline
\end{array}
\]

\textbf{Детектор гиперболичности:} \(\det J\neq 0\quad\Rightarrow\quad\text{линеаризация достаточна для локального типа.}\)

\textbf{Правила границ:} Границы \(\det=0\) или \(D=0\) — вне рамок M7 (негиперболика).

\subsection*{4. Применение алгоритма к объявленной задаче}

\textbf{Дано:} \(\dot{x}=2-2\sqrt{1+x+y}\), \(\dot{y}=\exp\!\bigl(\tfrac{5}{4}x+2y+y^{2}\bigr)-1\).

\paragraph{Шаг 0.} \textbf{Найти положения равновесия.}\\
\(f=0 \Rightarrow x+y=0\).
\(g=0 \Rightarrow \tfrac{5}{4}x+2y+y^2=0\).
Совместно: точки \((0,0)\) и \((\tfrac{3}{4},-\tfrac{3}{4})\).

\paragraph{Шаг 1.} \textbf{Вычислить матрицу Якоби.}\\
\[
f_x=f_y=-\frac{1}{\sqrt{1+x+y}},\quad
g_x=\tfrac{5}{4}e^{\Phi},\quad
g_y=(2+2y)e^{\Phi},\ \ \Phi=\tfrac{5}{4}x+2y+y^2.
\]
В равновесиях \(\sqrt{1+x+y}=1,\ e^{\Phi}=1\).

\paragraph{Шаг 2.} \textbf{Вычислить инварианты в каждой точке равновесия.}\\
Для \((0,0)\): \(J=\begin{pmatrix}-1&-1\\[2pt]\tfrac{5}{4}&2\end{pmatrix}\), \(\operatorname{tr}=1\), \(\det=-\tfrac{3}{4}<0\).\\
Для \((\tfrac{3}{4},-\tfrac{3}{4})\): \(J=\begin{pmatrix}-1&-1\\[2pt]\tfrac{5}{4}&\tfrac{1}{2}\end{pmatrix}\), \(\operatorname{tr}=-\tfrac{1}{2}\), \(\det=\tfrac{3}{4}>0\), \(D=\tfrac{1}{4}-3=-\tfrac{11}{4}<0\).

\paragraph{Шаг 3.} \textbf{Определить стабильность и направления.}\\
Для \((0,0)\): \(\det<0\) \(\Rightarrow\) \textbf{седло (неустойчивая)}.\\
Для \((\tfrac{3}{4},-\tfrac{3}{4})\): \(\det>0,\ D<0,\ \operatorname{tr}<0\) \(\Rightarrow\) \textbf{устойчивый фокус}.

\paragraph{Шаг 4.} \textbf{Нарисовать локальный эскиз.}\\
Седло в \((0,0)\): одна устойчивая и одна неустойчивая сепаратриса.
Фокус в \((\tfrac{3}{4},-\tfrac{3}{4})\): затухающие спирали.

\[
\boxed{\,\text{Две точки равновесия: седло }(0,0)\text{ и устойчивый фокус }(\tfrac{3}{4},-\tfrac{3}{4})\,}
\]
