\section{Нелинейные 2D-системы: равновесия, линеаризация, классификация}

\subsection*{1. Тип экзаменационной задачи (полное условие)}
\textbf{Условие.} Найдите положения равновесия автономной системы уравнений, определите их характер, и нарисуйте фазовые портреты в окрестности положений равновесия
\[
\begin{cases}
\dot{x}=2-2\sqrt{1+x+y},\\[4pt]
\dot{y}=\exp\!\Bigl(\tfrac{5}{4}x+2y+y^{2}\Bigr)-1.
\end{cases}
\]

\subsection*{2. Универсальный алгоритм (визуальные формулы и детерминированные шаги)}

\textbf{Исходные данные и обозначения (ввод).} Дана автономная система \(\dot{x}=f(x,y)\), \(\dot{y}=g(x,y)\), где \(f,g\in C^1(\mathbb{R}^2)\). Требуется найти положения равновесия \((x_0,y_0)\) такие, что \(f(x_0,y_0)=0\), \(g(x_0,y_0)=0\), и классифицировать их характер.

Вводим: \(J(x,y)\) — матрица Якоби; \(\operatorname{tr}J=f_x+g_y\) — след; \(\det J=f_x g_y-f_y g_x\) — определитель; \(D=\operatorname{tr}^2-4\det\) — дискриминант; \(\lambda_{1,2}\) — собственные значения \(J\).

\paragraph{Шаг 0.} \textbf{Найти положения равновесия.}\\
Решить систему \(f(x,y)=0\), \(g(x,y)=0\) и найти все точки \((x_0,y_0)\) такие, что \(f(x_0,y_0)=0\), \(g(x_0,y_0)=0\).

\paragraph{Шаг 1.} \textbf{Составить матрицу Якоби.}\\
Вычислить частные производные и составить
\[
J(x,y)=\begin{pmatrix} f_x & f_y\\[2pt] g_x & g_y \end{pmatrix}.
\]

\paragraph{Шаг 2.} \textbf{Вычислить инварианты в каждой точке равновесия.}\\
Для каждой точки \((x_0,y_0)\) вычислить:
\[
\operatorname{tr}J(x_0,y_0),\quad \det J(x_0,y_0),\quad D=\operatorname{tr}^2-4\det.
\]

\paragraph{Шаг 3.} \textbf{Классифицировать тип точки по детектору.}\\
Применить правила из таблицы классификации по знакам \(\det\), \(D\), \(\operatorname{tr}\).

\paragraph{Шаг 4.} \textbf{Определить устойчивость и направления.}\\
По знаку \(\operatorname{tr}\) и типу точки зафиксировать вход/выход; для седла отметить две сепаратрисы вдоль собственных направлений \(J\).

\paragraph{Шаг 5.} \textbf{Нарисовать фазовый портрет.}\\
Нанести типы точек и стрелки; при необходимости использовать изоклины \(f=0\), \(g=0\) для знаков \(\dot x\), \(\dot y\).

\subsection*{3. Сопроводительные материалы (таблицы и обозначения)}

\textbf{Таблица классификации равновесий:}
\[
\begin{array}{|c|c|c|}
\hline
\textbf{Условие} & \textbf{Тип точки} & \textbf{Устойчивость} \\
\hline
\det<0 & \text{седло} & \text{неустойчивая} \\
\hline
\det>0,\ D>0,\ \operatorname{tr}<0 & \text{узел} & \text{устойчивый} \\
\hline
\det>0,\ D>0,\ \operatorname{tr}>0 & \text{узел} & \text{неустойчивый} \\
\hline
\det>0,\ D<0,\ \operatorname{tr}<0 & \text{фокус} & \text{устойчивый} \\
\hline
\det>0,\ D<0,\ \operatorname{tr}>0 & \text{фокус} & \text{неустойчивый} \\
\hline
\det>0,\ D=0 \text{ или } \det=0 & \text{негиперболика} & \text{см. главу M10} \\
\hline
\end{array}
\]

\textbf{Быстрые производные (частые атомы):}
\[
\begin{aligned}
&f(x,y)=A-B\sqrt{\,\Phi(x,y)\,}\!:\quad
f_x=-\frac{B}{2}\Phi^{-1/2}\Phi_x,\ \ f_y=-\frac{B}{2}\Phi^{-1/2}\Phi_y;\\
&g(x,y)=e^{\Psi(x,y)}-1:\quad g_x=e^{\Psi}\Psi_x,\ \ g_y=e^{\Psi}\Psi_y.
\end{aligned}
\]

\textbf{Правила упрощения:} Если в равновесии \(g=0\), то \(e^{\Psi}=1\) и \(g_x=\Psi_x\), \(g_y=\Psi_y\); если \(1+x+y=1\), то \(\sqrt{1+x+y}=1\) и \(f_x=f_y=-1\).

\subsection*{4. Применение алгоритма к объявленной задаче}

\textbf{Дано:} \(\dot{x}=2-2\sqrt{1+x+y}\), \(\dot{y}=\exp\!\Bigl(\tfrac{5}{4}x+2y+y^{2}\Bigr)-1\).

\paragraph{Шаг 0.} \textbf{Найти положения равновесия.}\\
\(f=0\ \Rightarrow\ \sqrt{1+x+y}=1\ \Rightarrow\ x+y=0\). \quad
\(g=0\ \Rightarrow\ \tfrac{5}{4}x+2y+y^2=0\). Подставляя \(y=-x\):
\[
x^2-\tfrac{3}{4}x=0\ \Rightarrow\ x\in\{0,\tfrac{3}{4}\}.
\]
Точки равновесия: \((0,0)\) и \((\tfrac{3}{4},-\tfrac{3}{4})\).

\paragraph{Шаг 1.} \textbf{Составить матрицу Якоби.}\\
При \(x+y=0\) и \(\Psi=0\) имеем
\[
J(x,y)=
\begin{pmatrix}
-1 & -1\\[2pt]
\tfrac{5}{4} & 2+2y
\end{pmatrix}.
\]
Значит \(J(0,0)=\begin{pmatrix}-1&-1\\[2pt]\tfrac{5}{4}&2\end{pmatrix}\), \(J(\tfrac{3}{4},-\tfrac{3}{4})=\begin{pmatrix}-1&-1\\[2pt]\tfrac{5}{4}&\tfrac{1}{2}\end{pmatrix}\).

\paragraph{Шаг 2.} \textbf{Вычислить инварианты в каждой точке равновесия.}\\
\[
\begin{aligned}
&(0,0):\ \operatorname{tr}=1,\ \det=-\tfrac{3}{4}<0;\\
&(\tfrac{3}{4},-\tfrac{3}{4}):\ \operatorname{tr}=-\tfrac{1}{2},\ \det=\tfrac{3}{4}>0,\ 
D=\tfrac{1}{4}-3=-\tfrac{11}{4}<0.
\end{aligned}
\]

\paragraph{Шаг 3.} \textbf{Классифицировать тип точки по детектору.}\\
\[
\begin{aligned}
&(0,0):\ \det<0\ \Rightarrow\ \text{седло (неустойчивая)};\\
&(\tfrac{3}{4},-\tfrac{3}{4}):\ \det>0,\ D<0,\ \operatorname{tr}<0\ \Rightarrow\ \text{фокус устойчивый}.
\end{aligned}
\]

\paragraph{Шаг 4.} \textbf{Определить устойчивость и направления.}\\
В \((0,0)\) — две сепаратрисы по собственным направлениям \(J\); в \((\tfrac{3}{4},-\tfrac{3}{4})\) — спиральное вхождение.

\paragraph{Шаг 5.} \textbf{Нарисовать фазовый портрет.}\\
Эскиз: седло в \((0,0)\) с «крестом» сепаратрис; устойчивый фокус в \((\tfrac{3}{4},-\tfrac{3}{4})\) со стрелками внутрь. Изоклина \(x+y=0\) помогает ориентировать знаки \(\dot x\).

\[
\boxed{\,\text{Две точки равновесия: седло }(0,0)\text{ и устойчивый фокус }(\tfrac{3}{4},-\tfrac{3}{4})\,}
\]

