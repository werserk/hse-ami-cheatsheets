\section{Линейные ОДУ 2-го порядка: вронскиан, Абель, детектор линейности}

\subsection*{1. Тип экзаменационной задачи (полное условие)}
Дано уравнение
\[
(x+2y)\,y''-3\,y'+y\sqrt{1-x}=0.
\]
Пусть \(y_{1}(x)\) — решение, удовлетворяющее \(y_{1}(0)=0\), \(y_{1}'(0)=1\); \(y_{2}(x)\) — другое решение, удовлетворяющее \(y_{2}(0)=3\), \(y_{2}'(0)=2\). Составляют ли эти решения фундаментальную систему? Обоснуйте. Найдите \(W[y_1,y_2](-1)\).

\subsection*{2. Универсальный алгоритм (визуальные формулы и детерминированные шаги)}

\textbf{Исходные данные и обозначения (ввод).} Дано уравнение \(F(x,y,y',y'')=0\).
Вводим линейный канон: \(a_2(x)y''+a_1(x)y'+a_0(x)y=0\), где \(a_2\neq0\).
Вводим коэффициенты: \(p:=\frac{a_1}{a_2}\), \(q:=\frac{a_0}{a_2}\).
Вводим вронскиан: \(W[y_1,y_2]=\det\begin{pmatrix}y_1 & y_2\\ y_1' & y_2'\end{pmatrix}=y_1y_2'-y_2y_1'\).

\paragraph{Шаг 0.} \textbf{Проверить линейность уравнения.}\\
Проверить: \(\exists\,a_0,a_1,a_2\) (зависят только от \(x\)), что \(F\equiv a_2y''+a_1y'+a_0y\).
Эквивалентные условия:
\[
F_{y''y''}\equiv0,\quad \partial_{y}F_{y''}\equiv0,\ \partial_{y'}F_{y''}\equiv0,\quad
F-y''F_{y''}\ \text{линеен по }(y',y)\ \text{с коэффициентами от }x.
\]

\paragraph{Шаг 1.} \textbf{Привести к каноническому виду (если линейно).}\\
Записать \(y''+p(x)y'+q(x)y=0\), где \(p=\frac{a_1}{a_2}\), \(q=\frac{a_0}{a_2}\).

\paragraph{Шаг 2.} \textbf{Применить формулу Абеля для вронскиана (только линейный случай).}\\
Использовать:
\[
\boxed{\,W'=-pW\,},\qquad \boxed{\,W(x)=W(x_0)\exp\!\big(-\int_{x_0}^{x}p\big)\,}
\]

\paragraph{Шаг 3.} \textbf{Проверить фундаментальность системы решений (только линейный случай).}\\
Если \(y_1,y_2\) — решения линейного уравнения, то:
\[
\boxed{\,W(x_0)\neq0\ \Leftrightarrow\ \text{решения независимы на интервале}\,}
\]

\paragraph{Шаг 4.} \textbf{Обработать нелинейный случай.}\\
Если уравнение нелинейное, то:
\begin{itemize}
\item Понятие фундаментальной системы неприменимо
\item Формула Абеля неприменима
\item Вычислимо лишь \(W(x_0)=y_1(x_0)y_2'(x_0)-y_2(x_0)y_1'(x_0)\) локально
\end{itemize}

\paragraph{Шаг 5.} \textbf{Вычислить вронскиан в заданной точке.}\\
Использовать начальные данные: если \(y_1(x_0)=\alpha_1\), \(y_1'(x_0)=\beta_1\), \(y_2(x_0)=\alpha_2\), \(y_2'(x_0)=\beta_2\), то
\[
W(x_0)=\alpha_1\beta_2-\alpha_2\beta_1
\]

\subsection*{3. Сопроводительные материалы (таблицы и обозначения)}

\textbf{Детектор линейности/нелинейности:}
\begin{center}
\begin{tabular}{|c|c|}
\hline
\textbf{Признак} & \textbf{Вывод} \\ \hline
\(F_{y''y''}\not\equiv0\) & нелинейное \\ \hline
\(\partial_{y}F_{y''}\ne0\) или \(\partial_{y'}F_{y''}\ne0\) & нелинейное \\ \hline
\(\partial_{(y')^2}\big(F-y''F_{y''}\big)\ne0\) или \(\partial_{y^2}\big(F-y''F_{y''}\big)\ne0\) & нелинейное \\ \hline
или \(\partial_{y,y'}\big(F-y''F_{y''}\big)\ne0\) & нелинейное \\ \hline
\end{tabular}
\end{center}

\textbf{Формула Абеля (линейный канон):}
\[
W'=-pW,\qquad W(x)=W(x_0)\exp\!\Bigl(-\!\int_{x_0}^{x}p\Bigr),\qquad p=\frac{a_1}{a_2}
\]

\textbf{Вычисление вронскиана по начальным данным:}
Если \(y_1(x_0)=\alpha_1\), \(y_1'(x_0)=\beta_1\), \(y_2(x_0)=\alpha_2\), \(y_2'(x_0)=\beta_2\), то
\[
W(x_0)=\alpha_1\beta_2-\alpha_2\beta_1
\]

\subsection*{4. Применение алгоритма к объявленной задаче}

\[
(x+2y)\,y''-3\,y'+y\sqrt{1-x}=0
\]

\paragraph{Шаг 0.} \textbf{Проверить линейность уравнения.}\\
\(F=(x+2y)y''-3y'+y\sqrt{1-x}\), \(F_{y''}=x+2y\), \(\partial_yF_{y''}=2\ne0\).
\[
\boxed{\ \text{нелинейное}\ }
\]

\paragraph{Шаг 1.} \textbf{Привести к каноническому виду.}\\
Невозможно получить \(a_2(x)y''+a_1(x)y'+a_0(x)y=0\).

\paragraph{Шаг 2.} \textbf{Применить формулу Абеля.}\\
\(\boxed{\ \text{неприменимо для нелинейного уравнения}\ }\)

\paragraph{Шаг 3.} \textbf{Проверить фундаментальность системы решений.}\\
\(\boxed{\ \text{некорректно для нелинейного уравнения}\ }\)

\paragraph{Шаг 4.} \textbf{Обработать нелинейный случай.}\\
Понятие фундаментальной системы неприменимо; формула Абеля неприменима.

\paragraph{Шаг 5.} \textbf{Вычислить вронскиан в заданной точке.}\\
При \(x=0\): \(W(0)=\det\begin{pmatrix}0&3\\1&2\end{pmatrix}=-3\). 
Требуемое \(W(-1)\) без линейности не выводится из данных: \(\boxed{\,W(-1)\ \text{не определяется без явного решения}\,}\).

