\subsection{Минимальное ЛОУ: метод аннигиляторов}\label{subsec:min-lre}

\begin{center}
\fbox{\parbox{0.92\textwidth}{
\textbf{TL;DR:} минимальное ЛОУ (минимальная однородная линейная рекуррент с постоянными коэффициентами), которое имеет данные последовательности в качестве решений, строится так:
\begin{enumerate}
\item к каждой заданной последовательности приписать аннигилятор (многочлен от $E$);
\item взять НОК этих аннигиляторов как многочлен $L(\lambda)$;
\item развернуть $L(E)\,y=0$ в явную рекурренту. Степень $L$ — минимальный порядок.
\end{enumerate}
}}
\end{center}

\subsubsection*{Методика (детерминированно)}

Пусть даны частные решения $y^{(1)},\ldots,y^{(m)}$.

\paragraph{Шаг 1. Атом $\to$ аннигилятор}
Для каждой последовательности выпиши минимальный аннигилирующий многочлен:
\begin{table}[h!]
\centering
\caption{Атом $\to$ аннигилятор}
\label{tab:atom-to-annihilator}
\begin{tabular}{|l|l|}
\hline
\textbf{Атом (последовательность)} & \textbf{Минимальный аннигилятор $L(\lambda)$} \\
\hline
$r^t$ & $(\lambda - r)$ \\
\hline
$t^k r^t$ & $(\lambda - r)^{k+1}$ \\
\hline
$\rho^t \cos(\omega t)$, $\rho^t \sin(\omega t)$ & $Q_{\rho,\omega}(\lambda)=\lambda^2-2\rho\cos\omega\,\lambda+\rho^2$ \\
\hline
$t^k \, \rho^t \, \cos/\sin(\omega t)$ & $Q_{\rho,\omega}(\lambda)^{k+1}$ \\
\hline
$t^k$ & $(\lambda-1)^{k+1}$ \\
\hline
$(-1)^t$ & $(\lambda+1)$ \\
\hline
\end{tabular}
\end{table}

\paragraph{Шаг 2. Собрать общий аннигилятор}
Возьмём НОК (наименьший общий кратный) всех многочленов из шага 1:
\[
 L(\lambda)=\operatorname{lcm}\big(L_1(\lambda),\ldots,L_m(\lambda)\big).
\]
При одинаковых базах/частотах выбирается максимальная кратность (а не сумма).

\paragraph{Шаг 3. Развернуть в рекуррент}
Если $L(\lambda)=\lambda^k+c_1\lambda^{k-1}+\cdots+c_k$, то искомое уравнение:
\[
 \boxed{\,y_{t+k}+c_1 y_{t+k-1}+\cdots+c_k y_t=0\,}.
\]

\paragraph{Минимальность.}
Любой многочлен $P(E)$, который зануляет все данные последовательности, обязан делиться на $L(E)$. Поэтому $\deg L$ — минимально возможный порядок.

\bigskip

\begin{center}
\fbox{\parbox{0.92\textwidth}{
\subsubsection*{Простой пример}

Дано:
\[
y_t^{(1)}=3^t,\qquad y_t^{(2)}=(-2)^t.
\]

\paragraph{Шаг 1.} Аннигиляторы: $(\lambda-3)$ и $(\lambda+2)$.

\paragraph{Шаг 2.} НОК:
\[
(\lambda-3)(\lambda+2)=\lambda^2-\lambda-6.
\]

\paragraph{Шаг 3.} Рекуррентное соотношение (развёртка):
\[
\boxed{\,y_{t+2}-y_{t+1}-6y_t=0\,}.
\]

Проверка: последовательности $3^t$ и $(-2)^t$ действительно являются решениями; порядок $2$ минимален.
}}
\end{center}

\bigskip

\begin{center}
\fbox{\parbox{0.92\textwidth}{
\subsubsection*{Пример посложнее}

Дано:
\[
y^{(1)}_t=2^t,\qquad y^{(2)}_t=t2^t,\qquad y^{(3)}_t=(-1)^t,\qquad y^{(4)}_t=3^t\cos\frac{\pi t}{3}.
\]

\subsubsection*{Шаг 1. Аннигиляторы}
\begin{itemize}
  \item Для $2^t$: $(\lambda-2)$.
  \item Для $t2^t$: $(\lambda-2)^{2}$ (кратность на 1 больше из-за множителя $t$).
  \item Для $(-1)^t$: $(\lambda+1)$.
  \item Для $3^t\cos\frac{\pi t}{3}$:
  \[
  Q_{3,\pi/3}(\lambda)=\lambda^2-2\cdot 3\cos\frac{\pi}{3}\,\lambda+3^2
  =\lambda^2-3\lambda+9.
  \]
\end{itemize}

\subsubsection*{Шаг 2. НОК}
Учитываем максимальную кратность по базе $2$, значит
\[
L(\lambda)=(\lambda-2)^2(\lambda+1)(\lambda^2-3\lambda+9).
\]

\subsubsection*{Шаг 3. Развёртка}
Сначала
\[
(\lambda-2)^2(\lambda+1)=(\lambda^2-4\lambda+4)(\lambda+1)=\lambda^3-3\lambda^2+4.
\]
Умножаем на $\lambda^2-3\lambda+9$:
\[
L(\lambda)=(\lambda^3-3\lambda^2+4)(\lambda^2-3\lambda+9)
=\lambda^5-6\lambda^4+18\lambda^3-23\lambda^2-12\lambda+36.
\]

Соответствующая рекуррентная формула (коэффициенты берём по степеням $\lambda$) будет
\[
\boxed{\,y_{t+5}-6y_{t+4}+18y_{t+3}-23y_{t+2}-12y_{t+1}+36y_t=0\,.}
\]

\paragraph{Комментарий.} Это и есть минимальное ЛОУ, annihilator которого равен $L(\lambda)$.
}}
\end{center}
