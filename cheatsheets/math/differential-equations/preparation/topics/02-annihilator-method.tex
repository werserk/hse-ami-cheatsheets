\subsection{Минимальная ЛОРУ: метод аннигиляторов}\label{sec:min-lre}

\HSETldr{Минимальная ЛОРУ (линейное однородное разностное уравнение с постоянными коэффициентами), для которой данные последовательности являются решениями, строится так: \begin{enumerate} \item к каждой заданной последовательности приписать аннигилятор (многочлен от $E$); \item взять НОК этих аннигиляторов как многочлен $L(\lambda)$; \item развернуть $L(E)\,y=0$ в явную рекурренту. Степень $L$ — минимальный порядок. \end{enumerate}}

\subsubsection*{Методика}

Пусть даны частные решения $y^{(1)},\ldots,y^{(m)}$.

\textbf{Атом $\to$ аннигилятор.} Для каждой последовательности выпишите минимальный аннигилирующий многочлен:
\begin{table}[h!]
\centering
\caption{Атом $\to$ аннигилятор}
\label{tab:atom-to-annihilator}
\begin{tabular}{|l|l|}
\hline
\textbf{Атом (последовательность)} & \textbf{Минимальный аннигилятор $L(\lambda)$} \\
\hline
$r^t$ & $(\lambda - r)$ \\
\hline
$t^k r^t$ & $(\lambda - r)^{k+1}$ \\
\hline
$\rho^t \cos(\omega t)$, $\rho^t \sin(\omega t)$ & $Q_{\rho,\omega}(\lambda)=\lambda^2-2\rho\cos\omega\,\lambda+\rho^2$ \\
\hline
$t^k \, \rho^t \, \cos/\sin(\omega t)$ & $Q_{\rho,\omega}(\lambda)^{k+1}$ \\
\hline
$t^k$ & $(\lambda-1)^{k+1}$ \\
\hline
$(-1)^t$ & $(\lambda+1)$ \\
\hline
\end{tabular}
\end{table}

\textbf{Шаг 2. Собрать общий аннигилятор.} Возьмём НОК (наименьший общий кратный) всех многочленов из шага 1:
\[
 L(\lambda)=\operatorname{lcm}\big(L_1(\lambda),\ldots,L_m(\lambda)\big).
\]
При одинаковых базах/частотах выбирается максимальная кратность (а не сумма).

\textbf{Шаг 3. Развернуть в рекуррент.} Если $L(\lambda)=\lambda^k+c_1\lambda^{k-1}+\cdots+c_k$, то искомое уравнение:
\[
 \boxed{\,y_{t+k}+c_1 y_{t+k-1}+\cdots+c_k y_t=0\,}.
\]

\textbf{Минимальность.} Любой многочлен $P(E)$, который зануляет все данные последовательности, обязан делиться на $L(E)$. Поэтому $\deg L$ — минимально возможный порядок.

\bigskip

\textbf{Простой пример.} Дано: $y_t^{(1)}=3^t,\; y_t^{(2)}=(-2)^t$. \\
\begin{HSESteps}
  \item Аннигиляторы: $(\lambda-3)$ и $(\lambda+2)$.
  \item НОК: $(\lambda-3)(\lambda+2)=\lambda^2-\lambda-6$.
  \item Развёртка: $\boxed{y_{t+2}-y_{t+1}-6y_t=0}$.
\end{HSESteps}
Проверка: обе последовательности являются решениями; порядок $2$ минимален.

\bigskip

\textbf{Пример посложнее.} Дано: $y^{(1)}_t=2^t,\; y^{(2)}_t=t2^t,\; y^{(3)}_t=(-1)^t,\; y^{(4)}_t=3^t\cos\tfrac{\pi t}{3}$.

\textbf{Шаг 1. Аннигиляторы:} Для $2^t$: $(\lambda-2)$. Для $t2^t$: $(\lambda-2)^2$. Для $(-1)^t$: $(\lambda+1)$. Для $3^t\cos\tfrac{\pi t}{3}$: $Q_{3,\pi/3}(\lambda)=\lambda^2-3\lambda+9$.

\textbf{Шаг 2. НОК:} Учитываем максимальную кратность по базе $2$: $L(\lambda)=(\lambda-2)^2(\lambda+1)(\lambda^2-3\lambda+9)$.

\textbf{Шаг 3. Развёртка:} Сначала $(\lambda-2)^2(\lambda+1)=(\lambda^2-4\lambda+4)(\lambda+1)=\lambda^3-3\lambda^2+4$. Затем умножаем на $\lambda^2-3\lambda+9$ и получаем $L(\lambda)=\lambda^5-6\lambda^4+18\lambda^3-23\lambda^2-12\lambda+36$. Отсюда рекуррентное соотношение: $\boxed{y_{t+5}-6y_{t+4}+18y_{t+3}-23y_{t+2}-12y_{t+1}+36y_t=0}$.

\textit{Комментарий:} это и есть минимальная ЛОРУ, аннигилятор которой равен $L(\lambda)$.
