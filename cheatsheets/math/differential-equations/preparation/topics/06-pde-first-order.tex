\section{ПЧП первого порядка. Инварианты характеристик}\label{sec:pde-first-order}

\HSEDefinition{\Term{Линейное ПЧП первого порядка} в $\mathbb{R}^3$ — уравнение вида $a(x,y,z)\,u_x + b(x,y,z)\,u_y + c(x,y,z)\,u_z = 0$, где $a,b,c$ — непрерывные функции. Метод характеристик приводит к ОДУ-системе $\dot x = a(x,y,z),\ \dot y = b(x,y,z),\ \dot z = c(x,y,z)$.}

% Подключаем все подглавы по ПЧП первого порядка
\subsection{Вводная информация и единый алгоритм}\label{sec:pde-intro-algorithm}

Рассматриваем линейное ПЧП первого порядка в $\mathbb{R}^3$:
\[
a(x,y,z)\,u_x + b(x,y,z)\,u_y + c(x,y,z)\,u_z \;=\; 0,
\]
где $a,b,c$ — непрерывны. Метод характеристик приводит к ОДУ-системе
\[
\dot x = a(x,y,z),\quad \dot y = b(x,y,z),\quad \dot z = c(x,y,z).
\]

\HSEDefinition{\Term{Инвариант (первый интеграл)} $I(x,y,z)$ — это $C^1$-функция, постоянная вдоль характеристик, т.е.
\[
\frac{d}{ds}I(x(s),y(s),z(s))= \nabla I \cdot (a,b,c) = a I_x + b I_y + c I_z = 0.
\]
Общее решение ПЧП имеет вид $u=F(I_1,I_2)$, где $I_1,I_2$ — два независимых инварианта.}

\subsubsection{Единый алгоритм (5 шагов)}

\begin{enumerate}
\item \textbf{Цель.} Распознать форму и записать \emph{характеристики}.

    \textbf{Действие.} Выписать $\dot x=a,\ \dot y=b,\ \dot z=c$ и безпараметрические равенства
    $\displaystyle \frac{dx}{a}=\frac{dy}{b}=\frac{dz}{c}$.

\item \textbf{Цель.} Найти \emph{первый инвариант $I_1$}.

    \textbf{Действие.} Решить одну пару, напр. $\displaystyle \frac{dy}{dx}=\frac{b}{a}$, получить семейство кривых уровня
    $\Phi_1(x,y)=\text{const}$, положить $I_1=\Phi_1$ (или любой эквивалент).

\item \textbf{Цель.} Найти \emph{второй инвариант $I_2$}.

    \textbf{Действие.} Зафиксировать $I_1=\text{const}$ (т.е. связь $y=\Psi(x;I_1)$) и решить ОДУ по $z$ из
    $\displaystyle \frac{dz}{dx}=\frac{c(x,\Psi(x;I_1),z)}{a(x,\Psi(x;I_1))}$.
    \begin{itemize}
    \item Если $c\equiv 0$, взять $I_2=z$.
    \item Если $c=\alpha(x,y)\,z+\beta(x,y)$, получаем линейное ОДУ $z'=\tilde{\alpha}(x;I_1)z+\tilde{\beta}(x;I_1)$.
    \emph{Интегрирующий множитель} $M(x;I_1)=\exp\!\big(-\int \tilde{\alpha}\,dx\big)$, и
    \[
    \boxed{\,I_2 = z\,M(x;I_1) - \int \tilde{\beta}(x;I_1)\,M(x;I_1)\,dx\,}
    \]
    — константа вдоль характеристики.
    \item Если отделяется $z$ — разделить переменные и получить интегральный инвариант.
    \end{itemize}

\item \textbf{Цель.} Сформировать \emph{общее решение}.

    \textbf{Действие.} Записать $u=F(I_1,I_2)$.

\item \textbf{Цель.} Проверить \emph{независимость} $I_1,I_2$.

    \textbf{Действие.} Убедиться, что $dI_1\wedge dI_2\neq 0$ (или $\nabla I_1\times \nabla I_2\neq 0$) в рабочей области.
\end{enumerate}

\subsection{Как быстро находить $I_1$: детерминированные «детекторы»}\label{sec:pde-first-invariant}

Сначала можно убрать общий ненулевой множитель: $(a,b)\sim (\tilde a,\tilde b)$ дают те же инварианты.

\begin{itemize}
\item \textbf{Радиальный тип:} $(a,b)=(x,y)$ $\Rightarrow$ $\displaystyle \frac{dy}{dx}=\frac{y}{x}$ $\Rightarrow$ $I_1=\dfrac{y}{x}$.

\item \textbf{Вращение:} $(a,b)=(y,-x)$ $\Rightarrow$ $y\,dy=-x\,dx$ $\Rightarrow$ $I_1=x^2+y^2$.

\item \textbf{Диагональный линейный:} $(a,b)=(\alpha x,\beta y)$ $\Rightarrow$ $I_1=\dfrac{y}{x^{\beta/\alpha}}$.

\item \textbf{Общий линейный однородный:} $(a,b)=(\alpha x+\beta y,\ \gamma x+\delta y)$.
Через левые собственные векторы $M^\top$: взять $\xi=\mathbf w_1\!\cdot\!(x,y)$, $\eta=\mathbf w_2\!\cdot\!(x,y)$, тогда
$I_1=\eta/\xi^{\lambda_2/\lambda_1}$.
Альтернатива: подстановка $v=y/x$ всегда даёт $\dfrac{dv}{dx}=\dfrac{\Phi(v)}{x}$, интегрируется в логарифмах.

\item \textbf{Однородность одного порядка $d$:} если $a(\lambda x,\lambda y)=\lambda^d a$ и $b(\lambda x,\lambda y)=\lambda^d b$, то
$\,v=y/x$ ведёт к $I_1=x\,G(v)$.
\end{itemize}

\subsection{Как добирать $I_2$: стандартные ветки}\label{sec:pde-second-invariant}

\begin{itemize}
\item $c\equiv 0$ $\Rightarrow$ $I_2=z$.

\item $c=\mu(x,y)\,z$ $\Rightarrow$ $z'=\tilde{\alpha}(x;I_1)\,z$ $\Rightarrow$ $I_2=z\,\exp\!\big(-\int \tilde{\alpha}\,dx\big)$.

\item $c=\mu(x,y)\,z+\nu(x,y)$ $\Rightarrow$ линейное ОДУ, формула выше с $\tilde{\beta}\not\equiv0$.

\item $c=\mu(x,y)$ (не зависит от $z$) $\Rightarrow$ $I_2=z-\int \frac{\mu(x,\Psi(x;I_1))}{a(x,\Psi(x;I_1))}\,dx$.
\end{itemize}

