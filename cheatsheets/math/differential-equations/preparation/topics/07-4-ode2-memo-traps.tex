\subsection{Памятка и типичные ловушки}\label{sec:ode2-memo-traps}

\subsubsection{Памятка: что делать на экзамене (минимум выбора)}

\begin{enumerate}
\item Записать $p,q$ (после деления на коэффициент при $y''$).
\item Снять $y'$: $\phi=\exp(-\tfrac12\int p)$, найти $Q=q-\tfrac{p'}{2}-\tfrac{p^2}{4}$.
\item Нужен вронскиан? Сразу Абель: $W(x)=W(x_0)\exp(-\int p)$; если даны $y_i(x_0),y_i'(x_0)$ — подставить.
\item Нужен качественный вывод? Если $Q\le 0$ на данном промежутке — пишем «$\le 1$ нуля» (аргумент A).
Если дано $q<0$ — пишем «положительный максимум невозможен» (аргумент B).
\item Итог: короткая формулировка ($Q$, $W$, свойство).
\end{enumerate}

\subsubsection{Типичные ловушки}

\begin{itemize}
\item Забыли разделить на коэффициент при $y''$ — $p,q$ посчитаны неверно.
\item Ошибка знака в $Q$: внимательно к $-\tfrac{p'}{2}$ и $-\tfrac{p^2}{4}$.
\item Вронскиан: не пытайтесь дифференцировать $W$ напрямую — используйте Абеля.
\item Для утверждений о нулях не нужно решать уравнение: достаточно знака $Q$ (после снятия $y'$).
\end{itemize}
