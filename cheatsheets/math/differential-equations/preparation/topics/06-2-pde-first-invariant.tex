\subsection{Как быстро находить $I_1$: детерминированные «детекторы»}\label{sec:pde-first-invariant}

Сначала можно убрать общий ненулевой множитель: $(a,b)\sim (\tilde a,\tilde b)$ дают те же инварианты.

\begin{itemize}
\item \textbf{Радиальный тип:} $(a,b)=(x,y)$ $\Rightarrow$ $\displaystyle \frac{dy}{dx}=\frac{y}{x}$ $\Rightarrow$ $I_1=\dfrac{y}{x}$.

\item \textbf{Вращение:} $(a,b)=(y,-x)$ $\Rightarrow$ $y\,dy=-x\,dx$ $\Rightarrow$ $I_1=x^2+y^2$.

\item \textbf{Диагональный линейный:} $(a,b)=(\alpha x,\beta y)$ $\Rightarrow$ $I_1=\dfrac{y}{x^{\beta/\alpha}}$.

\item \textbf{Общий линейный однородный:} $(a,b)=(\alpha x+\beta y,\ \gamma x+\delta y)$.
Через левые собственные векторы $M^\top$: взять $\xi=\mathbf w_1\!\cdot\!(x,y)$, $\eta=\mathbf w_2\!\cdot\!(x,y)$, тогда
$I_1=\eta/\xi^{\lambda_2/\lambda_1}$.
Альтернатива: подстановка $v=y/x$ всегда даёт $\dfrac{dv}{dx}=\dfrac{\Phi(v)}{x}$, интегрируется в логарифмах.

\item \textbf{Однородность одного порядка $d$:} если $a(\lambda x,\lambda y)=\lambda^d a$ и $b(\lambda x,\lambda y)=\lambda^d b$, то
$\,v=y/x$ ведёт к $I_1=x\,G(v)$.
\end{itemize}
