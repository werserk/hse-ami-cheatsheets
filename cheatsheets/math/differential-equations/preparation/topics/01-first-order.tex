\section{Однородные линейные разностные уравнения}\label{sec:homogeneous}

\begin{center}
\fbox{\parbox{0.9\textwidth}{
\textbf{Пример.} Решите однородное линейное разностное уравнение:
\begin{align}
y_{t+3} - 3y_{t+2} + 6y_{t+1} - 4y_t = 0
\end{align}
}}
\end{center}

\textbf{Определение.} Линейное однородное разностное уравнение порядка $k$ с постоянными коэффициентами:
\begin{align}
a_t + c_1 a_{t-1} + c_2 a_{t-2} + \dots + c_k a_{t-k} = 0, \quad c_k \neq 0
\end{align}

Пара «уравнение + $k$ начальных условий» задаёт единственное решение.

\textbf{Идея решения -- метод характеристических корней.} Полагаем $a_t = r^t$ $\Rightarrow$
\begin{align}
r^t (1 + c_1 r^{-1} + c_2 r^{-2} + \dots + c_k r^{-k}) = 0 \iff r^k + c_1 r^{k-1} + \dots + c_k = 0
\end{align}

т.е. характеристический многочлен $\chi(r) = r^k + c_1 r^{k-1} + \dots + c_k$. Его корни целиком описывают форму общего решения.

\textit{Обозначения:} $P_j(t), Q_j(t)$ — полиномы по $t$ степени $\le j$.

\begin{table}[h!]
\centering
\caption{Выбор формы решения по типу корней характеристического многочлена}
\label{tab:form-choices}
\begin{tabular}{|l|l|}
\hline
\textbf{Условия на корни} & \textbf{Вклад в решение} \\
\hline
Действительный корень $r$ кратности $m\ge 1$ &
$P_{m-1}(t)\, r^{\,t}$ \\
\hline
Комплексно-сопряжённая пара $\rho e^{\pm i\theta}$ кратности $s\ge 1$ &
$\rho^{\,t}\big(P_{s-1}(t)\cos(\theta t)+Q_{s-1}(t)\sin(\theta t)\big)$ \\
\hline
\end{tabular}

\vspace{0.5em}
\emph{Итоговое общее решение -- сумма форм всех корней:}
\begin{align*}
a_t = \sum_{j} P_{m_j-1}(t)\, r_j^{\,t}
\; + \sum_{k} \rho_k^{\,t}\big(P_{s_k-1}(t)\cos(\theta_k t)+Q_{s_k-1}(t)\sin(\theta_k t)\big),
\end{align*}
где $r_j$ — действительные корни кратности $M_j$, $\rho_k e^{\pm i\theta_k}$ — комплексно-сопряжённые корни кратности $N_k$.
Сумма кратностей всех корней равна порядку $k$.
\end{table}

\textbf{Подгонка под начальные условия.} Подставляем $t = 0, 1, \dots, k - 1$ в общий вид, решаем линейную систему на $\alpha$-коэффициенты.

\begin{center}
\fbox{\parbox{0.9\textwidth}{
\textbf{Алгоритм.}
\begin{enumerate}
\item \textbf{Нормализация.} Привести уравнение к виду $a_t+\sum_{j=1}^{k} c_j a_{t-j}=0$, $c_k\neq 0$.
\item \textbf{Характеристический многочлен.} Записать $\chi(r)=r^{k}+c_1 r^{k-1}+\dots+c_k$.
\item \textbf{Корни и кратности.} Найти корни $r$ и их кратности $m$ (\,$\sum m = k$\,).
\item \textbf{Общий вид решения (см.~табл.~\ref{tab:form-choices}).}
Для каждого корня/пары взять соответствующий вклад из таблицы и сложить их.
\item \textbf{Подгонка под начальные условия.} Подставить $k$ заданных значений подряд и решить линейную систему для постоянных.
\end{enumerate}
}}
\end{center}

% Ссылки на задачи: добавьте URL/refs сюда
% Например: Сборник И. Иванов, гл. 2; задачи 1-20.
