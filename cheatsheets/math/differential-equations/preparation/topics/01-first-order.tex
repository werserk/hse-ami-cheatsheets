\section{Однородные линейные разностные уравнения}\label{sec:homogeneous}

\begin{center}
\fbox{\parbox{0.9\textwidth}{
\textbf{Пример.} Решите однородное линейное разностное уравнение:
\begin{align}
y_{t+3} - 3y_{t+2} + 6y_{t+1} - 4y_t = 0
\end{align}
}}
\end{center}

\textbf{Определение.} Линейное однородное разностное уравнение порядка $k$ с постоянными коэффициентами:
\begin{align}
a_t + c_1 a_{t-1} + c_2 a_{t-2} + \dots + c_k a_{t-k} = 0, \quad c_k \neq 0
\end{align}

Пара «уравнение + $k$ начальных условий» задаёт единственное решение.

\textbf{Идея решения -- метод характеристических корней.} Полагаем $a_t = r^t$ $\Rightarrow$
\begin{align}
r^t (1 + c_1 r^{-1} + c_2 r^{-2} + \dots + c_k r^{-k}) = 0 \iff r^k + c_1 r^{k-1} + \dots + c_k = 0
\end{align}

т.е. характеристический многочлен $\chi(r) = r^k + c_1 r^{k-1} + \dots + c_k$. Его корни целиком описывают форму общего решения.

\textbf{Три базовых случая (три формы решения):}

\textbf{1. Различные корни $r_1, \dots, r_k$:}
\begin{align}
a_t = \alpha_1 r_1^t + \dots + \alpha_k r_k^t
\end{align}

\textbf{2. Кратный корень $r$ кратности $m$:}
\begin{align}
a_t = (\alpha_0 + \alpha_1 t + \dots + \alpha_{m-1} t^{m-1}) r^t
\end{align}

\textbf{3. Комплексно-сопряжённые корни $r_{1,2} = \rho e^{\pm i\theta}$ (коэффициенты действительны):}
\begin{align}
a_t = \rho^t (A \cos(\theta t) + B \sin(\theta t))
\end{align}

Эти формы — стандартный результат теории ЛО-рекуррент (см. также «действительная форма» при комплексных корнях).

\textbf{Подгонка под начальные условия.} Подставляем $t = 0, 1, \dots, k - 1$ в общий вид, решаем линейную систему на $\alpha$-коэффициенты.

% Ссылки на задачи: добавьте URL/refs сюда
% Например: Сборник И. Иванов, гл. 2; задачи 1-20.


