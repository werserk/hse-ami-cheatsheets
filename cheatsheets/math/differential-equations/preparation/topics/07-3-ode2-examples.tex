\subsection{Примеры (одинаковая схема 5 шагов)}\label{sec:ode2-examples}

\subsubsection{Пример 1 (Эйлер—Коши; $x>0$)}

\[
y''+\frac{2}{x}y' - \frac{3}{x^2}y=0.
\]

\textbf{Шаг 1.} \textbf{Цель.} $p,q$. \textbf{Действие.} $p=\frac{2}{x}$, $q=-\frac{3}{x^2}$.

\textbf{Шаг 2.} \textbf{Цель.} $Q$. \textbf{Действие.} $\phi=x^{-1}$,
$
Q=q-\frac{p'}{2}-\frac{p^2}{4}
=-\frac{3}{x^2}-\Bigl(-\frac{1}{x^2}\Bigr)-\frac{1}{x^2}=-\frac{3}{x^2}.
$

\textbf{Шаг 3.} \textbf{Цель.} $W(x)$. \textbf{Действие.} $W(x)=W(1)\exp\!\bigl(-\int_1^x\frac{2}{t}dt\bigr)=W(1)/x^2$.

\textbf{Шаг 4.} \textbf{Цель.} Вывод о нулях. \textbf{Действие.} $Q\le0$ на $x>0$ $\Rightarrow$ у нетривиального решения $\le1$ нуля на $(0,\infty)$.

\textbf{Шаг 5.} \textbf{Цель.} Итог. \textbf{Действие.} Нормальная форма $z''-\frac{3}{x^2}z=0$, $W(x)=W(1)/x^2$, «$\le1$ нуля».

\subsubsection{Пример 2 (постоянные коэффициенты)}

\[
y''+4y'+3y=0.
\]

\textbf{Шаг 1.} $p=4,\ q=3$.

\textbf{Шаг 2.} \textbf{Цель.} $Q$. \textbf{Действие.} $\phi=e^{-2x}$, $Q=3-0-4=-1$, т.е. $z''-z=0$.

\textbf{Шаг 3.} \textbf{Цель.} $W(x)$. \textbf{Действие.} $W(x)=W(0)\,e^{-4x}$.

\textbf{Шаг 4.} \textbf{Цель.} Нули. \textbf{Действие.} $Q=-1<0$ $\Rightarrow$ у нетривиального решения $\le1$ нуля.

\textbf{Шаг 5.} \textbf{Цель.} Итог. \textbf{Действие.} $z=\alpha e^x+\beta e^{-x}$, $y=e^{-2x}z$, $W(x)=W(0)e^{-4x}$, «не осциллирует».

\subsubsection{Пример 3 (переменный коэффициент $p$, подобранный $q$)}

\[
y''+x\,y'+\Bigl(\frac{x^2}{4}-1\Bigr)\,y=0\qquad (x\in\mathbb R).
\]

\textbf{Шаг 1.} $p=x,\ q=\tfrac{x^2}{4}-1$.

\textbf{Шаг 2.} \textbf{Цель.} $Q$. \textbf{Действие.} $\phi=e^{-x^2/4}$,
\[
Q=q-\frac{p'}{2}-\frac{p^2}{4}
=\Bigl(\frac{x^2}{4}-1\Bigr)-\frac12-\frac{x^2}{4}=-\frac{3}{2}.
\]
Получаем $z''-\tfrac32 z=0$.

\textbf{Шаг 3.} \textbf{Цель.} $W(x)$. \textbf{Действие.} $W(x)=W(0)\,e^{-\int_0^x t\,dt}=W(0)\,e^{-x^2/2}$.

\textbf{Шаг 4.} \textbf{Цель.} Нули. \textbf{Действие.} $Q=-\tfrac32<0$ $\Rightarrow$ $\le1$ нуля.

\textbf{Шаг 5.} \textbf{Цель.} Итог. \textbf{Действие.} Нормальная форма постоянного знака, вронскиан гауссов, решение не осциллирует.

\subsubsection{Пример 4 (вронскиан в конкретной точке)}

\[
(x+2)\,y''-3\,y'+\sqrt{\,1-x\,}\,y=0,\qquad x\in(-2,1).
\]

Пусть $y_1,y_2$ — решения с начальными данными
\[
y_1(0)=0,\ y_1'(0)=1;\qquad y_2(0)=3,\ y_2'(0)=2.
\]

\textbf{Шаг 1.} \textbf{Цель.} $p,q$. \textbf{Действие.} Делим на $x+2$: $p(x)=-\dfrac{3}{x+2}$, $q(x)=\dfrac{\sqrt{1-x}}{x+2}$.

\textbf{Шаг 2.} \textbf{Цель.} $Q$. \textbf{Действие.} Не обязательно считать: достаточно для $W$.

\textbf{Шаг 3.} \textbf{Цель.} $W(x)$. \textbf{Действие.} $W(0)=\det\begin{pmatrix}0&3\\1&2\end{pmatrix}=-3\neq0$ (фундаментальная пара).
По Абелю
\[
W(-1)=W(0)\exp\!\Bigl(-\int_0^{-1}\!p(t)\,dt\Bigr)
= -3\exp\!\Bigl(-\int_0^{-1}\!\frac{-3}{t+2}\,dt\Bigr)
= -3\cdot\Bigl(\frac{1}{2}\Bigr)^{\!3}=-\frac{3}{8}.
\]

\textbf{Шаг 4.} \textbf{Цель.} Вывод. \textbf{Действие.} Пара фундаментальна, $W(-1)=-3/8$.

\textbf{Шаг 5.} \textbf{Цель.} Итог. \textbf{Действие.} Независимость подтверждена в любой точке интервала.
