\section{Нелинейные 2D-системы: равновесия, линеаризация, негиперболика}\label{sec:nonlinear-2d-systems}

\HSEDefinition{\Term{Нелинейная 2D-система} — автономная система вида $\dot{x} = f(x, y)$, $\dot{y} = g(x,y)$, где $f, g \in C^1$. Метод линеаризации применяется для анализа положений равновесия и их типов.}

\subsection{Где применяется метод линеаризации (признаки «наш случай»)}
\begin{itemize}
    \item Дано: автономная система $\dot{x} = f(x, y)$, $\dot{y} = g(x,y)$, $f, g \in C^1$.
    \item Спрашивают: положения равновесия, их тип и эскиз фазового портрета в окрестности.
    \item В точке(ах) равновесия Якоби $J = \begin{pmatrix} f_x & f_y \\ g_x & g_y \end{pmatrix}$ удовлетворяет $\det J \neq 0$ (гиперболическая точка).
    \item Если $\det J = 0$ или $D = 0$ --- это уже не МЗ (негиперболика/граница случаев).
\end{itemize}

\subsection{Единый 5-шаговый алгоритм (используем во всех примерах)}

\textbf{Шаг 1. Цель: найти все равновесия.}

Действие: решить $f(x, y) = 0, g(x, y) = 0$.

\textbf{Шаг 2. Цель: получить линеаризацию.}

Действие: в каждой найденной точке вычислить Якоби $J$.

\textbf{Шаг 3. Цель: классифицировать тип точки по числам.}

Действие (единственное ветвление строго по знакам):
\begin{itemize}
    \item Если $\det J < 0 \to$ седло (неустойч.).
    \item Если $\det J > 0$:
    \begin{itemize}
        \item посчитать $D = \text{tr}^2 - 4 \det$.
        \item если $D > 0$: $\text{tr} < 0 \to$ устойч. узел, $\text{tr} > 0 \to$ неустойч. узел;
        \item если $D < 0$: $\text{tr} < 0 \to$ устойч. фокус, $\text{tr} > 0 \to$ неустойч. фокус.
    \end{itemize}
\end{itemize}

\textbf{Шаг 4. Цель: зафиксировать направления и устойчивость.}

Действие: указать «куда текут» траектории (в/из точки) и, при седле, назвать две устойчивые/неустойчивые сепаратрисы (вдоль собственных направлений $J$).

\textbf{Шаг 5. Цель: нарисовать локальный эскиз.}

Действие: около каждой точки нанести тип (узел/фокус/седло), стрелки по устойчивости, грубо ориентируясь на нулевые изоклины $f = 0, g = 0$ для знаков $\dot{x}, \dot{y}$.

\HSENote{Этот алгоритм является универсальным для анализа нелинейных 2D-систем и позволяет систематически подходить к решению задач на классификацию равновесных точек.}
