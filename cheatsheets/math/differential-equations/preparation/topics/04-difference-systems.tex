\subsection{Системы разностных уравнений}\label{subsec:systems}

\begin{center}
\fbox{\parbox{0.9\textwidth}{
\textbf{Пример.} Решите систему разностных уравнений:
\begin{align}
\begin{cases}
x_{t+1} = 4x_t + y_t \\
y_{t+1} = 2y_t
\end{cases}
\quad \text{с начальными условиями } \mathbf{x}_0 = \begin{pmatrix} 1 \\ 1 \end{pmatrix}
\end{align}
}}
\end{center}

\textbf{Определение.} Система линейных разностных уравнений первого порядка с постоянными коэффициентами:
\begin{align}
\mathbf{x}_{t+1} = A\mathbf{x}_t, \quad \mathbf{x}_0 \text{ задано}
\end{align}
где $A$ — квадратная матрица $n \times n$, $\mathbf{x}_t \in \mathbb{R}^n$.

\textbf{Цель:} найти $\mathbf{x}_t = A^t\mathbf{x}_0$ для всех $t \geq 0$.

\textbf{Идея решения:} возведение матрицы $A$ в степень $t$. Для этого используем спектральное разложение матрицы.

\textit{Обозначения:} $\lambda_i$ — собственные значения, $\mathbf{v}_i$ — собственные векторы, $\chi_A(\lambda) = \det(\lambda I - A)$ — характеристический многочлен.

\begin{table}[h!]
\centering
\caption{Выбор метода решения по типу собственных значений}
\label{tab:method-selection}
\begin{tabular}{|l|l|}
\hline
\textbf{Условия на собственные значения} & \textbf{Рекомендуемый метод} \\
\hline
Разные действительные корни, полный базис собственных векторов & Диагонализация \\
\hline
Повторный корень, недостаточно собственных векторов & Жорданова форма \\
\hline
Комплексно-сопряжённая пара & Реальный блок поворота \\
\hline
Матрица $2 \times 2$ (любой случай) & Кэли–Гамильтон \\
\hline
\end{tabular}
\end{table}

\bigskip
\hrule
\bigskip

\subsubsection*{1. Диагонализация}

\textbf{Условие применения:} матрица $A$ имеет $n$ линейно независимых собственных векторов (диагонализуема).

\textbf{Теорема.} Если $A$ диагонализуема, то $A = S\Lambda S^{-1}$, где $\Lambda = \text{diag}(\lambda_1, \ldots, \lambda_n)$ — диагональная матрица собственных значений, $S = [\mathbf{v}_1, \ldots, \mathbf{v}_n]$ — матрица собственных векторов.

\begin{center}
\fbox{\parbox{0.92\textwidth}{
\textbf{Алгоритм диагонализации.}
\begin{enumerate}
\item \textbf{Характеристический многочлен.} Найти $\chi_A(\lambda) = \det(\lambda I - A)$.
\item \textbf{Собственные значения.} Решить $\chi_A(\lambda) = 0$, найти $\lambda_1, \ldots, \lambda_n$.
\item \textbf{Собственные векторы.} Для каждого $\lambda_i$ решить $(A - \lambda_i I)\mathbf{v}_i = \mathbf{0}$.
\item \textbf{Проверка диагонализуемости.} Убедиться, что $\det S \neq 0$ (собственные векторы линейно независимы).
\item \textbf{Диагонализация.} $A = S\Lambda S^{-1}$, где $\Lambda = \text{diag}(\lambda_1, \ldots, \lambda_n)$.
\item \textbf{Возведение в степень.} $\boxed{A^t = S\Lambda^t S^{-1}}$, где $\Lambda^t = \text{diag}(\lambda_1^t, \ldots, \lambda_n^t)$.
\item \textbf{Решение системы.} $\mathbf{x}_t = A^t\mathbf{x}_0$.
\end{enumerate}
}}
\end{center}

\begin{center}
\fbox{\parbox{0.92\textwidth}{
\textbf{Пример.} Решите систему:
\[
\begin{cases}
x_{t+1} = 4x_t + y_t \\
y_{t+1} = 2y_t
\end{cases}
\quad \text{с } \mathbf{x}_0 = \begin{pmatrix} 1 \\ 1 \end{pmatrix}
\]

\textbf{Решение.}

\textbf{Шаг 1.} Матрица системы: $A = \begin{pmatrix} 4 & 1 \\ 0 & 2 \end{pmatrix}$.

\textbf{Шаг 2.} Характеристический многочлен:
\[
\chi_A(\lambda) = \det\begin{pmatrix} \lambda-4 & -1 \\ 0 & \lambda-2 \end{pmatrix} = (\lambda-4)(\lambda-2) = 0
\]
Собственные значения: $\lambda_1 = 4$, $\lambda_2 = 2$.

\textbf{Шаг 3.} Собственные векторы:
\begin{itemize}
\item Для $\lambda_1 = 4$: $(A - 4I)\mathbf{v}_1 = \mathbf{0} \Rightarrow \begin{pmatrix} 0 & 1 \\ 0 & -2 \end{pmatrix}\mathbf{v}_1 = \mathbf{0} \Rightarrow \mathbf{v}_1 = \begin{pmatrix} 1 \\ 0 \end{pmatrix}$
\item Для $\lambda_2 = 2$: $(A - 2I)\mathbf{v}_2 = \mathbf{0} \Rightarrow \begin{pmatrix} 2 & 1 \\ 0 & 0 \end{pmatrix}\mathbf{v}_2 = \mathbf{0} \Rightarrow \mathbf{v}_2 = \begin{pmatrix} 1 \\ -2 \end{pmatrix}$
\end{itemize}

\textbf{Шаг 4.} Матрицы:
\[
S = \begin{pmatrix} 1 & 1 \\ 0 & -2 \end{pmatrix}, \quad \Lambda = \begin{pmatrix} 4 & 0 \\ 0 & 2 \end{pmatrix}, \quad S^{-1} = \begin{pmatrix} 1 & \frac{1}{2} \\ 0 & -\frac{1}{2} \end{pmatrix}
\]

\textbf{Шаг 5.} Возведение в степень:
\[
A^t = S\Lambda^t S^{-1} = \begin{pmatrix} 1 & 1 \\ 0 & -2 \end{pmatrix}\begin{pmatrix} 4^t & 0 \\ 0 & 2^t \end{pmatrix}\begin{pmatrix} 1 & \frac{1}{2} \\ 0 & -\frac{1}{2} \end{pmatrix} = \begin{pmatrix} 4^t & \frac{4^t - 2^t}{2} \\ 0 & 2^t \end{pmatrix}
\]

\textbf{Шаг 6.} Решение:
\[
\mathbf{x}_t = A^t\mathbf{x}_0 = \begin{pmatrix} 4^t & \frac{4^t - 2^t}{2} \\ 0 & 2^t \end{pmatrix}\begin{pmatrix} 1 \\ 1 \end{pmatrix} = \begin{pmatrix} 4^t + \frac{4^t - 2^t}{2} \\ 2^t \end{pmatrix} = \begin{pmatrix} \frac{3 \cdot 4^t - 2^t}{2} \\ 2^t \end{pmatrix}
\]

\textbf{Ответ:} $\boxed{x_t = \frac{3 \cdot 4^t - 2^t}{2}, \quad y_t = 2^t}$
}}
\end{center}

\bigskip
\hrule
\bigskip

\subsubsection*{2. Жорданова форма (повторный корень)}

\textbf{Условие применения:} матрица $A$ имеет повторное собственное значение, но недостаточно собственных векторов для диагонализации.

\textbf{Теорема.} Если $A$ имеет единственное собственное значение $\lambda$ кратности $n$, то $A = \lambda I + N$, где $N$ — нильпотентная матрица ($N^m = 0$ для некоторого $m \leq n$).

\textbf{Ключевая идея:} используем биномиальную формулу для $(I + \lambda^{-1}N)^t$.

\begin{center}
\fbox{\parbox{0.92\textwidth}{
\textbf{Алгоритм Жордановой формы.}
\begin{enumerate}
\item \textbf{Собственное значение.} Найти $\lambda$ (часто $\lambda = \frac{\text{tr}\,A}{n}$ для единственного корня).
\item \textbf{Нильпотентная матрица.} Вычислить $N = A - \lambda I$.
\item \textbf{Индекс нильпотентности.} Найти минимальное $m$ такое, что $N^m = 0$.
\item \textbf{Биномиальная формула.} $\boxed{A^t = \lambda^t \sum_{k=0}^{m-1} \binom{t}{k} (\lambda^{-1}N)^k}$.
\item \textbf{Решение системы.} $\mathbf{x}_t = A^t\mathbf{x}_0$.
\end{enumerate}
}}
\end{center}

\begin{center}
\fbox{\parbox{0.92\textwidth}{
\textbf{Пример.} Решите систему:
\[
\begin{cases}
x_{t+1} = 2x_t + y_t \\
y_{t+1} = 2y_t
\end{cases}
\quad \text{с } \mathbf{x}_0 = \begin{pmatrix} 1 \\ 1 \end{pmatrix}
\]

\textbf{Решение.}

\textbf{Шаг 1.} Матрица системы: $A = \begin{pmatrix} 2 & 1 \\ 0 & 2 \end{pmatrix}$.

\textbf{Шаг 2.} Собственное значение: $\lambda = \frac{\text{tr}\,A}{2} = \frac{4}{2} = 2$.

\textbf{Шаг 3.} Нильпотентная матрица:
\[
N = A - \lambda I = A - 2I = \begin{pmatrix} 0 & 1 \\ 0 & 0 \end{pmatrix}
\]

\textbf{Шаг 4.} Индекс нильпотентности: $N^2 = \begin{pmatrix} 0 & 1 \\ 0 & 0 \end{pmatrix}^2 = \begin{pmatrix} 0 & 0 \\ 0 & 0 \end{pmatrix} = 0$, значит $m = 2$.

\textbf{Шаг 5.} Биномиальная формула:
\[
A^t = 2^t \sum_{k=0}^{1} \binom{t}{k} (2^{-1}N)^k = 2^t \left[ \binom{t}{0} I + \binom{t}{1} \frac{N}{2} \right]
\]
\[
= 2^t \left[ I + \frac{t}{2} N \right] = 2^t \left[ \begin{pmatrix} 1 & 0 \\ 0 & 1 \end{pmatrix} + \frac{t}{2} \begin{pmatrix} 0 & 1 \\ 0 & 0 \end{pmatrix} \right]
\]
\[
= 2^t \begin{pmatrix} 1 & \frac{t}{2} \\ 0 & 1 \end{pmatrix}
\]

\textbf{Шаг 6.} Решение:
\[
\mathbf{x}_t = A^t\mathbf{x}_0 = 2^t \begin{pmatrix} 1 & \frac{t}{2} \\ 0 & 1 \end{pmatrix} \begin{pmatrix} 1 \\ 1 \end{pmatrix} = 2^t \begin{pmatrix} 1 + \frac{t}{2} \\ 1 \end{pmatrix}
\]

\textbf{Ответ:} $\boxed{x_t = 2^t \left(1 + \frac{t}{2}\right), \quad y_t = 2^t}$
}}
\end{center}

\bigskip
\hrule
\bigskip

\subsubsection*{3. Комплексная пара (реальный блок)}

\textbf{Условие применения:} матрица $A$ $2 \times 2$ имеет комплексно-сопряжённые собственные значения $\lambda = \rho e^{\pm i\theta}$.

\textbf{Теорема.} Для матрицы $A$ $2 \times 2$ с комплексными корнями $\lambda = \rho e^{\pm i\theta}$ справедливо:
\[
A^t = \rho^t \begin{pmatrix} \cos(t\theta) & -\sin(t\theta) \\ \sin(t\theta) & \cos(t\theta) \end{pmatrix}
\]

\textbf{Ключевая идея:} комплексные корни соответствуют повороту с масштабированием в вещественном пространстве.

\begin{center}
\fbox{\parbox{0.92\textwidth}{
\textbf{Алгоритм для комплексной пары.}
\begin{enumerate}
\item \textbf{Проверка комплексности.} Убедиться, что $(\text{tr}\,A)^2 - 4\det A < 0$.
\item \textbf{Модуль и аргумент.} Вычислить $\rho = \sqrt{\det A}$, $\cos\theta = \frac{\text{tr}\,A}{2\rho}$.
\item \textbf{Матрица поворота.} $\boxed{A^t = \rho^t \begin{pmatrix} \cos(t\theta) & -\sin(t\theta) \\ \sin(t\theta) & \cos(t\theta) \end{pmatrix}}$.
\item \textbf{Решение системы.} $\mathbf{x}_t = A^t\mathbf{x}_0$.
\end{enumerate}
}}
\end{center}

\begin{center}
\fbox{\parbox{0.92\textwidth}{
\textbf{Пример.} Решите систему:
\[
\begin{cases}
x_{t+1} = x_t - y_t \\
y_{t+1} = x_t + y_t
\end{cases}
\quad \text{с } \mathbf{x}_0 = \begin{pmatrix} 1 \\ 0 \end{pmatrix}
\]

\textbf{Решение.}

\textbf{Шаг 1.} Матрица системы: $A = \begin{pmatrix} 1 & -1 \\ 1 & 1 \end{pmatrix}$.

\textbf{Шаг 2.} Проверка комплексности:
\[
(\text{tr}\,A)^2 - 4\det A = 2^2 - 4 \cdot 2 = 4 - 8 = -4 < 0 \quad \checkmark
\]

\textbf{Шаг 3.} Модуль и аргумент:
\[
\rho = \sqrt{\det A} = \sqrt{2}, \quad \cos\theta = \frac{\text{tr}\,A}{2\rho} = \frac{2}{2\sqrt{2}} = \frac{1}{\sqrt{2}} = \cos\frac{\pi}{4}
\]
Значит, $\theta = \frac{\pi}{4}$.

\textbf{Шаг 4.} Матрица поворота:
\[
A^t = (\sqrt{2})^t \begin{pmatrix} \cos\frac{\pi t}{4} & -\sin\frac{\pi t}{4} \\ \sin\frac{\pi t}{4} & \cos\frac{\pi t}{4} \end{pmatrix}
\]

\textbf{Шаг 5.} Решение:
\[
\mathbf{x}_t = A^t\mathbf{x}_0 = (\sqrt{2})^t \begin{pmatrix} \cos\frac{\pi t}{4} & -\sin\frac{\pi t}{4} \\ \sin\frac{\pi t}{4} & \cos\frac{\pi t}{4} \end{pmatrix} \begin{pmatrix} 1 \\ 0 \end{pmatrix} = (\sqrt{2})^t \begin{pmatrix} \cos\frac{\pi t}{4} \\ \sin\frac{\pi t}{4} \end{pmatrix}
\]

\textbf{Ответ:} $\boxed{x_t = (\sqrt{2})^t \cos\frac{\pi t}{4}, \quad y_t = (\sqrt{2})^t \sin\frac{\pi t}{4}}$
}}
\end{center}

\bigskip
\hrule
\bigskip

\subsubsection*{4. Кэли–Гамильтон (универсальный метод)}

\textbf{Условие применения:} универсальный метод для матриц любого размера, особенно удобен для $2 \times 2$.

\textbf{Теорема Кэли–Гамильтона.} Матрица $A$ удовлетворяет своему характеристическому уравнению: $\chi_A(A) = 0$.

\textbf{Ключевая идея:} используем тождество $\chi_A(A) = 0$ для построения рекуррентного соотношения на степени матрицы.

\begin{center}
\fbox{\parbox{0.92\textwidth}{
\textbf{Алгоритм Кэли–Гамильтона.}
\begin{enumerate}
\item \textbf{Характеристический многочлен.} Найти $\chi_A(\lambda) = \lambda^2 - (\text{tr}\,A)\lambda + \det A$ для $2 \times 2$.
\item \textbf{Рекуррентное соотношение.} $A^{t+2} = (\text{tr}\,A)A^{t+1} - (\det A)A^t$.
\item \textbf{Представление.} $A^t = \alpha_t A + \beta_t I$ для $2 \times 2$.
\item \textbf{Система уравнений.} Решить систему для $\alpha_t, \beta_t$ используя начальные условия.
\item \textbf{Решение системы.} $\mathbf{x}_t = A^t\mathbf{x}_0$.
\end{enumerate}
}}
\end{center}

\textbf{Готовые формулы для $2 \times 2$:}
\begin{itemize}
\item \textbf{Разные корни} $\lambda_1 \neq \lambda_2$:
\[
\alpha_t = \frac{\lambda_1^t - \lambda_2^t}{\lambda_1 - \lambda_2}, \quad \beta_t = \frac{\lambda_1\lambda_2^t - \lambda_2\lambda_1^t}{\lambda_1 - \lambda_2}
\]
\item \textbf{Одинаковые корни} $\lambda_1 = \lambda_2 = \lambda$:
\[
\alpha_t = t\lambda^{t-1}, \quad \beta_t = (1-t)\lambda^t
\]
\end{itemize}

\begin{center}
\fbox{\parbox{0.92\textwidth}{
\textbf{Пример.} Решите систему:
\[
\begin{cases}
x_{t+1} = 3x_t + 2y_t \\
y_{t+1} = -2x_t - y_t
\end{cases}
\quad \text{с } \mathbf{x}_0 = \begin{pmatrix} 1 \\ 1 \end{pmatrix}
\]

\textbf{Решение.}

\textbf{Шаг 1.} Матрица системы: $A = \begin{pmatrix} 3 & 2 \\ -2 & -1 \end{pmatrix}$.

\textbf{Шаг 2.} Характеристический многочлен:
\[
\chi_A(\lambda) = \lambda^2 - (\text{tr}\,A)\lambda + \det A = \lambda^2 - 2\lambda + 1 = (\lambda - 1)^2
\]
Корни: $\lambda_1 = \lambda_2 = 1$ (кратность 2).

\textbf{Шаг 3.} Рекуррентное соотношение:
\[
A^{t+2} = 2A^{t+1} - A^t
\]

\textbf{Шаг 4.} Представление $A^t = \alpha_t A + \beta_t I$:
\[
\alpha_t = t \cdot 1^{t-1} = t, \quad \beta_t = (1-t) \cdot 1^t = 1-t
\]

\textbf{Шаг 5.} Матрица $A^t$:
\[
A^t = tA + (1-t)I = t\begin{pmatrix} 3 & 2 \\ -2 & -1 \end{pmatrix} + (1-t)\begin{pmatrix} 1 & 0 \\ 0 & 1 \end{pmatrix}
\]
\[
= \begin{pmatrix} 3t + (1-t) & 2t \\ -2t & -t + (1-t) \end{pmatrix} = \begin{pmatrix} 2t + 1 & 2t \\ -2t & 1 - 2t \end{pmatrix}
\]

\textbf{Шаг 6.} Решение:
\[
\mathbf{x}_t = A^t\mathbf{x}_0 = \begin{pmatrix} 2t + 1 & 2t \\ -2t & 1 - 2t \end{pmatrix} \begin{pmatrix} 1 \\ 1 \end{pmatrix} = \begin{pmatrix} 4t + 1 \\ 1 - 4t \end{pmatrix}
\]

\textbf{Ответ:} $\boxed{x_t = 4t + 1, \quad y_t = 1 - 4t}$
}}
\end{center}

\bigskip
\hrule
\bigskip

\begin{center}
\fbox{\parbox{0.92\textwidth}{
\textbf{Общий алгоритм решения систем разностных уравнений.}
\begin{enumerate}
\item \textbf{Анализ матрицы.} Вычислить $\text{tr}\,A$, $\det A$, $\chi_A(\lambda)$.
\item \textbf{Выбор метода.} По таблице~\ref{tab:method-selection} определить подходящий метод.
\item \textbf{Применение метода.} Использовать соответствующий алгоритм для нахождения $A^t$.
\item \textbf{Решение системы.} Вычислить $\mathbf{x}_t = A^t\mathbf{x}_0$.
\item \textbf{Проверка.} Убедиться, что $A^0 = I$, $A^1 = A$.
\end{enumerate}
}}
\end{center}

\textbf{Полезные проверки:}
\begin{itemize}
\item \textbf{Начальные условия:} $A^0 = I$, $A^1 = A$.
\item \textbf{Жорданова форма:} если $A = \lambda I + N$, проверить $N^m = 0$.
\item \textbf{Комплексная пара:} $\det A = \rho^2$, $\text{tr}\,A = 2\rho\cos\theta$.
\item \textbf{Биномиальные коэффициенты:} не забыть $\binom{t}{k}$ в формуле Жордана.
\end{itemize}

% Ссылки на задачи: добавьте URL/refs сюда
% Например: Сборник И. Иванов, гл. 3; задачи 1-15.
