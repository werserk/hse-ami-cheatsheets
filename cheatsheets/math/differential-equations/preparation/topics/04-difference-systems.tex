\subsection{Системы разностных уравнений}\label{subsec:systems}

\textbf{Пример (вход в тему).} Решите систему:
\[
\begin{cases}
x_{t+1} = 4x_t + y_t,\\
y_{t+1} = 2y_t,
\end{cases}
\qquad \mathbf{x}_0=\binom{1}{1}.
\]
Запишем в матричном виде: \(\ \mathbf{x}_{t+1}=A\mathbf{x}_t,\ \ A=\begin{pmatrix}4&1\\[2pt]0&2\end{pmatrix},\ \ \mathbf{x}_t=\binom{x_t}{y_t}.\)

\HSEDefinition{(Общий вид) Линейная система разностных уравнений первого порядка:}
\[
\boxed{\ \mathbf{x}_{t+1}=A\mathbf{x}_t+\mathbf{f}_t\ },\qquad
A\in\mathbb{R}^{n\times n},\ \ \mathbf{x}_t\in\mathbb{R}^n,\ \ \mathbf{f}_t\in\mathbb{R}^n.
\]
\emph{Цель:} найти \(\mathbf{x}_t\). Обозначим фундаментальную матрицу однородной части \(\Phi_t:=A^t\).

\HSEBox{Базовые формулы (запомнить!)}{
\[
\boxed{\ \mathbf{x}_t=\underbrace{A^t\mathbf{x}_0}_{\text{однородная часть}}+\underbrace{\sum_{k=0}^{t-1}A^{\,t-1-k}\,\mathbf{f}_k}_{\text{неоднородная свёртка}}\ }\quad (t\ge1).
\]
Частный случай \(\mathbf{f}_t\equiv\mathbf{b}\) (постоянный вектор):
\[
\boxed{\ \mathbf{x}_t=A^t\mathbf{x}_0+\Big(\sum_{k=0}^{t-1}A^k\Big)\mathbf{b}\ }=
\begin{cases}
A^t\mathbf{x}_0+(I-A^t)(I-A)^{-1}\mathbf{b}, & I-A\ \text{обратима},\\[4pt]
A^t\mathbf{x}_0+\text{(резонанс при }\lambda=1), & \text{иначе (см. ниже)}.
\end{cases}
\]
}

\paragraph{Дерево выбора метода для \(A^t\) (одинаково для однородных/неоднородных)}
\begin{itemize}
  \item Разные действительные корни \(\Rightarrow\) \textbf{Диагонализация}.
  \item Повторный корень, недостаточно собственных векторов \(\Rightarrow\) \textbf{Жордан}: \(A=\lambda I+N,\ N^m=0\).
  \item Комплексная пара \(\Rightarrow\) \textbf{Реальный поворот–масштаб}.
  \item Матрица \(2\times 2\) (любой случай) \(\Rightarrow\) часто быстрее \textbf{Кэли–Гамильтон}.
\end{itemize}

\subsubsection*{0. Неоднородные сразу: вариация постоянных (универсально)}
\HSEAlgorithm[Алгоритм вариации постоянных]{
\begin{enumerate}
  \item Найти \(\Phi_t=A^t\) (любой из разделов 1–4 ниже).
  \item Положить \(\mathbf{x}_t=\Phi_t\,\mathbf{c}_t\). Тогда \(\Phi_{t+1}\mathbf{c}_{t+1}=A\Phi_t\mathbf{c}_t+\mathbf{f}_t=\Phi_{t+1}\mathbf{c}_t+\mathbf{f}_t\).
  \item Получаем рекурренту на параметры: \(\boxed{\,\mathbf{c}_{t+1}-\mathbf{c}_t=\Phi_{t+1}^{-1}\mathbf{f}_t\,}\).
  \item Отсюда \(\ \mathbf{c}_t=\mathbf{c}_0+\sum_{k=0}^{t-1}\Phi_{k+1}^{-1}\mathbf{f}_k,\ \ \mathbf{c}_0=\mathbf{x}_0\).
  \item Итог: \(\boxed{\,\mathbf{x}_t=\Phi_t\mathbf{x}_0+\sum_{k=0}^{t-1}\Phi_t\Phi_{k+1}^{-1}\mathbf{f}_k
  =A^t\mathbf{x}_0+\sum_{k=0}^{t-1}A^{\,t-1-k}\mathbf{f}_k\,}\).
\end{enumerate}
}
\noindent\emph{Однородный случай} \(\mathbf{f}_t\equiv0\) получается автоматом: просто исчезает шаг 3 и суммирование.

\paragraph{Шорткаты по правой части (быстрое \(y^{(p)}\) как в скаляре)}
\begin{itemize}
  \item \(\mathbf{f}_t\equiv\mathbf{b}\). Если \(I-A\) обратима: \(\mathbf{x}^{(p)}=\mathbf{x}_*\) постоянно, где \((I-A)\mathbf{x}_*=\mathbf{b}\).
  \item \(\mathbf{f}_t=\lambda^t\mathbf{b}\). Пробуем \(\boxed{\ \mathbf{x}^{(p)}_t=\lambda^t\mathbf{y}\ }\). Если \(\det(\lambda I-A)\neq0\), то \((\lambda I-A)\mathbf{y}=\mathbf{b}\).
        Если \(\det(\lambda I-A)=0\) (\textbf{резонанс}), умножаем на \(t\): \(\mathbf{x}^{(p)}_t=t\,\lambda^t\mathbf{y}\) (для кратности \(1\)); при большей кратности — \(t^s\lambda^t\).
\end{itemize}

\bigskip\hrule\bigskip

\subsubsection*{1. Диагонализация (разные действительные корни)}
\textbf{Алгоритм.}
\(A=S\Lambda S^{-1},\ \Lambda=\mathrm{diag}(\lambda_i)\Rightarrow \boxed{A^t=S\Lambda^t S^{-1}},\ \Lambda^t=\mathrm{diag}(\lambda_i^t).\)

\textbf{Мини-пример.} \(A=\begin{psmallmatrix}4&1\\0&2\end{psmallmatrix}\Rightarrow
A^t=\begin{psmallmatrix}4^t&\tfrac{4^t-2^t}{2}\\[2pt]0&2^t\end{psmallmatrix}.\)
Тогда для \(\mathbf{f}_t=\lambda^t\mathbf{b}\) можно либо свёрткой, либо шорткатом: найти \(\mathbf{y}\) из \((\lambda I-A)\mathbf{y}=\mathbf{b}\).

\bigskip\hrule\bigskip

\subsubsection*{2. Жордан (повторный корень, недостаёт базиса)}
Если \(A=\lambda I+N,\ N^m=0\), то
\[
\boxed{\,A^t=\lambda^{\,t}\sum_{k=0}^{m-1}\binom{t}{k}\big(\lambda^{-1}N\big)^k\,}.
\]
Часто \(m=2\): \(A^t=\lambda^t\big(I+\tfrac{t}{\lambda}N\big)\).

\textbf{Мини-пример.} \(A=\begin{psmallmatrix}2&1\\0&2\end{psmallmatrix}\Rightarrow A^t=2^t\begin{psmallmatrix}1&\tfrac t2\\0&1\end{psmallmatrix}.\)
Для \(\mathbf{f}_t\equiv\mathbf{b}\): \(\sum_{k=0}^{t-1}A^k\) удобно считать той же формулой (бином по \(N\)).

\bigskip\hrule\bigskip

\subsubsection*{3. Комплексная пара (реальный поворот–масштаб)}
Если собственные \(\lambda=\rho e^{\pm i\theta}\) (для \(2\times2\): \((\operatorname{tr}A)^2-4\det A<0\)), то
\[
\boxed{\,A^t=\rho^t\begin{pmatrix}\cos t\theta&-\sin t\theta\\ \sin t\theta&\cos t\theta\end{pmatrix}\,}
\quad\text{(после приведения к блоку)}.
\]
Далее применяем общую свёртку или шорткаты по \(\mathbf{f}_t\).

\bigskip\hrule\bigskip

\subsubsection*{4. Кэли–Гамильтон (особенно быстро для \(2\times2\))}
Если \(\chi_A(\lambda)=\lambda^2-(\operatorname{tr}A)\lambda+\det A\), то
\[
\boxed{\,A^{t+2}=(\operatorname{tr}A)A^{t+1}-(\det A)A^{t}\,}.
\]
Ищем \(A^t=\alpha_t A+\beta_t I\) и решаем скалярную рекурренту (начальные \(\alpha_0{=}0,\beta_0{=}1;\ \alpha_1{=}1,\beta_1{=}0\)).
Готовые формулы: при \(\lambda_{1}\neq\lambda_{2}\),
\(\ \alpha_t=\frac{\lambda_1^t-\lambda_2^t}{\lambda_1-\lambda_2},\
\beta_t=\frac{\lambda_1\lambda_2^t-\lambda_2\lambda_1^t}{\lambda_1-\lambda_2};\)
при \(\lambda_1=\lambda_2=\lambda\): \(\ \alpha_t=t\lambda^{t-1},\ \beta_t=(1-t)\lambda^t.\)

\bigskip\hrule\bigskip

\subsubsection*{5. Готовый полноценный пример (неоднородный, все шаги)}

\paragraph{Геометрическая правая часть без резонанса.}
\[
\mathbf{x}_{t+1}=A\mathbf{x}_t+3^t\mathbf{b},\qquad
A=\begin{pmatrix}4&1\\ 0&2\end{pmatrix},\ \ \mathbf{b}=\binom{0}{1},\ \ \mathbf{x}_0=\binom{x_0}{y_0}.
\]

\textbf{Метод 1 (быстрый шорткат «экспонента»).}
Пробуем частное решение вида \(\mathbf{x}^{(p)}_t=3^t\mathbf{y}\).
Подставляем:
\[
3^{t+1}\mathbf{y}=A(3^t\mathbf{y})+3^t\mathbf{b}
\ \Leftrightarrow\
(3I-A)\mathbf{y}=\mathbf{b}.
\]
Проверка нерезонансности: \(\det(3I-A)=\det\begin{psmallmatrix}-1&-1\\ 0&1\end{psmallmatrix}\neq0\).
Решаем \((3I-A)\mathbf{y}=\mathbf{b}\):
\[
\begin{pmatrix}-1&-1\\ 0&1\end{pmatrix}\binom{y_1}{y_2}=\binom{0}{1}
\ \Rightarrow\ y_2=1,\; -y_1-y_2=0\Rightarrow y_1=-1.
\]
Значит \(\ \boxed{\ \mathbf{y}=\binom{-1}{\ 1}\ }\) и
\[
\boxed{\ \mathbf{x}^{(p)}_t=3^t\binom{-1}{\ 1}\ }.
\]
\emph{Общий вид решения:} \(\mathbf{x}_t=A^t\mathbf{C}+3^t\binom{-1}{1}\).
\emph{Подбор \(\mathbf{C}\) по начальному условию:} \(\mathbf{x}_0=\mathbf{C}+1\cdot\binom{-1}{1}\Rightarrow \mathbf{C}=\mathbf{x}_0-\binom{-1}{1}\).
Итог:
\[
\boxed{\ \mathbf{x}_t=A^t\!\left(\mathbf{x}_0-\binom{-1}{1}\right)+3^t\binom{-1}{1}\ }.
\]

\textbf{Явный вид через \(A^t\).}
Здесь
\[
A^t=\begin{pmatrix}4^t&\dfrac{4^t-2^t}{2}\\[4pt]0&2^t\end{pmatrix}.
\]
Отсюда (покомпонентно)
\[
\boxed{
\begin{aligned}
x_t&=4^t\!\Big(x_0+1\Big)+\frac{4^t-2^t}{2}\,\Big(y_0-1\Big)-3^t,\\[2pt]
y_t&=2^t\,\Big(y_0-1\Big)+3^t.
\end{aligned}}
\]
\emph{Проверка:} подстановка в \(\mathbf{x}_{t+1}=A\mathbf{x}_t+3^t\mathbf{b}\) даёт тождества.
