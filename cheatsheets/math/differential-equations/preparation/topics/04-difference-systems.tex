\subsection{Системы разностных уравнений}\label{subsec:systems}

\HSEExample{\textbf{Пример.} Решите систему разностных уравнений: $\begin{cases} x_{t+1} = 4x_t + y_t \\ y_{t+1} = 2y_t \end{cases}$, с начальными условиями $\mathbf{x}_0 = \begin{pmatrix} 1 \\ 1 \end{pmatrix}$.}

\HSEDefinition{Система линейных разностных уравнений первого порядка с постоянными коэффициентами: $\mathbf{x}_{t+1} = A\mathbf{x}_t$, где задано $\mathbf{x}_0$, $A \in \mathbb{R}^{n \times n}$, $\mathbf{x}_t \in \mathbb{R}^n$. Цель: найти $\mathbf{x}_t = A^t\mathbf{x}_0$.}

\textbf{Идея решения:} возведение матрицы $A$ в степень $t$. Для этого используем спектральное разложение матрицы.

\textit{Обозначения:} $\lambda_i$ — собственные значения, $\mathbf{v}_i$ — собственные векторы, $\chi_A(\lambda) = \det(\lambda I - A)$ — характеристический многочлен.

\begin{table}[h!]
\centering
\caption{Выбор метода решения по типу собственных значений}
\label{tab:method-selection}
\begin{tabular}{|l|l|}
\hline
\textbf{Условия на собственные значения} & \textbf{Рекомендуемый метод} \\
\hline
Разные действительные корни, полный базис собственных векторов & Диагонализация \\
\hline
Повторный корень, недостаточно собственных векторов & Жорданова форма \\
\hline
Комплексно-сопряжённая пара & Реальный блок поворота \\
\hline
Матрица $2 \times 2$ (любой случай) & Кэли–Гамильтон \\
\hline
\end{tabular}
\end{table}

\bigskip
\hrule
\bigskip

\subsubsection*{1. Диагонализация}

\textbf{Условие применения:} матрица $A$ имеет $n$ линейно независимых собственных векторов (диагонализуема).

\textbf{Теорема.} Если $A$ диагонализуема, то $A = S\Lambda S^{-1}$, где $\Lambda = \text{diag}(\lambda_1, \ldots, \lambda_n)$ — диагональная матрица собственных значений, $S = [\mathbf{v}_1, \ldots, \mathbf{v}_n]$ — матрица собственных векторов.

\HSEAlgorithm{\textbf{Алгоритм диагонализации.}\begin{enumerate}\item Характеристический многочлен: $\chi_A(\lambda) = \det(\lambda I - A)$.\item Собственные значения: решить $\chi_A(\lambda) = 0$.\item Собственные векторы: для каждого $\lambda_i$ решить $(A - \lambda_i I)\mathbf{v}_i = \mathbf{0}$.\item Проверка диагонализуемости: $\det S \neq 0$.\item Диагонализация: $A = S\Lambda S^{-1}$.\item Возведение в степень: $A^t = S\Lambda^t S^{-1}$.\item Решение: $\mathbf{x}_t = A^t\mathbf{x}_0$.\end{enumerate}}

\HSEExample{\textbf{Пример.} Та же система. Решение: \; $A = \begin{pmatrix} 4 & 1 \\ 0 & 2 \end{pmatrix}$. Найдём собственные значения и векторы, построим $S,\Lambda, S^{-1}$ и получим $A^t = \begin{pmatrix} 4^t & \frac{4^t - 2^t}{2} \\ 0 & 2^t \end{pmatrix}$, откуда $\mathbf{x}_t = \begin{pmatrix} \frac{3 \cdot 4^t - 2^t}{2} \\ 2^t \end{pmatrix}$.}

\bigskip
\hrule
\bigskip

\subsubsection*{2. Жорданова форма (повторный корень)}

\textbf{Условие применения:} матрица $A$ имеет повторное собственное значение, но недостаточно собственных векторов для диагонализации.

\textbf{Теорема.} Если $A$ имеет единственное собственное значение $\lambda$ кратности $n$, то $A = \lambda I + N$, где $N$ — нильпотентная матрица ($N^m = 0$ для некоторого $m \leq n$).

\textbf{Ключевая идея:} используем биномиальную формулу для $(I + \lambda^{-1}N)^t$.

\HSEAlgorithm{\textbf{Алгоритм Жордановой формы.}\begin{enumerate}\item Собственное значение $\lambda$.\item Нильпотентная матрица $N = A - \lambda I$.\item Индекс нильпотентности: $N^m=0$.\item Формула: $A^t = \lambda^t \sum_{k=0}^{m-1} \binom{t}{k} (\lambda^{-1}N)^k$.\item Решение: $\mathbf{x}_t = A^t\mathbf{x}_0$.\end{enumerate}}

\HSEExample{\textbf{Пример.} Система $\begin{cases} x_{t+1} = 2x_t + y_t \\ y_{t+1} = 2y_t \end{cases}$. Для $A = \begin{pmatrix} 2 & 1 \\ 0 & 2 \end{pmatrix}$ имеем $\lambda=2$, $N=\begin{pmatrix}0 & 1 \\ 0 & 0\end{pmatrix}$, $m=2$, поэтому $A^t = 2^t\begin{pmatrix} 1 & t/2 \\ 0 & 1 \end{pmatrix}$ и $\mathbf{x}_t = 2^t \begin{pmatrix} 1 + t/2 \\ 1 \end{pmatrix}$.}

\bigskip
\hrule
\bigskip

\subsubsection*{3. Комплексная пара (реальный блок)}

\textbf{Условие применения:} матрица $A$ $2 \times 2$ имеет комплексно-сопряжённые собственные значения $\lambda = \rho e^{\pm i\theta}$.

\textbf{Теорема.} Для матрицы $A$ $2 \times 2$ с комплексными корнями $\lambda = \rho e^{\pm i\theta}$ справедливо:
\[
A^t = \rho^t \begin{pmatrix} \cos(t\theta) & -\sin(t\theta) \\ \sin(t\theta) & \cos(t\theta) \end{pmatrix}
\]

\textbf{Ключевая идея:} комплексные корни соответствуют повороту с масштабированием в вещественном пространстве.

\HSEAlgorithm{\textbf{Алгоритм для комплексной пары.}\begin{enumerate}\item Проверка: $(\operatorname{tr}A)^2 - 4\det A < 0$.\item $\rho = \sqrt{\det A}$, $\cos\theta = \frac{\operatorname{tr}A}{2\rho}$.\item $A^t = \rho^t \begin{pmatrix} \cos(t\theta) & -\sin(t\theta) \\ \sin(t\theta) & \cos(t\theta) \end{pmatrix}$.\item Решение: $\mathbf{x}_t = A^t\mathbf{x}_0$.\end{enumerate}}

\HSEExample{\textbf{Пример.} $A = \begin{pmatrix} 1 & -1 \\ 1 & 1 \end{pmatrix}$: $(\operatorname{tr}A)^2-4\det A = -4 < 0$, $\rho=\sqrt{2}$, $\theta=\pi/4$. Тогда $A^t = (\sqrt{2})^t R(t\theta)$ и $\mathbf{x}_t = (\sqrt{2})^t \begin{pmatrix} \cos\tfrac{\pi t}{4} \\ \sin\tfrac{\pi t}{4} \end{pmatrix}$.}

\bigskip
\hrule
\bigskip

\subsubsection*{4. Кэли–Гамильтон (универсальный метод)}

\textbf{Условие применения:} универсальный метод для матриц любого размера, особенно удобен для $2 \times 2$.

\textbf{Теорема Кэли–Гамильтона.} Матрица $A$ удовлетворяет своему характеристическому уравнению: $\chi_A(A) = 0$.

\textbf{Ключевая идея:} используем тождество $\chi_A(A) = 0$ для построения рекуррентного соотношения на степени матрицы.

\HSEAlgorithm{\textbf{Алгоритм Кэли–Гамильтона (2×2).}\begin{enumerate}\item $\chi_A(\lambda) = \lambda^2 - (\operatorname{tr}A)\lambda + \det A$.\item Рекуррентное: $A^{t+2} = (\operatorname{tr}A)A^{t+1} - (\det A)A^t$.\item Представление: $A^t = \alpha_t A + \beta_t I$.\item Решить на $\alpha_t,\beta_t$ и получить $\mathbf{x}_t$.\end{enumerate}}

\HSEExample{\textbf{Пример.} Для $A = \begin{pmatrix} 3 & 2 \\ -2 & -1 \end{pmatrix}$: $\chi_A(\lambda) = (\lambda-1)^2$, откуда $A^{t+2}=2A^{t+1}-A^t$, $A^t = tA + (1-t)I$, и $\mathbf{x}_t = \begin{pmatrix} 4t + 1 \\ 1 - 4t \end{pmatrix}$.}

\HSEBox{Общий алгоритм решения систем разностных уравнений}{\begin{enumerate}\item Анализ матрицы: $\operatorname{tr}A$, $\det A$, $\chi_A(\lambda)$.\item Выбор метода по типу спектра.\item Получить $A^t$ соответствующим методом.\item Решить $\mathbf{x}_t = A^t\mathbf{x}_0$.\item Проверка: $A^0=I$, $A^1=A$.\end{enumerate}}

\textbf{Полезные проверки:}
\begin{itemize}
\item \textbf{Начальные условия:} $A^0 = I$, $A^1 = A$.
\item \textbf{Жорданова форма:} если $A = \lambda I + N$, проверить $N^m = 0$.
\item \textbf{Комплексная пара:} $\det A = \rho^2$, $\text{tr}\,A = 2\rho\cos\theta$.
\item \textbf{Биномиальные коэффициенты:} не забыть $\binom{t}{k}$ в формуле Жордана.
\end{itemize}

% Ссылки на задачи: добавьте URL/refs сюда
% Например: Сборник И. Иванов, гл. 3; задачи 1-15.
