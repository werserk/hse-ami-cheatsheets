\subsection{Вводная информация и единый алгоритм}\label{sec:ode2-intro-algorithm}

Рассматриваем однородное линейное уравнение второго порядка
\[
y''+p(x)\,y'(x)+q(x)\,y(x)=0, \qquad p,q\in C(I).
\]

Два основных инструмента:

\begin{itemize}
\item \textbf{Снятие $y'$} заменой $y=\phi z$, где
\[
\phi(x)=\exp\!\Bigl(-\frac12\int p(x)\,dx\Bigr),
\quad
z \text{ удовлетворяет } z''+Q(x)z=0,
\]
\[
\boxed{\,Q(x)=q(x)-\frac{p'(x)}{2}-\frac{p(x)^2}{4}\,}.
\]

\item \textbf{Формула Абеля (вронскиан)}: если $y_1,y_2$ — решения,
\[
\boxed{\,W(x):=\det\!\begin{pmatrix}y_1&y_2\\y_1'&y_2'\end{pmatrix}
= W(x_0)\exp\!\Bigl(-\int_{x_0}^{x} p(t)\,dt\Bigr)\,}.
\]
Отсюда: $W\not\equiv0 \iff y_1,y_2$ фундаментальны.
\end{itemize}

\subsubsection{Единый алгоритм (5 шагов)}

\begin{enumerate}
\item \textbf{Цель.} Привести к стандартному виду и зафиксировать $p,q$.

    \textbf{Действие.} При необходимости разделить исходное уравнение на коэффициент при $y''$; выписать $p,q$.

\item \textbf{Цель.} Снять $y'$ и получить нормальную форму.

    \textbf{Действие.} Положить $\displaystyle \phi=\exp\!\big(-\tfrac12\int p\,dx\big)$, $y=\phi z$.
    Тогда $z''+Qz=0$ с $Q=q-\tfrac{p'}{2}-\tfrac{p^2}{4}$.

\item \textbf{Цель.} Посчитать/сравнить вронскиан (линейная независимость).

    \textbf{Действие.} По Абелю $\displaystyle W(x)=W(x_0)\exp\!\bigl(-\int_{x_0}^{x} p\bigr)$.
    Если даны $y_1(x_0),y_1'(x_0),y_2(x_0),y_2'(x_0)$, то $W(x_0)$ считаем мгновенно.

\item \textbf{Цель.} Сделать краткий качественный вывод.

    \textbf{Действие.}
    \begin{itemize}
    \item \emph{«$\le 1$ нуля на интервале»}: если на интервале $J$ выполнено $Q\le 0$, то у нетривиального решения $z$ не может быть двух нулей в $J$ (интегральный аргумент, см. ниже) $\Rightarrow$ у $y=\phi z$ также $\le1$ нуля.
    \item \emph{«Положительный максимум невозможен»}: если $q(x)<0$ на $I$, то локальный максимум решения $y$ на $I$ не может быть $>0$ (экстремум-тест).
    \end{itemize}

\item \textbf{Цель.} Записать компактный итог.

    \textbf{Действие.} Указать $Q(x)$, формулу $W(x)$ (или значение), и сформулировать требуемое свойство/классификацию.
\end{enumerate}
