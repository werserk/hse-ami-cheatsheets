\documentclass[10pt,landscape,a4paper]{article}
\usepackage[utf8]{inputenc}
\usepackage[russian]{babel}
\usepackage{amsmath}
\usepackage{amsfonts}
\usepackage{amssymb}
\usepackage{geometry}
\usepackage{multicol}
\usepackage{graphicx}
\usepackage{xcolor}
\usepackage{titlesec}
\usepackage{enumitem}
\usepackage{booktabs}
\usepackage{array}

% Настройка страницы
\geometry{margin=1cm}
\setlength{\columnsep}{1cm}
\setlength{\columnseprule}{0.5pt}

% Настройка заголовков
\titleformat{\section}{\Large\bfseries\color{blue!70!black}}{\thesection}{1em}{}
\titleformat{\subsection}{\large\bfseries\color{blue!50!black}}{\thesubsection}{1em}{}
\titleformat{\subsubsection}{\normalsize\bfseries\color{blue!30!black}}{\thesubsubsection}{1em}{}

% Убираем номера страниц
\pagestyle{empty}

% Настройка списков
\setlist[itemize]{leftmargin=*,noitemsep,topsep=0pt}
\setlist[enumerate]{leftmargin=*,noitemsep,topsep=0pt}

\begin{document}

% Заголовок
\begin{center}
    {\Huge\bfseries Дифференциальные Уравнения}\\[0.5cm]
    {\large Краткий справочник по методам решения}\\[0.3cm]
    \rule{\textwidth}{1pt}
\end{center}

\begin{multicols}{2}

\section{Основные определения}

\subsection{Типы уравнений}
\begin{itemize}
    \item \textbf{Обыкновенные ДУ:} $F(x, y, y', y'', \ldots, y^{(n)}) = 0$
    \item \textbf{Порядок:} наивысшая производная
    \item \textbf{Степень:} степень старшей производной
    \item \textbf{Линейные:} $a_n(x)y^{(n)} + \ldots + a_0(x)y = f(x)$
\end{itemize}

\subsection{Начальные условия}
\begin{align}
    y(x_0) &= y_0 \\
    y'(x_0) &= y_1 \\
    &\vdots \\
    y^{(n-1)}(x_0) &= y_{n-1}
\end{align}

\section{Уравнения первого порядка}

\subsection{Разделяющиеся переменные}
\begin{align}
    \frac{dy}{dx} &= f(x)g(y) \\
    \int \frac{dy}{g(y)} &= \int f(x) dx + C
\end{align}

\subsection{Однородные уравнения}
\begin{align}
    \frac{dy}{dx} &= f\left(\frac{y}{x}\right) \\
    \text{Замена: } z &= \frac{y}{x}, \quad y = zx
\end{align}

\subsection{Линейные уравнения}
\begin{align}
    y' + p(x)y &= q(x) \\
    \text{Решение: } y &= e^{-\int p(x)dx}\left[\int q(x)e^{\int p(x)dx}dx + C\right]
\end{align}

\subsection{Уравнение Бернулли}
\begin{align}
    y' + p(x)y &= q(x)y^n \quad (n \neq 0, 1) \\
    \text{Замена: } z &= y^{1-n}
\end{align}

\section{Уравнения второго порядка}

\subsection{Понижение порядка}
\begin{itemize}
    \item $F(x, y', y'') = 0$: замена $z = y'$
    \item $F(y, y', y'') = 0$: замена $z = y'$, $y'' = z\frac{dz}{dy}$
\end{itemize}

\subsection{Линейные с постоянными коэффициентами}
\begin{align}
    ay'' + by' + cy &= 0 \\
    \text{Характеристическое: } ar^2 + br + c &= 0
\end{align}

\subsubsection{Случаи корней}
\begin{itemize}
    \item \textbf{Разные вещественные:} $y = C_1e^{r_1x} + C_2e^{r_2x}$
    \item \textbf{Кратные:} $y = (C_1 + C_2x)e^{rx}$
    \item \textbf{Комплексные:} $y = e^{\alpha x}(C_1\cos\beta x + C_2\sin\beta x)$
\end{itemize}

\section{Системы уравнений}

\subsection{Система линейных ДУ}
\begin{align}
    \frac{dx}{dt} &= a_{11}x + a_{12}y \\
    \frac{dy}{dt} &= a_{21}x + a_{22}y
\end{align}

\subsection{Метод решения}
\begin{enumerate}
    \item Найти собственные значения матрицы
    \item Найти собственные векторы
    \item Записать общее решение
\end{enumerate}

\section{Численные методы}

\subsection{Метод Эйлера}
\begin{align}
    y_{n+1} &= y_n + h \cdot f(x_n, y_n) \\
    x_{n+1} &= x_n + h
\end{align}

\subsection{Метод Рунге-Кутта 4-го порядка}
\begin{align}
    k_1 &= h \cdot f(x_n, y_n) \\
    k_2 &= h \cdot f(x_n + \frac{h}{2}, y_n + \frac{k_1}{2}) \\
    k_3 &= h \cdot f(x_n + \frac{h}{2}, y_n + \frac{k_2}{2}) \\
    k_4 &= h \cdot f(x_n + h, y_n + k_3) \\
    y_{n+1} &= y_n + \frac{1}{6}(k_1 + 2k_2 + 2k_3 + k_4)
\end{align}

\section{Специальные функции}

\subsection{Функции Бесселя}
\begin{align}
    x^2y'' + xy' + (x^2 - n^2)y &= 0 \\
    J_n(x) &= \sum_{m=0}^{\infty} \frac{(-1)^m}{m!(m+n)!}\left(\frac{x}{2}\right)^{2m+n}
\end{align}

\subsection{Функции Лежандра}
\begin{align}
    (1-x^2)y'' - 2xy' + n(n+1)y &= 0 \\
    P_n(x) &= \frac{1}{2^n n!}\frac{d^n}{dx^n}[(x^2-1)^n]
\end{align}

\section{Практические примеры}

\subsection{Пример 1: Разделяющиеся переменные}
\begin{align}
    \frac{dy}{dx} &= \frac{x}{y} \\
    y dy &= x dx \\
    \frac{y^2}{2} &= \frac{x^2}{2} + C \\
    y^2 &= x^2 + C_1
\end{align}

\subsection{Пример 2: Линейное уравнение}
\begin{align}
    y' + 2y &= e^{-x} \\
    \text{Интегрирующий множитель: } \mu &= e^{2x} \\
    y &= e^{-2x}\left[\int e^{2x} \cdot e^{-x} dx + C\right] \\
    y &= e^{-2x}\left[e^x + C\right] = e^{-x} + Ce^{-2x}
\end{align}

\section{Полезные формулы}

\subsection{Производные}
\begin{align}
    \frac{d}{dx}(e^{ax}) &= ae^{ax} \\
    \frac{d}{dx}(\ln x) &= \frac{1}{x} \\
    \frac{d}{dx}(\sin x) &= \cos x \\
    \frac{d}{dx}(\cos x) &= -\sin x
\end{align}

\subsection{Интегралы}
\begin{align}
    \int e^{ax} dx &= \frac{e^{ax}}{a} + C \\
    \int \frac{1}{x} dx &= \ln|x| + C \\
    \int \sin x dx &= -\cos x + C \\
    \int \cos x dx &= \sin x + C
\end{align}

\end{multicols}

\end{document}
