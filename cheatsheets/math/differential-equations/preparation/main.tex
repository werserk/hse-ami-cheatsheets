\documentclass[11pt,a4paper]{article}
\usepackage[utf8]{inputenc}
\usepackage[russian]{babel}
\usepackage{amsmath, amsfonts, amssymb}
\usepackage{geometry}
\usepackage{parskip}
\usepackage{hyperref}
\usepackage{fancyhdr}

% Цвета ссылок (чёрные)
\usepackage{xcolor}
\hypersetup{colorlinks=true, linkcolor=black, urlcolor=black, citecolor=black}

\geometry{margin=1.8cm}
\pagestyle{fancy}
\fancyhf{}
\fancyfoot[R]{\thepage}
\renewcommand{\headrulewidth}{0pt}

% --- Каталог тем для этого предмета ---
\newcommand{\topicsBase}{topics}

\begin{document}

\begin{center}
  {\Huge Подготовка: Дифференциальные уравнения}\\[0.4cm]
  {\large Полная версия с разборами тем и ссылками}\\[0.2cm]
  \rule{\textwidth}{0.6pt}
\end{center}

\vspace{0.5cm}

% --- Оглавление ---
\begin{center}
{\Large \textbf{Содержание}}
\end{center}

\vspace{0.5cm}

\begin{center}
\begin{tabular}{ll}
\hyperref[sec:homogeneous]{\textbf{1. Однородные линейные разностные уравнения}} & \pageref{sec:homogeneous} \\
\end{tabular}
\end{center}

\vspace{0.5cm}

% --- Подключайте темы ниже ---
% Примеры (создайте соответствующие файлы в preparation/topics/):
\section{Однородные линейные разностные уравнения}\label{sec:homogeneous}

\begin{center}
\fbox{\parbox{0.9\textwidth}{
\textbf{Пример.} Решите однородное линейное разностное уравнение:
\begin{align}
y_{t+3} - 3y_{t+2} + 6y_{t+1} - 4y_t = 0
\end{align}
}}
\end{center}

\textbf{Определение.} Линейное однородное разностное уравнение порядка $k$ с постоянными коэффициентами:
\begin{align}
a_t + c_1 a_{t-1} + c_2 a_{t-2} + \dots + c_k a_{t-k} = 0, \quad c_k \neq 0
\end{align}

Пара «уравнение + $k$ начальных условий» задаёт единственное решение.

\textbf{Идея решения -- метод характеристических корней.} Полагаем $a_t = r^t$ $\Rightarrow$
\begin{align}
r^t (1 + c_1 r^{-1} + c_2 r^{-2} + \dots + c_k r^{-k}) = 0 \iff r^k + c_1 r^{k-1} + \dots + c_k = 0
\end{align}

т.е. характеристический многочлен $\chi(r) = r^k + c_1 r^{k-1} + \dots + c_k$. Его корни целиком описывают форму общего решения.

\begin{table}[h!]
\centering
\caption{Выбор формы решения по типу корней характеристического многочлена}
\label{tab:form-choices}
\begin{tabular}{|l|l|}
\hline
\textbf{Условия на корни} & \textbf{Вклад в решение} \\
\hline
Простой действительный корень $r\in\mathbb{R}$ &
$\alpha\, r^{\,t}$ \\
\hline
Действительный корень $r$ кратности $m\ge 2$ &
$(\alpha_{0}+\alpha_{1}t+\cdots+\alpha_{m-1}t^{m-1})\, r^{\,t}$ \\
\hline
Комплексно-сопряжённая пара $r,\overline{r}=\rho e^{\pm i\theta}$ кратности $m$ &
$\displaystyle \sum_{\ell=0}^{m-1} t^{\ell}\,\rho^{\,t}\big(A_{\ell}\cos(\theta t)+B_{\ell}\sin(\theta t)\big)$ \\
\hline
\end{tabular}

\vspace{0.5em}
\emph{Итоговое общее решение} — сумма вкладов по всем корням/парам; сумма кратностей равна порядку $k$.
\end{table}

\textbf{Подгонка под начальные условия.} Подставляем $t = 0, 1, \dots, k - 1$ в общий вид, решаем линейную систему на $\alpha$-коэффициенты.

\begin{center}
\fbox{\parbox{0.9\textwidth}{
\textbf{Алгоритм.}
\begin{enumerate}
\item \textbf{Нормализация.} Приведите уравнение к виду $a_t+\sum_{j=1}^{k} c_j a_{t-j}=0$, $c_k\neq 0$.
\item \textbf{Характеристический многочлен.} $\chi(r)=r^{k}+c_1 r^{k-1}+\dots+c_k$.
\item \textbf{Корни и кратности.} Найдите корни $r$ и их кратности $m$ (\,$\sum m = k$\,).
\item \textbf{Общий вид решения (см.~табл.~\ref{tab:form-choices}).}
Для каждого корня/пары возьмите соответствующий вклад из таблицы и сложите их.
\item \textbf{Подгонка под начальные условия.} Подставьте $k$ заданных значений подряд и решите линейную систему для постоянных.
\end{enumerate}
}}
\end{center}

% Ссылки на задачи: добавьте URL/refs сюда
% Например: Сборник И. Иванов, гл. 2; задачи 1-20.



\section{Неоднородные линейные разностные уравнения}\label{sec:inhomogeneous}

\begin{center}
\fbox{\parbox{0.9\textwidth}{
\textbf{Пример.} Решите неоднородное линейное разностное уравнение:
\begin{align}
y_{t+3} - 3y_{t+2} + 6y_{t+1} - 4y_t = 2^t + t
\end{align}
}}
\end{center}

\textbf{Определение.} Линейное неоднородное разностное уравнение порядка $k$ с постоянными коэффициентами:
\begin{align}
a_t + c_1 a_{t-1} + c_2 a_{t-2} + \dots + c_k a_{t-k} = f(t), \quad c_k \neq 0
\end{align}

где $f(t)$ — заданная функция (неоднородность).

\textbf{Структура общего решения:} $a_t = a_t^{(h)} + a_t^{(p)}$, где:
\begin{itemize}
\item $a_t^{(h)}$ — общее решение однородного уравнения (см. раздел~\ref{sec:homogeneous})
\item $a_t^{(p)}$ — частное решение неоднородного уравнения
\end{itemize}

\textbf{Метод неопределённых коэффициентов для $a_t^{(p)}$.}

Пусть характеристический многочлен однородного уравнения:
\[
\chi(r) = r^k + c_1 r^{k-1} + \dots + c_k
\quad \text{и} \quad
\chi(r) = \prod_i (r-r_i)^{m_i} \prod_\ell Q_{\rho_\ell, \theta_\ell}(r)^{s_\ell},
\]
где
\[
Q_{\rho,\theta}(r) = (r - \rho e^{i\theta})(r - \rho e^{-i\theta}) = r^2 - 2\rho \cos\theta \, r + \rho^2.
\]

\textbf{Правило «множитель $\to$ вклад» (однородная часть):}

\begin{itemize}
\item Линейный $(r-r_0)^m \;\Rightarrow\; \sum_{j=0}^{m-1} \alpha_j t^j r_0^t.$

\item Квадратный $Q_{\rho,\theta}(r)^s \;\Rightarrow\;
\rho^t \Bigl( \sum_{j=0}^{s-1} t^j \bigl(A_j \cos(\theta t) + B_j \sin(\theta t)\bigr)\Bigr).$
\end{itemize}

\textbf{Итог:} $a_t^{(h)}$ — сумма всех таких вкладов по всем множителям $\chi$.

\textbf{Выбор формы частного решения $a_t^{(p)}$:}

\textit{Обозначения:} $P_n(t)$ — полином степени $n$; $Q_n(t), R_n(t)$ — полиномы; $\lambda\in\mathbb{C}$; $s$ — кратность резонанса (кратность соответствующего множителя в $\chi$).

\begin{table}[h!]
\centering
\caption{Выбор формы частного решения и проверка резонанса}
\label{tab:particular-form}
\begin{tabular}{|l|l|l|}
\hline
\textbf{Неоднородность $f(t)$} & \textbf{Проверка резонанса} & \textbf{Базовая форма $a_t^{(p)}$} \\
\hline
$P_n(t)\,\lambda^t$ & $\chi(\lambda)=0?$ & $Q_n(t)\,\lambda^t$ \\
\hline
$\rho^t\cos(\theta t)$, $\rho^t\sin(\theta t)$ & $Q_{\rho,\theta}(r)\mid \chi(r)?$ & $\rho^t\big(A\cos(\theta t)+B\sin(\theta t)\big)$ \\
\hline
$P_n(t)\,\rho^t\cos(\theta t)$ (или $\sin$) & $Q_{\rho,\theta}(r)\mid \chi(r)?$ & $\rho^t\big(Q_n(t)\cos(\theta t)+R_n(t)\sin(\theta t)\big)$ \\
\hline
Чистый полином $P_n(t)$ & $\chi(1)=0?$ & $Q_n(t)$ \\
\hline
\end{tabular}
\end{table}

\textbf{Правило резонанса:} если проверка даёт резонанс кратности $s$, домножьте базовую форму на $t^{s}$.

\begin{center}
\fbox{\parbox{0.9\textwidth}{
\textbf{Алгоритм решения неоднородного уравнения.}
\begin{enumerate}
\item \textbf{Однородная часть.} Найти $a_t^{(h)}$ методом характеристических корней (см. раздел~\ref{sec:homogeneous}).
\item \textbf{Форма частного решения.} По таблице~\ref{tab:particular-form} выбрать форму $a_t^{(p)}$ с учётом правила резонанса.
\item \textbf{Подстановка.} Подставить $a_t^{(p)}$ в исходное неоднородное уравнение и найти неопределённые коэффициенты.
\item \textbf{Общее решение.} $a_t = a_t^{(h)} + a_t^{(p)}$.
\item \textbf{Начальные условия.} Подставить $k$ заданных значений и найти константы в $a_t^{(h)}$.
\end{enumerate}
}}
\end{center}

% \input{\topicsBase/03-systems}
% \input{\topicsBase/04-numerical}

% Команда для отображения by werserk (чёрный цвет, без ссылки)
\newcommand{\byWerserk}{\textbf{by werserk}}

% Футер с разделительной линией (только на последней странице)
\thispagestyle{fancy}
\fancyfoot[L]{\byWerserk}
\fancyfoot[R]{\thepage}
\vspace{1cm}
\rule{\textwidth}{0.4pt}
\vspace{0.3cm}
{\noindent \byWerserk}

\end{document}


