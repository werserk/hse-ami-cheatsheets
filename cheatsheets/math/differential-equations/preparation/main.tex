\documentclass[11pt,a4paper]{article}
\usepackage[utf8]{inputenc}
\usepackage[russian]{babel}
\usepackage{amsmath, amsfonts, amssymb}
\usepackage{geometry}
\usepackage{parskip}
\usepackage{hyperref}
\usepackage{fancyhdr}

% Цвета ссылок (чёрные)
\usepackage{xcolor}
\hypersetup{colorlinks=true, linkcolor=black, urlcolor=black, citecolor=black}

\geometry{margin=1.8cm}
\pagestyle{fancy}
\fancyhf{}
\fancyfoot[R]{\thepage}
\renewcommand{\headrulewidth}{0pt}

% --- Каталог тем для этого предмета ---
\newcommand{\topicsBase}{topics}

\begin{document}

\begin{center}
  {\Huge Подготовка: Дифференциальные уравнения}\\[0.4cm]
  {\large Полная версия с разборами тем и ссылками}\\[0.2cm]
  \rule{\textwidth}{0.6pt}
\end{center}

\vspace{0.5cm}

% --- Оглавление ---
\begin{center}
{\Large \textbf{Содержание}}
\end{center}

\vspace{0.5cm}

\begin{center}
\begin{tabular}{ll}
\hyperref[sec:homogeneous]{\textbf{1. Однородные линейные разностные уравнения}} & \pageref{sec:homogeneous} \\
\end{tabular}
\end{center}

\vspace{0.5cm}

% --- Подключайте темы ниже ---
% Примеры (создайте соответствующие файлы в preparation/topics/):
\section{Уравнения первого порядка}
Ключевые типы: разделяющиеся, однородные, линейные, Бернулли.

\subsection{Разделяющиеся переменные}
\begin{align}
\frac{dy}{dx} = f(x)g(y) \quad \Rightarrow \quad \int \frac{dy}{g(y)} = \int f(x)dx + C
\end{align}

\subsection{Линейные уравнения}
\begin{align}
y' + p(x)y = q(x),\quad \mu = e^{\int p(x)dx}
\end{align}

\subsection{Бернулли}
\begin{align}
y' + p(x)y = q(x) y^n,\quad z = y^{1-n}
\end{align}

% Ссылки на задачи: добавьте URL/refs сюда
% Например: Сборник И. Иванов, гл. 2; задачи 1-20.



\section{Уравнения второго порядка}
\subsection{Постоянные коэффициенты}
\begin{align}
ay'' + by' + cy = 0; \quad ar^2 + br + c = 0
\end{align}

\subsection{Понижение порядка}
Идеи замен: $z = y'$, либо $y'' = z\,dz/dy$.



% \input{\topicsBase/03-systems}
% \input{\topicsBase/04-numerical}

% Команда для отображения by werserk (чёрный цвет, без ссылки)
\newcommand{\byWerserk}{\textbf{by werserk}}

% Футер с разделительной линией (только на последней странице)
\thispagestyle{fancy}
\fancyfoot[L]{\byWerserk}
\fancyfoot[R]{\thepage}
\vspace{1cm}
\rule{\textwidth}{0.4pt}
\vspace{0.3cm}
{\noindent \byWerserk}

\end{document}


