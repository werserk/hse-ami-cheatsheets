\documentclass[12pt]{article}
\usepackage[utf8]{inputenc}
\usepackage[russian]{babel}
\usepackage{amsmath,amssymb}
\usepackage{enumitem}
\usepackage{geometry}
\geometry{margin=2.2cm}

\begin{document}

\section*{План тем (1--12)}

\begin{enumerate}[leftmargin=*,label=\textbf{\arabic*.}]
\item Разностные линейные уравнения с постоянными коэффициентами (ЛОС) и неоднородностью:
      характеристический многочлен, частное решение для $f(t)$ вида $\alpha^t$, полином, синусы/косинусы, правило резонанса ($t^s$).

\item Синтез разностного уравнения по заданным частным решениям:
      восстановление минимального порядка по корням характеристического многочлена (в т.ч. комплексные парами).

\item Системы разностных уравнений:
      фундаментальная матрица $\Phi_t=A^t$, спектральный/Jordan разбор, неоднородный случай (дискретная вариация постоянных).

\item Однородные ПЧП первого порядка:
      метод характеристик, поиск двух независимых инвариантов $I_1,I_2$, общее решение $u=F(I_1,I_2)$.

\item ПЧП первого порядка с задачей Коши:
      нехарактеристичность начповерхности, построение $F$ по данным, локальная разрешимость/единственность.

\item Первые интегралы в 3D-системах ОДУ:
      быстрые инварианты (например, для подсистемы $\dot y=z,\ \dot z=-y$), поиск двух независимых интегралов и их проверка.

\item Нелинейные 2D-системы:
      равновесия, линеаризация (матрица Якоби), классификация по $\operatorname{tr}J,\det J$, набросок фазового портрета.

\item Вращающиеся системы и полярные координаты:
      переход $(x,y)\mapsto (r,\theta)$, уравнения на $\dot r,\dot\theta$, роль параметра $a$ и классификация траекторий.

\item Линейные ОДУ второго порядка:
      фундаментальная система, вронскиан и формула Абеля, «детектор линейности» постановки (корректность требований).

\item Приведение к нормальной форме $z''+q(x)z=0$:
      выбор подстановки $y=u\phi(x)$, снятие первого производного, оценки нулей решений (идеи теории Штурма).

\item Периодические коэффициенты:
      $y''+q(x)y=0$, $q(x+T)=q(x)$; сдвиг-оператор, монодромия, мультипликаторы Флоке, следствия для нулей/роста.

\item Механические системы и устойчивость:
      $\ddot{\mathbf x}=-\nabla V(\mathbf x)$, энергия как первый интеграл, устойчивость минимума потенциала (Ляпунов по $V$).
\end{enumerate}

\end{document}
