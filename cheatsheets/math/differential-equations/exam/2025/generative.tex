\documentclass[12pt]{article}
\usepackage[utf8]{inputenc}
\usepackage[russian]{babel}
\usepackage{amsmath,amssymb,mathtools}
\usepackage{geometry}
\geometry{margin=2.3cm}

\begin{document}

\begin{center}
\Large\textbf{Тренировочные задачи по темам M1--M13}\\[2mm]
\small (по 3 задачи на тему; уровень: экзамен и выше)
\end{center}

% ===================== M1 =====================
\section*{M1. Разностные ЛОС с постоянными коэффициентами (неоднородность)}
\begin{enumerate}
\item Найдите общее решение
\[
y_{t+4}-2y_{t+3}-y_{t+2}+2y_{t+1}=3\cdot 2^{t}+(t^2-1)(-1)^t+5.
\]
Укажите, какие слагаемые правой части требуют сдвига степени (резонанс), и какого именно.

\item Решите с начальными условиями $y_0=1,\ y_1=0,\ y_2=2,\ y_3=3$:
\[
y_{t+5}+y_{t+4}-6y_{t+3}-6y_{t+2}+8y_{t+1}+8y_t
=2^{t}\cos\frac{\pi t}{2}+t\,3^{t}.
\]

\item Найдите общее решение
\[
y_{t+3}-3y_{t+2}+3y_{t+1}-y_t
= (t^2+4)\cdot 1^t + t\,(-2)^t.
\]
(Корень $r=1$ имеет кратность $3$; аккуратно обработайте резонанс с полиномом.)
\end{enumerate}

% ===================== M2 =====================
\section*{M2. Синтез разностного уравнения по заданным решениям}
\begin{enumerate}
\item Постройте линейное однородное разностное уравнение минимального порядка, частными решениями которого являются
\[
y_t^{(1)}=2^t,\qquad
y_t^{(2)}=t\,2^t,\qquad
y_t^{(3)}=(-2)^t\sin\!\frac{\pi t}{3}.
\]

\item Найдите минимальное ЛОС, для которого все функции
\[
3^t,\quad t\,3^t,\quad 2^t\cos\!\frac{\pi t}{4},\quad 2^t\sin\!\frac{\pi t}{4}
\]
являются решениями. Укажите его порядок и характеристический многочлен.

\item Постройте уравнение минимального порядка, имеющее решения
\[
(-1)^t,\quad t(-1)^t,\quad t^2(-1)^t,\quad 5^t.
\]
Поясните, какая кратность у соответствующих корней характеристического многочлена.
\end{enumerate}

% ===================== M3 =====================
\section*{M3. Нелинейные 2D-системы: равновесия, линеаризация, фазовый портрет (гиперболика)}
\begin{enumerate}
\item Исследуйте систему
\[
\dot x = y - x(1-x-y),\qquad
\dot y = -x + y(2-x).
\]
Найдите все равновесия, классифицируйте их по $\operatorname{tr}J$ и $\det J$, набросайте локальные фазовые портреты.

\item Для параметризованной системы
\[
\dot x = a\,y + x(x^2+y^2-1),\qquad
\dot y = -a\,x + y(x^2+y^2-1),
\]
классифицируйте начало координат в зависимости от $a\in\mathbb{R}$ и опишите типы траекторий в окрестности.

\item Исследуйте
\[
\dot x = (y-1)(x+2)-xy,\qquad
\dot y = (x+1)(y-2)-xy.
\]
Найдите равновесия, типы и локальные эскизы.
\end{enumerate}

% ===================== M4 =====================
\section*{M4. Линейные ОДУ второго порядка: снятие $y'$, вронскиан, нули}
\begin{enumerate}
\item На $x>0$ рассмотрите
\[
y''+\frac{2}{x}y' -\Bigl(\frac{5}{x^2}+1\Bigr)y=0.
\]
(a) Приведите к $z''+Q(x)z=0$. (б) Выведите формулу для $W(x)$ через $W(1)$. (в) Докажите, что всякое нетривиальное решение имеет не более одного нуля на $(0,\infty)$.

\item Рассмотрите
\[
y''+4y'+\bigl(3+e^{-x}\bigr)y=0.
\]
(a) Снимите $y'$. (б) Найдите $W(x)$ через $W(0)$. (в) Покажите, что нетривиальное решение имеет не более одного нуля на $\mathbb{R}$.

\item Эйлера--Коши:
\[
x^2y''+\alpha x y'+\beta y=0,\qquad x>0.
\]
(a) Снимите $y'$ общей формулой. (б) Выразите $W(x)$ через $W(x_0)$. (в) Укажите условия на $(\alpha,\beta)$, гарантирующие «не более одного нуля» на $(0,\infty)$.
\end{enumerate}

% ===================== M5 =====================
\section*{M5. ПЧП первого порядка: общее решение $u=F(I_1,I_2)$}
\begin{enumerate}
\item Найдите общее решение
\[
(x+y)\,u_x+(2y-x)\,u_y+0\cdot u_z=0.
\]

\item Найдите общее решение
\[
x\,u_x+y\,u_y+(x+y)\,z\,u_z=0.
\]

\item Найдите общее решение
\[
(2xy)\,u_x+(y^2-x^2)\,u_y+(x-y)\,u_z=0.
\]
(Подсказка: начните с подстановки $v=y/x$ для пары $(x,y)$.)
\end{enumerate}

% ===================== M6 =====================
\section*{M6. Первые интегралы в 3D-системах (поиск двух независимых)}
\begin{enumerate}
\item Система
\[
\dot x = yz,\qquad \dot y = zx,\qquad \dot z = xy.
\]
Найдите два независимых первых интеграла и опишите траектории в непустой области $x^2\ne y^2\ne z^2$.

\item Система
\[
\dot x = y^2-z^2,\qquad \dot y = xz,\qquad \dot z = xy.
\]
Найдите два независимых первых интеграла и опишите интегральные поверхности.

\item Система
\[
\dot x = y+z,\qquad \dot y = z+x,\qquad \dot z = x+y.
\]
Найдите два независимых первых интеграла (подсказка: диагонализуйте линейную часть и используйте комбинации координат).
\end{enumerate}

% ===================== M7 =====================
\section*{M7. Нелинейные 2D: равновесия, линеаризация, портрет (шире набора M3)}
\begin{enumerate}
\item Исследуйте
\[
\dot x = x(1-x)-y(1+y),\qquad
\dot y = 2x -3y + xy.
\]
Полный набор равновесий, типы, локальные эскизы.

\item С параметром $b$:
\[
\dot x = y - x(x^2+b),\qquad
\dot y = -x - y(y^2+b).
\]
Определите тип начала координат и режимы при $b<0$, $b=0$, $b>0$.

\item Исследуйте
\[
\dot x = (x-y)(1-xy),\qquad
\dot y = (x+y)(1+x^2).
\]
Определите все равновесия, их характер, локальные эскизы.
\end{enumerate}

% ===================== M8 =====================
\section*{M8. Переход в полярные: вращающиеся/радиальные системы, интегрирование}
\begin{enumerate}
\item Для
\[
\dot x = a\,y + x(x^2+y^2-1),\qquad
\dot y = -a\,x + y(x^2+y^2-1),
\]
перейдите в полярные и исследуйте уравнения на $\dot r,\dot\theta$; опишите типы траекторий при $a\ne0$.

\item Для
\[
\dot x = x(1-r^2)+\omega y,\qquad
\dot y = y(1-r^2)-\omega x,\quad r^2=x^2+y^2,
\]
в полярных найдите стационарные радиусы и опишите динамику по $r$ и $\theta$.

\item Для
\[
\dot x = (r^2-2)x + \Omega y,\qquad
\dot y = (r^2-2)y - \Omega x,\quad r^2=x^2+y^2,
\]
решите радиальное уравнение и классифицируйте траектории в зависимости от стартового $r(0)$.
\end{enumerate}

% ===================== M9 =====================
\section*{M9. Линейные ОДУ-2: фундаментальная система, вронскиан, корректность}
\begin{enumerate}
\item Пусть $y_1,y_2$ --- решения
\[
y''+\Bigl(\frac{2}{x}+e^{-x}\Bigr)y' + \Bigl(1+\frac{1}{x^2}\Bigr)y=0,\quad x>0,
\]
где $y_1(1)=0,\ y_1'(1)=1,\ y_2(1)=1,\ y_2'(1)=0$. Составляют ли они фундаментальную систему на $(0,\infty)$? Найдите $W(x)$.

\item Рассмотрите
\[
y''+p(x)y'+q(x)y=0,\qquad p,q\in C[0,1].
\]
Пусть $y_1(0)=0,\ y_1'(0)=1$ и $y_2(0)=1,\ y_2'(0)=\alpha$. Для каких $\alpha$ пара $(y_1,y_2)$ фундаментальна на $[0,1]$? Выразите $W(1)$ через $\alpha$ и $p$.

\item Для
\[
y''+4y'+(3+\sin x)\,y=0
\]
покажите, что любые два решения, линейно независимые в одной точке, остаются независимыми на всей $\mathbb{R}$. Найдите явную формулу для $W(x)$.
\end{enumerate}

% ===================== M10 =====================
\section*{M10. Приведение к $z''+q(x)z=0$ заменою $y=u\phi(x)$; оценки нулей}
\begin{enumerate}
\item Для
\[
y''+\frac{2}{x}y'+\Bigl(\frac{1}{3}x^2-\frac{7}{x^2}\Bigr)y=0,\quad x>0,
\]
подберите $\phi(x)$, приведите к $z''+Q(x)z=0$ и докажите, что нетривиальное решение имеет не более одного нуля на $(0,\infty)$.

\item Для
\[
y''+2\tanh x\, y'+\bigl(1-\operatorname{sech}^2 x\bigr)y=0,
\]
снимите $y'$, вычислите $Q(x)$ и сделайте вывод о числе нулей.

\item На $x>0$:
\[
x^2y''+(\alpha+1)xy'+\Bigl(\beta-\frac{\gamma}{x^2}\Bigr)y=0.
\]
Приведите к нормальной форме и укажите условия на параметры для «$\le1$ нуля».
\end{enumerate}

% ===================== M11 =====================
\section*{M11. Периодические коэффициенты, монодромия (Флоке)}
\begin{enumerate}
\item Пусть $q\in C(\mathbb{R})$, $q(x+T)=q(x)$, и $y$ --- нетривиальное решение $y''+q(x)y=0$ с $y(0)=y(T)=0$. Докажите существование $C\ne0$ такое, что $y(x+T)=C\,y(x)$; выразите $C$ через $y'(T)$ и $y'(0)$.

\item Для
\[
y''+\bigl(2+\cos x\bigr)y=0
\]
рассмотрите фундаментальную матрицу за период $2\pi$ и решите задачу о виде множителей Флоке (без их численного значения).

\item Пусть $q\in C(\mathbb{R})$, $q(x+\pi)=q(x)$ и $y$ --- решение $y''+q(x)y=0$ с $y(0)=0$, $y'(\pi)=0$. Покажите, что $y(x+\pi)=-y(x)$ либо $y(x+\pi)=y(x)$; обсудите, когда возможно каждое из двух.
\end{enumerate}

% ===================== M12 =====================
\section*{M12. Механические системы и устойчивость по Ляпунову через потенциал}
\begin{enumerate}
\item Пусть частица движется по $\ddot{\mathbf x}=-\nabla V(\mathbf x)$, $V\in C^2$, причём $V(\mathbf 0)=0$, $V(\mathbf x)>0$ при $\mathbf x\ne 0$, а $\nabla^2 V(\mathbf 0)$ положительно определена. Докажите устойчивость равновесия в $\mathbf 0$ и обсудите, почему асимптотическая устойчивость вообще невозможна без диссипации.

\item Рассмотрите $V(x,y)=\tfrac14(x^2+y^2-1)^2+\varepsilon xy$ при малом $|\varepsilon|$. Найдите все равновесия и классифицируйте их по характеру (минимум/седло) для $\varepsilon$ в окрестности нуля.

\item Для системы с малой линейной вязкостью
\[
\ddot{\mathbf x}+\gamma \dot{\mathbf x} = -\nabla V(\mathbf x),\qquad \gamma>0,
\]
покажите, что при $V(\mathbf x)\ge c\|\mathbf x\|^2$ вблизи нуля равновесие асимптотически устойчиво. Укажите функцию Ляпунова.
\end{enumerate}

% ===================== M13 =====================
\section*{M13. Системы разностных уравнений: вариация постоянных, задача Коши}
\begin{enumerate}
\item Решите задачу Коши
\[
\begin{cases}
\mathbf x_{t+1}=A\,\mathbf x_t+\mathbf b\,2^t,\quad
A=\begin{pmatrix}3&-1\\2&0\end{pmatrix},\ \ \mathbf b=\begin{pmatrix}1\\-1\end{pmatrix},\\[2mm]
\mathbf x_0=\begin{pmatrix}0\\1\end{pmatrix}.
\end{cases}
\]
Найдите $A^t$ и примените дискретную вариацию постоянных.

\item Пусть
\[
\mathbf x_{t+1}=A\,\mathbf x_t,\quad
A=\begin{pmatrix}2&1&0\\0&2&1\\0&0&2\end{pmatrix}.
\]
(a) Постройте фундаментальную матрицу $\Phi_t=A^t$. (б) Запишите общее решение. (в) Обсудите рост норм решения и минимальный полином $A$.

\item Неоднородная система
\[
\mathbf x_{t+1}=\tfrac12
\begin{pmatrix}
1&3\\
3&1
\end{pmatrix}\mathbf x_t+\begin{pmatrix}(-1)^t\\ t\,2^t\end{pmatrix},\qquad \mathbf x_0=\mathbf 0.
\]
(а) Найдите диагонализацию/Жордан для матрицы. (б) Выпишите явные формулы для компонент решения.
\end{enumerate}

\end{document}
