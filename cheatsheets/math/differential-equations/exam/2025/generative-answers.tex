\documentclass[12pt]{article}
\usepackage[utf8]{inputenc}
\usepackage[russian]{babel}
\usepackage{amsmath,amssymb,mathtools}
\usepackage{geometry}
\geometry{margin=2.3cm}
\begin{document}

\begin{center}
\Large\textbf{Короткие ответы к задачам M1--M5}
\end{center}

% ============== M1 ==============
\section*{M1. Разностные ЛОС с постоянными коэффициентами}

\subsection*{Задача 1}
\[
y_{t+4}-2y_{t+3}-y_{t+2}+2y_{t+1}=3\cdot 2^{t}+(t^2-1)(-1)^t+5.
\]
\textbf{Характеристический многочлен: } $(r-2)(r-1)(r+1)r$.
\[
\begin{aligned}
y_t&=C_0\,0^t+C_1\,1^t+C_2(-1)^t+C_3\,2^t\\
&\quad+\tfrac14\,t\,2^t
+(-1)^t\!\left(\tfrac1{18}t^3-\tfrac7{18}t^2+\tfrac{35}{54}t\right)
-\tfrac52\,t.
\end{aligned}
\]

\subsection*{Задача 2}
\[
\begin{aligned}
&y_{t+5}+y_{t+4}-6y_{t+3}-6y_{t+2}+8y_{t+1}+8y_t\\
&\qquad=2^{t}\cos\frac{\pi t}{2}+t\,3^{t}.
\end{aligned}
\]
\textbf{Корни ЛОС: } $r\in\{2,-1,-2,\sqrt2,-\sqrt2\}$.
\[
\begin{aligned}
y_t&=C_1\,2^t+C_2(-1)^t+C_3(-2)^t+C_4(\sqrt2)^t+C_5(-\sqrt2)^t\\
&\quad+2^t\!\left(\tfrac1{240}\cos\frac{\pi t}{2}+\tfrac1{120}\sin\frac{\pi t}{2}\right)
+3^t\!\left(\tfrac1{140}\,t-\tfrac{969}{19600}\right).
\end{aligned}
\]
\textit{Примечание.} Для однозначности решения нужна ещё одна нач. величина (порядок $5$).

\subsection*{Задача 3}
\[
y_{t+3}-3y_{t+2}+3y_{t+1}-y_t=(t^2+4)\cdot1^t+t(-2)^t.
\]
Левая часть $(E-1)^3$. Общее решение:
\[
\begin{aligned}
y_t&=A+Bt+Ct^2
+\left(\tfrac1{60}t^5-\tfrac18 t^4+t^3\right)
+(-2)^t\!\left(-\tfrac1{27}t+\tfrac{2}{27}\right).
\end{aligned}
\]

% ============== M2 ==============
\section*{M2. Синтез разностного уравнения}

\subsection*{Задача 1}
Частные решения: $2^t,\ t2^t,\ (-2)^t\sin\frac{\pi t}{3}$. \\
\textbf{Ответ: } характеристический многочлен
\[
(r-2)^2\,(r^2+2r+4),
\]
уравнение минимального порядка $4$:
\[
y_{t+4}-2y_{t+3}-8y_{t+1}+16y_t=0.
\]

\subsection*{Задача 2}
Решения: $3^t,\ t3^t,\ 2^t\cos\frac{\pi t}{4},\ 2^t\sin\frac{\pi t}{4}$.\\
\textbf{Ответ: } многочлен
\[
(r-3)^2\,(r^2-2\sqrt2\,r+4),
\]
порядок $4$ (коэффициенты допускают $\sqrt2$).

\subsection*{Задача 3}
Решения: $(-1)^t,\ t(-1)^t,\ t^2(-1)^t,\ 5^t$.\\
\textbf{Ответ: } многочлен
\[
(r+1)^3(r-5)=r^4-2r^3-12r^2-14r-5,
\]
соответствующее ЛОС:
\[
y_{t+4}-2y_{t+3}-12y_{t+2}-14y_{t+1}-5y_t=0.
\]

% ============== M3 ==============
\section*{M3. Нелинейные 2D: равновесия и типы (гиперболика)}

\subsection*{Задача 1}
\[
\dot x=y-x+x^2+xy,\qquad \dot y=-x+2y-xy.
\]
Равновесия: 
\[
(0,0),\quad (2-\sqrt3,\ 2/\sqrt3-1),\quad (2+\sqrt3,\ -1-2/\sqrt3).
\]
Типы:
\[
(0,0)\ \text{--- седло};\quad
(2-\sqrt3,2/\sqrt3-1)\ \text{--- неустойчивый фокус};
\]
\[
(2+\sqrt3,-1-2/\sqrt3)\ \text{--- седло}.
\]

\subsection*{Задача 2}
\[
\dot x=a\,y+x(r^2-1),\quad \dot y=-a\,x+y(r^2-1),\quad r^2=x^2+y^2.
\]
В начале координат: $J=\begin{pmatrix}-1&a\\-a&-1\end{pmatrix}$,
$\operatorname{tr}=-2$, $\det=1+a^2>0$, $D<0$ $\Rightarrow$ \textbf{устойчивый фокус} при любом $a$.

\subsection*{Задача 3}
\[
\dot x=2y-x-2,\qquad \dot y=-2x+y-2.
\]
Единственное равновесие $\bigl(-\tfrac23,\tfrac23\bigr)$.
Линеаризация: $J=\begin{pmatrix}-1&2\\-2&1\end{pmatrix}$,
$\operatorname{tr}=0$, $\det=3>0$, $D<0$ $\Rightarrow$ \textbf{центр}.

% ============== M4 ==============
\section*{M4. Линейные ОДУ-2: снятие $y'$, вронскиан, нули}

\subsection*{Задача 1}
\[
y''+\frac{2}{x}y'-\Bigl(\frac{5}{x^2}+1\Bigr)y=0,\quad x>0.
\]
$\phi=x^{-1}$, \(z=y/\phi\), \(z''+Qz=0\) c \(Q=-\tfrac{5}{x^2}-1\le0\). \\
\(W(x)=W(1)\,x^{-2}\). Любое нетривиальное решение имеет $\le1$ нуль на $(0,\infty)$.

\subsection*{Задача 2}
\[
y''+4y'+(3+e^{-x})y=0.
\]
$\phi=e^{-2x}$, \(Q=e^{-x}-1\le0\). \\
\(W(x)=W(0)\,e^{-4x}\). $\Rightarrow$ у решения $\le1$ нуль на $\mathbb R$.

\subsection*{Задача 3}
\[
x^2y''+\alpha x y'+\beta y=0,\quad x>0.
\]
$\phi=x^{-\alpha/2}$, \(Q=\dfrac{4\beta+2\alpha-\alpha^2}{4x^2}\). \\
\(W(x)=W(x_0)\,(x_0/x)^{\alpha}\). Условие «$\le1$ нуль»: \(4\beta+2\alpha-\alpha^2\le0\).

% ============== M5 ==============
\section*{M5. ПЧП первого порядка: $u=F(I_1,I_2)$}

\subsection*{Задача 1}
\[
(x+y)u_x+(2y-x)u_y=0.
\]
Инварианты: из $\dfrac{dy}{dx}=\dfrac{2y-x}{x+y}$ при $v=y/x$ получаем
\[
\int \frac{1+v}{v^2-v+1}\,dv=-\ln|x|+C\quad\Rightarrow\quad
I_1= x\,\exp\!\Bigl(\tfrac12\ln(v^2-v+1)+\sqrt3\,\arctan\!\tfrac{2v-1}{\sqrt3}\Bigr).
\]
Второй инвариант $I_2=z$. \quad Итог: \(u=F(I_1,I_2)\).

\subsection*{Задача 2}
\[
x\,u_x+y\,u_y+(x+y)z\,u_z=0.
\]
Инварианты: \(I_1=\dfrac{y}{x}\) (масштабность), \(I_2=z\,e^{-(x+y)}\).
Итог: \(u=F\!\left(\dfrac{y}{x},\,z\,e^{-(x+y)}\right)\).

\subsection*{Задача 3}
\[
(2xy)u_x+(y^2-x^2)u_y+(x-y)u_z=0.
\]
Инварианты: \(I_1=\dfrac{x^2+y^2}{x}\) (при $v=y/x$ получаем $d\ln(v^2+1)=-d\ln x$),
и \(I_2=z+\ln|x+y|\) (так как $z'=(x-y)$ и $(x+y)'=2xy+y^2-x^2=(y-x)(x+y)$).
Итог: \(u=F\!\left(\dfrac{x^2+y^2}{x},\ z+\ln|x+y|\right)\).

\end{document}
