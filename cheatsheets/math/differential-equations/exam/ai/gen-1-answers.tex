\documentclass[12pt]{article}
\usepackage[utf8]{inputenc}
\usepackage[russian]{babel}
\usepackage{amsmath,amssymb,mathtools}
\usepackage{geometry}
\geometry{margin=2.3cm}
\begin{document}

\begin{center}
\Large\textbf{Короткие ответы к задачам Gen-1}
\end{center}

% ============== M1 ==============
\section*{M1. Разностные ЛОС с постоянными коэффициентами}

\subsection*{Задача 1}
\[
y_{t+4}-2y_{t+3}-y_{t+2}+2y_{t+1}=3\cdot 2^{t}+(t^2-1)(-1)^t+5.
\]
\textbf{Характеристический многочлен: } $(r-2)(r-1)(r+1)r$.
\[
\begin{aligned}
y_t&=C_0\,0^t+C_1\,1^t+C_2(-1)^t+C_3\,2^t\\
&\quad+\tfrac14\,t\,2^t
+(-1)^t\!\left(\tfrac1{18}t^3-\tfrac7{18}t^2+\tfrac{35}{54}t\right)
-\tfrac52\,t.
\end{aligned}
\]

\subsection*{Задача 2}
\[
\begin{aligned}
&y_{t+5}+y_{t+4}-6y_{t+3}-6y_{t+2}+8y_{t+1}+8y_t\\
&\qquad=2^{t}\cos\frac{\pi t}{2}+t\,3^{t}.
\end{aligned}
\]
\textbf{Корни ЛОС: } $r\in\{2,-1,-2,\sqrt2,-\sqrt2\}$.
\[
\begin{aligned}
y_t&=C_1\,2^t+C_2(-1)^t+C_3(-2)^t+C_4(\sqrt2)^t+C_5(-\sqrt2)^t\\
&\quad+2^t\!\left(\tfrac1{240}\cos\frac{\pi t}{2}+\tfrac1{120}\sin\frac{\pi t}{2}\right)
+3^t\!\left(\tfrac1{140}\,t-\tfrac{969}{19600}\right).
\end{aligned}
\]
\textit{Примечание.} Для однозначности решения нужна ещё одна нач. величина (порядок $5$).

\subsection*{Задача 3}
\[
y_{t+3}-3y_{t+2}+3y_{t+1}-y_t=(t^2+4)\cdot1^t+t(-2)^t.
\]
Левая часть $(E-1)^3$. Общее решение:
\[
\begin{aligned}
y_t&=A+Bt+Ct^2
+\left(\tfrac1{60}t^5-\tfrac18 t^4+t^3\right)
+(-2)^t\!\left(-\tfrac1{27}t+\tfrac{2}{27}\right).
\end{aligned}
\]

% ============== M2 ==============
\section*{M2. Синтез разностного уравнения}

\subsection*{Задача 1}
Частные решения: $2^t,\ t2^t,\ (-2)^t\sin\frac{\pi t}{3}$. \\
\textbf{Ответ: } характеристический многочлен
\[
(r-2)^2\,(r^2+2r+4),
\]
уравнение минимального порядка $4$:
\[
y_{t+4}-2y_{t+3}-8y_{t+1}+16y_t=0.
\]

\subsection*{Задача 2}
Решения: $3^t,\ t3^t,\ 2^t\cos\frac{\pi t}{4},\ 2^t\sin\frac{\pi t}{4}$.\\
\textbf{Ответ: } многочлен
\[
(r-3)^2\,(r^2-2\sqrt2\,r+4),
\]
порядок $4$ (коэффициенты допускают $\sqrt2$).

\subsection*{Задача 3}
Решения: $(-1)^t,\ t(-1)^t,\ t^2(-1)^t,\ 5^t$.\\
\textbf{Ответ: } многочлен
\[
(r+1)^3(r-5)=r^4-2r^3-12r^2-14r-5,
\]
соответствующее ЛОС:
\[
y_{t+4}-2y_{t+3}-12y_{t+2}-14y_{t+1}-5y_t=0.
\]

% ============== M3 ==============
\section*{M3. Нелинейные 2D: равновесия и типы (гиперболика)}

\subsection*{Задача 1}
\[
\dot x=y-x+x^2+xy,\qquad \dot y=-x+2y-xy.
\]
Равновесия: 
\[
(0,0),\quad (2-\sqrt3,\ 2/\sqrt3-1),\quad (2+\sqrt3,\ -1-2/\sqrt3).
\]
Типы:
\[
(0,0)\ \text{--- седло};\quad
(2-\sqrt3,2/\sqrt3-1)\ \text{--- неустойчивый фокус};
\]
\[
(2+\sqrt3,-1-2/\sqrt3)\ \text{--- седло}.
\]

\subsection*{Задача 2}
\[
\dot x=a\,y+x(r^2-1),\quad \dot y=-a\,x+y(r^2-1),\quad r^2=x^2+y^2.
\]
В начале координат: $J=\begin{pmatrix}-1&a\\-a&-1\end{pmatrix}$,
$\operatorname{tr}=-2$, $\det=1+a^2>0$, $D<0$ $\Rightarrow$ \textbf{устойчивый фокус} при любом $a$.

\subsection*{Задача 3}
\[
\dot x=2y-x-2,\qquad \dot y=-2x+y-2.
\]
Единственное равновесие $\bigl(-\tfrac23,\tfrac23\bigr)$.
Линеаризация: $J=\begin{pmatrix}-1&2\\-2&1\end{pmatrix}$,
$\operatorname{tr}=0$, $\det=3>0$, $D<0$ $\Rightarrow$ \textbf{центр}.

% ============== M4 ==============
\section*{M4. Линейные ОДУ-2: снятие $y'$, вронскиан, нули}

\subsection*{Задача 1}
\[
y''+\frac{2}{x}y'-\Bigl(\frac{5}{x^2}+1\Bigr)y=0,\quad x>0.
\]
$\phi=x^{-1}$, \(z=y/\phi\), \(z''+Qz=0\) c \(Q=-\tfrac{5}{x^2}-1\le0\). \\
\(W(x)=W(1)\,x^{-2}\). Любое нетривиальное решение имеет $\le1$ нуль на $(0,\infty)$.

\subsection*{Задача 2}
\[
y''+4y'+(3+e^{-x})y=0.
\]
$\phi=e^{-2x}$, \(Q=e^{-x}-1\le0\). \\
\(W(x)=W(0)\,e^{-4x}\). $\Rightarrow$ у решения $\le1$ нуль на $\mathbb R$.

\subsection*{Задача 3}
\[
x^2y''+\alpha x y'+\beta y=0,\quad x>0.
\]
$\phi=x^{-\alpha/2}$, \(Q=\dfrac{4\beta+2\alpha-\alpha^2}{4x^2}\). \\
\(W(x)=W(x_0)\,(x_0/x)^{\alpha}\). Условие «$\le1$ нуль»: \(4\beta+2\alpha-\alpha^2\le0\).

% ============== M5 ==============
\section*{M5. ПЧП первого порядка: $u=F(I_1,I_2)$}

\subsection*{Задача 1}
\[
(x+y)u_x+(2y-x)u_y=0.
\]
Инварианты: из $\dfrac{dy}{dx}=\dfrac{2y-x}{x+y}$ при $v=y/x$ получаем
\[
\int \frac{1+v}{v^2-v+1}\,dv=-\ln|x|+C\quad\Rightarrow\quad
I_1= x\,\exp\!\Bigl(\tfrac12\ln(v^2-v+1)+\sqrt3\,\arctan\!\tfrac{2v-1}{\sqrt3}\Bigr).
\]
Второй инвариант $I_2=z$. \quad Итог: \(u=F(I_1,I_2)\).

\subsection*{Задача 2}
\[
x\,u_x+y\,u_y+(x+y)z\,u_z=0.
\]
Инварианты: \(I_1=\dfrac{y}{x}\) (масштабность), \(I_2=z\,e^{-(x+y)}\).
Итог: \(u=F\!\left(\dfrac{y}{x},\,z\,e^{-(x+y)}\right)\).

\subsection*{Задача 3}
\[
(2xy)u_x+(y^2-x^2)u_y+(x-y)u_z=0.
\]
Инварианты: \(I_1=\dfrac{x^2+y^2}{x}\) (при $v=y/x$ получаем $d\ln(v^2+1)=-d\ln x$),
и \(I_2=z+\ln|x+y|\) (так как $z'=(x-y)$ и $(x+y)'=2xy+y^2-x^2=(y-x)(x+y)$).
Итог: \(u=F\!\left(\dfrac{x^2+y^2}{x},\ z+\ln|x+y|\right)\).

% ===================== M6 =====================
\section*{M6. ПЧП первого порядка: задача Коши}

\subsection*{Задача 1}
$y\,z_x - x\,z_y = 0$ с данными $z = 2y$ при $x = 1$. \\
\textbf{Характеристики: } $x^2+y^2=C$. \\
\textbf{Тест нехарактеристичности: } $\Delta = y \cdot 1 - (-1) \cdot 0 = y$. В $(1,0)$: $\Delta=0$ (характеристично). \\
\textbf{Решение: } $z = \pm 2\sqrt{x^2+y^2-1}$ (неединственность).

\subsection*{Задача 2}
$y\,z_x - x\,z_y = 0$ с данными $z = 2y$ при $x = 1+y$. \\
\textbf{Тест: } $\Delta = y + x = 1 \neq 0$ в $(1,0)$ (нехарактеристично). \\
\textbf{Решение: } $z = -1 + \sqrt{2(x^2+y^2)-1}$ (единственно).

\subsection*{Задача 3}
$(x+y)\,u_x + (2y-x)\,u_y = 0$ с данными $u = x^2$ на $y = x^2$. \\
\textbf{Инвариант: } $I_1 = x^2+y^2$. \\
\textbf{Тест: } $\Delta = (x+y) \cdot 2x - (2y-x) \cdot 1 = 2x^2+2xy-2y+x \neq 0$ в $(0,0)$. \\
\textbf{Решение: } $u = F(x^2+y^2)$, где $F$ определяется из $F(x^2+x^4) = x^2$.

% ===================== M7 =====================
\section*{M7. Нелинейные 2D: равновесия, линеаризация, портрет}

\subsection*{Задача 1}
$\dot x=x-x^2-y-y^2,\ \dot y=2x-3y+xy$. \\
\textbf{Равновесия: } $(0,0)$ и $\bigl(x_*,y_*\bigr)\approx(0.1911,\ 0.1361)$. \\
Якоби в $(0,0)$: $\begin{pmatrix}1&-1\\2&-3\end{pmatrix}$, собств. значения $-1\pm\sqrt2$ $\Rightarrow$ \textbf{седло}. \\
В $(x_*,y_*)$: собственные $\approx(-1.5627,\ -0.6284)$ $\Rightarrow$ \textbf{устойчивый узел}.

\subsection*{Задача 2}
$\dot x=y-x(x^2+b),\ \dot y=-x-y(y^2+b)$. \\
$J(0,0)=\begin{pmatrix}-b&1\\-1&-b\end{pmatrix}$, $\lambda=-b\pm i$. \\
\textbf{Классификация: } $b>0$ — устойчивый фокус; $b<0$ — неустойчивый фокус; $b=0$ — линейно центр, но нелинейные кубики дают $\dot r=-(x^4+y^4)/r<0$ при $r\ne0$ $\Rightarrow$ \textbf{асимптотически устойчивый фокус}.

\subsection*{Задача 3}
$\dot x=(x-y)(1-xy),\ \dot y=(x+y)(1+x^2)$. \\
\textbf{Единственное равновесие: } $(0,0)$. \\
$J(0,0)=\begin{pmatrix}1&-1\\1&1\end{pmatrix}$, $\lambda=1\pm i$ $\Rightarrow$ \textbf{неустойчивый фокус}.

% ===================== M8 =====================
\section*{M8. Полярные координаты, $\dot r,\dot\theta$}

\subsection*{Задача 1}
$\dot x=a\,y+x(r^2-1),\ \dot y=-a\,x+y(r^2-1)$. \\
В полярных: $\dot r=r(r^2-1)$, $\dot\theta=-a$. \\
\textbf{Динамика: } $r=0$ устойчив, $r=1$ неустойчивый цикл; при $r>1$ — уход на бесконечность; угловая скорость постоянна.

\subsection*{Задача 2}
$\dot x=x(1-r^2)+\omega y,\ \dot y=y(1-r^2)-\omega x$. \\
В полярных: $\dot r=r(1-r^2)$, $\dot\theta=-\omega$. \\
\textbf{Динамика: } $r=0$ неустойчив, $r=1$ \textbf{устойчивый предельный цикл}.

\subsection*{Задача 3}
$\dot x=(r^2-2)x+\Omega y,\ \dot y=(r^2-2)y-\Omega x$. \\
В полярных: $\dot r=r(r^2-2)$, $\dot\theta=-\Omega$. \\
\textbf{Динамика: } $r=0$ устойчив, $r=\sqrt2$ неустойчивый цикл; $r>\sqrt2$ — разлёт.

% ===================== M9 =====================
\section*{M9. Первые интегралы в 3D-ОДУ}

\subsection*{Задача 1}
$\dot x=yz,\ \dot y=zx,\ \dot z=xy$. \\
\textbf{Интегралы: } $I_1=x^2-y^2,\ I_2=x^2-z^2$ (постоянны, т.к. $\frac{d}{dt}(x^2-y^2)=2xyz-2xyz=0$, аналогично для $x^2-z^2$). \\
\textbf{Интегральные поверхности: } пересечения квадрик $x^2-y^2=C_1,\ x^2-z^2=C_2$.

\subsection*{Задача 2}
$\dot x=y^2-z^2,\ \dot y=zx,\ \dot z=xy$. \\
\textbf{Интеграл 1: } $I_1=y^2-z^2$ (как в задаче 1). Тогда $\dot x=I_1=\text{const}$. \\
\textbf{Интеграл 2: } 
\[
I_2=\ln\!\frac{y-z}{\,y+z\,}+\frac{x^2}{\,y^2-z^2\,},
\]
т.к. $\frac{d}{dt}\ln\frac{y-z}{y+z}=-2x$, а $\frac{d}{dt}\!\bigl(\frac{x^2}{I_1}\bigr)=\frac{2x\dot x}{I_1}=2x$. \\
\textbf{Поверхности: } $y^2-z^2=C_1$, \ $\ln\frac{y-z}{y+z}+\frac{x^2}{C_1}=C_2$.

\subsection*{Задача 3}
$\dot x=y+z,\ \dot y=z+x,\ \dot z=x+y$. \\
Собственный базис: $v_1=(1,1,1)$, $\lambda_1=2$; $v_2=(1,-1,0)$, $v_3=(1,0,-1)$, $\lambda_{2,3}=-1$. \\
Координаты: $\xi=\frac{x+y+z}{3},\ \eta=\frac{x-2y+z}{3},\ \zeta=\frac{x+y-2z}{3}$. \\
\textbf{Интегралы: } $I_1=\dfrac{\zeta}{\eta}$, \quad $I_2=\eta^2\,\xi$ \ (т.к. $\dot\eta=-\eta,\ \dot\zeta=-\zeta,\ \dot\xi=2\xi$). \\
В явном виде: $I_1=\dfrac{x+y-2z}{\,x-2y+z\,}$, \ $I_2=\dfrac{(x-2y+z)^2(x+y+z)}{27}$.

% ===================== M10 =====================
\section*{M10. Периодические коэффициенты, монодромия}

\subsection*{Задача 1}
$q(x+T)=q(x)$, $y(0)=y(T)=0$. Тогда $y_1(x)=y(x+T)$ тоже решение и $y_1(0)=0$. \\
Пространство решений с $y(0)=0$ одномерно $\Rightarrow y(x+T)=C\,y(x)$. \\
\textbf{Константа: } $C=\dfrac{y'(T)}{y'(0)}\ne0$.

\subsection*{Задача 2}
$y''+(2+\cos x)y=0$, период $2\pi$. Фундаментальная матрица $\Phi(2\pi)$ имеет $\det=1$. \\
\textbf{Множители Флоке: } корни $\mu_{1,2}$ уравнения $\mu^2-\Delta\,\mu+1=0$, где $\Delta=\operatorname{tr}\Phi(2\pi)\in\mathbb{R}$. \\
\textbf{Виды: } 
(i) $|\Delta|<2$: $\mu=e^{\pm i\theta}$ (устойчивый, «эллиптический»); 
(ii) $|\Delta|>2$: вещественные взаимно обратные; 
(iii) $|\Delta|=2$: кратный $\pm1$.

\subsection*{Задача 3}
$q(x+\pi)=q(x)$, $y(0)=0$, $y'(\pi)=0$. Тогда вектор $(y(\pi),0)$ — результат действия монодромии на $(0,y'(0))$. \\
На краях зон спектра монодромия имеет $\mu=\pm1$ $\Rightarrow$ \textbf{$y(x+\pi)=\pm y(x)$}. Оба варианта возможны (в зависимости от знака \(\mu\)).

% ===================== M11 =====================
\section*{M11. Доказательные мини-кейсы: нули Бесселя, энергетическая устойчивость}

\subsection*{Задача 1}
Докажите, что функция Бесселя $J_0(x)$ имеет бесконечно много нулей на $(0,\infty)$. \\
\textbf{Доказательство: } $J_0$ удовлетворяет $x^2y''+xy'+x^2y=0$. При $x\to\infty$ уравнение асимптотически близко к $y''+y=0$, решения которого осциллируют. \\
\textbf{Теорема сравнения: } если $q(x)\ge 1$ при больших $x$, то решения $y''+q(x)y=0$ имеют бесконечно много нулей.

\subsection*{Задача 2}
$y''+\frac{1}{x}y'+\bigl(1-\frac{\nu^2}{x^2}\bigr)y=0$, $\nu\ge1$. \\
\textbf{Нормальная форма: } $z=y\sqrt{x}$, $z''+Q(x)z=0$ где $Q(x)=1-\frac{\nu^2-1/4}{x^2}$. \\
При $\nu\ge1$: $Q(x)\le 1$ и $Q(x)\to 1$ при $x\to\infty$. \\
\textbf{Вывод: } решение имеет не более одного нуля на $(0,\infty)$.

\subsection*{Задача 3}
$\ddot x + \omega^2 x + \varepsilon x^3 = 0$, $\varepsilon>0$. \\
\textbf{Функция Ляпунова: } $E=\frac{1}{2}\dot x^2 + \frac{\omega^2}{2}x^2 + \frac{\varepsilon}{4}x^4$. \\
$\dot E = \dot x(\ddot x + \omega^2 x + \varepsilon x^3) = 0$ $\Rightarrow$ \textbf{устойчивость по Ляпунову}.

% ===================== M12 =====================
\section*{M12. Потенциальные системы и устойчивость}

\subsection*{Задача 1}
$\ddot{\mathbf x}=-\nabla V(\mathbf x)$, $V\ge0$, минимум в $\mathbf0$. \\
Ляпунов: $E=\tfrac12\|\dot{\mathbf x}\|^2+V(\mathbf x)$, $\dot E=0$ $\Rightarrow$ \textbf{устойчивость}. \\
\textbf{Асимптотическая устойчивость} невозможна без диссипации: $E$ сохраняется.

\subsection*{Задача 2}
$V=\tfrac14(r^2-1)^2+\varepsilon xy$, $r^2=x^2+y^2$, $|\varepsilon|\ll1$. \\
Критические точки: $r^2=1\pm\varepsilon$, при $r^2=1-\varepsilon$ имеем $y=x$, при $r^2=1+\varepsilon$ — $y=-x$. \\
Точки: $(\pm a,\pm a)$, $a=\sqrt{\tfrac{1-\varepsilon}{2}}$; \ $(\pm b,\mp b)$, $b=\sqrt{\tfrac{1+\varepsilon}{2}}$. \\
Гессиан в $(\pm a,\pm a)$: собственные $2(1-\varepsilon)$ и $-2\varepsilon$. \\
В $(\pm b,\mp b)$: собственные $2(1+\varepsilon)$ и $2\varepsilon$. \\
\textbf{Классика: } при $\varepsilon>0$: $(\pm b,\mp b)$ — \textbf{минимумы}, $(\pm a,\pm a)$ — \textbf{седла}; при $\varepsilon<0$ — наоборот.

\subsection*{Задача 3}
$\ddot{\mathbf x}+\gamma\dot{\mathbf x}=-\nabla V(\mathbf x)$, $\gamma>0$, $V(\mathbf x)\ge c\|\mathbf x\|^2$ близ нуля. \\
Ляпунов: $E=\tfrac12\|\dot{\mathbf x}\|^2+V(\mathbf x)$, $\dot E=-\gamma\|\dot{\mathbf x}\|^2\le0$. \\
С учётом $V\ge c\|\mathbf x\|^2$ и инвариантности по Ляпунову–ЛаСаллю $\Rightarrow$ \textbf{асимптотически устойчиво}.

% ===================== M13 =====================
\section*{M13. Системы разностных: вариация постоянных, $A^t$}

\subsection*{Задача 1}
$A=\begin{pmatrix}3&-1\\2&0\end{pmatrix}$, \ $\mathbf b=\binom{1}{-1}$, \ $\mathbf x_{t+1}=A\mathbf x_t+\mathbf b\,2^t$, \ $\mathbf x_0=\binom{0}{1}$. \\
$A^t=\begin{pmatrix}2\cdot2^t-1&\ 1-2^t\\[2pt]2\cdot2^t-2&\ 2-2^t\end{pmatrix}$. \\
\[
\mathbf x_t=A^t\mathbf x_0+\sum_{k=0}^{t-1}A^{t-1-k}\mathbf b\,2^k
=\binom{\frac{3}{2}\,2^t t-3\cdot 2^t+3}{\ \frac{3}{2}\,2^t t-5\cdot 2^t+6\ }.
\]

\subsection*{Задача 2}
$A=\begin{pmatrix}2&1&0\\0&2&1\\0&0&2\end{pmatrix}=2I+N$, $N^3=0$. \\
\[
A^t=2^t\Bigl(I+\tfrac{t}{2}N+\tfrac{t(t-1)}{8}N^2\Bigr)
=
2^t\begin{pmatrix}
1&\tfrac{t}{2}&\tfrac{t(t-1)}{8}\\
0&1&\tfrac{t}{2}\\
0&0&1
\end{pmatrix}.
\]
\textbf{Общее решение: } $\mathbf x_t=A^t\mathbf x_0$. Рост нормы $\sim C\,2^t t^2$. \\
\textbf{Мин. полином: } $(\lambda-2)^3$.

\subsection*{Задача 3}
$\mathbf x_{t+1}=\tfrac12\!\begin{pmatrix}1&3\\3&1\end{pmatrix}\mathbf x_t+\binom{(-1)^t}{t\,2^t}$, \ $\mathbf x_0=\mathbf 0$. \\
Собств. значения матрицы: $2$ и $-1$ (резонанс с правой частью). \\
\[
\mathbf x_t=\binom{-\frac12(-1)^t t-\frac{5}{18}(-1)^t+\frac18\,2^t t^2-\frac{7}{24}\,2^t t+\frac{5}{18}\,2^t}{
\ \ \frac12(-1)^t t-\frac{1}{18}(-1)^t+\frac18\,2^t t^2+\frac{1}{24}\,2^t t+\frac{1}{18}\,2^t\ }.
\]

\end{document}

