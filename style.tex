% style.tex — Formalized style guide with macros and instructions
% Project: HSE AMI Cheatsheets
% Author: werserk
% Version: 1.0

% This file contains formalized style rules and macros for consistent
% document formatting across the HSE AMI Cheatsheets project.

\NeedsTeXFormat{LaTeX2e}
\ProvidesPackage{style}[2025/01/27 v1.0 HSE AMI Cheatsheets Style Guide]

% =============================================================================
% CORE REQUIREMENTS
% =============================================================================

% Required packages for style compliance
\RequirePackage{amsmath}
\RequirePackage{amssymb}
\RequirePackage{mathtools}
\RequirePackage{booktabs}
\RequirePackage{enumitem}
\RequirePackage{xparse}

% =============================================================================
% VARIABLE NAMING CONVENTIONS
% =============================================================================

% Time indices: ONLY t or T (NO n allowed)
% Usage: \timeIndex{t} or \timeIndex{T}
\NewDocumentCommand{\timeIndex}{m}{#1}

% Discrete step: k (only in local reasoning with annotation)
% Usage: \discreteStep{k} with comment
\NewDocumentCommand{\discreteStep}{m}{#1}

% Dimensions: m, d, N (use only when necessary)
% Usage: \dimension{m}, \dimension{d}, \dimension{N}
\NewDocumentCommand{\dimension}{m}{#1}

% =============================================================================
% MATHEMATICAL NOTATION STANDARDS
% =============================================================================

% Standard mathematical sets
\newcommand{\reals}{\mathbb{R}}           % Real numbers
\newcommand{\naturals}{\mathbb{N}}        % Natural numbers
\newcommand{\complex}{\mathbb{C}}         % Complex numbers
\newcommand{\integers}{\mathbb{Z}}        % Integers
\newcommand{\rationals}{\mathbb{Q}}       % Rational numbers

% Common operators (extend as needed)
\DeclareMathOperator{\grad}{grad}
\DeclareMathOperator{\divergence}{div}
\DeclareMathOperator{\curl}{curl}
\DeclareMathOperator{\laplacian}{\Delta}
\DeclareMathOperator{\trace}{tr}
\DeclareMathOperator{\rank}{rank}
\DeclareMathOperator{\nullity}{nullity}

% Linear algebra operators
\newcommand{\linop}{\mathcal{L}}          % Linear operator
\newcommand{\algebra}{\mathcal{A}}        % Algebra
\newcommand{\sigmafield}{\mathcal{F}}     % Sigma-field
\newcommand{\expectation}{\mathbb{E}}     % Expectation
\newcommand{\variance}{\mathbb{V}}        % Variance

% =============================================================================
% CONTENT STRUCTURE MACROS
% =============================================================================

% Problem structure components
\NewDocumentCommand{\ProblemCondition}{m}{%
  \textbf{Условие:} #1
}

\NewDocumentCommand{\ProblemHint}{m}{%
  \textbf{Подсказка:} #1
}

\NewDocumentCommand{\ProblemSolution}{m}{%
  \textbf{Решение:} #1
}

\NewDocumentCommand{\ProblemAnswer}{m}{%
  \textbf{Ответ:} #1
}

\NewDocumentCommand{\ProblemRemark}{m}{%
  \textbf{Замечание:} #1
}

% Steps environment with Russian labels
\NewDocumentEnvironment{ProblemSteps}{ O{} }{%
  \begin{enumerate}[label=\textbf{Шаг~\arabic*:}, #1]
}{\end{enumerate}}

% =============================================================================
% TERMINOLOGY AND LANGUAGE STANDARDS
% =============================================================================

% Standard Russian mathematical terms
\newcommand{\proof}{\textbf{Доказательство:}}
\newcommand{\example}{\textbf{Пример:}}
\newcommand{\exercise}{\textbf{Упражнение:}}
\newcommand{\remark}{\textbf{Замечание:}}
\newcommand{\definition}{\textbf{Определение:}}
\newcommand{\theorem}{\textbf{Теорема:}}
\newcommand{\lemma}{\textbf{Лемма:}}
\newcommand{\corollary}{\textbf{Следствие:}}

% Formal language helpers
\newcommand{\required}{\textbf{требуется}}
\newcommand{\mustbe}{\textbf{должно быть}}
\newcommand{\necessary}{\textbf{необходимо}}
\newcommand{\sufficient}{\textbf{достаточно}}

% Avoid informal language
% Use \required instead of "надо"
% Use \mustbe instead of "нужно"

% =============================================================================
% LABELING CONVENTIONS
% =============================================================================

% Standard label prefixes
\newcommand{\seclabel}[1]{\label{sec:#1}}
\newcommand{\eqlabel}[1]{\label{eq:#1}}
\newcommand{\figlabel}[1]{\label{fig:#1}}
\newcommand{\thmlabel}[1]{\label{thm:#1}}
\newcommand{\exlabel}[1]{\label{ex:#1}}
\newcommand{\tablabel}[1]{\label{tab:#1}}
\newcommand{\alglabel}[1]{\label{alg:#1}}

% Reference commands with proper prefixes
\newcommand{\secref}[1]{\ref{sec:#1}}
\newcommand{\eqref}[1]{\eqref{eq:#1}}
\newcommand{\figref}[1]{\ref{fig:#1}}
\newcommand{\thmref}[1]{\ref{thm:#1}}
\newcommand{\exref}[1]{\ref{ex:#1}}
\newcommand{\tabref}[1]{\ref{tab:#1}}
\newcommand{\algref}[1]{\ref{alg:#1}}

% =============================================================================
% CODE FORMATTING STANDARDS
% =============================================================================

% Code block with language specification
\NewDocumentCommand{\CodeBlock}{m m}{%
  \begin{verbatim}
# RUN: #2
#1
  \end{verbatim}
}

% Inline code
\newcommand{\code}[1]{\texttt{#1}}

% =============================================================================
% TABLE FORMATTING STANDARDS
% =============================================================================

% Standard table environment with booktabs
\NewDocumentEnvironment{StandardTable}{O{h!} m m}{%
  \begin{table}[#1]
  \centering
  \caption{#2}
  \label{tab:#3}
  \begin{tabular}{#4}
  \toprule
}{%
  \bottomrule
  \end{tabular}
  \end{table}
}

% =============================================================================
% NOTATION DECLARATION ENVIRONMENT
% =============================================================================

% Environment for declaring new mathematical symbols
\NewDocumentEnvironment{NotationSection}{}{%
  \subsection*{Обозначения}
  \begin{itemize}[leftmargin=*,noitemsep,topsep=0pt]
}{%
  \end{itemize}
}

% Helper for notation entries
\NewDocumentCommand{\NotationEntry}{m m}{%
  \item $#1$ — #2
}

% =============================================================================
% QUALITY CONTROL MACROS
% =============================================================================

% Check for forbidden variable usage
\newcommand{\checkVariables}{%
  % This would be implemented as a script
  % For now, just a reminder comment
  % TODO: Implement variable checking
}

% Validate label consistency
\newcommand{\validateLabels}{%
  % This would be implemented as a script
  % For now, just a reminder comment
  % TODO: Implement label validation
}

% =============================================================================
% STYLE COMPLIANCE INSTRUCTIONS
% =============================================================================

% Instructions for style compliance
\newcommand{\StyleInstructions}{%
  \begin{itemize}
    \item Используйте только \texttt{t} или \texttt{T} для индексов времени
    \item Объявляйте новые символы в разделе «Обозначения»
    \item Структурируйте задачи: Условие → Решение → Ответ
    \item Используйте формальную терминологию
    \item Проверяйте соответствие меток и ссылок
  \end{itemize}
}

% =============================================================================
% DOCUMENTATION
% =============================================================================

% This package provides:
% 1. Standardized variable naming (timeIndex, discreteStep, dimension)
% 2. Mathematical notation standards (reals, naturals, complex, etc.)
% 3. Content structure macros (ProblemCondition, ProblemSolution, etc.)
% 4. Labeling conventions (seclabel, eqlabel, etc.)
% 5. Code formatting standards (CodeBlock, code)
% 6. Table formatting standards (StandardTable)
% 7. Notation declaration environment (NotationSection)
% 8. Quality control reminders (checkVariables, validateLabels)

% Usage:
% \usepackage{style}
%
% In document:
% \timeIndex{t} for time indices
% \reals, \naturals, \complex for standard sets
% \ProblemCondition{...} for problem structure
% \seclabel{name} for section labels
% \CodeBlock{code}{language} for code blocks

\endinput
