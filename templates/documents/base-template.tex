% base-template.tex — Unified base template for all HSE AMI Cheatsheets documents
% Project: HSE AMI Cheatsheets
% Author: werserk
% Version: 2.0
%
% This template provides a unified foundation for all document types:
% - Cheatsheets (single and two-column)
% - Exam variants
% - Preparation materials
% - Modular content
%
% Usage: Copy this file and customize the document class and options as needed

% =============================================================================
% DOCUMENT CLASS AND BASIC PACKAGES
% =============================================================================

% Choose appropriate document class based on document type:
% - article: for preparation materials and exam variants
% - article with landscape: for cheatsheets
\documentclass[11pt,a4paper]{article}
% For cheatsheets, use: \documentclass[10pt,landscape,a4paper]{article}

% Basic LaTeX packages
\usepackage[utf8]{inputenc}
\usepackage[T2A]{fontenc}
\usepackage[russian]{babel}

% =============================================================================
% UNIFIED STYLING SYSTEM
% =============================================================================

% Load the unified styling system
\usepackage{../../assets/styles/hse-unified}

% =============================================================================
% DOCUMENT-SPECIFIC CONFIGURATION
% =============================================================================

% Geometry settings (customize as needed)
\usepackage{geometry}
\geometry{margin=1.8cm}
% For cheatsheets: \geometry{margin=1cm}

% Page style
\usepackage{fancyhdr}
\pagestyle{fancy}
\fancyhf{}
\fancyfoot[L]{\textbf{by werserk}}
\fancyfoot[R]{\thepage}
\renewcommand{\headrulewidth}{0pt}

% For cheatsheets, use: \pagestyle{empty}

% =============================================================================
% DOCUMENT METADATA
% =============================================================================

% Document title and subtitle (customize these)
\newcommand{\DocumentTitle}{Название документа}
\newcommand{\DocumentSubtitle}{Краткое описание}

% Contributors (optional)
\newcommand{\contributors}{}

% Accent color (optional, defaults to FF7517)
% \HSESetAccent{FF7517}

% =============================================================================
% DOCUMENT CONTENT
% =============================================================================

\begin{document}

% Document title
\HSETitle{\DocumentTitle}{\DocumentSubtitle}

% Optional: Table of contents
% {
%     \hypersetup{linkcolor=black}
%     \tableofcontents
% }
% \newpage

% =============================================================================
% MAIN CONTENT AREA
% =============================================================================

% For cheatsheets, wrap content in two-column environment:
% \HSETwoColsBegin
% ... content ...
% \HSETwoColsEnd

% For exam variants, use:
% \ExamVariantTitle{Вариант А}
% \begin{examproblems}
% \item Задача 1
% \item Задача 2
% \end{examproblems}

% For preparation materials, use standard sections:
% \section{Название раздела}\seclabel{section-name}
% ... content ...

% =============================================================================
% EXAMPLE CONTENT STRUCTURE
% =============================================================================

\section{Основные понятия}\seclabel{basic-concepts}

\HSEDefinition{\Term{Ключевое понятие} — формальное определение понятия.}

\begin{NotationSection}
\NotationEntry{x_t}{последовательность значений}
\NotationEntry{f(x)}{функция от переменной x}
\end{NotationSection}

\subsection{Примеры}\seclabel{examples}

\HSEExample{\example Решите уравнение $y_{t+2} - 5y_{t+1} + 6y_t = 0$.

\textbf{Решение:}
\begin{ProblemSteps}
\item Характеристическое уравнение: $\lambda^2 - 5\lambda + 6 = 0$
\item Корни: $\lambda_1 = 2$, $\lambda_2 = 3$
\item Общее решение: $y_t = C_1 \cdot 2^t + C_2 \cdot 3^t$
\end{ProblemSteps}

\textbf{Ответ:} $y_t = C_1 \cdot 2^t + C_2 \cdot 3^t$, где $C_1, C_2 \in \reals$.}

\subsection{Алгоритмы}\seclabel{algorithms}

\HSEAlgorithm{\textbf{Алгоритм решения.}
\begin{ProblemSteps}
\item Шаг 1: Описание первого шага
\item Шаг 2: Описание второго шага
\item Шаг 3: Описание третьего шага
\end{ProblemSteps}
}

% =============================================================================
% TABLES AND FORMULAS
% =============================================================================

\begin{StandardTable}[h!]{Пример таблицы}{example-table}
{c|c|c}
Параметр & Значение & Описание \\
\midrule
$\alpha$ & 0.05 & Уровень значимости \\
$\beta$ & 0.95 & Доверительный уровень \\
$\gamma$ & 1.4 & Коэффициент \\
\end{StandardTable}

% Mathematical formulas
\begin{equation}\eqlabel{main-formula}
y_t = \sum_{k=0}^{n} a_k \cdot r_k^t
\end{equation}

% =============================================================================
% FOOTER
% =============================================================================

% For cheatsheets and preparation materials
\HSEFooter

% For exam variants, omit footer or use custom footer

\end{document}
