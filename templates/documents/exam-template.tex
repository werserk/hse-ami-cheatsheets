% exam-template.tex — Template for exam variants
% Project: HSE AMI Cheatsheets
% Author: werserk
% Version: 2.0
%
% This template is specifically designed for exam variants

% =============================================================================
% DOCUMENT CLASS AND BASIC PACKAGES
% =============================================================================

\documentclass[12pt]{article}

% Basic LaTeX packages
\usepackage[utf8]{inputenc}
\usepackage[T2A]{fontenc}
\usepackage[russian]{babel}

% =============================================================================
% UNIFIED STYLING SYSTEM
% =============================================================================

% Load the unified styling system
\usepackage{../../assets/styles/hse-unified}

% =============================================================================
% EXAM-SPECIFIC CONFIGURATION
% =============================================================================

% Geometry settings for exams
\usepackage{geometry}
\geometry{margin=2.5cm}

% =============================================================================
% DOCUMENT METADATA
% =============================================================================

% Exam metadata (customize these)
\SetExamSubject{Дифференциальные уравнения}
\SetExamLecturer{Букин К.А.}

% Variant title
\newcommand{\VariantTitle}{Вариант А}

% =============================================================================
% DOCUMENT CONTENT
% =============================================================================

\begin{document}

% Exam variant title
\ExamVariantTitle{\VariantTitle}

% =============================================================================
% EXAM PROBLEMS
% =============================================================================

\begin{examproblems}

\item Решите разностное уравнение
\[
y_{t+3}-y_{t+2}+4y_{t+1}-4y_{t} = 26 \cdot 3^{t} + 10t + 9.
\]

\item Решите систему разностных уравнений
\[
\begin{pmatrix}
x_{t+1}\\[2pt]
y_{t+1}\\[2pt]
z_{t+1}
\end{pmatrix}
=
A
\begin{pmatrix}
x_{t}\\[2pt]
y_{t}\\[2pt]
z_{t}
\end{pmatrix},
\qquad
A=
\begin{pmatrix}
4 & 2 & -2\\
1 & 3 & -1\\
3 & 3 & -1
\end{pmatrix},
\]
собственное значение имеет кратность $3$.

\item Пусть функция $q(x)$ непрерывна на всей оси и периодична с периодом. Пусть $y(x)$ нетривиальное решение уравнения
\[
y'' + q(x)y(x)=0,
\]
удовлетворяющее условию $y(0)=y(1)=0$. Докажите, что $y(x+1)=Cy(x)$ при некотором $C$.

\item Найдите положения равновесия автономной системы уравнений, определите их характер, и нарисуйте фазовые портреты в окрестности положений равновесия
\[
\begin{cases}
\dot{x}=2-2\sqrt{1+x+y},\\[4pt]
\dot{y}=\exp\!\Bigl(\tfrac{5}{4}x+2y+y^{2}\Bigr)-1.
\end{cases}
\]

\item Решите однородное уравнение в частных производных первого порядка
\[
|x^{2}\,\frac{\partial u}{\partial x}
+(2xy-y^{2})\,\frac{\partial u}{\partial y}
+(2xz-yz)\,\frac{\partial u}{\partial z}=0.
\]

\end{examproblems}

\end{document}
