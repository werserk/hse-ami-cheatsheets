% cheatsheet-template.tex — Template for cheatsheets
% Project: HSE AMI Cheatsheets
% Author: werserk
% Version: 2.0
%
% This template is specifically designed for cheatsheets with two-column layout

% =============================================================================
% DOCUMENT CLASS AND BASIC PACKAGES
% =============================================================================

\documentclass[10pt,landscape,a4paper]{article}

% Basic LaTeX packages
\usepackage[utf8]{inputenc}
\usepackage[T2A]{fontenc}
\usepackage[russian]{babel}

% =============================================================================
% UNIFIED STYLING SYSTEM
% =============================================================================

% Load the unified styling system
\usepackage{../../assets/styles/hse-unified}

% =============================================================================
% CHEATSHEET-SPECIFIC CONFIGURATION
% =============================================================================

% Geometry settings for cheatsheets
\usepackage{geometry}
\geometry{margin=1cm}
\setlength{\columnsep}{1cm}
\setlength{\columnseprule}{0.5pt}

% Page style - no page numbers for cheatsheets
\pagestyle{empty}

% =============================================================================
% DOCUMENT METADATA
% =============================================================================

% Document title and subtitle (customize these)
\newcommand{\DocumentTitle}{Название Cheatsheet}
\newcommand{\DocumentSubtitle}{Краткое описание темы}

% Contributors (optional)
\newcommand{\contributors}{}

% Accent color (optional, defaults to FF7517)
% \HSESetAccent{FF7517}

% =============================================================================
% DOCUMENT CONTENT
% =============================================================================

\begin{document}

% Document title
\HSETitle{\DocumentTitle}{\DocumentSubtitle}

% Two-column layout for cheatsheets
\HSETwoColsBegin

% =============================================================================
% LEFT COLUMN
% =============================================================================

\section{Основные понятия}

\subsection{Определения}
\begin{itemize}
    \item \textbf{Понятие 1:} Определение...
    \item \textbf{Понятие 2:} Определение...
\end{itemize}

\subsection{Формулы}
\begin{align}
    E &= mc^2 \\
    F &= ma \\
    \int_a^b f(x) dx &= F(b) - F(a)
\end{align}

\subsection{Таблица данных}
\begin{center}
\small
\begin{tabular}{@{}ll@{}}
\toprule
Параметр & Значение \\
\midrule
$\pi$ & 3.14159... \\
$e$ & 2.71828... \\
$c$ & 299,792,458 м/с \\
\bottomrule
\end{tabular}
\end{center}

\section{Практические примеры}

\subsection{Пример 1}
Описание примера...

\subsection{Пример 2}
Описание примера...

% =============================================================================
% RIGHT COLUMN
% =============================================================================

\section{Дополнительные формулы}

\subsection{Важные константы}
\begin{itemize}
    \item $\pi \approx 3.14159$
    \item $e \approx 2.71828$
    \item $c = 299,792,458$ м/с
    \item $h = 6.626 \times 10^{-34}$ Дж·с
\end{itemize}

\subsection{Методы решения}
\begin{enumerate}
    \item Метод 1: Описание
    \item Метод 2: Описание
    \item Метод 3: Описание
\end{enumerate}

\subsection{Полезные тождества}
\begin{align}
    \sin^2 x + \cos^2 x &= 1 \\
    e^{i\pi} + 1 &= 0 \\
    \sum_{k=1}^n k &= \frac{n(n+1)}{2}
\end{align}

\section{Общие замечания}

\subsection{Применение}
Краткое описание области применения...

\subsection{Ограничения}
Важные ограничения и условия...

\subsection{Ссылки}
\begin{itemize}
    \item Дополнительная литература
    \item Полезные ресурсы
\end{itemize}

% End two-column layout
\HSETwoColsEnd

% =============================================================================
% FOOTER
% =============================================================================

\HSEFooter

\end{document}
