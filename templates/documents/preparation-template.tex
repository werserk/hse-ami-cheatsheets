% preparation-template.tex — Template for preparation materials
% Project: HSE AMI Cheatsheets
% Author: werserk
% Version: 2.0
%
% This template is specifically designed for preparation materials with modular content

% =============================================================================
% DOCUMENT CLASS AND BASIC PACKAGES
% =============================================================================

\documentclass[11pt,a4paper]{article}

% Basic LaTeX packages
\usepackage[utf8]{inputenc}
\usepackage[T2A]{fontenc}
\usepackage[russian]{babel}

% =============================================================================
% UNIFIED STYLING SYSTEM
% =============================================================================

% Load the unified styling system
\usepackage{../../assets/styles/hse-unified}

% =============================================================================
% PREPARATION-SPECIFIC CONFIGURATION
% =============================================================================

% Geometry settings for preparation materials
\usepackage{geometry}
\geometry{margin=1.8cm}

% Page style
\usepackage{fancyhdr}
\pagestyle{fancy}
\fancyhf{}
\fancyfoot[L]{\textbf{by werserk}}
\fancyfoot[R]{\thepage}
\renewcommand{\headrulewidth}{0pt}

% =============================================================================
% DOCUMENT METADATA
% =============================================================================

% Document title and subtitle (customize these)
\newcommand{\DocumentTitle}{Подготовка: Название предмета}
\newcommand{\DocumentSubtitle}{Полная версия с разборами тем и ссылками}

% Contributors (optional)
\newcommand{\contributors}{}

% Topics base path (customize as needed)
\newcommand{\topicsBase}{preparation/topics}

% =============================================================================
% DOCUMENT CONTENT
% =============================================================================

\begin{document}

% Document title
\HSETitle{\DocumentTitle}{\DocumentSubtitle}

\vspace{1cm}

% Table of contents
{
    \hypersetup{linkcolor=black}
    \tableofcontents
}

\newpage

% =============================================================================
% MODULAR CONTENT INCLUSION
% =============================================================================

% Include topic files (customize as needed)
% Each topic should be a separate .tex file in the topics/ directory
% Use descriptive filenames with leading zeros for ordering

% \input{\topicsBase/00-introduction}
% \input{\topicsBase/01-basic-concepts}
% \input{\topicsBase/02-methods}
% \input{\topicsBase/03-examples}
% \input{\topicsBase/04-practice}

% =============================================================================
% EXAMPLE TOPIC STRUCTURE
% =============================================================================

\section{Введение в предмет}\seclabel{introduction}

\HSEDefinition{\Term{Предмет} — краткое определение предмета и его основных понятий.}

\begin{NotationSection}
\NotationEntry{x_t}{последовательность значений}
\NotationEntry{f(x)}{функция от переменной x}
\NotationEntry{\reals}{множество действительных чисел}
\end{NotationSection}

\subsection{Основные понятия}\seclabel{basic-concepts}

\HSEExample{\example Решите простое уравнение $y_{t+1} = 2y_t$.

\textbf{Решение:}
\begin{ProblemSteps}
\item Запишем характеристическое уравнение: $\lambda = 2$
\item Общее решение: $y_t = C \cdot 2^t$
\end{ProblemSteps}

\textbf{Ответ:} $y_t = C \cdot 2^t$, где $C \in \reals$.}

\subsection{Методы решения}\seclabel{solution-methods}

\HSEAlgorithm{\textbf{Алгоритм решения.}
\begin{ProblemSteps}
\item Шаг 1: Анализ уравнения
\item Шаг 2: Поиск характеристических корней
\item Шаг 3: Запись общего решения
\item Шаг 4: Определение констант из начальных условий
\end{ProblemSteps}
}

% =============================================================================
% FOOTER
% =============================================================================

\HSEFooter

\end{document}
