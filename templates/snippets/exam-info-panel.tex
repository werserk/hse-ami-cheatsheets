\documentclass[12pt]{article}
\usepackage[utf8]{inputenc}
\usepackage[russian]{babel}
\usepackage{amsmath,amssymb,mathtools}
\usepackage{geometry}
\geometry{margin=2.5cm}

\begin{document}

% Информационная панель для экзаменационных вариантов
% Использование: \examinfopanel{год}{вариант}{время}{баллы}
% Размещается в начале документа после заголовка

\newcommand{\examinfopanel}[4]{%
\vspace{0.5em}
\noindent\rule{\textwidth}{0.4pt}
\vspace{0.3em}
\noindent\textbf{Дифференциальные уравнения} \hfill \textbf{#1 год} \hfill \textbf{#2}\\
\noindent\textit{Лектор: Букин К.А.} \hfill \textit{Время: #3} \hfill \textit{Всего баллов: #4}
\vspace{0.3em}
\noindent\rule{\textwidth}{0.4pt}
\vspace{0.5em}
}

% Примеры использования:
% \examinfopanel{2025}{Демонстрационный вариант А}{120 минут}{100}
% \examinfopanel{2025}{Вариант $\phi$}{120 минут}{100}
% \examinfopanel{2024}{Вариант $\epsilon$}{120 минут}{100}

% Демонстрация макроса:
\begin{center}
\textbf{Пример использования макроса}
\end{center}

\examinfopanel{2025}{Демонстрационный вариант А}{120 минут}{100}

\end{document}
