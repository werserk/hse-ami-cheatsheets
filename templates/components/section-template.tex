% section_template.tex — Шаблон раздела с инструкциями по стилю
% Project: HSE AMI Cheatsheets
% Usage: Копируйте этот файл и заполните согласно инструкциям

% =============================================================================
% ИНСТРУКЦИИ ПО ИСПОЛЬЗОВАНИЮ
% =============================================================================

% 1. ЗАМЕНИТЕ все заглушки [ЗАМЕНИТЬ] на реальный контент
% 2. УДАЛИТЕ все комментарии-инструкции после заполнения
% 3. ПРОВЕРЬТЕ соответствие style-guide.md
% 4. УБЕДИТЕСЬ что все метки уникальны и соответствуют конвенциям

% =============================================================================
% ЗАГОЛОВОК РАЗДЕЛА
% =============================================================================

\section{[ЗАМЕНИТЬ: Название раздела]}\seclabel{[ЗАМЕНИТЬ: section-name]}

% ПРАВИЛА ЗАГОЛОВКОВ:
% - Используйте единственное число для терминов
% - Избегайте разговорных выражений
% - Примеры: "Разностные уравнения", "Системы дифференциальных уравнений"

% =============================================================================
% ОПРЕДЕЛЕНИЯ И ОСНОВНЫЕ ПОНЯТИЯ
% =============================================================================

\HSEDefinition{\Term{[ЗАМЕНИТЬ: Ключевое понятие]} — [ЗАМЕНИТЬ: формальное определение].}

% ПРАВИЛА ОПРЕДЕЛЕНИЙ:
% - Используйте \Term{} для выделения термина при первом использовании
% - Определение должно быть формальным и точным
% - Избегайте разговорных выражений

% =============================================================================
% ОБОЗНАЧЕНИЯ (если нужны новые символы)
% =============================================================================

\begin{NotationSection}
\NotationEntry{[ЗАМЕНИТЬ: символ]}{[ЗАМЕНИТЬ: описание]}
\NotationEntry{[ЗАМЕНИТЬ: символ]}{[ЗАМЕНИТЬ: описание]}
\end{NotationSection}

% ПРАВИЛА ОБОЗНАЧЕНИЙ:
% - Объявляйте ВСЕ новые математические символы
% - Используйте стандартные обозначения: \reals, \naturals, \complex
% - Индекс времени: только \timeIndex{t} или \timeIndex{T} (НЕ n!)

% =============================================================================
% ПОДРАЗДЕЛЫ
% =============================================================================

\subsection{[ЗАМЕНИТЬ: Название подраздела]}\seclabel{[ЗАМЕНИТЬ: subsection-name]}

% Содержимое подраздела...

\subsubsection{[ЗАМЕНИТЬ: Название подподраздела]}\seclabel{[ЗАМЕНИТЬ: subsubsection-name]}

% Содержимое подподраздела...

% =============================================================================
% ПРИМЕРЫ
% =============================================================================

\HSEExample{\example [ЗАМЕНИТЬ: Краткое описание примера]

[ЗАМЕНИТЬ: Условие примера]

\textbf{Решение:}
\begin{ProblemSteps}
\item [ЗАМЕНИТЬ: Шаг 1]
\item [ЗАМЕНИТЬ: Шаг 2]
\item [ЗАМЕНИТЬ: Шаг 3]
\end{ProblemSteps}

\textbf{Ответ:} [ЗАМЕНИТЬ: Итоговый ответ]
}

% ПРАВИЛА ПРИМЕРОВ:
% - Структура: Условие → Решение (пошагово) → Ответ
% - Используйте \begin{ProblemSteps} для пошагового решения
% - Нумерация шагов автоматическая

% =============================================================================
% АЛГОРИТМЫ
% =============================================================================

\HSEAlgorithm{\textbf{Алгоритм [ЗАМЕНИТЬ: название алгоритма].}
\begin{ProblemSteps}
\item [ЗАМЕНИТЬ: Шаг 1 алгоритма]
\item [ЗАМЕНИТЬ: Шаг 2 алгоритма]
\item [ЗАМЕНИТЬ: Шаг 3 алгоритма]
\end{ProblemSteps}
}

% ПРАВИЛА АЛГОРИТМОВ:
% - Четкая последовательность шагов
% - Используйте \begin{ProblemSteps} для структурирования
% - Каждый шаг должен быть конкретным и выполнимым

% =============================================================================
% ТЕОРЕМЫ И ЛЕММЫ
% =============================================================================

\begin{theorem}\thmlabel{[ЗАМЕНИТЬ: theorem-name]}
[ЗАМЕНИТЬ: Формулировка теоремы]
\end{theorem}

\begin{proof}
[ЗАМЕНИТЬ: Доказательство теоремы]
\end{proof}

% ПРАВИЛА ТЕОРЕМ:
% - Используйте стандартные окружения: theorem, lemma, corollary
% - Обязательно добавляйте метки \thmlabel{}
% - Доказательства оформляйте в окружении proof

% =============================================================================
% ТАБЛИЦЫ
% =============================================================================

\begin{StandardTable}[h!]{[ЗАМЕНИТЬ: Название таблицы]}{[ЗАМЕНИТЬ: table-name]}
{[ЗАМЕНИТЬ: c|c|c]} % Замените на нужное выравнивание колонок
[ЗАМЕНИТЬ: Заголовок 1] & [ЗАМЕНИТЬ: Заголовок 2] & [ЗАМЕНИТЬ: Заголовок 3] \\
\midrule
[ЗАМЕНИТЬ: Данные 1] & [ЗАМЕНИТЬ: Данные 2] & [ЗАМЕНИТЬ: Данные 3] \\
[ЗАМЕНИТЬ: Данные 1] & [ЗАМЕНИТЬ: Данные 2] & [ЗАМЕНИТЬ: Данные 3] \\
\end{StandardTable}

% ПРАВИЛА ТАБЛИЦ:
% - Используйте \begin{StandardTable} для единообразия
% - Обязательно добавляйте \tablabel{} для ссылок
% - Используйте booktabs стиль (\toprule, \midrule, \bottomrule)

% =============================================================================
% МАТЕМАТИЧЕСКИЕ ФОРМУЛЫ
% =============================================================================

% Простые формулы
[ЗАМЕНИТЬ: $y = f(x)$]

% Нумерованные уравнения
\begin{equation}\eqlabel{[ЗАМЕНИТЬ: equation-name]}
[ЗАМЕНИТЬ: Математическая формула]
\end{equation}

% Системы уравнений
\begin{align}
[ЗАМЕНИТЬ: Уравнение 1] \eqlabel{[ЗАМЕНИТЬ: eq1-name]} \\
[ЗАМЕНИТЬ: Уравнение 2] \eqlabel{[ЗАМЕНИТЬ: eq2-name]}
\end{align}

% ПРАВИЛА ФОРМУЛ:
% - Все важные формулы должны быть пронумерованы
% - Используйте \eqlabel{} для ссылок на уравнения
% - Индекс времени: только \timeIndex{t} или \timeIndex{T}

% =============================================================================
% КОД (если нужен)
% =============================================================================

\CodeBlock{[ЗАМЕНИТЬ: Код программы]}{[ЗАМЕНИТЬ: Язык программирования]}

% ПРАВИЛА КОДА:
% - Используйте \CodeBlock{код}{язык} для блоков кода
% - Комментарии в коде на русском языке
% - Включайте префикс # RUN для исполняемых команд

% =============================================================================
% ЗАМЕЧАНИЯ И ДОПОЛНИТЕЛЬНАЯ ИНФОРМАЦИЯ
% =============================================================================

\HSENote{[ЗАМЕНИТЬ: Важное замечание или дополнительная информация]}

% =============================================================================
% ССЫЛКИ НА ДРУГИЕ РАЗДЕЛЫ
% =============================================================================

% Ссылки на другие разделы документа
См. также \secref{[ЗАМЕНИТЬ: other-section-name]}.

% Ссылки на уравнения
Уравнение \eqref{[ЗАМЕНИТЬ: equation-name]} показывает...

% Ссылки на таблицы
В таблице \tabref{[ЗАМЕНИТЬ: table-name]} приведены...

% ПРАВИЛА ССЫЛОК:
% - Используйте соответствующие команды: \secref, \eqref, \tabref
% - Все ссылки должны иметь соответствующие метки
% - Проверяйте, что все ссылки разрешены

% =============================================================================
% ПРОВЕРКА СООТВЕТСТВИЯ СТИЛЮ
% =============================================================================

% ПЕРЕД ЗАВЕРШЕНИЕМ ПРОВЕРЬТЕ:
% □ Все [ЗАМЕНИТЬ] заменены на реальный контент
% □ Используется только \timeIndex{t} или \timeIndex{T} (НЕ n!)
% □ Все новые символы объявлены в разделе "Обозначения"
% □ Структура задач: Условие → Решение → Ответ
% □ Все метки уникальны и соответствуют конвенциям
% □ Используется формальная терминология
% □ Все ссылки (\ref, \cite) разрешены
% □ Документ успешно компилируется

% =============================================================================
% КОНЕЦ ШАБЛОНА
% =============================================================================
