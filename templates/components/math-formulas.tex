\documentclass[10pt,landscape,a4paper]{article}
\usepackage[utf8]{inputenc}
\usepackage[russian]{babel}
\usepackage{amsmath}
\usepackage{amsfonts}
\usepackage{amssymb}
\usepackage{geometry}
\usepackage{multicol}
\usepackage{graphicx}
\usepackage{xcolor}
\usepackage{titlesec}
\usepackage{enumitem}
\usepackage{booktabs}
\usepackage{array}
\usepackage{mathtools}

% Настройка страницы
\geometry{margin=1cm}
\setlength{\columnsep}{1cm}
\setlength{\columnseprule}{0.5pt}

% Настройка заголовков
\titleformat{\section}{\Large\bfseries\color{red!70!black}}{\thesection}{1em}{}
\titleformat{\subsection}{\large\bfseries\color{red!50!black}}{\thesubsection}{1em}{}
\titleformat{\subsubsection}{\normalsize\bfseries\color{red!30!black}}{\thesubsubsection}{1em}{}

% Убираем номера страниц
\pagestyle{empty}

% Настройка списков
\setlist[itemize]{leftmargin=*,noitemsep,topsep=0pt}
\setlist[enumerate]{leftmargin=*,noitemsep,topsep=0pt}

% Математические операторы
\DeclareMathOperator{\grad}{grad}
\DeclareMathOperator{\divergence}{div}
\DeclareMathOperator{\curl}{curl}
\DeclareMathOperator{\laplacian}{\Delta}

\begin{document}

% Заголовок
\begin{center}
    {\Huge\bfseries Математические Формулы}\\[0.5cm]
    {\large Справочник по основным математическим формулам}\\[0.3cm]
    \rule{\textwidth}{1pt}
\end{center}

\begin{multicols}{2}

\section{Алгебра}

\subsection{Квадратные уравнения}
\begin{align}
    ax^2 + bx + c &= 0 \\
    x_{1,2} &= \frac{-b \pm \sqrt{b^2 - 4ac}}{2a} \\
    D &= b^2 - 4ac
\end{align}

\subsection{Логарифмы}
\begin{align}
    \log_a(xy) &= \log_a x + \log_a y \\
    \log_a\left(\frac{x}{y}\right) &= \log_a x - \log_a y \\
    \log_a(x^n) &= n \log_a x \\
    \log_a a &= 1, \quad \log_a 1 = 0
\end{align}

\section{Тригонометрия}

\subsection{Основные тождества}
\begin{align}
    \sin^2 x + \cos^2 x &= 1 \\
    \tan x &= \frac{\sin x}{\cos x} \\
    \cot x &= \frac{\cos x}{\sin x} \\
    1 + \tan^2 x &= \sec^2 x \\
    1 + \cot^2 x &= \csc^2 x
\end{align}

\subsection{Формулы сложения}
\begin{align}
    \sin(a \pm b) &= \sin a \cos b \pm \cos a \sin b \\
    \cos(a \pm b) &= \cos a \cos b \mp \sin a \sin b \\
    \tan(a \pm b) &= \frac{\tan a \pm \tan b}{1 \mp \tan a \tan b}
\end{align}

\section{Дифференциальное исчисление}

\subsection{Производные}
\begin{align}
    \frac{d}{dx}(c) &= 0 \\
    \frac{d}{dx}(x^n) &= nx^{n-1} \\
    \frac{d}{dx}(e^x) &= e^x \\
    \frac{d}{dx}(\ln x) &= \frac{1}{x} \\
    \frac{d}{dx}(\sin x) &= \cos x \\
    \frac{d}{dx}(\cos x) &= -\sin x
\end{align}

\subsection{Правила дифференцирования}
\begin{align}
    (f \pm g)' &= f' \pm g' \\
    (fg)' &= f'g + fg' \\
    \left(\frac{f}{g}\right)' &= \frac{f'g - fg'}{g^2} \\
    (f(g(x)))' &= f'(g(x)) \cdot g'(x)
\end{align}

\section{Интегральное исчисление}

\subsection{Основные интегралы}
\begin{align}
    \int k \, dx &= kx + C \\
    \int x^n \, dx &= \frac{x^{n+1}}{n+1} + C \quad (n \neq -1) \\
    \int e^x \, dx &= e^x + C \\
    \int \frac{1}{x} \, dx &= \ln|x| + C \\
    \int \sin x \, dx &= -\cos x + C \\
    \int \cos x \, dx &= \sin x + C
\end{align}

\subsection{Методы интегрирования}
\begin{align}
    \int f(g(x))g'(x) \, dx &= F(g(x)) + C \\
    \int u \, dv &= uv - \int v \, du \\
    \int_a^b f(x) \, dx &= F(b) - F(a)
\end{align}

\section{Векторная алгебра}

\subsection{Скалярное произведение}
\begin{align}
    \vec{a} \cdot \vec{b} &= |\vec{a}||\vec{b}|\cos\theta \\
    \vec{a} \cdot \vec{b} &= a_1b_1 + a_2b_2 + a_3b_3
\end{align}

\subsection{Векторное произведение}
\begin{align}
    \vec{a} \times \vec{b} &= |\vec{a}||\vec{b}|\sin\theta \hat{n} \\
    \vec{a} \times \vec{b} &= \begin{vmatrix}
        \hat{i} & \hat{j} & \hat{k} \\
        a_1 & a_2 & a_3 \\
        b_1 & b_2 & b_3
    \end{vmatrix}
\end{align}

\section{Комплексные числа}

\subsection{Основные операции}
\begin{align}
    z &= a + bi \\
    |z| &= \sqrt{a^2 + b^2} \\
    \overline{z} &= a - bi \\
    z_1 + z_2 &= (a_1 + a_2) + i(b_1 + b_2) \\
    z_1 \cdot z_2 &= (a_1a_2 - b_1b_2) + i(a_1b_2 + a_2b_1)
\end{align}

\subsection{Тригонометрическая форма}
\begin{align}
    z &= r(\cos\theta + i\sin\theta) \\
    z^n &= r^n(\cos(n\theta) + i\sin(n\theta)) \\
    \sqrt[n]{z} &= \sqrt[n]{r}\left(\cos\frac{\theta + 2k\pi}{n} + i\sin\frac{\theta + 2k\pi}{n}\right)
\end{align}

\end{multicols}

\end{document}
