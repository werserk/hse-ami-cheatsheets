\documentclass[10pt,landscape,a4paper]{article}
\usepackage[utf8]{inputenc}
\usepackage[russian]{babel}
\usepackage{amsmath}
\usepackage{amsfonts}
\usepackage{amssymb}
\usepackage{geometry}
\usepackage{multicol}
\usepackage{graphicx}
\usepackage{xcolor}
\usepackage{titlesec}
\usepackage{enumitem}
\usepackage{booktabs}
\usepackage{array}
\usepackage{hyperref}
\usepackage{../../assets/styles/hse-cheatsheets}

% Настройка страницы
\geometry{margin=1cm}
\setlength{\columnsep}{1cm}
\setlength{\columnseprule}{0.5pt}

% Опционально: \HSESetAccent{FF7517}

% Настройка заголовков (без акцентного цвета)
\titleformat{\section}{\Large\bfseries}{\thesection}{1em}{}
\titleformat{\subsection}{\large\bfseries}{\thesubsection}{1em}{}
\titleformat{\subsubsection}{\normalsize\bfseries}{\thesubsubsection}{1em}{}

% Убираем номера страниц
\pagestyle{empty}

% Настройка списков
\setlist[itemize]{leftmargin=*,noitemsep,topsep=0pt}
\setlist[enumerate]{leftmargin=*,noitemsep,topsep=0pt}

\begin{document}

% Заголовок
\HSETitle{Название Cheatsheet}{Краткое описание темы}

% Опционально: \newcommand{\contributors}{Ivan Petrov; Botkin DS; @nick}

\HSETwoColsBegin

\section{Основные понятия}

\subsection{Определения}
\begin{itemize}
    \item \textbf{Понятие 1:} Определение...
    \item \textbf{Понятие 2:} Определение...
\end{itemize}

\subsection{Формулы}
\begin{align}
    E &= mc^2 \\
    F &= ma \\
    \int_a^b f(x) dx &= F(b) - F(a)
\end{align}

\section{Практические примеры}

\subsection{Пример 1}
Описание примера...

\subsection{Пример 2}
Описание примера...

\section{Таблицы и списки}

\subsection{Важные значения}
\begin{center}
\begin{tabular}{@{}ll@{}}
\toprule
Параметр & Значение \\
\midrule
$\pi$ & 3.14159... \\
$e$ & 2.71828... \\
$c$ & 299,792,458 м/с \\
\bottomrule
\end{tabular}
\end{center}

\subsection{Список формул}
\begin{enumerate}
    \item Формула 1: $y = mx + b$
    \item Формула 2: $a^2 + b^2 = c^2$
    \item Формула 3: $\sin^2(x) + \cos^2(x) = 1$
\end{enumerate}

\section{Дополнительная информация}

\subsection{Полезные ссылки}
\begin{itemize}
    \item Ссылка 1
    \item Ссылка 2
\end{itemize}

\subsection{Примечания}
Дополнительные замечания и комментарии...

\HSETwoColsEnd

% Футер: by werserk + contributors
\HSEFooter

\end{document}
