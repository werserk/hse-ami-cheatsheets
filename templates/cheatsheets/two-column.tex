\documentclass[10pt,landscape,a4paper]{article}
\usepackage[utf8]{inputenc}
\usepackage[russian]{babel}
\usepackage{amsmath}
\usepackage{amsfonts}
\usepackage{amssymb}
\usepackage{geometry}
\usepackage{multicol}
\usepackage{graphicx}
\usepackage{xcolor}
\usepackage{titlesec}
\usepackage{enumitem}
\usepackage{booktabs}
\usepackage{array}
\usepackage{parskip}
\usepackage{hyperref}

% Настройка страницы
\geometry{margin=1.5cm}
\setlength{\columnsep}{1.5cm}
\setlength{\columnseprule}{0.8pt}

% Акцентный цвет
\definecolor{accent}{HTML}{FF7517}
% Цвета ссылок
\hypersetup{colorlinks=true, linkcolor=accent, urlcolor=accent, citecolor=accent}

% Настройка заголовков (без акцентного цвета)
\titleformat{\section}{\Large\bfseries}{\thesection}{1em}{}
\titleformat{\subsection}{\large\bfseries}{\thesubsection}{1em}{}
\titleformat{\subsubsection}{\normalsize\bfseries}{\thesubsubsection}{1em}{}

% Убираем номера страниц
\pagestyle{empty}

% Настройка списков
\setlist[itemize]{leftmargin=*,noitemsep,topsep=0pt}
\setlist[enumerate]{leftmargin=*,noitemsep,topsep=0pt}

% Настройка отступов между параграфами
\setlength{\parskip}{0.3em}

\begin{document}

% Заголовок
\begin{center}
    {\Huge\bfseries Двухколоночный Cheatsheet}\\[0.5cm]
    {\large Оптимизированный для плотной информации}\\[0.3cm]
    \vspace{0.3cm}
    \rule{\textwidth}{1pt}
\end{center}

% Опционально: перед сборкой можно определить список вкладчиков, например:
% \newcommand{\contributors}{Ivan Petrov; Botkin DS; @nick}
% По умолчанию оставлено пустым
\providecommand{\contributors}{}

\begin{multicols}{2}

\section{Левая колонка}

\subsection{Основные понятия}
\begin{itemize}
    \item \textbf{Термин 1:} Краткое определение
    \item \textbf{Термин 2:} Краткое определение
    \item \textbf{Термин 3:} Краткое определение
\end{itemize}

\subsection{Формулы}
\begin{align}
    y &= ax + b \\
    z &= \sqrt{x^2 + y^2} \\
    f(x) &= \int_0^x g(t) dt
\end{align}

\subsection{Таблица данных}
\begin{center}
\small
\begin{tabular}{@{}ll@{}}
\toprule
Параметр & Значение \\
\midrule
$\alpha$ & 0.05 \\
$\beta$ & 0.95 \\
$\gamma$ & 1.4 \\
\bottomrule
\end{tabular}
\end{center}

\subsection{Список правил}
\begin{enumerate}
    \item Правило 1
    \item Правило 2
    \item Правило 3
    \item Правило 4
\end{enumerate}

\section{Правая колонка}

\subsection{Дополнительные формулы}
\begin{align}
    E &= \frac{1}{2}mv^2 \\
    P &= \frac{V^2}{R} \\
    \nabla \cdot \vec{E} &= \frac{\rho}{\varepsilon_0}
\end{align}

\subsection{Важные константы}
\begin{itemize}
    \item $\pi \approx 3.14159$
    \item $e \approx 2.71828$
    \item $c = 299,792,458$ м/с
    \item $h = 6.626 \times 10^{-34}$ Дж·с
\end{itemize}

\subsection{Методы решения}
\begin{enumerate}
    \item Метод 1: Описание
    \item Метод 2: Описание
    \item Метод 3: Описание
\end{enumerate}

\subsection{Полезные тождества}
\begin{align}
    \sin^2 x + \cos^2 x &= 1 \\
    e^{i\pi} + 1 &= 0 \\
    \sum_{k=1}^n k &= \frac{n(n+1)}{2}
\end{align}

\section{Общие замечания}

\subsection{Применение}
Краткое описание области применения...

\subsection{Ограничения}
Важные ограничения и условия...

\subsection{Ссылки}
\begin{itemize}
    \item Дополнительная литература
    \item Полезные ресурсы
\end{itemize}

\end{multicols}

% Футер: by werserk (ссылка, акцентный цвет) + contributors
{\noindent by \href{https://werserk.com}{\textcolor{accent}{\textbf{werserk}}} \hfill contributors: \texttt{\contributors}}

\end{document}
